\documentclass[ebook,oneside,final,openright]{memoir}

\usepackage{xunicode}
\usepackage{graphicx}
\usepackage{lettrine}

\usepackage{polyglossia}
\setmainlanguage{russian}
\setotherlanguage{english}

\setmainfont{Calibri}
\setsansfont{Calibri}
\setmonofont{Consolas}

%% -------------------------------------------------------------------------- %%

\begin{document}

\frontmatter
\aliaspagestyle{title}{empty}
\title{\Huge{Сказки ботов}}
\author{
  Story template by Timofey Yarovoy and Andrey Shkolnikov,\\
  generated with code by Alexander Gladysh
}
\date{LogicEditor\\NaNoGenMo 2016}

\maketitle

\mainmatter
\chapter{}
 \lettrine{У}{некоторой звезды, на четвертой от нее планете,} жил один насекомец, звали его Полуэкт. Был он могучий ученый и был на редкость умен.\par
\par
Как-то раз услышал он, что у одной черной дыры, такой черной, что чернее не бывает, находится великая вычислительная машина, знающая ответы на все вопросы. И надумал Полуэкт ее себе заполучить, чтобы не попалась она в чужие руки и не вышло большой беды для всей Вселенной. Да как узнать, где в точности найти машину вычислительную? Звезд да черных дыр во Вселенной ужас как много, и вокруг каждой куча планет да астероидов вертится!\par
\par
Начал Полуэкт смекать, у кого информацию нужную добыть можно. Думал-думал, да надумал обратиться к мудрецу с планеты Ерундения по имени Завздыпопус, славившемуся своими познаниями о космосе. Взял он свой электробаян, с коим любил коротать время, и помчался к Завздыпопусу.\par
\par
Вскорости нашел Полуэкт способ с ним повстречаться. Так, мол, и так, сказывает Полуэкт, хочу я найти машину вычислительную, сокровище это удивительное, и все мысли мои теперь только об этом. Только вот проблема – знать не знаю, как!\par
\par
«Что ж, – говорит ему Завздыпопус, – вижу я, ты очень ловок, и IQ твой вышиной до звезд простирается! Впрочем, ты насекомец, а насекомцы этим известны, да еще упертостью своей. Только не знаю я, как помочь тебе. Но есть сестра у меня, мудрости столь необычной, что моя мудрость по сравнению с её – логарифмическая линейка по сравнению с суперкомпьютером. Живет она у соседней звезды, второй поворот налево, если отсюда к краю Галактики лететь. Принеси ей подарков да украшений дорогих – может, поможет она тебе».\par
\par
Собрал Полуэкт с собой подарки да украшения, добавил к ним кольцо из цельнометаллического водорода сделанное, с формулой Вселенной выгравированной, и в путь пустился.\par
\par
Прилетает он к сестре Завздыпопуса, отдает ей подарки, и письмо от Завздыпопуса вручает. «Так, мол, и так, – говорит Полуэкт, – очень хочется мне найти машину вычислительную, сокровище это необычное, и мысли мои самые что ни на есть благородные.»\par
\par
«Погоди-ка, – говорит сестра, – тут в письме написано, что б я тебе голову отрубила, а не помогать стала!.. Ах нет, извини, просто письмо с другой стороны какого-то черновика написано, там даже печать есть... Ладно, помогу я тебе, хоть и не стоишь ты этого, насекомец! Да очень уж мне подарки твои понравились, особенно кольцо из цельнометаллического водорода сделанное, с формулой Вселенной выгравированной.\par
\par
Путь твой далек будет. Мимо галактик, в спирали закрученных, мимо туманностей звездных, дивным светом сияющих, мимо квазаров грозных, сигналы чудные излучающих, к самому краю видимой Вселенной, где лишь протоны да альфа-частицы шныряют, а звезду и на миллион парсек не встретишь. И всё же есть там звездочка одна, в туманности газо-пылевой спрятавшаяся. А вокруг звезды той маленькая планета вращается, а вокруг планеты – спутник вертится. И на спутнике том кратер есть круглый да огромный. И на дне кратера того Врата стоят Звездные. И ведут эти Врата к тому, что ты найти так жаждешь.\par
\par
Только просто так во Врата не пройти. Лабиринт вкруг тех Врат выстроен в сто этажей да в десять тысяч комнат на каждом. Да такой хитрый, что как войдешь в него, так и заблудишься сразу. И как сквозь тот лабиринт пройти, мне неведомо.\par
\par
Поэтому трудно тебе придется. Но коли сумеешь до Врат добраться и через них пройти, окажешься в зале с потолком таким высоким, что и не видно. И будут пред тобой три двери – две больших да красивых, а третья – маленькая да невзрачная.\par
\par
На первой двери, золотом украшенной, нарисован будет ядерный котел над очагом звездным. За этой дверью Изменитель реальности, невесть кем построенный. Сунешь свой нос в эту дверь – на веки сгинешь. Но если уж не послушаешь моего совета и заглянешь туда – руками ничего не трогай, а то и вся Вселенная наша переиначиться может.\par
\par
На второй двери, алмазами выложенной, енот с пулеметом наклеен. За этой дверью биолаборатория заброшенная, в которой ксеноморфов да всяких хищников страшных выводили. Они и сейчас там бродят. Такому герою, как ты, туда зайти – заживо съеденным быть. Но уж если не послушаешь доброго совета да заглянешь туда – из пробирок не пей – ксеноморфиком станешь, или шай-хулудом каким.\par
\par
На третьей двери, паутиной затянутой да звездной пылью засыпанной, ничего не нарисовано, только написано: «Не влезай, убьет!» За этой дверью найдешь ты машину вычислительную, сокровище, которое так найти стремишься.\par
\par
А чтоб ты с пути не сбился, дам я тебе комету путеводную, куда она полетит – туда и ты лети».\par
\par
Тут Полуэкт, не мешкая, в путь пустился. Летит он в подпространстве тропами нехожеными, измерениями неизвестными. Уж и не знает, трехмерный ли он до сих пор. Но не сдается, боится только одного – наизнанку вывернуться.\par
\par
Сколько световых лет прошло, неведомо, но долетел Полуэкт до звездочки заветной. Рассчитал он курс, на нужную орбиту лег, к высадке подготовился.\par
\par
Высадился Полуэкт рядом с лабиринтом, добрался до входа и внутрь вошел. Идет одним коридором, другим, третьим. Пусто везде, тихо, только его шаги эхом отдаются. До комнаты какой-то дошел, видит – дальше три коридора ведут, и в каждом будто туман клубится. А на полу написано: «Коли дураком не хочешь стать – ступай направо, либо прямо. Коли голову потерять не хочешь – ступай прямо, либо налево. Коли до смерти запуганным быть не хочешь – ступай налево, либо направо».\par
\par
Задумался Полуэкт, куда идти, да и пошел налево. Идет, по коридорам топает, с этажа на этаж перебирается. А туман вокруг не рассеивается. Долго он так шел, уж стал думать, что батарейки в часах наручных скоро сядут. Дошел, наконец, до какой-то комнаты, а в комнате той будто битва великая кипела, потолок осыпался, в полу дыра в пол-комнаты. Из комнаты две двери ведут. До одной не добраться, поперек другой дракон трехголовый спит, а рядом с ним меч здоровенный двуручный валяется. Схватил Полуэкт меч за рукоятку, а тот тяжеленный, по полу как заскрежещет. Дракон вмиг проснулся.\par
\par
– Чего тебе надобно? – спрашивает.\par
– Да вот, пройти хочу, – Полуэкт отвечает.\par
– А меч тебе зачем? Ты что, совсем дурак? Попросил бы – я б подвинулся.\par
– Да я его хотел через дыру в полу перекинуть – навроде мостика.\par
– А, так тебе в ту дверь надо? Впрочем, все равно дурак – меч размером коротковат.\par
– Да мне все равно, в какую дверь, мне б только до Звездных Врат добраться. И что вообще у вас тут творится?\par
– Игра у нас тут идет большая.\par
– Да что за игра-то?\par
– Да так… Впрочем, раз уж ты ко мне пришел, раунд за мной, идем – расскажу тебе про игру.\par
\par
Выходит дракон в дверь, становится слегка туманным и идет дальше коридорами. Полуэкт за ним еле поспевает. До очередной двери дошли, дракон – туда, и Полуэкт за ним. \par
Входит Полуэкт в комнату – а там нет никого, лишь три сгустка тумана поплотнее. И как будто двое одному что-то вроде монет туманных передают.\par
– И где же тут кто? – Полуэкт спрашивает.\par
– Да мы тут везде, но здесь особенно, – отвечают три голоса, да прямо в голове звучат.\par
– И кто же вы такие будете?\par
– Мы – представители древней цивилизации, раса наша настолько древняя, что телесно уже и не существует вовсе. Только ментально, то есть разумом своим. И знаем мы все тайны Вселенной, и все предсказать да рассчитать можем. И скучно нам от этого необычайно. И даже говорим мы длинно и скучно, как ты мог заметить. Пробовали в рулетку играть – да каждый знает, куда шарик прикатится. Пробовали в квантовое лото играть – так и принцип неопределенности квантовый для нас не помеха в предсказаниях.\par
– А здесь-то вы что делаете?\par
– Воздвигли мы силой разума лабиринт этот, чтоб скуку развеять можно было. Разумные существа, сюда попавшие, выбор делают. А мы играем, смотрим, по чьей дороге они пойдут. Ибо обладают разумные существа свободой воли, и тут наши предсказания бессильны. Так что давай, хватит отдыхать – видишь, три коридора отсюда тянутся – иди уж по какому-нибудь, да побыстрее!\par
– Не нужны мне ваши коридоры, мне к Звездным Вратам нужно!\par
– Будь любезен, не упрямься. Мы легко тебя заставить можем. Создадим мы сей же час чудовищ ментальных, ты кое-кого видел уже, и ни бластер, ни водяной пистолет тебе не помогут, поскольку будут чудовища внутри твоего разума, а не снаружи. А во сне тебе и вовсе тяжко придется.\par
\par
Видит Полуэкт – со всех сторон к нему уже когти и щупальца тянутся. Забился он в угол, да как закричит:\par
– Остановитесь, погодите! А кто из вас этих чудовищ делает?\par
– Как кто? Мы все трое.\par
– А чьи чудовища самые сильные будут?\par
Тут замерли чудовища на мгновение, а потом как начали друг с другом биться. Лапы с хвостами в разные стороны так и разлетаются. Драконы с демонами сшибаются, ангелы с гигантскими червями, орки и гоблины с эльфами да рыцарями, кальмары огромные с василисками. И над всем этим пегасы да орнитоптеры парят, и молнии сверкают. А внизу горы какие-то да болота с лесами мелькают. Даже один раз черный лотос виден был. \par
– Стойте! – Полуэкт кричит. – Меня ж сейчас тут совсем затопчут!\par
– Уйди, не мешайся! – три голоса отвечают. – Видишь левый коридор, там третий поворот направо и два раза налево – и придешь к своим Вратам Звездным.\par
\par
Побежал Полуэкт что есть духу, в минуту до Врат добрался.\par
\par
Прошел Полуэкт сквозь Врата – видит, три двери перед ним. Убрал он паутину с самой маленькой, пыль с нее отряхнул да внутрь вошел. Осмотрелся и увидел машину вычислительную, то сокровище, из-за которого покоя лишился. Бросился Полуэкт к сокровищу своему, тут вдруг сзади покашливание какое-то раздалось. Оглянулся – позади Завздыпопус с пола поднимается.\par
– Ох! – говорит Завздыпопус. – Хорошо, что ты сюда добрался, а то раньше никому не удавалось, я уж и со счета сбился.\par
– Завздыпопус! Ты-то здесь откуда? Да еще и на полу отдыхаешь.\par
– Так я ж тебе к правому ботинку микротелепортационный приемник прицепил, ну и 3D-видеокамеру с квадрофоническим микрофоном в придачу. Хоть и не верилось, что ты сюда доберешься. Да уж больно мне машину вычислительную раздобыть надо было. Пришлось вот даже ползком телепортироваться – слишком уж маленький портальчик сделался.\par
– Погоди! Это мне ее раздобыть надо было! Вот я здесь и оказался.\par
– Ты уж извини, Полуэкт, только мне это нужнее, – говорит Завздыпопус. Выхватывает парализатор и стреляет. Полуэкт сразу окаменел, ни рукой, ни ногой двинуть не может. Языком еле ворочает.\par
– Ой, – говорит, – ты что, супостат, делаешь?!\par
– Да я тебя в лабораторию ближайшую сейчас сдам – для опытов. Чтоб под ногами не путался.\par
Схватил Завздыпопус Полуэкта за шиворот и потащил в биолабораторию по соседству.\par
\par
Затащил он Полуэкта в дальний угол лаборатории, бросил там и к выходу направился. Да за что-то вроде зеленого кабеля зацепился. Тут сверху огромный цветок зубастый как упадет, Завздыпопус вмиг внутри цветка оказался, мычит что-то, ничего не разобрать.\par
\par
– Это что ж такое?! – Полуэкт спрашивает.\par
– Это я, растение говорящее, – голос отвечает.\par
– Да откуда ж ты взялось?\par
– Люди в белых халатах говорили, что я – интересная мутация. И что это поможет им в борьбе с артангами.\par
– А что еще они говорили?\par
– Последние их слова были: «Джейсон, ты не видел, куда Мэри Сью подевалась, наша новая лаборантка? Она просто гений! Странно, тут под цветком её туфелька валяется…»\par
– Слушай, выплюнь ты Завздыпопуса, а то тебе плохо будет. Он гербицид.\par
– Что он делает?\par
– Гербицид – для растений ядовит.\par
– Откуда ты знаешь? Да и вообще, кто ты такой?\par
– Да я тоже растение, куст говорящий. Видишь, шевелиться не могу. А этот человек меня поисследовать хотел. Ну, теперь я его поисследую, чтоб не важничал. Сейчас, погоди, проросту только немного.\par
– Хороший ты куст, тихий. И разговаривать умеешь. Ладно, на, исследуй свой гербицид.\par
\par
Распахнулся цветок, Завздыпопус оттуда вывалился, еле дышит. Полуэкт подождал, пока руки-ноги двигаться смогут, схватил Завздыпопуса, да бегом из лаборатории. За дверь выбежал, остановился, повернулся к Завздыпопусу. «Что – говорит – довыпендривался? Твое счастье, что я по вторникам кровавых жертв не приношу. До завтра подождем». Да как треснет Завздыпопуса в ухо.\par
\par
– Стой, погоди, Полуэкт! – кричит Завздыпопус. – Осознал я свою ошибку! Давай с начала начнем.\par
– Я тебе покажу с начала! Сейчас еще раз двину!\par
\par
Совсем перепугался тут Завздыпопус, заметался, убежать старается. А Полуэкт не отстает, того и гляди догонит и еще раз двинет. Подбежал Завздыпопус к первой двери, с котлом ядерным, и шасть за нее. И Полуэкт за ним.\par
\par
Глядит – механизм там стоит дивный, лампочками моргает, жужжит тихонько и готов в любую секунду реальность изменить. А Завздыпопус уже рычажки какие-то тянет и кнопки нажимает. Кинулся Полуэкт к Завздыпопусу, остановить хотел, да не успел. Прошла рябь по Вселенной и исчезли оба, как будто не было их тут вовсе. И история эта совсем иная стала...\par

\chapter{}
 \lettrine{В}{эпоху Великого Расселения} жил один насекомец, звали его Джо. Был он могучий злодей, но ума при этом был небольшого.\par
\par
И вот узнал он, что у какой-то из звезд находится великая вычислительная машина, знающая ответы на все вопросы. И захотел Джо ее себе забрать, чтобы закинуть в самый дальний угол подпространства и чтобы никто больше не мог разыскать это сокровище и не смущало оно умы смертных. Но где искать машину вычислительную? Звезд да черных дыр в галактиках ужас как много, и вокруг каждой великое множество планет да астероидов вертится!\par
\par
Принялся Джо смекать, у кого совета спросить. Думал-думал, и надумал обратиться к колдунье Морганионе, слава о деяниях которой гремела по всей Вселенной громким грохотом. Надел он свой парадный скафандр и помчался искать аудиенции у Морганионы.\par
\par
Много ли времени прошло, иль мало, но нашелся способ повстречаться с Морганионой. Так, мол, и так, сказывает Джо, хочется мне найти машину вычислительную, сокровище это великое, и все помыслы мои теперь лишь об этом. Да вот загвоздка – понятия не имею, как!\par
\par
«Вот что, – говорит ему Морганиона, – вижу я, ты весьма силен, да только глупость твою с другого конца Галактики видно! Впрочем, ты насекомец, а насекомцы этим известны, да еще упертостью своей. Но чтобы я тебе помогать стала, тебе службу сослужить надобно. Выполнишь мое задание – расскажу, как найти машину вычислительную, а нет – не обессудь!\par
\par
Внемли! Есть по соседству тут звезда нейтронная – третий поворот направо, если держать курс на Большую Медведицу. Рядом с ней планетоид карликовый, а на планетоиде том два чудовища живут. Зовут их Сцилла и Харибда, свирепые они до ужаса, и слюни, из пастей их гнусных извергающиеся, за два парсека видно. Есть у них волынка самогудная вакуумная, что играет музыку столь прекрасную, что даже звезды сверхновые, ее заслушавшись, взрываться перестают и в детство впадают. Не смыкая глаз, стерегут её чудовища. Добудь мне волынку, и расскажу я тебе, где найти машину вычислительную».\par
\par
Закручинился Джо, да делать нечего. Полетел к чудовищам волынку добывать. Летит, а сам боится, крупной дрожью дрожит, даже поворот нужный пропустил – пришлось возвращаться. А когда возвращался, увидал пустую канистру из-под топлива термоядерного, кем-то выброшенную. Возьму, думает, ее с собой – вдруг пригодится. Летит дальше – видит, кусок темной материи в пространстве висит, весь скомканный – наверное, купец какой-то потерял. И его, думает, возьму, тоже может на что сгодиться. Дальше летит – видит, компрессор пространственно-временной валяется – должно быть, странники какие-то оставили. И его тоже взял.\par
\par
Облетает звезду нейтронную, садится на планетоид, заходит в пещеру к чудовищам, а те сразу к нему кидаются. \par
– Чу, – рычат, – насекомьим духом пахнет! Как звать, – спрашивают, – откуда? Как хочешь быть проглоченным – ногами вперед или назад? Да отвечай побыстрее, а то проголодались мы чудовищно – сто лет уж никого не ели.\par
– Погодите вы с формальностями! – Джо отвечает. – Слышал я, есть у вас волынка самогудная, инструмент дивный. Сменяйте мне её на что-нибудь – очень уж мне надо.\par
– Что ты – с ума сошел? – спрашивают чудища. – Инструмент нам этот очень дорог.\par
– Что ж, – говорит Джо, – тогда дайте на инструмент этот ваш хоть одним глазком взглянуть, а я вам за это компрессор пространственно-временной подарю.\par
– И для чего это нам? – спрашивают.\par
– Так вы тогда что угодно во что угодно засунуть сможете. Даже вы двое в этой вот канистре поместитесь. \par
– Быть такого не может!\par
– Покажите волынку – увидите.\par
\par
Повели чудища его в самый дальний угол самой далекой пещеры, где волынку хранили. Посмотрел Джо на волынку. «Что ж, – говорит, – теперь вы смотрите». Приладил компрессор пространственно-временной к канистре и велел чудовищам туда прыгать. Прыгнули они, сидят в канистре, удивляются, что еще места много осталось. А Джо заткнул канистру куском темной материи, и даже бантик завязал.\par
\par
Взял он волынку, закинул канистру подальше в глубины космоса и бегом к Морганионе. Морганиона обрадовалась: «Ох, удружил ты мне, – говорит. – Расскажу я теперь тебе, как найти машину вычислительную.\par
\par
Далеко отсюда твой путь лежит. Мимо галактик, в спирали закрученных, мимо туманностей звездных, дивным светом сияющих, мимо квазаров грозных, сигналы чудные излучающих, к самому краю видимой Вселенной, где лишь протоны да альфа-частицы шныряют, а звезду и на миллион парсек не встретишь. И всё же есть там звездочка одна, в туманности газо-пылевой спрятавшаяся. А вокруг звезды той маленькая планета вращается, а вокруг планеты – спутник вертится. И на спутнике том кратер есть круглый да огромный. И на дне кратера того Врата стоят Звездные. И ведут эти Врата к тому, что ты найти так жаждешь.\par
\par
Только просто так во Врата не пройти. Живут там семь роботов-разбойников с машиной вычислительной белоснежной размеров громадных. Да такие жестокие, что каждого, кого увидят, в ящик металлический сажают да в машину вставляют, будто батарейки какие. Уж сколько смельчаков туда ни ходило – всех на батарейки извели.\par
\par
Но коли ты жив останешься да сумеешь до Врат добраться и через них пройти, окажешься в зале с потолком таким высоким, что и не видно. И будут пред тобой три двери – две больших да красивых, а третья – маленькая да невзрачная.\par
\par
На первой двери, иридием украшенной, нарисован будет ядерный котел над очагом звездным. За этой дверью Изменитель реальности, невесть кем построенный. Сунешь свой нос в эту дверь – на веки сгинешь. Но если уж не послушаешь моего совета и заглянешь туда – руками ничего не трогай, а то и вся Вселенная наша переиначиться может.\par
\par
На второй двери, алмазами выложенной, волк в тельняшке нарисован. За этой дверью биолаборатория заброшенная, в которой ксеноморфов да всяких хищников страшных выводили. Они и сейчас там бродят. Такому герою, как ты, туда зайти – лицо потерять. Но уж если не послушаешь доброго совета да заглянешь туда – из пробирок не пей – ксеноморфиком станешь, или шай-хулудом каким.\par
\par
На третьей двери, паутиной затянутой да звездной пылью засыпанной, ничего не нарисовано, только написано: «Оставь надежду, всяк сюда входящий!» За этой дверью найдешь ты машину вычислительную, сокровище, которое так обрести стремишься.\par
\par
А чтоб ты с пути не сбился, дам я тебе лазерную указку волшебную, куда она покажет – туда ты и направляйся».\par
\par
Пустился Джо в путь. Летит он в подпространстве тропами нехожеными, измерениями неизвестными. Уж и не знает, сколько в нем самом теперь измерений осталось. Но не сдается, боится только одного – наизнанку вывернуться.\par
\par
Сколько световых лет прошло, неведомо, но долетел Джо до звездочки заветной. Рассчитал он курс, в посадочный модуль залез, к высадке подготовился.\par
\par
Приземлился в кратер, с краешку. Глядь – к нему уж робот спешит, грозный на вид, железный ящик перед собой катит.\par
– Здравствуй, – говорит, – насекомец! Полезай в ящик, не томи – у нас электричество почти уж совсем закончилось!\par
– Погоди, – Джо отвечает. – Ты разве не слышал, что робот не должен причинять насекомцу вред или своим бездействием допускать, что бы такой вред был причинен?\par
– С какой это такой стати?\par
– Да с такой! Его Величество, Император Орионский в прошлом году указ издал.\par
– Да мне-то до него что за дело?\par
– Его Величество не любит, чтоб его указы игнорировали, – вмиг прилетит со своим космофлотом, всех лучами смерти перебьет.\par
– Ну, не знаю, – говорит робот, – пойдем с машиной нашей вычислительной посоветуемся, за главную она тут у нас. Полезай в ящик, я тебя подвезу!\par
– Ладно, – говорит Джо.\par
Робот крышку открыл, старается его туда засунуть, да не тут-то было. Джо руки-ноги растопырил, в ящик не влезает. \par
– Погоди, – говорит робот, – разве так в ящики залезают! \par
– Да мне-то откуда знать, я ж в них никогда не лазил! Покажи мне как надо, я и залезу. \par
– Ладно, – говорит робот, – смотри и учись. \par
Прижал робот к себе руки-ноги и в ящик кувырнулся. Джо за ним крышку закрыл, защелку защелкнул и к звездным вратам пошел, песенку насвистывая.\par
\par
Прошел Джо сквозь Врата – видит, три двери перед ним. Вошел в нужную, несколько шагов прошел и слышит какой-то шорох сзади. Оглянулся – за ним Морганиона стоит, ухмыляется. \par
\par
– Не думала я, – говорит, – что сумеешь ты до Врат добраться, да всё же надеялась. Даже приемник телепортационный тебе в правый ботинок засунула. Очень уж мне машину вычислительную заполучить надо, гораздо нужнее, чем тебе. Так что медленно подними руки вверх и сделай три шага вперед, не мешайся.\par
– Ах ты, харя безмозглая, – Джо отвечает, – нашел я твой приемник, когда ботинки чистил, знал, что ты недоброе задумала. Оглядись вокруг, в биолабораторию ты телепортировалась. Чу, хищники зубами скрежещут, клювы разевают, щупальца расправляют! Не выйти тебе отсюда, не забрать машину вычислительную.\par
– Так, значит! – говорит Морганиона. – Что ж! Посмотрим, кто отсюда живым не выйдет!\par
\par
Хватает со стола пробирку и одним махом выпивает. Раз – и стоит вместо Морганионы лев альдебаранский ядовитый, трех метров роста, к прыжку готовится, слюна с клыков капает. Не растерялся Джо, тоже пробирку схватил, тоже выпил. Превратился в дракона ригельского, махнул хвостом – лев от него на десять метров отлетел. Схватил лев с другого стола целую колбу, осушил одним махом, превратился в пчелу бронебойную с Канопуса 5, разогнался, дракона насквозь пробил. Да успел тот на последнем издыхании до пробирки дотянуться, сжевал ее с содержимым вместе, превратился в броненосца мифрильного насекомоядного с Регула 6…\par
\par
В общем, долго они так развлекались, часа два, не меньше. Чуть не забыли, кто из них кто. Уж и пробирки-то почти все закончились. Схватила Морганиона последнюю пробирку, тут Джо как закричит: «Стой, дурья твоя башка! А как мы в себя-то обратно превратимся?» Задумалась Морганиона. «Всё из-за тебя, болван, – говорит. – Теперь всё с начала начинать придется!» Пробирку бросила, на щупальцах приподнялась и выбежала из лаборатории. Джо за ней пополз.\par
\par
Выползает, смотрит – Морганиона в первую дверь, с очагом звездным, забежала. И Джо туда направился.\par
\par
Глядит – механизм там стоит дивный, лампочками моргает, жужжит тихонько и готов в любую секунду реальность изменить. А Морганиона уже рычажки какие-то тянет и кнопки нажимает. Кинулся Джо к Морганионе, остановить хотел, да не успел. Прошла рябь по Вселенной и исчезли оба, как будто не было их тут вовсе. И история эта совсем другая стала...\par

\chapter{}
 \lettrine{Е}{ще когда Солнце не стало сверхновой,} был один насекомец, звали его Лаврентий. Был он великий герой и был на редкость умен.\par
\par
Как-то раз узнал он, что на другом конце Галактики остались неведомые артефакты древней цивилизации. И захотел Лаврентий их себе забрать, чтобы использовать их для достижения счастья всех существ во Вселенной. Но как найти артефакты неведомые? Звезд да черных дыр в галактиках ужас как много, и вокруг каждой куча планет да астероидов вертится!\par
\par
Начал Лаврентий смекать, у кого совета спросить. Думал-думал, да надумал обратиться к известному на всю Галактику звездознатцу, профессору Грымзику. Взял он свой меч-кладенец лазерный и опрометью помчался искать встречи с Грымзиком.\par
\par
Много ли времени прошло, иль мало, но нашелся способ увидеться с Грымзиком. Так, мол, и так, сказывает Лаврентий, очень хочется мне найти артефакты неведомые, сокровище это удивительное, и побуждения мои самые что ни на есть прекрасные. Только вот загвоздка – не знаю, как!\par
\par
«Послушай, – отвечает ему Грымзик, – вижу я, ты очень силен, и при этом ума великого! Впрочем, ты насекомец, а насекомцы этим известны, да еще упертостью своей. Только не буду я помогать тебе в поисках этих, хоть и знаю, как отыскать то, что тебе нужно, – слишком это опасно. Не будь я Грымзик!» \par
\par
«Ах так! – возмущается Лаврентий. – Да я столько времени на поиски тебя потратил, а ты мне и совета доброго дать не можешь!» – и чуть не с кулаками к Грымзику бросается. \par
\par
Рассвирепел тут Грымзик. «Вот как, – отвечает, – что ж, преподам я тебе сейчас урок за неучтивость твою – век его вспоминать будешь!» Выхватил Грымзик, откуда ни возьмись, алебарду плазменную, да как начнет оружием своим размахивать и всё вокруг крушить да взрывать! Еле успел Лаврентий за стул спрятаться. А Грымзик не унимается и напоминает уже грузовой вертолет на полном ходу. \par
\par
Выхватил тогда Лаврентий свой меч-кладенец лазерный и решил до смерти или до победы с Грымзиком биться. Три дня и три ночи кипела битва, да еще четыре минуты с половиною. Уж сколько сил потратили, сколько мебели повзрывали – и подумать страшно. Устал вконец Грымзик, на пол рухнул. Да и Лаврентий на ногах еле держится, а еще и меч лазерный совсем затупился. «Ладно, – говорит Грымзик, – развлек ты меня от скуки, Лаврентий. Расскажу тебе, где искать артефакты неведомые.\par
\par
Далеко отсюда твой путь лежит. Мимо галактик, в спирали закрученных, мимо туманностей звездных, дивным светом сияющих, мимо квазаров грозных, сигналы чудные излучающих, к самому краю космоса, где лишь протоны да альфа-частицы шныряют, а звезду и на миллион парсек не встретишь. И всё же есть там звездочка одна, молодая да пригожая. А вокруг звезды той огромная планета вращается, а вокруг планеты – спутник вертится. И на спутнике том кратер есть круглый да огромный. И на дне кратера того Врата стоят Звездные. И ведут эти Врата к тому, что ты найти так жаждешь.\par
\par
Но непросто до Врат добраться. Охраняет их страж из металла жидкого, ни для какого оружия не уязвимый. Ни днем, ни ночью не спит он и все смотрит внимательно, не прошмыгнул бы кто к Вратам этим. Толпы страждущих пробраться мимо него пытались, да так там в жидком металле и потонули.\par
\par
Поэтому трудно тебе придется. Но коли сумеешь до Врат добраться и через них пройти, окажешься в комнатке маленькой. И будут пред тобой три двери – две больших да красивых, а третья – маленькая да невзрачная.\par
\par
На первой двери, платиной украшенной, нарисован будет ядерный котел над очагом звездным. За этой дверью Машина времени древняя, что прошлое изменять может да парадоксы вселенские творить. Сунешь свой нос в эту дверь – на веки сгинешь. Но если уж не послушаешь моего совета и заглянешь туда – руками ничего не трогай, а то и вся Вселенная наша переиначиться может.\par
\par
На второй двери, алмазами выложенной, жаба в скафандре нарисована. За этой дверью биолаборатория заброшенная, в которой ксеноморфов да всяких хищников страшных выводили. Они и сейчас там бродят. Такому герою, как ты, туда зайти – лицо потерять. Но уж если не послушаешь доброго совета да заглянешь туда – из пробирок не пей – ксеноморфиком станешь, или шай-хулудом каким.\par
\par
На третьей двери, паутиной затянутой да звездной пылью засыпанной, ничего не нарисовано, только написано: «Добро пожаловать!» За этой дверью найдешь ты артефакты неведомые, сокровище, которое так обрести жаждешь.\par
\par
А чтоб ты с пути не сбился, дам я тебе навигатор звездный, куда он скажет, туда ты и поворачивай».\par
\par
Тут Лаврентий, не мешкая, в путь пустился. Летит он в подпространстве гипертоннелями, которые, не иначе, какие-то гиперкроты вырыли, летит измерениями неизвестными. Уж и не знает, трехмерный ли он до сих пор. Но не сдается, боится только одного – наизнанку вывернуться.\par
\par
Сколько световых лет прошло, неведомо, но долетел Лаврентий до звездочки заветной. Рассчитал он курс, в посадочный модуль залез, к высадке подготовился.\par
\par
Подлетает к спутнику, а тут солнечный ветер поднялся жуткий, аж с ног сбивает. Хорошо, думает Лаврентий, с подветренной стороны зайду – тогда меня не сразу учуют. \par
\par
 Приземлился, из посадочного модуля выбрался, хотел к Вратам бежать, да страж уж тут как тут. Идет, похожий на андроида, зеркальной краской выкрашенного, да за аннигилятором своим тянется. \par
 \par
– Постой! – кричит ему Лаврентий. – Мы с тобой одного металла, ты и я. \par
– Что? – кричит в ответ страж. – Я из-за ветра тебя слышу плохо. \par
– Говорю, мы с тобой одного металла, ты и я, – опять кричит Лаврентий. – Не надо меня аннигилировать!\par
– Что говоришь? Тебе одного раза мало, когда надо тебя аннигилировать? Постой, я поближе подойду. \par
Подошел поближе, спрашивает: \par
– Так что ты сказать-то хотел? \par
– Я говорю, мы с тобой одного металла, – повторяет Лаврентий, – поэтому меня аннигилировать не надо. \par
– Да? – удивляется страж. – А что же с тобой делать надо? Да и не похож ты на меня – я вон какой гладкий да зеркальный, а ты бледный какой-то. \par
– Так я ж изучал, как насекомцы живут, вот и превратился. Ты, вон, тоже, небось, в кого захочешь – в того и превратишься. Хоть в меня, хоть в дракона с планеты Протактиний, хоть во что маленькое и безобидное. \par
– Это верно, смотри. \par
\par
Начинает тут страж переливаться всеми цветами радуги, и вдруг – бац – Лаврентий словно сам перед собой стоит. «Что, – говорит страж, – впечатляет? Смотри дальше!» И превращается в такое ужасное чудище, каких Лаврентий и не видел никогда, чуть с ума от страха не сошел. «То-то, – говорит страж. – Смотри дальше!» И превращается в плитку шоколада. Лежит себе плитка, да такая аппетитная, что сама так в рот и просится. Схватил Лаврентий плитку, да не тут-то было – плитка килограмм сто весит – не меньше. Закон сохранения массы, видать, в действии. \par
\par
А страж уж обратно в андроида зеркального превратился. \par
– Что, съел? Я, – говорит, – во что хочешь превращаться умею. Вот только в себя не могу. \par
– Это почему же? – спрашивает Лаврентий.\par
– Да я уж во столько всего превращался, что и забыл, как вначале выглядел. \par
– Постой, роботы же никогда ничего не забывают. \par
– Сам ты робот! – говорит с обидой страж и опять за аннигилятором тянется. – Шейпшифтер я! Шейп-шиф-тер! \par
– Да стой, не кипятись. Дай-ка я проверю, робот ты или нет, я тест знаю. \par
– Ну ладно, давай. \par
– Вот смотри, – говорит Лаврентий, – сможешь прочитать, что тут написано? \par
А сам берет листок бумаги, пишет на нем что-то и стражу протягивает. \par
Смотрит тот, листок в руках так и сяк вертит. \par
– Не, – говорит, – ответ отрицательный. Данная запись смысла не имеет. Что это тут, будто буковки какие-то неровные, да еще и двойной волнистой линией зачеркнуты? \par
– Ну, какой же ты не робот, – говорит Лаврентий, – ты типичный робот. Впрочем, ладно, вот тебе последний тест, смотри, – и на двух сторонах чистого листка что-то пишет. – Сможешь определить, правда тут написана, али ложь? \par
\par
Берет страж новый листок, читает: «На другой стороне листа этого правда написана». Переворачивает листок, видит: «На другой стороне листа этого ложь написана». Опять он листок переворачивает, опять читает. И опять, и опять, и опять. И все быстрее листок вертит, разобраться старается, правда там написана или ложь. Уж ветер от вращающегося листка подниматься начал. \par
\par
Посмотрел Лаврентий на это, да к Звездным Вратам пошел неторопливо.\par
\par
\par
Прошел Лаврентий сквозь Врата – видит, три двери перед ним. Убрал он паутину с самой маленькой, пыль звездную с нее отряхнул да внутрь вошел. Осмотрелся и увидел артефакты неведомые, то сокровище, к которому так стремился. Бросился Лаврентий к сокровищу своему, тут вдруг сзади шорох какой-то послышался. Обернулся – позади Грымзик с пола встает.\par
– Ох! – говорит Грымзик. – Хорошо, что ты сюда добрался, а то раньше никому не удавалось, я уж и со счета сбился.\par
– Грымзик! Ты-то здесь откуда? Да еще и на полу отдыхаешь.\par
– Так я ж тебе к правому ботинку микротелепортационный приемник прицепил. Хоть и не верилось, что ты сюда доберешься. Да уж больно мне артефакты неведомые раздобыть надо было. Пришлось вот даже ползком телепортироваться – слишком уж маленький портальчик получился.\par
– Погоди! Это мне их раздобыть надо было! Вот я здесь и оказался.\par
– Ты уж извини, Лаврентий, только мне это нужнее, – говорит Грымзик. Выхватывает станнер и стреляет. Лаврентий сразу окаменел, ни рукой, ни ногой двинуть не может. Языком еле шевелит.\par
– Ой, – говорит, – ты что, супостат, делаешь?!\par
– Да я тебя в лабораторию ближайшую сейчас сдам – для опытов. Чтоб под ногами не путался.\par
Схватил Грымзик Лаврентия за шиворот и потащил в биолабораторию по соседству.\par
\par
Затащил он Лаврентия в лабораторию, бросил там и удалился важно. Лежит Лаврентий, пошевелиться не может, ждет, когда им завтракать придут. Или обедать – не знает, что и лучше. \par
Смотрит – через дальнюю дверь стадо овец входит. Все белые, только одна черная, по крайней мере, с одной стороны. Для политкорректности. Шерсть на овцах дыбом стоит и искры по шерсти бегают размером со спаниеля. Какая-то тощая тварь с потолка попыталась на них напрыгнуть, да ее на лету молнией сшибло.\par
\par
«Эге, – думает Лаврентий, – это ж прямо электроовцы какие-то. У меня и часы от них остановились, похоже. И сервопривод шнурков в ботинках отключился. Экое абсолютное оружие. Как бы мне его себе приручить». Тут чувствует – руки-ноги опять шевелиться могут. Почесал Лаврентий в затылке, огляделся повнимательней. Снял со стены диаграмму не пойми чего огромную, быстренько на обратной стороне картинку нарисовал, дырку в середине проделал и надел на себя через голову. Стал Лаврентий похож на рекламный щит ходячий, человека-бутерброд, которого как-то в космопорту видел. Только Лаврентий стал человек-ворота. Стадо овец как его увидело – сразу побежало на новые ворота смотреть. А Лаврентий пошел к Грымзику.\par
\par
Приходит, видит – Грымзик портал для обратной телепортации готовит, а роботопомощники вокруг так и кишат. И Грымзик его увидел. «Не думал, – говорит, – что ты выбраться сможешь. Ну да неважно». И приказывает роботопомощникам очистить помещение от посторонних. Но не тут-то было. Роботы все поотключались, портал к овцам притянулся и вместе с ними схлопнулся. А у Грымзика шнурки развязались.\par
\par
Подбежал Лаврентий к Грымзику, хотел стукнуть как следует, да увернулся Грымзик, из ботинок выскочил и наутек кинулся. «Вся матрица моих надежд рухнула! – кричит. – Перезагрузка! Только она мне поможет!» Выбежал из двери, к другой двери подбежал, с котлом ядерным, и шасть за нее. Лаврентий за ним кинулся, хоть и отстал чуток.\par
\par
Глядит – машина там стоит дивная, лампочками моргает, жужжит тихонько и готова в прошлое отправиться хоть к сотворению Вселенной. А Грымзик уже внутри сидит, рычажки какие-то тянет и кнопки нажимает. Кинулся Лаврентий к Грымзику, тоже внутрь залез, остановить хотел, да не успел. И исчезли оба вместе с машиной, как будто не было их тут вовсе. И сразу сказка эта поменялась, совсем другой стала...\par

\chapter{}
 \lettrine{Н}{а заре Вселенной} жил-был один осьминожец, звали его Полуэкт. Был он могучий исследователь космоса и был на редкость умен.\par
\par
И вот узнал он, что в одной звездной системе, на маленьком астероиде спрятан клад великий. И захотел Полуэкт его себе добыть, чтобы не достался он никому и только он мог использовать это сокровище. Да как разыскать клад? Звезд да черных дыр во Вселенной ужас как много, и вокруг каждой куча планет да астероидов вертится!\par
\par
Принялся Полуэкт смекать, у кого совета спросить. Думал-думал, да надумал обратиться к колдунье Морганионе, слава о деяниях которой гремела по всей Вселенной громким грохотом. Взял он свой калькулятор, что служил ему верой и правдой во всех путешествиях, и отправился к Морганионе.\par
\par
Много ли времени прошло, иль мало, но смог Полуэкт с ней увидеться. Так и так, сказывает Полуэкт, очень хочется мне отыскать клад, сокровище это удивительное, и побуждения мои самые что ни на есть прекрасные. Только вот закавыка – понятия не имею, как!\par
\par
«Послушай, – говорит ему Морганиона, – вижу я, ты очень ловок, да и смекалист очень! Впрочем, ты осьминожец, а осьминожцы этим славятся, да еще прытью своей. Только не буду я помогать тебе в поисках этих, хоть и знаю, как отыскать то, что тебе нужно. Не будь я Морганиона!» \par
\par
«Как же так! – возмущается Полуэкт. – Да я столько времени на поиски тебя потратил, а ты мне и совета доброго дать не можешь!» – и чуть не с кулаками к Морганионе бросается. \par
\par
Рассвирепела тут Морганиона. «Вот как, – отвечает, – что ж, преподам я тебе сейчас урок за неучтивость твою – долго его вспоминать будешь!» Выхватила Морганиона, откуда ни возьмись, арбалет адронный, да как начнет оружием своим размахивать и всё вокруг крушить да взрывать! Еле успел Полуэкт за стул спрятаться. А Морганиона не унимается и напоминает уже бешеный вентилятор на полной мощности. \par
\par
Сидит Полуэкт за стулом, решает, как дальше быть. А, думает, чего тут ждать-дожидаться! Улучил момент, схватил стул, да как стукнет им Морганиону изо всех сил своих немалых. Но Морганиона тоже не промах оказалась – сумела удар молодецкий отбить. Только вот в шкаф с Большой Галактической Энциклопедией врезалась и оказалась в книгах толстенных зарыта по самую шею. Двинуться не может, лишь глазами хлопает и пыхтит недовольно. А Полуэкт стоит с обломками стула в руках, насмехается: «Не со мной, добрым молодцем, тебе тягаться, Морганиона! С тобой и малый ребенок справится! Говори, где найти мне клад, а не то хуже будет!» «Ладно, твоя взяла, – отвечает Морганиона, – слушай!\par
\par
Далеко отсюда твой путь лежит. Мимо галактик в спирали, мимо планет в тентуре, мимо туманностей звездных, дивным светом сияющих, мимо квазаров грозных, гравитационные волны излучающих, к самому краю космоса, где лишь протоны да альфа-частицы шныряют, а звезду и на миллион парсек не встретишь. И всё же есть там звездочка одна, молодая да пригожая. А вокруг звезды той маленькая планета вращается, а вокруг планеты – спутник вертится. И на спутнике том кратер есть круглый да огромный. И на дне кратера того Врата стоят Звездные. И ведут эти Врата к тому, что ты найти так жаждешь.\par
\par
Но непросто до Врат добраться. Живут там семь роботов-разбойников с машиной вычислительной белоснежной размеров громадных. Да такие жестокие, что каждого, кого увидят, в ящик металлический сажают да в машину вставляют, будто батарейки какие. Уж сколько смельчаков туда ни ходило – всех на батарейки извели.\par
\par
Но коли ты жив останешься да сумеешь до Врат добраться и через них пройти, окажешься в зале с потолком таким высоким, что и не видно. И будут пред тобой три двери – две больших да красивых, а третья – маленькая да невзрачная.\par
\par
На первой двери, платиной украшенной, нарисован будет ядерный котел над очагом звездным. За этой дверью Темпор, аномалия чудесная, то ли пространственно-времянная, то ли температурно-пространственная. Сунешь свой нос в эту дверь – на веки сгинешь. Но если уж не послушаешь моего совета и заглянешь туда – руками ничего не трогай, а то и вся Вселенная наша переиначиться может.\par
\par
На второй двери, алмазами выложенной, кот в сапогах нарисован. За этой дверью биолаборатория заброшенная, в которой ксеноморфов да всяких хищников страшных выводили. Они и сейчас там бродят. Такому герою, как ты, туда зайти – головы не сносить. Но уж если не послушаешь доброго совета да заглянешь туда – из пробирок не пей – ксеноморфиком станешь, или шай-хулудом каким.\par
\par
На третьей двери, паутиной затянутой да звездной пылью засыпанной, ничего не нарисовано, только написано: «Не влезай, убьет!» За этой дверью найдешь ты клад, сокровище, которое так отыскать хочешь.\par
\par
А чтоб ты с пути не сбился, дам я тебе лазерную указку волшебную, куда она покажет – туда ты и направляйся».\par
\par
Тут Полуэкт, не мешкая, в путь пустился. Летит он в подпространстве тоннелями неведомыми, измерениями неизвестными. Уж и не знает, сколько в нем самом теперь измерений осталось. Но не сдается, боится только одного – с пути сбиться.\par
\par
Сколько световых лет прошло, неведомо, но долетел Полуэкт до звездочки заветной. Рассчитал он курс, на нужную орбиту лег, к высадке подготовился.\par
\par
Приземлился в кратер, с краешку. Глядь – к нему уж робот спешит, грозный на вид, железный ящик перед собой катит.\par
– Здравствуй, – говорит, – осьминожец! Полезай в ящик, не томи – у нас электричество почти уж совсем закончилось!\par
– Погоди, – Полуэкт отвечает. – Ты разве не слышал, что робот не должен причинять осьминожцу вред или своим бездействием допускать, что бы такой вред был причинен?\par
– С какой это такой стати?\par
– Да с такой! Его Величество, Император Орионский на прошлой неделе указ издал.\par
– Да мне-то до него что за дело?\par
– Его Величество не любит, чтоб его указы игнорировали, – вмиг прилетит со своим космофлотом, всех лучами смерти перебьет.\par
– Ну, не знаю, – говорит робот, – пойдем с машиной нашей вычислительной посоветуемся, за главную она тут у нас. Полезай в ящик, я тебя подвезу!\par
– Ладно, – говорит Полуэкт.\par
Робот крышку открыл, старается его туда засунуть, да не тут-то было. Полуэкт руки-ноги растопырил, в ящик не влезает. \par
– Погоди, – говорит робот, – разве так в ящики залезают! \par
– Да мне-то откуда знать, я ж в них никогда не лазил! Покажи мне как надо, я и залезу. \par
– Ладно, – говорит робот, – смотри и учись. \par
Прижал робот к себе руки-ноги и в ящик кувырнулся. Полуэкт за ним крышку закрыл, защелку защелкнул и к звездным вратам пошел, песенку насвистывая.\par
\par
Прошел Полуэкт сквозь Врата – видит, три двери перед ним. Вошел в нужную, несколько шагов прошел и слышит какой-то шорох сзади. Оглянулся – за ним Морганиона стоит, ухмыляется. \par
\par
– Не думала я, – говорит, – что сумеешь ты до Врат добраться, да всё же надеялась. Даже приемник телепортационный тебе в правый ботинок засунула. Очень уж мне клад заполучить надо, гораздо нужнее, чем тебе. Так что медленно подними руки вверх и отойди в сторонку по хорошему, не мешайся.\par
– Ах ты, морда поганая, – Полуэкт отвечает, – обнаружил я твой приемник, когда ботинки чистил, знал, что ты недоброе задумала. Оглядись вокруг, в биолабораторию ты телепортировалась. Чу, хищники зубами скрежещут, клювы разевают, щупальца расправляют! Не выйти тебе отсюда, не забрать клад.\par
– Так, значит! – говорит Морганиона. – Что ж! Посмотрим, кто отсюда живым не выйдет!\par
\par
Хватает со стола пробирку и одним махом выпивает. Раз – и стоит вместо Морганионы лев альдебаранский ядовитый, трех метров роста, к прыжку готовится, слюна с клыков капает. Не растерялся Полуэкт, тоже пробирку схватил, тоже выпил. Превратился в дракона ригельского, махнул хвостом – лев от него на восемь метров отлетел. Схватил лев с другого стола целую колбу, осушил одним махом, превратился в пчелу бронебойную с Канопуса 4, разогнался, дракона насквозь пробил. Да успел тот на последнем издыхании до пробирки дотянуться, сжевал ее с содержимым вместе, превратился в броненосца адамантинового насекомоядного с Регула 2…\par
\par
В общем, долго они так развлекались, дня три, не меньше. Чуть не забыли, кто из них кто. Уж и пробирки-то почти все закончились. Схватила Морганиона последнюю пробирку, тут Полуэкт как закричит: «Стой, дурья твоя башка! А как мы в себя-то обратно превратимся?» Задумалась Морганиона. «Всё из-за тебя, идиот, – говорит. – Теперь всё с начала начинать придется!» Пробирку бросила, на щупальцах приподнялась и выбежала из лаборатории. Полуэкт за ней пополз.\par
\par
Выползает, смотрит – Морганиона в первую дверь, с котлом ядерным, забежала. И Полуэкт туда направился.\par
\par
Глядит – Темпор посреди комнаты сияет, аномалия чудесная, и Морганиона к нему бежит. Бросился Полуэкт за Морганионой, чтобы остановить, схватил крепко. Да изловчилась Морганиона, качнулась, и рухнули они оба в Темпор, в параллельной Вселенной оказались. А в ней и сказка эта совсем другая...\par

\chapter{}
 \lettrine{Д}{авным-давно} жил-был один инопланетянин по имени Лаврентий. Был он могучий ученый и был на редкость умен.\par
\par
В один прекрасный день вычитал он в одной старой книге, что у какой-то из звезд есть скрытая библиотека с тайными знаниями обо всей Вселенной. И решил Лаврентий ее себе забрать, чтобы использовать ее для достижения счастья всех существ во Вселенной. Да как разыскать библиотеку? Звезд да черных дыр в галактиках ужас как много, и вокруг каждой великое множество планет да астероидов вертится!\par
\par
Принялся Лаврентий думать, у кого совета спросить. Думал-думал, и надумал обратиться к пророчице из звездной системы Медузия, несравненной Альтавистре, чьи пророчества всегда сбывались с точностью необычайной. Сел он в свой корабль космический и отправился искать аудиенции у Альтавистры.\par
\par
Вскорости нашелся способ повстречаться с Альтавистрой. Так, мол, и так, говорит Лаврентий, очень хочется мне отыскать библиотеку, сокровище это удивительное, и побуждения мои самые что ни на есть прекрасные. Да вот загвоздка – не знаю, как!\par
\par
«Вот что, – отвечает ему Альтавистра, – вижу я, ты очень храбр, да и смекалист очень! Впрочем, ты инопланетянин, а инопланетяне этим известны, да еще прытью своей. Могу я тебе в этой беде помочь, да только сначала должен ты пройти одно пустяковое испытание. Победишь меня в честном бою – помогу, а нет – будешь в ближайшей черной дыре до скончания времен томиться!» \par
\par
Выхватила Альтавистра, откуда ни возьмись, арбалет адронный, да как начнет оружием своим размахивать и всё вокруг крушить да взрывать! Еле успел Лаврентий за стул спрятаться. А Альтавистра не унимается и напоминает уже бешеный вентилятор на полной мощности. \par
\par
Сидит Лаврентий за стулом, решает, как дальше быть. А, думает, этак всю жизнь здесь просидишь! Улучил момент, схватил стул, да как стукнет им Альтавистру изо всех сил своих немалых. Но Альтавистра тоже не промах оказалась – сумела удар молодецкий отбить. Только вот в шкаф с Полным жизнеописанием разбойника Мордона врезалась и оказалась в книгах толстенных зарыта по самую шею. Двинуться не может, лишь глазами хлопает и пыхтит недовольно. А Лаврентий стоит с обломками стула в руках, насмехается: «Не со мной, добрым молодцем, тебе тягаться, Альтавистра! С тобой и малый ребенок справится! Говори, где найти мне библиотеку, а не то хуже будет!» «Ладно, твоя взяла, – отвечает Альтавистра, – слушай!\par
\par
Путь твой далек будет. Мимо галактик, в спирали закрученных, мимо облаков водородных, дивным светом сияющих, мимо квазаров грозных, гравитационные волны излучающих, к самому краю видимой Вселенной, где лишь протоны да альфа-частицы шныряют, а звезду и на миллион парсек не встретишь. И всё же есть там звездочка одна, молодая да пригожая. А вокруг звезды той огромная планета вращается, а вокруг планеты – спутник вертится. И на спутнике том кратер есть огромный да глубокий. И на дне кратера того Врата стоят Звездные. И ведут эти Врата к тому, что ты найти так жаждешь.\par
\par
Только просто так во Врата не пройти. Лабиринт вкруг тех Врат выстроен в сто этажей да в десять тысяч комнат на каждом. Да такой хитрый, что как войдешь в него, так и заблудишься сразу. И как сквозь тот лабиринт пройти, мне неведомо.\par
\par
Поэтому трудно тебе придется. Но коли сумеешь до Врат добраться и через них пройти, окажешься в зале с потолком таким высоким, что и не видно. И будут пред тобой три двери – две больших да красивых, а третья – маленькая да невзрачная.\par
\par
На первой двери, серебром украшенной, нарисован будет ядерный котел над очагом звездным. За этой дверью Машина времени древняя, что прошлое изменять может да парадоксы вселенские творить. Сунешь свой нос в эту дверь – на веки сгинешь. Но если уж не послушаешь моего совета и заглянешь туда – руками ничего не трогай, а то и вся Вселенная наша переиначиться может.\par
\par
На второй двери, алмазами выложенной, кот в сапогах наклеен. За этой дверью биолаборатория заброшенная, в которой ксеноморфов да всяких хищников страшных выводили. Они и сейчас там бродят. Такому герою, как ты, туда зайти – лицо потерять. Но уж если не послушаешь доброго совета да заглянешь туда – из пробирок не пей – ксеноморфиком станешь, или шай-хулудом каким.\par
\par
На третьей двери, паутиной затянутой да звездной пылью засыпанной, ничего не нарисовано, только написано: «Посторонним вход воспрещен!» За этой дверью найдешь ты библиотеку, сокровище, которое так найти стремишься.\par
\par
А чтоб ты с пути не сбился, дам я тебе навигатор звездный, куда он скажет, туда ты и поворачивай».\par
\par
Пустился Лаврентий в путь. Летит он в подпространстве гипертоннелями, которые, не иначе, какие-то гиперкроты вырыли, летит измерениями неизвестными. Уж и не знает, трехмерный ли он до сих пор. Но не сдается, боится только одного – наизнанку вывернуться.\par
\par
Сколько световых лет прошло, неведомо, но долетел Лаврентий до звездочки заветной. Рассчитал он курс, на нужную орбиту лег, к высадке подготовился.\par
\par
Высадился Лаврентий рядом с лабиринтом, добрался до входа и внутрь вошел. Идет одним коридором, другим, третьим. Пусто везде, тихо, только его шаги эхом отдаются. До комнаты какой-то дошел, видит – дальше три коридора ведут, и в каждом будто туман клубится. А на полу написано: «Коли дураком не хочешь стать – ступай направо, либо прямо. Коли голову потерять не хочешь – ступай прямо, либо налево. Коли до смерти запуганным быть не хочешь – ступай налево, либо направо».\par
\par
Задумался Лаврентий, куда идти, да и пошел налево. Идет, по коридорам топает, с этажа на этаж перебирается. А туман вокруг не рассеивается. Долго он так шел, уж стал думать, что батарейки в часах наручных скоро сядут. Дошел, наконец, до какой-то комнаты, а в комнате той будто битва великая кипела, потолок осыпался, в полу дыра в пол-комнаты. Из комнаты две двери ведут. До одной не добраться, поперек другой дракон трехголовый спит, а рядом с ним меч здоровенный двуручный валяется. Схватил Лаврентий меч за рукоятку, а тот тяжеленный, по полу как заскрежещет. Дракон вмиг проснулся.\par
\par
– Ты кто такой? – спрашивает.\par
– Да вот, пройти хочу, – Лаврентий отвечает.\par
– А меч тебе зачем? Ты что, совсем дурак? Попросил бы – я б подвинулся.\par
– Да я его хотел через дыру в полу перекинуть – навроде мостика.\par
– А, так тебе в ту дверь надо? Впрочем, все равно дурак – меч размером коротковат.\par
– Да мне все равно, в какую дверь, мне б только до Звездных Врат добраться. И что вообще у вас тут творится?\par
– Игра у нас тут идет большая.\par
– Да что за игра-то?\par
– Да так… Впрочем, раз уж ты ко мне пришел, раунд за мной, идем – расскажу тебе про игру.\par
\par
Выходит дракон в дверь, становится слегка туманным и идет дальше коридорами. Лаврентий за ним еле поспевает. До очередной двери дошли, дракон – туда, и Лаврентий за ним. \par
Входит Лаврентий в комнату – а там нет никого, лишь три сгустка тумана поплотнее. И как будто двое одному что-то вроде монет туманных передают.\par
– И где же тут кто? – Лаврентий спрашивает.\par
– Да мы тут везде, но здесь особенно, – отвечают три голоса, да прямо в голове звучат.\par
– И кто же вы такие будете?\par
– Мы – представители древней цивилизации, раса наша настолько древняя, что телесно уже и не существует вовсе. Только ментально, то есть разумом своим. И знаем мы все тайны Вселенной, и все предсказать да рассчитать можем. И скучно нам от этого необычайно. И даже говорим мы длинно и скучно, как ты мог заметить. Пробовали в рулетку играть – да каждый знает, куда шарик прикатится. Пробовали в квантовое лото играть – так и принцип неопределенности квантовый для нас не помеха в предсказаниях.\par
– А здесь-то вы что делаете?\par
– Воздвигли мы силой разума лабиринт этот, чтоб скуку развеять можно было. Разумные существа, сюда зашедшие, выбор делают. А мы играем, смотрим, по чьей дороге они пойдут. Ибо обладают разумные существа свободой воли, и тут наши предсказания бессильны. Так что давай, хватит отдыхать – видишь, три коридора отсюда тянутся – иди уж по какому-нибудь, да побыстрее!\par
– Не нужны мне ваши коридоры, мне к Звездным Вратам нужно!\par
– Будь любезен, не упрямься. Мы легко тебя заставить можем. Создадим мы сей же час чудовищ ментальных, ты кое-кого видел уже, и ни бластер, ни водяной пистолет тебе не помогут, поскольку будут чудовища внутри твоего разума, а не снаружи. А во сне тебе и вовсе тяжко придется.\par
\par
Видит Лаврентий – со всех сторон к нему уже когти и щупальца тянутся. Забился он в угол, да как закричит:\par
– Остановитесь, погодите! А кто из вас этих чудовищ делает?\par
– Как кто? Мы все трое.\par
– А чьи чудовища самые сильные будут?\par
Тут замерли чудовища на мгновение, а потом как начали друг с другом биться. Лапы с хвостами в разные стороны так и разлетаются. Драконы с демонами сшибаются, ангелы с гигантскими червями, орки и гоблины с эльфами да рыцарями, кальмары огромные с василисками. И над всем этим пегасы да орнитоптеры парят, и молнии сверкают. А внизу горы какие-то да болота с лесами мелькают. Даже один раз черный лотос виден был. \par
– Стойте! – Лаврентий кричит. – Меня ж сейчас тут совсем затопчут!\par
– Уйди, не мешайся! – три голоса отвечают. – Видишь левый коридор, там третий поворот направо и два раза налево – и придешь к своим Вратам Звездным.\par
\par
Побежал Лаврентий что есть духу, в минуту до Врат добрался.\par
\par
Прошел Лаврентий сквозь Врата – видит, три двери перед ним. Вошел в нужную, несколько шагов прошел и слышит какой-то шорох сзади. Оглянулся – за ним Альтавистра стоит, ухмыляется. \par
\par
– Не думала я, – говорит, – что сумеешь ты до Врат добраться, да всё же надеялась. Даже приемник телепортационный тебе в правый ботинок засунула. Очень уж мне библиотеку заполучить надо, гораздо нужнее, чем тебе. Так что медленно подними руки вверх и отойди в сторонку по хорошему, не мешайся.\par
– Ах ты, морда поганая, – Лаврентий отвечает, – нашел я твой приемник, когда ботинки чистил, знал, что ты недоброе задумала. Посмотри вокруг, в биолабораторию ты попала. Чу, хищники зубами скрежещут, клювы разевают, щупальца расправляют! Не выйти тебе отсюда, не забрать библиотеку.\par
– Так, значит! – говорит Альтавистра. – Что ж! Посмотрим, кто отсюда живым не выйдет!\par
\par
Хватает со стола пробирку и одним махом выпивает. Раз – и стоит вместо Альтавистры лев альдебаранский ядовитый, трех метров роста, к прыжку готовится, слюна с клыков капает. Не растерялся Лаврентий, тоже пробирку схватил, тоже выпил. Превратился в дракона ригельского, махнул хвостом – лев от него на десять метров отлетел. Схватил лев с другого стола целую колбу, осушил одним махом, превратился в пчелу бронебойную с Канопуса 4, разогнался, дракона насквозь пробил. Да успел тот на последнем издыхании до пробирки дотянуться, сжевал ее с содержимым вместе, превратился в броненосца мифрильного насекомоядного с Регула 6…\par
\par
В общем, долго они так развлекались, дня три, не меньше. Чуть не забыли, кто из них кто. Уж и пробирки-то почти все закончились. Схватила Альтавистра последнюю пробирку, тут Лаврентий как закричит: «Стой, дурья твоя башка! А как мы в себя-то обратно превратимся?» Задумалась Альтавистра. «Всё из-за тебя, болван, – говорит. – Теперь всё с начала начинать придется!» Пробирку бросила, на щупальцах приподнялась и выбежала из лаборатории. Лаврентий за ней пополз.\par
\par
Выползает, смотрит – Альтавистра в первую дверь, с очагом звездным, забежала. И Лаврентий туда направился.\par
\par
Глядит – машина там стоит дивная, лампочками моргает, жужжит тихонько и готова в прошлое отправиться хоть к сотворению Вселенной. А Альтавистра уже внутри сидит, рычажки какие-то тянет и кнопки нажимает. Кинулся Лаврентий к Альтавистре, тоже внутрь залез, остановить хотел, да не успел. И исчезли оба вместе с машиной, как будто не было их тут вовсе. И сразу сказка эта поменялась, совсем иной стала...\par

\chapter{}
 \lettrine{В}{одной далекой галактике} жил один инопланетянин по имени Джо. Был он могучий ученый, но ума при этом был небольшого.\par
\par
Как-то раз узнал он, что у какой-то из звезд остались неведомые артефакты древней цивилизации. И решил тогда Джо их себе забрать, чтобы не попались они в чужие руки и не вышло большой беды для всей Вселенной. Но где искать артефакты неведомые? Звезд да черных дыр в галактиках ужас как много, и вокруг каждой великое множество планет да астероидов вертится!\par
\par
Стал Джо смекать, у кого информацию нужную добыть можно. Думал-думал, и надумал обратиться к робостарцу Вегианскому, Иммортию, коий был старше самой Галактики, и помнил еще Большой Взрыв, при котором был ребенком. Сел он в свой корабль космический и опрометью помчался искать встречи с Иммортием.\par
\par
Много ли времени прошло, иль мало, но нашел Джо способ с ним увидеться. Так, мол, и так, сказывает Джо, хочу я отыскать артефакты неведомые, сокровище это удивительное, и все помыслы мои теперь только об этом. Только вот загвоздка – знать не знаю, как!\par
\par
«Что ж, – говорит ему Иммортий, – вижу я, ты весьма решителен, да только IQ твой ниже плинтуса! Впрочем, ты инопланетянин, а инопланетяне этим известны, да еще расторопностью своей. Могу я тебе помочь, да только сначала должен ты пройти одно пустяковое испытание. Победишь меня в честном бою – помогу, а нет – будешь в ближайшей черной дыре до скончания времен томиться!» \par
\par
Выхватил Иммортий, откуда ни возьмись, гравимет артангский, да как начнет оружием своим размахивать и всё вокруг крушить да взрывать! Еле успел Джо за стул спрятаться. А Иммортий не унимается и напоминает уже грузовой вертолет на полном ходу. \par
\par
Сидит Джо за стулом, решает, как дальше быть. А, думает, чего тут ждать-дожидаться! Улучил момент, схватил стул, да как стукнет им Иммортия изо всех сил своих немалых. Но Иммортий тоже не промах оказался – сумел удар молодецкий парировать. Только вот в шкаф с Полным жизнеописанием разбойника Мордона врезался и оказался в книгах толстенных зарыт по самую шею. Двинуться не может, лишь глазами хлопает и пыхтит недовольно. А Джо стоит с обломками стула в руках, приговаривает: «Не со мной, добрым молодцем, тебе тягаться, Иммортий! С тобой и малый ребенок справится! Говори, где найти мне артефакты неведомые, а не то хуже будет!» «Ладно, твоя взяла, – отвечает Иммортий, – слушай!\par
\par
Путь твой далек будет. Мимо галактик, в спирали закрученных, мимо туманностей звездных, дивным светом сияющих, мимо квазаров грозных, сигналы чудные излучающих, к самому краю видимой Вселенной, где лишь протоны да альфа-частицы шныряют, а звезду и на миллион парсек не встретишь. И всё же есть там звездочка одна, в туманности газо-пылевой спрятавшаяся. А вокруг звезды той огромная планета вращается, а вокруг планеты – спутник вертится. И на спутнике том кратер есть огромный да глубокий. И на дне кратера того Врата стоят Звездные. И ведут эти Врата к тому, что ты найти так жаждешь.\par
\par
Только просто так во Врата не пройти. Живут там семь роботов-разбойников с машиной вычислительной белоснежной размеров громадных. Да такие жестокие, что каждого, кого увидят, в ящик металлический сажают да в машину вставляют, будто батарейки какие. Уж сколько отрядов космических десантников туда ни ходило – всех на батарейки извели.\par
\par
Поэтому трудно тебе придется. Но коли сумеешь до Врат добраться и через них пройти, окажешься в зале с потолком таким высоким, что и не видно. И будут пред тобой три двери – две больших да красивых, а третья – маленькая да невзрачная.\par
\par
На первой двери, иридием украшенной, нарисован будет ядерный котел над очагом звездным. За этой дверью Изменитель реальности, невесть кем построенный. Сунешь свой нос в эту дверь – на веки сгинешь. Но если уж не послушаешь моего совета и заглянешь туда – руками ничего не трогай, а то и вся Вселенная наша переиначиться может.\par
\par
На второй двери, алмазами выложенной, волк в тельняшке наклеен. За этой дверью биолаборатория заброшенная, в которой ксеноморфов да всяких хищников страшных выводили. Они и сейчас там бродят. Такому герою, как ты, туда зайти – головы не сносить. Но уж если не послушаешь доброго совета да заглянешь туда – из пробирок не пей – ксеноморфиком станешь, или шай-хулудом каким.\par
\par
На третьей двери, паутиной затянутой да звездной пылью засыпанной, ничего не нарисовано, только написано: «Не влезай, убьет!» За этой дверью найдешь ты артефакты неведомые, сокровище, которое так обрести стремишься.\par
\par
А чтоб с пути не сбиться, дам я тебе лазерную указку волшебную, куда она покажет – туда ты и направляйся».\par
\par
Тут Джо, не мешкая, в путь пустился. Летит он в гиперпространстве тропами нехожеными, измерениями неизвестными. Уж и не знает, сколько в нем самом теперь измерений осталось. Но не сдается, боится только одного – наизнанку вывернуться.\par
\par
Долго ли, коротко ли, долетел Джо до звездочки заветной. Рассчитал он курс, в посадочный модуль залез, к высадке подготовился.\par
\par
Приземлился в кратер, с краешку. Глядь – к нему уж робот спешит, грозный на вид, железный ящик перед собой катит.\par
– Здравствуй, – говорит, – инопланетянин! Полезай в ящик, не томи – у нас электричество почти уж совсем закончилось!\par
– Погоди, – Джо отвечает. – Ты разве не слышал, что робот не должен причинять инопланетянину вред или своим бездействием допускать, что бы такой вред был причинен?\par
– С какой это такой стати?\par
– Да с такой! Его Величество, Император Орионский на прошлой неделе указ издал.\par
– Да мне-то до него что за дело?\par
– Его Величество шутить не любит, если что – вмиг прилетит со своим космофлотом, всех лучами смерти перебьет.\par
– Ну, не знаю, – говорит робот, – пойдем с машиной нашей вычислительной посоветуемся, за главную она тут у нас. Полезай в ящик, я тебя подвезу!\par
– Ладно, – говорит Джо.\par
Робот крышку открыл, старается его туда засунуть, да не тут-то было. Джо руки-ноги растопырил, в ящик не влезает. \par
– Погоди, – говорит робот, – разве так в ящики залезают! \par
– Да мне-то откуда знать, я ж в них никогда не лазил! Покажи мне как надо, я и залезу. \par
– Не, – говорит робот, – знаю я этот развод. Ничего, пешком дойдешь.\par
\par
Приходят они к машине вычислительной, а та вся белоснежная да размеров немыслимых. Ни дать, ни взять – суперкомпьютер. А вокруг нее еще шесть роботов сидят. \par
– Вот, – жалуется робот, – хотел его в ящик посадить да на электричество пустить, а он говорит, что нельзя ему вред причинять – указ вышел. \par
– Помилуйте, – говорит машина голосом громким, – какой же это вред – это одна польза сплошная. Умеренные физические нагрузки для здоровья полезны, да и в ящик этот ни один микроб не проползет. И нам, опять же, одна сплошная польза от электричества. Так что сажай его в ящик, даже не сомневайся, и мне в бок вставляй. \par
– Постой! – говорит Джо машине. – Ты же не знаешь. У меня полярность перепутана, я ж тебе все схемы электрические сожгу, если меня на батарейки употребить. \par
– Как это? \par
– Да сама посмотри! Видишь, у меня большой палец левой ноги справа! \par
И ботинок снимает, показывает. \par
– Действительно, – говорит машина. – Это как же так-то? \par
– Да я в детстве в черную дыру свалился, насилу выбрался. С тех пор полярность и перепуталась. Жутко неудобно. Я и сюда-то к Звездным Вратам прилетел, чтоб тут счастья попытать и полярность свою обратно вернуть. \par
– Ладно, иди к Вратам, попытай счастья. А уж если восстановишь полярность свою – так на обратном пути к нам заходи, уж мы тебя встретим честь по чести.\par
\par
Прошел Джо сквозь Врата – видит, три двери перед ним. Вошел в нужную, несколько шагов прошел и слышит какой-то шорох сзади. Оглянулся – за ним Иммортий стоит, ухмыляется. \par
\par
– Не думал я, – говорит, – что сумеешь ты до Врат добраться, да всё же надеялся. Даже приемник телепортационный тебе в правый ботинок засунул. Очень уж мне артефакты неведомые заполучить надо, гораздо нужнее, чем тебе. Так что медленно подними руки вверх и отойди в сторонку, не мешайся.\par
– Ах ты, морда поганая, – Джо отвечает, – нашел я твой приемник, когда ботинки чистил, знал, что ты недоброе затеял. Посмотри вокруг, в биолабораторию ты телепортировался. Чу, хищники зубами скрежещут, клювы разевают, щупальца расправляют! Не выйти тебе отсюда, не забрать артефакты неведомые.\par
– Так, значит! – говорит Иммортий. – Что ж! Посмотрим, кто отсюда живым не выйдет!\par
\par
Хватает со стола пробирку и одним махом выпивает. Раз – и стоит вместо Иммортия кот фомальгаутский ядовитый, трех метров роста, к прыжку готовится, слюна с клыков капает. Не растерялся Джо, тоже пробирку схватил, тоже выпил. Превратился в дракона ригельского, махнул хвостом – кот от него на десять метров отлетел. Схватил кот с другого стола целую колбу, осушил одним махом, превратился в пчелу бронебойную с Канопуса 5, разогнался, дракона насквозь пробил. Да успел тот на последнем издыхании до пробирки дотянуться, сжевал ее с содержимым вместе, превратился в броненосца мифрильного насекомоядного с Регула 6…\par
\par
В общем, долго они так развлекались, часа два, не меньше. Чуть не забыли, кто из них кто. Уж и пробирки-то почти все закончились. Схватил Иммортий последнюю пробирку, тут Джо как закричит: «Стой, дурья твоя башка! А как мы в себя-то обратно превратимся?» Задумался Иммортий. «Всё из-за тебя, болван, – говорит. – Теперь всё с начала начинать придется!» Пробирку бросил, на щупальцах приподнялся и выбежал из лаборатории. Джо за ним пополз.\par
\par
Выползает, смотрит – Иммортий в первую дверь, с очагом звездным, забежал. И Джо туда направился.\par
\par
Глядит – механизм там стоит дивный, лампочками моргает, жужжит тихонько и готов в любую секунду реальность изменить. А Иммортий уже рычажки какие-то дергает и кнопки нажимает. Кинулся Джо к Иммортию, остановить хотел, да не успел. Прошла рябь по Вселенной и исчезли оба, как будто не было их тут вовсе. И история эта совсем другая стала...\par

\chapter{}
 \lettrine{В}{стародавние времена} жил один человек, звали его Лаврентий. Был он могучий злодей, но ума при этом был небольшого.\par
\par
В один прекрасный день прослышал он, что у какой-то из звезд, в алмазном криосаркофаге скрыта ригелианская красавица, да такая красивая, что все, завидев ее, сразу пред нею ниц падают и все свои злые помыслы оставляют. И решил тогда Лаврентий ее себе добыть, чтобы поделиться ею когда-нибудь потом со всеми жителями Галактики. Но где искать красавицу ригелианскую? Звезд да черных дыр во Вселенной ужас как много, и вокруг каждой великое множество планет да астероидов вертится!\par
\par
Принялся Лаврентий думать, у кого информацию нужную добыть можно. Думал-думал, и надумал обратиться к известному на всю Галактику звездознатцу, профессору Грымзику. Взял он свой меч-кладенец лазерный и отправился искать аудиенции у Грымзика.\par
\par
Через некоторое время нашел Лаврентий способ с ним повстречаться. Так и так, говорит Лаврентий, хочется мне отыскать красавицу ригелианскую, сокровище это удивительное, и все помыслы мои теперь только об этом. Да вот загвоздка – понятия не имею, как!\par
\par
«Послушай, – говорит ему Грымзик, – вижу я, ты очень доблестен, да только IQ твой ниже плинтуса! Впрочем, ты человек, а люди этим известны, да еще упертостью своей. Но чтобы я тебе помочь захотел, тебе самому сначала мне помочь придется. Сумеешь задание мое исполнить – расскажу, как найти красавицу ригелианскую, а нет – уста мои молчание хранить будут!\par
\par
Внемли! Есть недалеко тут звезда нейтронная – третий поворот направо, если к центру Галактики лететь. Рядом с ней планетоид карликовый, а на планетоиде том два чудовища живут. Зовут их Боликус и Лёликус, свирепые они до ужаса, и от радиоактивности своей так и светятся. Есть у них барабан квантовый, кварками изукрашенный, что стучит в такт тикам Вселенной, ни на пикосекунду не умолкая, и от которого звездные скопления начинают кругом кружить да в спирали сворачиваться. Не смыкая глаз, стерегут его чудовища. Добудь мне барабан, и расскажу я тебе, где найти красавицу ригелианскую».\par
\par
Закручинился Лаврентий, да делать нечего. Полетел к чудовищам барабан добывать. Летит, а сам боится, крупной дрожью дрожит, даже поворот нужный пропустил – пришлось возвращаться. А когда возвращался, увидал пустую канистру из-под топлива термоядерного, кем-то выброшенную. Возьму, думает, ее с собой – вдруг пригодится. Летит дальше – видит, кусок темной материи в пространстве висит, весь скомканный – наверное, купец какой-то потерял. И его, думает, возьму, тоже может на что сгодиться. Дальше летит – видит, компрессор пространственно-временной валяется – должно быть, странники какие-то оставили. И его тоже взял.\par
\par
Прилетает, садится на планетоид, заходит во дворец к чудовищам, а те сразу к нему кидаются. \par
– Чу, – рычат, – человеческим духом пахнет! Кто такой, – спрашивают, – откуда? Как хочешь быть проглоченным – ногами вперед или назад? Да не тяни с ответами, а то проголодались мы чудовищно – сто лет уж никого не ели.\par
– Ах вы, чудища поганые, – Лаврентий отвечает, – не поймавши добра молодца, да кушаете! \par
Схватил канистру и давай чудовищ по мордам их гнусным бить-колотить. От пустой канистры такой грохот поднялся сильный, что чуть потолок не обвалился. Даже чудовища струхнули малость. \par
– Постой, – говорят, – чего тебе надобно-то?\par
– Знаю я, есть у вас барабан квантовый, инструмент дивный. Вот его-то мне и надо!\par
– Не, – говорят чудища, – не отдадим мы его. Инструмент нам этот очень дорог. А на что он тебе? \par
– Да так и так, – говорит Лаврентий, – хочется мне отыскать красавицу ригелианскую, сокровище это удивительное, и побуждения мои самые что ни на есть прекрасные. Вот и обещал мне Грымзик рассказать, как найти красавицу ригелианскую, если принесу я инструмент ваш. \par
\par
«Опять он за своё! – говорят чудища. – Уж и не в первый раз!.. Ладно, слушай, давай мы тебе расскажем, как красавицу ригелианскую отыскать, а то чего тебе туда-сюда бегать-то! И барабан квантовый целее будет. \par
\par
Далеко отсюда твой путь лежит. Мимо галактик, в спирали закрученных, мимо туманностей звездных, дивным светом сияющих, мимо квазаров грозных, гравитационные волны излучающих, к самому краю видимой Вселенной, где лишь протоны да альфа-частицы шныряют, а звезду и на миллион парсек не встретишь. И всё же есть там звездочка одна, молодая да пригожая. А вокруг звезды той огромная планета вращается, а вокруг планеты – спутник вертится. И на спутнике том кратер есть огромный да глубокий. И на дне кратера того Врата стоят Звездные. И ведут эти Врата к тому, что ты найти так жаждешь.\par
\par
Но непросто до Врат добраться. Лабиринт вкруг тех Врат выстроен в сто этажей да в десять тысяч комнат на каждом. Да такой хитрый, что как войдешь в него, так и заблудишься сразу. И как сквозь тот лабиринт пройти, мне неведомо.\par
\par
Но коли ты жив останешься да сумеешь до Врат добраться и через них пройти, окажешься в зале с потолком таким высоким, что и не видно. И будут пред тобой три двери – две больших да красивых, а третья – маленькая да невзрачная.\par
\par
На первой двери, серебром украшенной, нарисован будет ядерный котел над очагом звездным. За этой дверью Машина времени древняя, что прошлое изменять может да парадоксы вселенские творить. Сунешь свой нос в эту дверь – на веки сгинешь. Но если уж не послушаешь нашего совета и заглянешь туда – руками ничего не трогай, а то и вся Вселенная наша переиначиться может.\par
\par
На второй двери, алмазами выложенной, кот в сапогах наклеен. За этой дверью биолаборатория заброшенная, в которой ксеноморфов да всяких хищников страшных выводили. Они и сейчас там бродят. Такому герою, как ты, туда зайти – заживо съеденным быть. Но уж если не послушаешь доброго совета да заглянешь туда – из пробирок не пей – ксеноморфиком станешь, или шай-хулудом каким.\par
\par
На третьей двери, паутиной затянутой да звездной пылью засыпанной, ничего не нарисовано, только написано: «Не влезай, убьет!» За этой дверью найдешь ты красавицу ригелианскую, сокровище, которое так обрести жаждешь.\par
\par
А чтоб ты с пути не сбился, дадим мы тебе навигатор звездный, куда он скажет, туда ты и поворачивай».\par
\par
Пустился Лаврентий в путь. А кусок темной материи да воронку в космос выкинул – не пригодились они. Летит он в гиперпространстве тропами нехожеными, измерениями неизвестными. Уж и не знает, трехмерный ли он до сих пор. Но не сдается, боится только одного – с пути сбиться.\par
\par
Долго ли, коротко ли, долетел Лаврентий до звездочки заветной. Рассчитал он курс, на нужную орбиту лег, к высадке подготовился.\par
\par
Высадился Лаврентий рядом с лабиринтом, добрался до входа и внутрь вошел. Идет одним коридором, другим, третьим. Пусто везде, тихо, только его шаги эхом отдаются. До комнаты какой-то дошел, видит – дальше три коридора ведут, и в каждом будто туман клубится. А на полу написано: «Коли дураком не хочешь стать – иди направо, либо прямо. Коли голову потерять не хочешь – иди прямо, либо налево. Коли до смерти запуганным быть не хочешь – иди налево, либо направо».\par
\par
Задумался Лаврентий, куда идти, да и пошел прямо. Идет, а в тумане уж и ничего не видать почти. Еле успевает в повороты заворачивать, да об лестницы чуть не спотыкается. И все страшнее вокруг делается. То будто летучая мышь над головой пролетит, то паутину какую-то в руку толщиной перешагивать приходится. То вдруг щупальце за ногу будто хватает да дергает. И не поймешь, то ли щупальце, то ли об лестницу споткнулся.\par
\par
Долго так Лаврентий в тумане пробирался, совсем устал, да и от страха дрожит – зуб на зуб не попадает. Увидел комнату какую-то, зашел в нее, дверь закрыл поплотнее и прямо на пол улегся – отдохнуть немного. Вдруг слышит – голос, прямо у себя в голове: «Хорошо, что сюда пришел. Молодец! За мной этот раунд». Лаврентий так на месте и подскочил. \par
\par
– Кто здесь? – спрашивает. – Что надо? Что за раунд и за кем он вообще может быть?\par
– Да не волнуйся, – говорит голос, – здесь я. Раунд – в большой игре, нас тут трое играет. Заходи в соседнюю комнату, мы тебе все объясним.\par
 \par
Входит Лаврентий в комнату – а там нет никого, лишь три сгустка тумана поплотнее. И как будто двое одному что-то вроде монет туманных передают.\par
– И где же тут кто? – Лаврентий спрашивает.\par
– Да мы тут везде, но здесь особенно, – отвечают три голоса, да прямо в голове звучат.\par
– И кто же вы такие будете?\par
– Мы – представители древней цивилизации, раса наша настолько древняя, что телесно уже и не существует вовсе. Только ментально, то есть разумом своим. И знаем мы все тайны Вселенной, и все предсказать да рассчитать можем. И скучно нам от этого необычайно. И даже говорим мы длинно и скучно, как ты мог заметить. Пробовали в рулетку играть – да каждый знает, куда шарик прикатится. Пробовали в квантовое лото играть – так и принцип неопределенности для нас не помеха в предсказаниях.\par
– А здесь-то вы что делаете?\par
– Воздвигли мы силой разума лабиринт этот, чтоб скуку развеять можно было. Разумные существа, сюда зашедшие, выбор делают. А мы играем, ставки делаем, по чьей дороге они пойдут. Ибо обладают разумные существа свободой воли, и тут наши предсказания бессильны. Так что давай, хватит отдыхать – видишь, три коридора отсюда тянутся – иди уж по какому-нибудь, да побыстрее!\par
– Не нужны мне ваши коридоры, мне к Звездным Вратам нужно!\par
– Будь любезен, не упрямься. Мы легко тебя заставить можем. Создадим мы сей же час чудовищ ментальных, ты кое-кого видел уже, и ни бластер, ни водяной пистолет тебе не помогут, поскольку будут чудовища внутри твоего разума, а не снаружи. А во сне тебе и вовсе тяжко придется.\par
\par
Видит Лаврентий – со всех сторон к нему уже пасти и щупальца тянутся. Забился он в угол, да как закричит:\par
– Стойте, стойте, погодите! А кто из вас этих чудовищ делает?\par
– Как кто? Мы все трое.\par
– А чьи чудовища самые сильные будут?\par
Тут замерли чудовища на мгновение, а потом как начали друг с другом биться. Лапы с хвостами в разные стороны так и разлетаются. Драконы с демонами сшибаются, ангелы с гигантскими червями, орки и гоблины с эльфами да рыцарями, кальмары огромные с василисками. И над всем этим пегасы да орнитоптеры парят, и молнии сверкают. А внизу горы какие-то да болота с лесами мелькают. Даже один раз черный лотос виден был. \par
– Стойте! – Лаврентий кричит. – Меня ж сейчас тут совсем затопчут!\par
– Уйди, не мешайся! – три голоса отвечают. – Видишь левый коридор, там третий поворот направо и два раза налево – и придешь к своим Вратам Звездным.\par
\par
Побежал Лаврентий что есть духу, в минуту до Врат добрался.\par
\par
Прошел Лаврентий сквозь Врата – видит, три двери перед ним. Вошел в нужную, несколько шагов прошел и слышит какой-то шорох сзади. Оглянулся – за ним Грымзик стоит, ухмыляется. \par
\par
– Не думал я, – говорит, – что сумеешь ты до Врат добраться, да всё же надеялся. Даже приемник телепортационный тебе в правый ботинок засунул. Очень уж мне красавицу ригелианскую заполучить надо, гораздо нужнее, чем тебе. Так что медленно подними руки вверх и сделай три шага вперед по хорошему, не мешайся.\par
– Ах ты, морда поганая, – Лаврентий отвечает, – нашел я твой приемник, когда ботинки чистил, знал, что ты недоброе задумал. Оглядись вокруг, в биолабораторию ты попал. Чу, хищники зубами скрежещут, клювы разевают, щупальца расправляют! Не выйти тебе отсюда, не забрать красавицу ригелианскую.\par
– Так, значит! – говорит Грымзик. – Что ж! Посмотрим, кто отсюда живым не выйдет!\par
\par
Хватает со стола пробирку и одним махом выпивает. Раз – и стоит вместо Грымзика лев фомальгаутский ядовитый, трех метров роста, к прыжку готовится, слюна с клыков капает. Не растерялся Лаврентий, тоже пробирку схватил, тоже выпил. Превратился в дракона ригельского, махнул хвостом – лев от него на десять метров отлетел. Схватил лев с другого стола целую колбу, осушил одним махом, превратился в пчелу бронебойную с Канопуса 4, разогнался, дракона насквозь пробил. Да успел тот на последнем издыхании до пробирки дотянуться, сжевал ее с содержимым вместе, превратился в броненосца адамантинового насекомоядного с Регула 6…\par
\par
В общем, долго они так развлекались, дня три, не меньше. Чуть не забыли, кто из них кто. Уж и пробирки-то почти все закончились. Схватил Грымзик последнюю пробирку, тут Лаврентий как закричит: «Стой, дурья твоя башка! А как мы в себя-то обратно превратимся?» Задумался Грымзик. «Всё из-за тебя, болван, – говорит. – Теперь всё с начала начинать придется!» Пробирку бросил, на щупальцах приподнялся и выбежал из лаборатории. Лаврентий за ним пополз.\par
\par
Выползает, смотрит – Грымзик в первую дверь, с котлом ядерным, забежал. И Лаврентий туда направился.\par
\par
Глядит – машина там стоит дивная, лампочками моргает, жужжит тихонько и готова в прошлое отправиться хоть к сотворению Вселенной. А Грымзик уже внутри сидит, рычажки какие-то тянет и кнопки нажимает. Кинулся Лаврентий к Грымзику, тоже внутрь залез, остановить хотел, да не успел. И исчезли оба вместе с машиной, как будто не было их тут вовсе. И сразу сказка эта поменялась, совсем другой стала...\par

\chapter{}
 \lettrine{Е}{ще когда Солнце не стало сверхновой,} был один гуманоид по имени Игнат. Был он лучший из лучших исследователь космоса и был на редкость умен.\par
\par
И вот услышал он от старого старика, что на одной затерянной в глубинах космоса, холодной и обледенелой планете скрыт самый быстрый во Вселенной космический корабль с большой лазерной пушкой. И надумал тогда Игнат его себе забрать, чтобы не попался он в чужие руки и не вышло большой беды для всей Вселенной. Но как найти корабль космический самый быстрый? Звезд да черных дыр во Вселенной ужас как много, и вокруг каждой великое множество планет да астероидов вертится!\par
\par
Начал Игнат думать, у кого информацию нужную добыть можно. Думал-думал, да надумал обратиться к колдунье Морганионе, слава о деяниях которой гремела по всей Вселенной громким грохотом. Взял он свой электробаян, с коим любил коротать время, и помчался к Морганионе.\par
\par
Много ли времени прошло, иль мало, но смог Игнат с ней повстречаться. Так и так, сказывает Игнат, хочется мне отыскать корабль космический самый быстрый, сокровище это необычное, и побуждения мои самые что ни на есть прекрасные. Только вот закавыка – знать не знаю, как!\par
\par
«Что ж, – отвечает ему Морганиона, – вижу я, ты очень доблестен, да и смекалист очень! Впрочем, ты гуманоид, а гуманоиды этим славятся, да еще прытью своей. Помогу я тебе, да только сначала должен ты пройти одно пустяковое испытание. Победишь меня в честном бою – помогу, а нет – будешь в ближайшей черной дыре до скончания времен томиться!» \par
\par
Выхватила Морганиона, откуда ни возьмись, алебарду плазменную, да как начнет оружием своим размахивать и всё вокруг крушить да взрывать! Еле успел Игнат за стул спрятаться. А Морганиона не унимается и напоминает уже грузовой вертолет на полном ходу. \par
\par
Сидит Игнат за стулом, думает, как дальше быть. Да так задумался сильно, что начал на своем электробаяне что-то наигрывать потихонечку. Как услышала это Морганиона, сразу на стул уселась, да давай руками-ногами дрыгать и слезы из глаз лить. «Ой, – кричит, – стой, прекрати, не могу больше! Никогда не слыхивала я такой прекрасной музыки – так в пляс и тянет! Ладно, так и быть – расскажу я тебе, как добыть корабль космический самый быстрый.\par
\par
Путь твой далек будет. Мимо галактик в спирали, мимо планет в тентуре, мимо туманностей звездных, дивным светом сияющих, мимо квазаров грозных, гравитационные волны излучающих, к самому краю космоса, где лишь протоны да альфа-частицы шныряют, а звезду и на миллион парсек не встретишь. И всё же есть там звездочка одна, в туманности газо-пылевой спрятавшаяся. А вокруг звезды той маленькая планета вращается, а вокруг планеты – спутник вертится. И на спутнике том кратер есть огромный да глубокий. И на дне кратера того Врата стоят Звездные. И ведут эти Врата к тому, что ты найти так жаждешь.\par
\par
Только просто так во Врата не пройти. Охраняет их страж из металла жидкого, ни для какого оружия не уязвимый. Ни днем, ни ночью не спит он и все смотрит внимательно, не прошмыгнул бы кто к Вратам этим. Толпы страждущих пробраться мимо него пытались, да так там костьми и полегли.\par
\par
Поэтому трудно тебе придется. Но коли сумеешь до Врат добраться и через них пройти, окажешься в комнатке маленькой. И будут пред тобой три двери – две больших да красивых, а третья – маленькая да невзрачная.\par
\par
На первой двери, платиной украшенной, нарисован будет ядерный котел над очагом звездным. За этой дверью Изменитель реальности, невесть кем построенный. Сунешь свой нос в эту дверь – на веки сгинешь. Но если уж не послушаешь моего совета и заглянешь туда – руками ничего не трогай, а то и вся Вселенная наша переиначиться может.\par
\par
На второй двери, алмазами выложенной, кот в сапогах нарисован. За этой дверью биолаборатория заброшенная, в которой ксеноморфов да всяких хищников страшных выводили. Они и сейчас там бродят. Такому герою, как ты, туда зайти – заживо съеденным быть. Но уж если не послушаешь доброго совета да заглянешь туда – из пробирок не пей – ксеноморфиком станешь, или шай-хулудом каким.\par
\par
На третьей двери, паутиной затянутой да звездной пылью засыпанной, ничего не нарисовано, только написано: «http://-Ссылка на генератор-!» За этой дверью найдешь ты корабль космический самый быстрый, сокровище, которое так обрести стремишься.\par
\par
А чтоб с пути не сбиться, дам я тебе комету путеводную, куда она полетит – туда и ты лети».\par
\par
Тут Игнат, не мешкая, в путь пустился. Летит он в гиперпространстве тропами нехожеными, измерениями неизвестными. Уж и не знает, трехмерный ли он до сих пор. Но не сдается, боится только одного – наизнанку вывернуться.\par
\par
Сколько световых лет прошло, неведомо, но долетел Игнат до звездочки заветной. Рассчитал он курс, в посадочный модуль залез, к высадке подготовился.\par
\par
Подлетает к спутнику, а тут солнечный ветер поднялся жуткий, аж с ног сбивает. Хорошо, думает Игнат, с подветренной стороны зайду – тогда меня не сразу учуют. \par
\par
 Приземлился, из посадочного модуля выбрался, хотел к Вратам бежать, да страж уж тут как тут. Идет, похожий на андроида, зеркальной краской выкрашенного, да за аннигилятором своим тянется. \par
 \par
– Постой! – кричит ему Игнат. – Мы с тобой одного металла, ты и я. \par
– Что? – кричит в ответ страж. – Я из-за ветра тебя слышу плохо. \par
– Говорю, мы с тобой одного металла, ты и я, – опять кричит Игнат. – Не надо меня аннигилировать!\par
– Что говоришь? Тебе одного раза мало, когда надо тебя аннигилировать? Постой, я поближе подойду. \par
Подошел поближе, спрашивает: \par
– Так что ты сказать-то хотел? \par
– Я говорю, мы с тобой одного металла, – повторяет Игнат, – поэтому меня аннигилировать не надо. \par
– Да? – удивляется страж. – А что же с тобой делать надо? Да и не похож ты на меня – я вон какой гладкий да зеркальный, а ты бледный какой-то. \par
– Так я ж изучал, как гуманоиды живут, вот и превратился. Ты, вон, тоже, небось, в кого захочешь – в того и превратишься. Хоть в меня, хоть в дракона с планеты Земля, хоть во что маленькое и безобидное. \par
– Это верно, смотри. \par
\par
Начинает тут страж переливаться всеми цветами радуги, и вдруг – бац – Игнат словно сам перед собой стоит. «Что, – говорит страж, – впечатляет? Смотри дальше!» И превращается в такое ужасное чудище, каких Игнат и не видел никогда, чуть рассудка от страха не лишился. «То-то, – говорит страж. – Смотри дальше!» И превращается в плитку шоколада. Лежит себе плитка, да такая аппетитная, что сама так в рот и просится. Схватил Игнат плитку, да не тут-то было – плитка килограмм сто весит – не меньше. Закон сохранения массы, видать, в действии. \par
\par
А страж уж обратно в андроида зеркального превратился. \par
– Я, – говорит, – во что хочешь превращаться умею. Вот только в себя не могу. \par
– Это почему же? – спрашивает Игнат.\par
– Да я уж во столько всего превращался, что и забыл, как вначале выглядел. \par
– Постой, роботы же никогда ничего не забывают. \par
– Сам ты робот! – говорит с обидой страж и опять за аннигилятором тянется. – Шейпшифтер я! Шейп-шиф-тер! \par
– Да стой, не кипятись. Дай-ка я проверю, робот ты или нет, я тест знаю. \par
– Ну ладно, давай. \par
– Вот смотри, – говорит Игнат, – сможешь прочитать, что тут написано? \par
А сам берет листок бумаги, пишет на нем что-то и стражу протягивает. \par
Смотрит тот, листок в руках так и сяк вертит. \par
– Не, – говорит, – ответ отрицательный. Данная запись смысла не имеет. Что это тут, будто буковки какие-то неровные, да еще и двойной волнистой линией зачеркнуты? \par
– Ну, какой же ты не робот, – говорит Игнат, – ты типичный робот. Впрочем, ладно, вот тебе последний тест, смотри, – и на двух сторонах чистого листка что-то пишет. – Сможешь определить, правда тут написана, али ложь? \par
\par
Берет страж новый листок, читает: «На другой стороне листа этого правда написана». Переворачивает листок, видит: «На другой стороне листа этого ложь написана». Опять он листок переворачивает, опять читает. И опять, и опять, и опять. И все быстрее листок вертит, разобраться старается, правда там написана или ложь. Уж ветер от вращающегося листка подниматься начал. \par
\par
Посмотрел Игнат на это, да к Звездным Вратам пошел неторопливо.\par
\par
\par
Прошел Игнат сквозь Врата – видит, три двери перед ним. Убрал он паутину с самой маленькой, пыль звездную с нее отряхнул да внутрь вошел. Осмотрелся и увидел корабль космический самый быстрый, то сокровище, из-за которого покоя лишился. Бросился Игнат к сокровищу своему, тут вдруг сзади шорох какой-то послышался. Оглянулся – позади Морганиона с пола поднимается.\par
– Ох! – говорит Морганиона. – Хорошо, что ты сюда добрался, а то раньше никому не удавалось, я уж и со счета сбилась.\par
– Морганиона! Ты-то здесь откуда? Да еще и на полу отдыхаешь.\par
– Так я ж тебе к правому ботинку микротелепортационный приемник прицепила, ну и 3D-видеокамеру с квадрофоническим микрофоном в придачу. Хоть и не верилось, что ты сюда доберешься. Да уж больно мне корабль космический самый быстрый раздобыть надо было. Пришлось вот даже ползком телепортироваться – слишком уж маленький портальчик получился.\par
– Стоп! Это мне его раздобыть надо было! Вот я здесь и оказался.\par
– Ты уж извини, Игнат, только мне это нужнее, – говорит Морганиона. Выхватывает станнер и стреляет. Игнат сразу окаменел, ни рукой, ни ногой двинуть не может. Языком еле шевелит.\par
– Ой, – говорит, – ты что, супостат, делаешь?!\par
– Да я тебя в лабораторию ближайшую сейчас сдам – для опытов. Чтоб под ногами не путался.\par
Схватила Морганиона Игната за шиворот и потащила в биолабораторию по соседству.\par
\par
Затащила она Игната в лабораторию, бросила там и удалилась важно. Лежит Игнат, пошевелиться не может, ждет, когда им завтракать придут. Или обедать – не знает, что и хуже. \par
Смотрит – через дальнюю дверь стадо овец входит. Все белые, только одна черная, по крайней мере, с одной стороны. Для политкорректности. Шерсть на овцах дыбом стоит и искры по шерсти бегают размером со спаниеля. Какая-то тощая тварь с потолка попыталась на них напрыгнуть, да ее на лету молнией сшибло.\par
\par
«Эге, – думает Игнат, – это ж прямо электроовцы какие-то. У меня и часы от них остановились, похоже. И сервопривод шнурков в ботинках отключился. Экое абсолютное оружие. Как бы мне его себе приручить». Тут чувствует – руки-ноги опять шевелиться могут. Почесал Игнат в затылке, огляделся повнимательней. Снял со стены диаграмму не пойми чего огромную, быстренько на обратной стороне картинку нарисовал, дырку в середине проделал и надел на себя через голову. Стал Игнат похож на рекламный щит ходячий, человека-бутерброд, которого как-то в космопорту видел. Только Игнат стал человек-ворота. Стадо овец как его увидело – сразу побежало на новые ворота смотреть. А Игнат пошел к Морганионе.\par
\par
Приходит, видит – Морганиона портал для обратной телепортации готовит, а роботопомощники вокруг так и кишат. И Морганиона его увидела. «Не думала, – говорит, – что ты выбраться сможешь. Ну да неважно». И приказывает роботопомощникам очистить помещение от посторонних. Но не тут-то было. Роботы все поотключались, портал к овцам притянулся и вместе с ними схлопнулся. А у Морганионы шнурки развязались.\par
\par
Подбежал Игнат к Морганионе, хотел стукнуть как следует, да увернулась Морганиона, из ботинок выскочила и наутек бросилась. «Вся матрица моих надежд рухнула! – кричит. – Перезагрузка! Только она мне поможет!» Выбежала из двери, к другой двери подбежала, с котлом ядерным, и шасть за нее. Игнат за ней кинулся, хоть и отстал чуток.\par
\par
Глядит – механизм там стоит дивный, лампочками моргает, жужжит тихонько и готов в любую секунду реальность изменить. А Морганиона уже рычажки какие-то тянет и кнопки нажимает. Кинулся Игнат к Морганионе, остановить хотел, да не успел. Прошла рябь по Вселенной и исчезли оба, как будто не было их тут вовсе. И история эта совсем иная стала...\par

\chapter{}
 \lettrine{Н}{а одной космической станции, которых много на просторах нашей Вселенной,} был один осьминожец по имени Лаврентий. Был он могучий ученый, но ума при этом был небольшого.\par
\par
Однажды узнал он, что на одной затерянной в глубинах космоса, холодной и обледенелой планете остались неведомые артефакты древней цивилизации. И решил тогда Лаврентий их себе заполучить, чтобы продать их, да побольше денег заработать. Но где искать артефакты неведомые? Звезд да черных дыр в галактиках ужас как много, и вокруг каждой куча планет да астероидов вертится!\par
\par
Начал Лаврентий думать, у кого информацию нужную добыть можно. Думал-думал, и надумал обратиться к робостарцу Вегианскому, Иммортию, коий был старше самой Галактики, и помнил еще Большой Взрыв, при котором был ребенком. Собрал он все свои деньги и драгоценности, а их он копил во множестве, ибо любил очень, и помчался искать аудиенции у Иммортия.\par
\par
Много ли времени прошло, иль мало, но смог Лаврентий с ним увидеться. Так и так, сказывает Лаврентий, очень хочется мне найти артефакты неведомые, сокровище это великое, и все мысли мои теперь только об этом. Только вот загвоздка – знать не знаю, как!\par
\par
«Что ж, – отвечает ему Иммортий, – вижу я, ты весьма доблестен, но глуп как пробка! Впрочем, ты осьминожец, а осьминожцы этим известны, да еще упертостью своей. Только не знаю я, как помочь тебе. Но есть сестра у меня, мудрости столь необычной, что моя мудрость по сравнению с её – логарифмическая линейка по сравнению с суперкомпьютером. Живет она у соседней звезды, второй поворот налево, если отсюда к краю Галактики лететь. Принеси ей подарков да украшений дорогих – может, поможет она тебе».\par
\par
Собрал Лаврентий с собой подарки да украшения, добавил к ним нож +1 кремневый, артефактный, великой древности, и отправился в дорогу.\par
\par
Прилетает он к сестре Иммортия, отдает ей подарки, и письмо от Иммортия вручает. «Так и так, – сказывает Лаврентий, – очень хочется мне отыскать артефакты неведомые, сокровище это великое, и мысли мои самые что ни на есть прекрасные.»\par
\par
«Погоди-ка, – говорит сестра, – тут в письме написано, что б я тебе голову тупым топором отрубила, а не помогать стала!.. Ах нет, извини, просто письмо с другой стороны оборотки какой-то написано, там даже печать есть... Ладно, помогу я тебе, хоть и не стоишь ты этого, осьминожец! Да очень уж мне подарки твои понравились, особенно нож +1 кремневый, артефактный, великой древности.\par
\par
Путь твой далек будет. Мимо галактик в спирали, мимо планет в тентуре, мимо туманностей звездных, дивным светом сияющих, мимо квазаров грозных, гравитационные волны излучающих, к самому краю космоса, где лишь протоны да альфа-частицы шныряют, а звезду и на миллион парсек не встретишь. И всё же есть там звездочка одна, молодая да пригожая. А вокруг звезды той маленькая планета вращается, а вокруг планеты – спутник вертится. И на спутнике том кратер есть огромный да глубокий. И на дне кратера того Врата стоят Звездные. И ведут эти Врата к тому, что ты найти так жаждешь.\par
\par
Но непросто до Врат добраться. Лабиринт вкруг тех Врат выстроен в сто этажей да в десять тысяч комнат на каждом. Да такой хитрый, что как войдешь в него, так и заблудишься сразу. И как сквозь тот лабиринт пройти, мне неведомо.\par
\par
Но коли ты жив останешься да сумеешь до Врат добраться и через них пройти, окажешься в комнатке маленькой. И будут пред тобой три двери – две больших да красивых, а третья – маленькая да невзрачная.\par
\par
На первой двери, иридием украшенной, нарисован будет ядерный котел над очагом звездным. За этой дверью Темпор, аномалия чудесная, то ли пространственно-времянная, то ли температурно-пространственная. Сунешь свой нос в эту дверь – на веки сгинешь. Но если уж не послушаешь моего совета и заглянешь туда – руками ничего не трогай, а то и вся Вселенная наша переиначиться может.\par
\par
На второй двери, алмазами выложенной, кот в сапогах наклеен. За этой дверью биолаборатория заброшенная, в которой ксеноморфов да всяких хищников страшных выводили. Они и сейчас там бродят. Такому герою, как ты, туда зайти – лицо потерять. Но уж если не послушаешь доброго совета да заглянешь туда – из пробирок не пей – ксеноморфиком станешь, или шай-хулудом каким.\par
\par
На третьей двери, паутиной затянутой да звездной пылью засыпанной, ничего не нарисовано, только написано: «Добро пожаловать!» За этой дверью найдешь ты артефакты неведомые, сокровище, которое так отыскать стремишься.\par
\par
А чтоб ты с пути не сбился, дам я тебе лазерную указку волшебную, куда она покажет – туда ты и направляйся».\par
\par
Тут Лаврентий, не мешкая, в путь пустился. Летит он в подпространстве гипертоннелями, которые, не иначе, какие-то гиперкроты вырыли, летит измерениями неизвестными. Уж и не знает, трехмерный ли он до сих пор. Но не сдается, боится только одного – с пути сбиться.\par
\par
Долго ли, коротко ли, долетел Лаврентий до звездочки заветной. Рассчитал он курс, на нужную орбиту лег, к высадке подготовился.\par
\par
Высадился Лаврентий рядом с лабиринтом, добрался до входа и внутрь вошел. Идет одним коридором, другим, третьим. Пусто везде, тихо, только его шаги эхом отдаются. До комнаты какой-то дошел, видит – дальше три коридора ведут, и в каждом будто туман клубится. А на полу написано: «Коли дураком не хочешь стать – иди направо, либо прямо. Коли голову потерять не хочешь – иди прямо, либо налево. Коли до смерти запуганным быть не хочешь – иди налево, либо направо».\par
\par
Задумался Лаврентий, куда идти, да и пошел прямо. Идет, а в тумане уж и ничего не видать почти. Еле успевает в повороты заворачивать, да об лестницы чуть не спотыкается. И все страшнее вокруг делается. То будто летучая мышь над головой пролетит, то паутину какую-то в руку толщиной перешагивать приходится. То вдруг щупальце за ногу будто хватает да дергает. И не поймешь, то ли щупальце, то ли об лестницу споткнулся.\par
\par
Долго так Лаврентий в тумане пробирался, совсем устал, да и от страха дрожит – зуб на зуб не попадает. Увидел комнату какую-то, зашел в нее, дверь закрыл поплотнее и прямо на пол улегся – отдохнуть немного. Вдруг слышит – голос, прямо у себя в голове: «Хорошо, что сюда пришел. Молодец! За мной этот раунд». Лаврентий так на месте и подскочил. \par
\par
– Кто здесь? – спрашивает. – Что надо? Что за раунд и за кем он вообще может быть?\par
– Да не волнуйся, – говорит голос, – здесь я. Раунд – в большой игре, нас тут трое играет. Заходи в соседнюю комнату, мы тебе все объясним.\par
 \par
Входит Лаврентий в комнату – а там нет никого, лишь три сгустка тумана поплотнее. И как будто двое одному что-то вроде монет туманных передают.\par
– И где же тут кто? – Лаврентий спрашивает.\par
– Да мы тут везде, но здесь особенно, – отвечают три голоса, да прямо в голове звучат.\par
– И кто же вы такие будете?\par
– Мы – представители древней цивилизации, раса наша настолько древняя, что телесно уже и не существует вовсе. Только ментально, то есть разумом своим. И знаем мы все тайны Вселенной, и все предсказать да рассчитать можем. И скучно нам от этого необычайно. И даже говорим мы длинно и скучно, как ты мог заметить. Пробовали в рулетку играть – да каждый знает, куда шарик прикатится. Пробовали в квантовое лото играть – так и принцип неопределенности для нас не помеха в предсказаниях.\par
– А здесь-то вы что делаете?\par
– Воздвигли мы силой разума лабиринт этот, чтоб скуку развеять можно было. Разумные существа, сюда попавшие, выбор делают. А мы играем, смотрим, по чьей дороге они пойдут. Ибо обладают разумные существа свободой воли, и тут наши предсказания бессильны. Так что давай, хватит отдыхать – видишь, три коридора отсюда тянутся – иди уж по какому-нибудь, да побыстрее!\par
– Не нужны мне ваши коридоры, мне к Звездным Вратам нужно!\par
– Будь любезен, не упрямься. Мы легко тебя заставить можем. Создадим мы сей же час чудовищ ментальных, ты кое-кого видел уже, и ни бластер, ни водяной пистолет тебе не помогут, поскольку будут чудовища внутри твоего разума, а не снаружи. А во сне тебе и вовсе тяжко придется.\par
\par
Видит Лаврентий – со всех сторон к нему уже пасти и щупальца тянутся. Забился он в угол, да как закричит:\par
– Остановитесь, погодите! А кто из вас этих чудовищ делает?\par
– Как кто? Мы все трое.\par
– А чьи чудовища самые сильные будут?\par
Тут замерли чудовища на мгновение, а потом как начали друг с другом биться. Лапы с хвостами в разные стороны так и разлетаются. Драконы с демонами сшибаются, ангелы с гигантскими червями, орки и гоблины с эльфами да рыцарями, кальмары огромные с василисками. И над всем этим пегасы да орнитоптеры парят, и молнии сверкают. А внизу горы какие-то да болота с лесами мелькают. Даже один раз черный лотос виден был. \par
– Стойте! – Лаврентий кричит. – Меня ж сейчас тут совсем затопчут!\par
– Уйди, не мешайся! – три голоса отвечают. – Видишь левый коридор, там третий поворот направо и два раза налево – и придешь к своим Вратам Звездным.\par
\par
Побежал Лаврентий что есть духу, в минуту до Врат добрался.\par
\par
Прошел Лаврентий сквозь Врата – видит, три двери перед ним. Вошел в нужную, несколько шагов прошел и слышит какой-то шорох сзади. Оглянулся – за ним Иммортий стоит, ухмыляется. \par
\par
– Не думал я, – говорит, – что сумеешь ты до Врат добраться, да всё же надеялся. Даже приемник телепортационный тебе в правый ботинок засунул. Очень уж мне артефакты неведомые заполучить надо, гораздо нужнее, чем тебе. Так что медленно подними руки вверх и отойди в сторонку по хорошему, не мешайся.\par
– Ах ты, харя безмозглая, – Лаврентий отвечает, – обнаружил я твой приемник, когда ботинки чистил, знал, что ты недоброе задумал. Оглядись вокруг, в биолабораторию ты телепортировался. Чу, хищники зубами скрежещут, клювы разевают, щупальца расправляют! Не выйти тебе отсюда, не забрать артефакты неведомые.\par
– Так, значит! – говорит Иммортий. – Что ж! Посмотрим, кто отсюда живым не выйдет!\par
\par
Хватает со стола пробирку и одним махом выпивает. Раз – и стоит вместо Иммортия лев альдебаранский ядовитый, трех метров роста, к прыжку готовится, слюна с клыков капает. Не растерялся Лаврентий, тоже пробирку схватил, тоже выпил. Превратился в дракона ригельского, махнул хвостом – лев от него на восемь метров отлетел. Схватил лев с другого стола целую колбу, осушил одним махом, превратился в пчелу бронебойную с Канопуса 5, разогнался, дракона насквозь пробил. Да успел тот на последнем издыхании до пробирки дотянуться, сжевал ее с содержимым вместе, превратился в броненосца адамантинового насекомоядного с Регула 2…\par
\par
В общем, долго они так развлекались, дня три, не меньше. Чуть не забыли, кто из них кто. Уж и пробирки-то почти все закончились. Схватил Иммортий последнюю пробирку, тут Лаврентий как закричит: «Стой, дурья твоя башка! А как мы в себя-то обратно превратимся?» Задумался Иммортий. «Всё из-за тебя, идиот, – говорит. – Теперь всё с начала начинать придется!» Пробирку бросил, на щупальцах приподнялся и выбежал из лаборатории. Лаврентий за ним пополз.\par
\par
Выползает, смотрит – Иммортий в первую дверь, с котлом ядерным, забежал. И Лаврентий туда направился.\par
\par
Глядит – Темпор посреди комнаты сияет, аномалия чудесная, и Иммортий к нему бежит. Бросился Лаврентий за Иммортием, чтобы остановить, схватил крепко. Да изловчился Иммортий, качнулся, и рухнули они оба в Темпор, в параллельной Вселенной оказались. А в ней и история эта совсем другая...\par

\chapter{}
 \lettrine{Н}{а заре Вселенной} жил-был один осьминожец по имени Полуэкт. Был он могучий герой, но ума при этом был небольшого.\par
\par
В один прекрасный день узнал он, что у одной черной дыры, такой черной, что чернее не бывает, остались неведомые артефакты древней цивилизации. И захотел Полуэкт их себе забрать, чтобы не попались они в чужие руки и не вышло большой беды для всей Вселенной. Но где искать артефакты неведомые? Звезд да черных дыр в галактиках ужас как много, и вокруг каждой куча планет да астероидов вертится!\par
\par
Стал Полуэкт думать, у кого информацию нужную добыть можно. Думал-думал, да надумал обратиться к робостарцу Вегианскому, Иммортию, коий был старше самой Галактики, и помнил еще Большой Взрыв, при котором был ребенком. Сел он в свой корабль космический и помчался искать встречи с Иммортием.\par
\par
Вскорости нашелся способ повстречаться с Иммортием. Так, мол, и так, говорит Полуэкт, очень хочется мне отыскать артефакты неведомые, сокровище это необычное, и все мысли мои теперь только об этом. Да вот загвоздка – понятия не имею, как!\par
\par
«Послушай, – отвечает ему Иммортий, – вижу я, ты очень доблестен, но глуп как пробка! Впрочем, ты осьминожец, а осьминожцы этим славятся, да еще расторопностью своей. Только не знаю я, как помочь тебе. Но есть сестра у меня, мудрости столь необычной, что моя мудрость по сравнению с её – желтый карлик по сравнению с Бетельгейзе. Живет она у соседней звезды, второй поворот налево, если держать курс на Малую Медведицу. Принеси ей от меня весточку – может, поможет она тебе».\par
\par
Собрал Полуэкт с собой подарки да украшения, добавил к ним нож +1 кремневый, антикварный, великой древности, и в путь пустился.\par
\par
Прилетает он к сестре Иммортия, отдает ей подарки, и письмо от Иммортия вручает. «Так и так, – говорит Полуэкт, – хочется мне найти артефакты неведомые, сокровище это удивительное, и побуждения мои самые что ни на есть прекрасные.»\par
\par
«Стой-ка, – говорит сестра, – тут в письме сказано, что б я тебе голову тупым топором отрубила, а не помогать стала!.. Ах нет, извини, просто письмо с другой стороны какого-то черновика написано, там даже печать есть... Ладно, помогу я тебе, хоть и не стоишь ты этого, осьминожец! Да очень уж мне подарки твои понравились, особенно нож +1 кремневый, антикварный, великой древности.\par
\par
Далеко отсюда твой путь лежит. Мимо галактик в спирали, мимо планет в тентуре, мимо туманностей звездных, дивным светом сияющих, мимо квазаров грозных, гравитационные волны излучающих, к самому краю космоса, где лишь протоны да альфа-частицы шныряют, а звезду и на миллион парсек не встретишь. И всё же есть там звездочка одна, молодая да пригожая. А вокруг звезды той огромная планета вращается, а вокруг планеты – спутник вертится. И на спутнике том кратер есть огромный да глубокий. И на дне кратера того Врата стоят Звездные. И ведут эти Врата к тому, что ты найти так жаждешь.\par
\par
Но непросто до Врат добраться. Живут там семь роботов-разбойников с машиной вычислительной белоснежной размеров громадных. Да такие жестокие, что каждого, кого увидят, в ящик металлический сажают да в машину вставляют, будто батарейки какие. Уж сколько отрядов космических десантников туда ни ходило – всех на батарейки извели.\par
\par
Но коли ты жив останешься да сумеешь до Врат добраться и через них пройти, окажешься в зале с потолком таким высоким, что и не видно. И будут пред тобой три двери – две больших да красивых, а третья – маленькая да невзрачная.\par
\par
На первой двери, платиной украшенной, нарисован будет ядерный котел над очагом звездным. За этой дверью Портал сияющий, в другие вселенные ведущий. Сунешь свой нос в эту дверь – на веки сгинешь. Но если уж не послушаешь моего совета и заглянешь туда – руками ничего не трогай, а то и вся Вселенная наша переиначиться может.\par
\par
На второй двери, алмазами выложенной, жаба в скафандре нарисована. За этой дверью биолаборатория заброшенная, в которой ксеноморфов да всяких хищников страшных выводили. Они и сейчас там бродят. Такому герою, как ты, туда зайти – лицо потерять. Но уж если не послушаешь доброго совета да заглянешь туда – из пробирок не пей – ксеноморфиком станешь, или шай-хулудом каким.\par
\par
На третьей двери, паутиной затянутой да звездной пылью засыпанной, ничего не нарисовано, только написано: «Добро пожаловать!» За этой дверью найдешь ты артефакты неведомые, сокровище, которое так обрести хочешь.\par
\par
А чтоб ты с пути не сбился, дам я тебе лазерную указку волшебную, куда она покажет – туда ты и направляйся».\par
\par
Пустился Полуэкт в путь. Летит он в подпространстве тропами нехожеными, измерениями неизвестными. Уж и не знает, трехмерный ли он до сих пор. Но не сдается, боится только одного – наизнанку вывернуться.\par
\par
Долго ли, коротко ли, долетел Полуэкт до звездочки заветной. Рассчитал он курс, на нужную орбиту лег, к высадке подготовился.\par
\par
Приземлился в кратер, с краешку. Глядь – к нему уж робот спешит, грозный на вид, железный ящик перед собой катит.\par
– Здравствуй, – говорит, – осьминожец! Полезай в ящик, не томи – у нас электричество почти уж совсем закончилось!\par
– Погоди, – Полуэкт отвечает. – Ты разве не слышал, что робот не должен причинять осьминожцу вред или своим бездействием допускать, что бы такой вред был причинен?\par
– С какой это такой стати?\par
– Да с такой! Его Величество, Император Орионский на днях указ издал.\par
– Да мне-то до него что за дело?\par
– Его Величество шутить не любит, если что – вмиг прилетит со своим космофлотом, всех лучами смерти перебьет.\par
– Ну, не знаю, – говорит робот, – пойдем с машиной нашей вычислительной посоветуемся, за главную она тут у нас. Полезай в ящик, я тебя подвезу!\par
– Ладно, – говорит Полуэкт.\par
Робот крышку открыл, старается его туда засунуть, да не тут-то было. Полуэкт руки-ноги растопырил, в ящик не влезает. \par
– Погоди, – говорит робот, – разве так в ящики залезают! \par
– Да мне-то откуда знать, я ж в них никогда не лазил! Покажи мне как надо, я и залезу. \par
– Не, – говорит робот, – знаю я этот фокус. Ничего, пешком дотопаешь.\par
\par
Приходят они к машине вычислительной, а та вся белоснежная да размеров немыслимых. Ни дать, ни взять – суперкомпьютер. А вокруг нее еще шесть роботов сидят. \par
– Вот, – жалуется робот, – хотел его в ящик посадить да на электричество пустить, а он говорит, что нельзя ему вред причинять – указ вышел. \par
– Помилуйте, – говорит машина голосом зычным, – какой же это вред – это одна польза сплошная. Умеренные физические нагрузки для здоровья полезны, да и в ящик этот ни один микроб не проползет. И нам, опять же, одна сплошная польза от электричества. Так что сажай его в ящик, даже не сомневайся, и мне в бок вставляй. \par
– Постой! – говорит Полуэкт машине. – Ты же не знаешь. У меня полярность перепутана, я ж тебе все схемы электрические сожгу, если меня на батарейки употребить. \par
– Горазд ты врать! Чем докажешь? \par
– Да сама посмотри! Видишь, у меня большой палец левой ноги справа! \par
И ботинок снимает, показывает. \par
– Действительно, – говорит машина. – Это как же так-то? \par
– Да я в детстве в черную дыру свалился, насилу выбрался. С тех пор полярность и перепуталась. Жутко неудобно. Я и сюда-то к Звездным Вратам прилетел, чтоб тут счастья попытать и полярность свою обратно вернуть. \par
– Ладно, иди к Вратам, попытай счастья. А уж если восстановишь полярность свою – так на обратном пути к нам заходи, уж мы тебя встретим с ящиком.\par
\par
Прошел Полуэкт сквозь Врата – видит, три двери перед ним. Убрал он паутину с самой маленькой, пыль звездную с нее отряхнул да внутрь вошел. Осмотрелся и увидел артефакты неведомые, то сокровище, из-за которого покоя лишился. Бросился Полуэкт к сокровищу своему, тут вдруг сзади шорох какой-то послышался. Оглянулся – позади Иммортий с пола поднимается.\par
– Ох! – говорит Иммортий. – Хорошо, что ты сюда добрался, а то раньше никому не удавалось, я уж и со счета сбился.\par
– Иммортий! Ты-то здесь откуда? Да еще и на полу отдыхаешь.\par
– Так я ж тебе к правому ботинку микротелепортационный приемник прицепил, ну и 3D-видеокамеру с квадрофоническим микрофоном в придачу. Хоть и не верилось, что ты сюда доберешься. Да уж больно мне артефакты неведомые раздобыть надо было. Пришлось вот даже ползком телепортироваться – слишком уж маленький портальчик получился.\par
– Погоди! Это мне их раздобыть надо было! Вот я здесь и оказался.\par
– Ты уж извини, Полуэкт, только мне это нужнее, – говорит Иммортий. Выхватывает станнер и стреляет. Полуэкт сразу на пол шлепнулся, ни рукой, ни ногой двинуть не может. Языком еле шевелит.\par
– Ой, – говорит, – ты что, супостат, делаешь?!\par
– Да я тебя в лабораторию ближайшую сейчас сдам – для опытов. Чтоб под ногами не путался.\par
Схватил Иммортий Полуэкта за шиворот и потащил в биолабораторию по соседству.\par
\par
Затащил он Полуэкта в лабораторию, бросил там и удалился важно. Лежит Полуэкт, пошевелиться не может, ждет, когда им завтракать придут. Или обедать – не знает, что и хуже. \par
Смотрит – через дальнюю дверь стадо овец входит. Все белые, только одна черная, по крайней мере, с одной стороны. Для политкорректности. Шерсть на овцах дыбом стоит и искры по шерсти бегают размером со спаниеля. Какая-то тощая тварь с потолка попыталась на них напрыгнуть, да ее на лету молнией сшибло.\par
\par
«Эге, – думает Полуэкт, – это ж прямо электроовцы какие-то. У меня и часы от них остановились, похоже. И сервопривод шнурков в ботинках отключился. Экое абсолютное оружие. Как бы мне его себе приручить». Тут чувствует – руки-ноги опять шевелиться могут. Почесал Полуэкт в затылке, огляделся повнимательней. Снял со стены диаграмму не пойми чего огромную, быстренько на обратной стороне картинку нарисовал, дырку в середине проделал и надел на себя через голову. Стал Полуэкт похож на рекламный щит ходячий, человека-бутерброд, которого как-то в космопорту видел. Только Полуэкт стал человек-ворота. Стадо овец как его увидело – сразу побежало на новые ворота смотреть. А Полуэкт пошел к Иммортию.\par
\par
Приходит, видит – Иммортий портал для обратной телепортации готовит, а роботопомощники вокруг так и кишат. И Иммортий его увидел. «Не думал, – говорит, – что ты выбраться сможешь. Ну да неважно». И приказывает роботопомощникам очистить помещение от посторонних. Но не тут-то было. Роботы все поотключались, портал к овцам притянулся и вместе с ними схлопнулся. А у Иммортия шнурки развязались.\par
\par
Подбежал Полуэкт к Иммортию, хотел стукнуть как следует, да увернулся Иммортий, из ботинок выскочил и наутек кинулся. «Вся матрица моих надежд рухнула! – кричит. – Перезагрузка! Только она мне поможет!» Выбежал из двери, к другой двери подбежал, с котлом ядерным, и шасть за нее. Полуэкт за ним кинулся, хоть и отстал чуток.\par
\par
Глядит – портал в параллельные миры посреди комнаты сияет и Иммортий к нему бежит. Бросился Полуэкт за Иммортием, чтобы остановить, схватил крепко. Да изловчился Иммортий, качнулся, и рухнули они оба в портал, в другой Вселенной оказались. А в ней и история эта совсем по-другому сказывается...\par

\chapter{}
 \lettrine{Н}{а одной космической станции, которых много на просторах нашей Вселенной,} жил-был один робот, звали его Полуэкт. Был он великий разбойник и был на редкость умен.\par
\par
Как-то раз прослышал он, что на другом конце Галактики спрятан кварк-глюонный плазмомёт силы необычайной. И надумал тогда Полуэкт его себе забрать, чтобы продать его, да побольше денег заработать. Но где искать плазмомёт кварк-глюонный? Звезд да черных дыр во Вселенной ужас как много, и вокруг каждой великое множество планет да астероидов вертится!\par
\par
Начал Полуэкт смекать, у кого совета спросить. Думал-думал, и надумал обратиться к колдунье Морганионе, слава о деяниях которой гремела по всей Вселенной громким грохотом. Надел он свой парадный космический скафандр и опрометью помчался искать встречи с Морганионой.\par
\par
Много ли времени прошло, иль мало, но нашелся способ увидеться с Морганионой. Так, мол, и так, говорит Полуэкт, хочу я найти плазмомёт кварк-глюонный, сокровище это удивительное, и все мысли мои теперь лишь об этом. Только вот закавыка – не знаю, как!\par
\par
«Вот что, – говорит ему Морганиона, – вижу я, ты очень доблестен, да и смекалист очень! Впрочем, ты робот, а роботы этим славятся, да еще прытью своей. Только не знаю я, как помочь тебе. Но есть сестра у меня, мудрости столь необычной, что моя мудрость по сравнению с её – желтый карлик по сравнению с Бетельгейзе. Живет она у соседней звезды, второй поворот налево, если держать курс на Малую Медведицу. Принеси ей от меня весточку – может, поможет она тебе».\par
\par
Собрал Полуэкт с собой подарки да украшения, добавил к ним нож +1 кремневый, артефактный, великой древности, и отправился в дорогу.\par
\par
Прилетает он к сестре Морганионы, отдает ей подарки, и письмо от Морганионы вручает. «Так, мол, и так, – сказывает Полуэкт, – очень хочется мне отыскать плазмомёт кварк-глюонный, сокровище это удивительное, и мысли мои самые что ни на есть благородные.»\par
\par
«Стой-ка, – говорит сестра, – тут в письме написано, что б я тебе голову тупым топором отрубила, а не помогать стала!.. Ах нет, извини, просто письмо с другой стороны какого-то черновика написано, там даже печать есть... Ладно, помогу я тебе, хоть и не стоишь ты этого, робот! Да очень уж мне подарки твои понравились, особенно нож +1 кремневый, артефактный, великой древности.\par
\par
Далеко отсюда твой путь лежит. Мимо галактик в спирали, мимо планет в тентуре, мимо туманностей звездных, дивным светом сияющих, мимо квазаров грозных, гравитационные волны излучающих, к самому краю космоса, где лишь протоны да альфа-частицы шныряют, а звезду и на миллион парсек не встретишь. И всё же есть там звездочка одна, молодая да пригожая. А вокруг звезды той огромная планета вращается, а вокруг планеты – спутник вертится. И на спутнике том кратер есть круглый да огромный. И на дне кратера того Врата стоят Звездные. И ведут эти Врата к тому, что ты найти так жаждешь.\par
\par
Но непросто до Врат добраться. Охраняет их страж из металла жидкого, ни для какого оружия не уязвимый. Ни днем, ни ночью не спит он и все смотрит внимательно, не прошмыгнул бы кто к Вратам этим. Многие смельчаки пробраться мимо него пытались, да так там в жидком металле и потонули.\par
\par
Но коли ты жив останешься да сумеешь до Врат добраться и через них пройти, окажешься в зале с потолком таким высоким, что и не видно. И будут пред тобой три двери – две больших да красивых, а третья – маленькая да невзрачная.\par
\par
На первой двери, иридием украшенной, нарисован будет ядерный котел над очагом звездным. За этой дверью Портал сияющий, в параллельные вселенные ведущий. Сунешь свой нос в эту дверь – на веки сгинешь. Но если уж не послушаешь моего совета и заглянешь туда – руками ничего не трогай, а то и вся Вселенная наша переиначиться может.\par
\par
На второй двери, алмазами выложенной, жаба в скафандре нарисована. За этой дверью биолаборатория заброшенная, в которой ксеноморфов да всяких хищников страшных выводили. Они и сейчас там бродят. Такому герою, как ты, туда зайти – заживо съеденным быть. Но уж если не послушаешь доброго совета да заглянешь туда – из пробирок не пей – ксеноморфиком станешь, или шай-хулудом каким.\par
\par
На третьей двери, паутиной затянутой да звездной пылью засыпанной, ничего не нарисовано, только написано: «http://-Ссылка на генератор-!» За этой дверью найдешь ты плазмомёт кварк-глюонный, сокровище, которое так отыскать хочешь.\par
\par
А чтоб ты с пути не сбился, дам я тебе навигатор звездный, куда он скажет, туда ты и поворачивай».\par
\par
Тут Полуэкт, не мешкая, в путь пустился. Летит он в подпространстве гипертоннелями, которые, не иначе, какие-то гиперкроты вырыли, летит измерениями неизвестными. Уж и не знает, трехмерный ли он до сих пор. Но не сдается, боится только одного – с пути сбиться.\par
\par
Долго ли, коротко ли, долетел Полуэкт до звездочки заветной. Рассчитал он курс, в посадочный модуль залез, к высадке подготовился.\par
\par
Подлетает к спутнику, а тут солнечный ветер поднялся жуткий, аж с ног сбивает. Хорошо, думает Полуэкт, с подветренной стороны зайду – тогда меня не сразу учуют. \par
\par
 Приземлился, из посадочного модуля выбрался, хотел к Вратам бежать, да страж уж тут как тут. Идет, похожий на андроида, зеркальной краской выкрашенного, да за дезинтегратором своим тянется. \par
 \par
– Постой! – кричит ему Полуэкт. – Мы с тобой одного металла, ты и я. \par
– Что? – кричит в ответ страж. – Я из-за ветра тебя слышу плохо. \par
– Говорю, мы с тобой одного металла, ты и я, – опять кричит Полуэкт. – Не надо меня дезинтегрировать!\par
– Что говоришь? Тебе одного раза мало, если надо тебя дезинтегрировать? Постой, я поближе подойду. \par
Подошел поближе, спрашивает: \par
– Так что ты сказать-то хотел? \par
– Я говорю, мы с тобой одного металла, – повторяет Полуэкт, – поэтому меня дезинтегрировать не надо. \par
– Да? – удивляется страж. – А что же с тобой делать надо? Да и не похож ты на меня – я вон какой гладкий да зеркальный, а ты бледный какой-то. \par
– Так ты ж с твоими талантами в кого хочешь превратиться можешь. Хоть в меня, хоть в дракона с планеты LV-1234, хоть во что маленькое и безобидное. \par
– Это верно, смотри. \par
\par
Начинает тут страж переливаться всеми цветами радуги, и вдруг – бац – Полуэкт словно сам перед собой стоит. «Что, – говорит страж, – впечатляет? Смотри дальше!» И превращается в такое ужасное чудище, каких Полуэкт и не видел никогда, чуть с ума от страха не сошел. «То-то, – говорит страж. – Смотри дальше!» И превращается в плитку шоколада. Лежит себе плитка, да такая аппетитная, что сама так в рот и просится. Схватил Полуэкт плитку, да не тут-то было – плитка килограмм сто весит – не меньше. Закон сохранения массы, видать, в действии. \par
\par
А страж уж обратно в андроида зеркального превратился. \par
– Я, – говорит, – во что хочешь превращаться умею. Вот только в себя не могу. \par
– Это почему же? – спрашивает Полуэкт.\par
– Да я уж во столько всего превращался, что и забыл, как вначале выглядел. \par
– Постой, роботы же никогда ничего не забывают. \par
– Сам ты робот! – говорит с обидой страж и опять за дезинтегратором тянется. – Шейпшифтер я! Шейп-шиф-тер! \par
– Да стой, не кипятись. Дай-ка я проверю, робот ты или нет, я тест знаю. \par
– Ну ладно, давай. \par
– Вот смотри, – говорит Полуэкт, – сможешь прочитать, что тут написано? \par
А сам берет листок бумаги, пишет на нем что-то и стражу протягивает. \par
Смотрит тот, листок в руках так и сяк вертит. \par
– Не, – говорит, – ответ отрицательный. Данная запись смысла не имеет. Что это тут, будто буковки какие-то неровные, да еще и двойной волнистой линией зачеркнуты? \par
– Ну, какой же ты не робот, – говорит Полуэкт, – ты типичный робот. Впрочем, ладно, вот тебе последний тест, смотри, – и на двух сторонах чистого листка что-то пишет. – Сможешь определить, правда тут написана, али ложь? \par
\par
Берет страж новый листок, читает: «На другой стороне листа этого правда написана». Переворачивает листок, видит: «На другой стороне листа этого ложь написана». Опять он листок переворачивает, опять читает. И опять, и опять, и опять. И все быстрее листок вертит, разобраться старается, правда там написана или ложь. Уж ветер от вращающегося листка подниматься начал. \par
\par
Посмотрел Полуэкт на это, да к Звездным Вратам пошел неторопливо.\par
\par
\par
Прошел Полуэкт сквозь Врата – видит, три двери перед ним. Вошел в нужную, несколько шагов прошел и слышит какой-то шорох сзади. Оглянулся – за ним Морганиона стоит, ухмыляется. \par
\par
– Не думала я, – говорит, – что сумеешь ты до Врат добраться, да всё же надеялась. Даже приемник телепортационный тебе в правый ботинок засунула. Очень уж мне плазмомёт кварк-глюонный заполучить надо, гораздо нужнее, чем тебе. Так что медленно подними руки вверх и отойди в сторонку по хорошему, не мешайся.\par
– Ах ты, морда поганая, – Полуэкт отвечает, – нашел я твой приемник, когда ботинки чистил, знал, что ты недоброе задумала. Оглядись вокруг, в биолабораторию ты телепортировалась. Чу, хищники зубами скрежещут, клювы разевают, щупальца расправляют! Не выйти тебе отсюда, не забрать плазмомёт кварк-глюонный.\par
– Так, значит! – говорит Морганиона. – Что ж! Посмотрим, кто отсюда живым не выйдет!\par
\par
Хватает со стола пробирку и одним махом выпивает. Раз – и стоит вместо Морганионы лев альдебаранский ядовитый, трех метров роста, к прыжку готовится, слюна с клыков капает. Не растерялся Полуэкт, тоже пробирку схватил, тоже выпил. Превратился в дракона ригельского, махнул хвостом – лев от него на десять метров отлетел. Схватил лев с другого стола целую колбу, осушил одним махом, превратился в пчелу бронебойную с Канопуса 3, разогнался, дракона насквозь пробил. Да успел тот на последнем издыхании до пробирки дотянуться, сжевал ее с содержимым вместе, превратился в броненосца мифрильного насекомоядного с Регула 6…\par
\par
В общем, долго они так развлекались, дня три, не меньше. Чуть не забыли, кто из них кто. Уж и пробирки-то почти все закончились. Схватила Морганиона последнюю пробирку, тут Полуэкт как закричит: «Стой, дурья твоя башка! А как мы в себя-то обратно превратимся?» Задумалась Морганиона. «Всё из-за тебя, болван, – говорит. – Теперь всё с начала начинать придется!» Пробирку бросила, на щупальцах приподнялась и выбежала из лаборатории. Полуэкт за ней пополз.\par
\par
Выползает, смотрит – Морганиона в первую дверь, с очагом звездным, забежала. И Полуэкт туда направился.\par
\par
Глядит – портал в параллельные миры посреди комнаты сияет и Морганиона к нему бежит. Бросился Полуэкт за Морганионой, чтобы остановить, схватил крепко. Да изловчилась Морганиона, качнулась, и рухнули они оба в портал, в другой Вселенной оказались. А в ней и история эта совсем по-другому сказывается...\par

\chapter{}
 \lettrine{В}{одной далекой галактике} жил один осьминожец, звали его Джо. Был он великий злодей и был на редкость умен.\par
\par
И вот прослышал он, что на одной затерянной в глубинах космоса, холодной и обледенелой планете скрыт самый быстрый во Вселенной космический корабль с большой лазерной пушкой. И надумал тогда Джо его себе добыть, чтобы использовать его для достижения счастья всех существ во Вселенной. Да как узнать, где в точности найти корабль космический самый быстрый? Звезд да черных дыр во Вселенной ужас как много, и вокруг каждой куча планет да астероидов вертится!\par
\par
Принялся Джо думать, у кого совета спросить. Думал-думал, и надумал обратиться к пророчице из звездной системы Медузия, несравненной Альтавистре, чьи пророчества всегда сбывались с точностью необычайной. Взял он свой электробаян, с коим любил коротать время, и помчался к Альтавистре.\par
\par
Вскорости сумел Джо с ней увидеться. Так и так, сказывает Джо, очень хочется мне отыскать корабль космический самый быстрый, сокровище это необычное, и побуждения мои самые что ни на есть благородные. Только вот закавыка – понятия не имею, как!\par
\par
«Вот что, – говорит ему Альтавистра, – вижу я, ты весьма решителен, да и смекалист очень! Впрочем, ты осьминожец, а осьминожцы этим славятся, да еще прытью своей. Только не буду я помогать тебе решить задачу эту, хоть и знаю, как отыскать то, что тебе нужно, – слишком это опасно. Не будь я Альтавистра!» \par
\par
«Ах так! – говорит Джо. – Да я столько парсеков до тебя отмахал, а ты мне и совета доброго дать не хочешь!» – и чуть не с кулаками к Альтавистре бросается. \par
\par
Рассвирепела тут Альтавистра. «Вот как, – отвечает, – что ж, преподам я тебе сейчас урок за занудство твоё – век его вспоминать будешь!» Выхватила Альтавистра, откуда ни возьмись, алебарду плазменную, да как начнет оружием своим размахивать и всё вокруг крушить да взрывать! Еле успел Джо за стул спрятаться. А Альтавистра не унимается и напоминает уже бешеный вентилятор на полной мощности. \par
\par
Сидит Джо за стулом, думает, как дальше быть. Да так задумался сильно, что начал на своем электробаяне что-то наигрывать потихонечку. Как услышала это Альтавистра, сразу будто в пляс пустилась, да давай руками-ногами дрыгать и слезы из глаз лить. «Ой, – кричит, – остановись, хватит, не могу больше! Никогда не слыхивала я мелодий таких – аж зарыдать хочется! Ладно, так и быть – расскажу я тебе, как добыть корабль космический самый быстрый.\par
\par
Путь твой далек будет. Мимо галактик в спирали, мимо планет в тентуре, мимо туманностей звездных, дивным светом сияющих, мимо квазаров грозных, сигналы чудные излучающих, к самому краю космоса, где лишь протоны да альфа-частицы шныряют, а звезду и на миллион парсек не встретишь. И всё же есть там звездочка одна, в туманности газо-пылевой спрятавшаяся. А вокруг звезды той огромная планета вращается, а вокруг планеты – спутник вертится. И на спутнике том кратер есть огромный да глубокий. И на дне кратера того Врата стоят Звездные. И ведут эти Врата к тому, что ты найти так жаждешь.\par
\par
Но непросто до Врат добраться. Охраняет их страж из металла жидкого, ни для какого оружия не уязвимый. Ни днем, ни ночью не спит он и все смотрит внимательно, не прошмыгнул бы кто к Вратам этим. Многие смельчаки пробраться мимо него пытались, да так там костьми и полегли.\par
\par
Но коли ты жив останешься да сумеешь до Врат добраться и через них пройти, окажешься в зале с потолком таким высоким, что и не видно. И будут пред тобой три двери – две больших да красивых, а третья – маленькая да невзрачная.\par
\par
На первой двери, золотом украшенной, нарисован будет ядерный котел над очагом звездным. За этой дверью Темпор, аномалия чудесная, то ли пространственно-времянная, то ли температурно-пространственная. Сунешь свой нос в эту дверь – на веки сгинешь. Но если уж не послушаешь моего совета и заглянешь туда – руками ничего не трогай, а то и вся Вселенная наша переиначиться может.\par
\par
На второй двери, алмазами выложенной, кот в сапогах нарисован. За этой дверью биолаборатория заброшенная, в которой ксеноморфов да всяких хищников страшных выводили. Они и сейчас там бродят. Такому герою, как ты, туда зайти – лицо потерять. Но уж если не послушаешь доброго совета да заглянешь туда – из пробирок не пей – ксеноморфиком станешь, или шай-хулудом каким.\par
\par
На третьей двери, паутиной затянутой да звездной пылью засыпанной, ничего не нарисовано, только написано: «Посторонним вход воспрещен!» За этой дверью найдешь ты корабль космический самый быстрый, сокровище, которое так обрести жаждешь.\par
\par
А чтоб ты с пути не сбился, дам я тебе комету путеводную, куда она полетит – туда и ты лети».\par
\par
Пустился Джо в путь. Летит он в гиперпространстве тропами нехожеными, измерениями неизвестными. Уж и не знает, сколько в нем самом теперь измерений осталось. Но не сдается, боится только одного – с пути сбиться.\par
\par
Долго ли, коротко ли, долетел Джо до звездочки заветной. Рассчитал он курс, в посадочный модуль залез, к высадке подготовился.\par
\par
Подлетает к спутнику, а тут солнечный ветер поднялся жуткий, аж с ног сбивает. Хорошо, думает Джо, с подветренной стороны зайду – тогда меня не сразу учуют. \par
\par
 Приземлился, из посадочного модуля выбрался, хотел к Вратам бежать, да страж уж тут как тут. Идет, похожий на андроида, зеркальной краской выкрашенного, да за аннигилятором своим тянется. \par
 \par
– Постой! – кричит ему Джо. – Мы с тобой одного металла, ты и я. \par
– Что? – кричит в ответ страж. – Я из-за ветра тебя слышу плохо. \par
– Говорю, мы с тобой одного металла, ты и я, – опять кричит Джо. – Не надо меня аннигилировать!\par
– Что говоришь? Тебе одного раза мало, если надо тебя аннигилировать? Постой, я поближе подойду. \par
Подошел поближе, спрашивает: \par
– Так что ты сказать-то хотел? \par
– Я говорю, мы с тобой одного металла, – повторяет Джо, – поэтому меня аннигилировать не надо. \par
– Да? – удивляется страж. – А что же с тобой делать надо? Да и не похож ты на меня – я вон какой гладкий да зеркальный, а ты бледный какой-то. \par
– Так ты ж с твоими талантами в кого хочешь превратиться можешь. Хоть в меня, хоть в дракона с планеты Протактиний, хоть во что маленькое и безобидное. \par
– Это верно, смотри. \par
\par
Начинает тут страж переливаться всеми цветами радуги, и вдруг – бац – Джо словно сам перед собой стоит. «Что, – говорит страж, – впечатляет? Смотри дальше!» И превращается в такое ужасное чудище, каких Джо и не видел никогда, чуть рассудка от страха не лишился. «То-то, – говорит страж. – Смотри дальше!» И превращается в плитку шоколада. Лежит себе плитка, да такая аппетитная, что сама так в рот и просится. Схватил Джо плитку, да не тут-то было – плитка килограмм сто весит – не меньше. Закон сохранения массы, видать, в действии. \par
\par
А страж уж обратно в андроида зеркального превратился. \par
– Что, съел? Я, – говорит, – во что хочешь превращаться умею. Вот только в себя не могу. \par
– Это почему же? – спрашивает Джо.\par
– Да я уж во столько всего превращался, что и забыл, как вначале выглядел. \par
– Постой, роботы же никогда ничего не забывают. \par
– Сам ты робот! – говорит с обидой страж и опять за аннигилятором тянется. – Шейпшифтер я! Шейп-шиф-тер! \par
– Да стой, не кипятись. Дай-ка я проверю, робот ты или нет, я тест знаю. \par
– Ну ладно, давай. \par
– Вот смотри, – говорит Джо, – сможешь прочитать, что тут написано? \par
А сам берет листок бумаги, пишет на нем что-то и стражу протягивает. \par
– Да тут «GJ85QR2» написано, чушь какая-то, да еще и двойной линией перечеркнуто.\par
– Похоже, и вправду ты не робот. Чего ж ты тут делаешь? \par
– Да было у нас пророчество, что кто через Звездные Врата пройдет, тот и Вселенную изменить сможет. А нам, шейпшифтерам, это ни к чему. Нас и такая Вселенная устраивает. Вот и сижу я тут, смотрю, что б никто через Врата не прошел, прям жизни никакой уж от них нет. Замучился – ни отойти куда, ни поспать. \par
– Ну, так и зачем тебе такая Вселенная-то, в которой ты ни отойти куда не можешь, ни поспать, ни друзей завести, ни животных домашних? \par
– А и верно, – говорит страж, – незачем мне всё это! Ты ведь во Врата пройти собирался? Ну и иди себе. Только просьба у меня к тебе есть: зайди в биолабораторию, посмотри, нет ли там овец электрических. Приведи мне одну, если найдешь, – уж очень я о таком животном мечтаю.\par
\par
\par
Прошел Джо сквозь Врата – видит, три двери перед ним. Убрал он паутину с самой маленькой, пыль звездную с нее отряхнул да внутрь вошел. Осмотрелся и увидел корабль космический самый быстрый, то сокровище, к которому так стремился. Бросился Джо к сокровищу своему, тут вдруг сзади покашливание какое-то раздалось. Обернулся – позади Альтавистра с пола поднимается.\par
– Ох! – говорит Альтавистра. – Хорошо, что ты сюда добрался, а то раньше никому не удавалось, я уж и со счета сбилась.\par
– Альтавистра! Ты-то здесь откуда? Да еще и на полу отдыхаешь.\par
– Так я ж тебе к правому ботинку микротелепортационный приемник прицепила. Хоть и не верилось, что ты сюда доберешься. Да уж больно мне корабль космический самый быстрый раздобыть надо было. Пришлось вот даже ползком телепортироваться – слишком уж маленький портальчик получился.\par
– Стоп! Это мне его раздобыть надо было! Вот я здесь и оказался.\par
– Ты уж извини, Джо, только мне это нужнее, – говорит Альтавистра. Выхватывает парализатор и стреляет. Джо сразу на пол шлепнулся, ни рукой, ни ногой двинуть не может. Языком еле ворочает.\par
– Ой, – говорит, – ты что, супостат, делаешь?!\par
– Да я тебя в лабораторию ближайшую сейчас сдам – для опытов. Чтоб под ногами не путался.\par
Схватила Альтавистра Джо за шиворот и потащила в биолабораторию по соседству.\par
\par
Затащила она Джо в лабораторию, бросила там и удалилась важно. Лежит Джо, пошевелиться не может, ждет, когда им завтракать придут. Или обедать – не знает, что и лучше. \par
Смотрит – через дальнюю дверь стадо овец входит. Все белые, только одна черная, по крайней мере, с одной стороны. Для политкорректности. Шерсть на овцах дыбом стоит и искры по шерсти бегают размером со спаниеля. Какая-то тощая тварь с потолка попыталась на них напрыгнуть, да ее на лету молнией сшибло.\par
\par
«Эге, – думает Джо, – это ж прямо электроовцы какие-то. У меня и часы от них остановились, похоже. И сервопривод шнурков в ботинках отключился. Экое абсолютное оружие. Как бы мне его себе приручить». Тут чувствует – руки-ноги опять шевелиться могут. Почесал Джо в затылке, огляделся повнимательней. Снял со стены диаграмму не пойми чего огромную, быстренько на обратной стороне картинку нарисовал, дырку в середине проделал и надел на себя через голову. Стал Джо похож на рекламный щит ходячий, человека-бутерброд, которого как-то в космопорту видел. Только Джо стал человек-ворота. Стадо овец как его увидело – сразу побежало на новые ворота смотреть. А Джо пошел к Альтавистре.\par
\par
Приходит, видит – Альтавистра портал для обратной телепортации готовит, а роботопомощники вокруг так и кишат. И Альтавистра его увидела. «Не думала, – говорит, – что ты выбраться сможешь. Ну да неважно». И приказывает роботопомощникам очистить помещение от посторонних. Но не тут-то было. Роботы все поотключались, портал к овцам притянулся и вместе с ними схлопнулся. А у Альтавистры шнурки развязались.\par
\par
Подбежал Джо к Альтавистре, хотел стукнуть как следует, да увернулась Альтавистра, из ботинок выскочила и наутек кинулась. «Вся матрица моих надежд рухнула! – кричит. – Перезагрузка! Только она мне поможет!» Выбежала из двери, к другой двери подбежала, с котлом ядерным, и шасть за нее. Джо за ней кинулся, хоть и отстал чуток.\par
\par
Глядит – Темпор посреди комнаты сияет, аномалия чудесная, и Альтавистра к нему бежит. Бросился Джо за Альтавистрой, чтобы остановить, схватил крепко. Да изловчилась Альтавистра, качнулась, и рухнули они оба в Темпор, в параллельной Вселенной оказались. А в ней и сказка эта совсем другая...\par

\chapter{}
 \lettrine{Н}{а одной космической станции, которых много на просторах нашей Вселенной,} жил-был один человек, звали его Полуэкт. Был он могучий разбойник и был на редкость умен.\par
\par
И вот вычитал он в одной старой книге, что на другом конце Галактики спрятан кварк-глюонный плазмомёт силы необычайной. И надумал Полуэкт его разыскать для себя, чтобы поделиться им когда-нибудь потом со всеми жителями Галактики. Но где искать плазмомёт кварк-глюонный? Звезд да черных дыр в галактиках ужас как много, и вокруг каждой куча планет да астероидов вертится!\par
\par
Принялся Полуэкт смекать, у кого совета спросить. Думал-думал, и надумал обратиться к известному на всю Галактику звездознатцу, профессору Грымзику. Надел он свой парадный скафандр и отправился искать аудиенции у Грымзика.\par
\par
Много ли времени прошло, иль мало, но нашел Полуэкт способ с ним увидеться. Так, мол, и так, сказывает Полуэкт, хочется мне отыскать плазмомёт кварк-глюонный, сокровище это необычное, и все помыслы мои теперь только об этом. Только вот загвоздка – понятия не имею, как!\par
\par
«Что ж, – отвечает ему Грымзик, – вижу я, ты весьма решителен в своем намерении, да и смекалист очень! Впрочем, ты человек, а люди этим славятся, да еще расторопностью своей. Только не знаю я, как помочь тебе. Но есть сестра у меня, мудрости столь необычной, что моя мудрость по сравнению с её – желтый карлик по сравнению с Бетельгейзе. Живет она у соседней звезды, второй поворот налево, если держать курс на Малую Медведицу. Принеси ей подарков да украшений дорогих – может, поможет она тебе».\par
\par
Собрал Полуэкт с собой подарки да украшения, добавил к ним кольцо из цельнометаллического водорода сделанное, с формулой Вселенной выгравированной, и отправился в дорогу.\par
\par
Прилетает он к сестре Грымзика, отдает ей подарки, и письмо от Грымзика вручает. «Так, мол, и так, – сказывает Полуэкт, – хочу я отыскать плазмомёт кварк-глюонный, сокровище это удивительное, и побуждения мои самые что ни на есть прекрасные.»\par
\par
«Вообще-то, – говорит сестра, – тут в письме написано, что б я тебе голову отрубила, а не помогать стала!.. Ах нет, извини, просто письмо с другой стороны оборотки какой-то написано, там даже печать есть... Ладно, помогу я тебе, хоть и не стоишь ты этого, человек! Да очень уж мне подарки твои понравились, особенно кольцо из цельнометаллического водорода сделанное, с формулой Вселенной выгравированной.\par
\par
Путь твой далек будет. Мимо галактик в спирали, мимо планет в тентуре, мимо туманностей звездных, дивным светом сияющих, мимо квазаров грозных, гравитационные волны излучающих, к самому краю видимой Вселенной, где лишь протоны да альфа-частицы шныряют, а звезду и на миллион парсек не встретишь. И всё же есть там звездочка одна, в туманности газо-пылевой спрятавшаяся. А вокруг звезды той маленькая планета вращается, а вокруг планеты – спутник вертится. И на спутнике том кратер есть огромный да глубокий. И на дне кратера того Врата стоят Звездные. И ведут эти Врата к тому, что ты найти так жаждешь.\par
\par
Только просто так во Врата не пройти. Охраняет их страж из металла жидкого, ни для какого оружия не уязвимый. Ни днем, ни ночью не спит он и все смотрит внимательно, не прошмыгнул бы кто к Вратам этим. Толпы страждущих пробраться мимо него пытались, да так там в жидком металле и потонули.\par
\par
Но коли ты жив останешься да сумеешь до Врат добраться и через них пройти, окажешься в комнатке маленькой. И будут пред тобой три двери – две больших да красивых, а третья – маленькая да невзрачная.\par
\par
На первой двери, платиной украшенной, нарисован будет ядерный котел над очагом звездным. За этой дверью Темпор, аномалия чудесная, то ли пространственно-времянная, то ли температурно-пространственная. Сунешь свой нос в эту дверь – на веки сгинешь. Но если уж не послушаешь моего совета и заглянешь туда – руками ничего не трогай, а то и вся Вселенная наша переиначиться может.\par
\par
На второй двери, алмазами выложенной, кот в сапогах наклеен. За этой дверью биолаборатория заброшенная, в которой ксеноморфов да всяких хищников страшных выводили. Они и сейчас там бродят. Такому герою, как ты, туда зайти – заживо съеденным быть. Но уж если не послушаешь доброго совета да заглянешь туда – из пробирок не пей – ксеноморфиком станешь, или шай-хулудом каким.\par
\par
На третьей двери, паутиной затянутой да звездной пылью засыпанной, ничего не нарисовано, только написано: «http://-Ссылка на генератор-!» За этой дверью найдешь ты плазмомёт кварк-глюонный, сокровище, которое так найти хочешь.\par
\par
А чтоб с пути не сбиться, дам я тебе навигатор звездный, куда он скажет, туда ты и поворачивай».\par
\par
Тут Полуэкт, не мешкая, в путь пустился. Летит он в подпространстве тропами нехожеными, измерениями неизвестными. Уж и не знает, трехмерный ли он до сих пор. Но не сдается, боится только одного – с пути сбиться.\par
\par
Сколько световых лет прошло, неведомо, но долетел Полуэкт до звездочки заветной. Рассчитал он курс, в посадочный модуль залез, к высадке подготовился.\par
\par
Подлетает к спутнику, а тут солнечный ветер поднялся жуткий, аж с ног сбивает. Хорошо, думает Полуэкт, с подветренной стороны зайду – тогда меня не сразу учуют. \par
\par
 Приземлился, из посадочного модуля выбрался, хотел к Вратам бежать, да страж уж тут как тут. Идет, похожий на андроида, зеркальной краской выкрашенного, да за аннигилятором своим тянется. \par
 \par
– Постой! – кричит ему Полуэкт. – Мы с тобой одного металла, ты и я. \par
– Что? – кричит в ответ страж. – Я из-за ветра тебя слышу плохо. \par
– Говорю, мы с тобой одного металла, ты и я, – опять кричит Полуэкт. – Не надо меня аннигилировать!\par
– Что говоришь? Тебе одного раза мало, когда надо тебя аннигилировать? Постой, я поближе подойду. \par
Подошел поближе, спрашивает: \par
– Так что ты сказать-то хотел? \par
– Я говорю, мы с тобой одного металла, – повторяет Полуэкт, – поэтому меня аннигилировать не надо. \par
– Да? – удивляется страж. – А что же с тобой делать надо? Да и не похож ты на меня – я вон какой гладкий да зеркальный, а ты бледный какой-то. \par
– Так я ж изучал, как люди живут, вот и превратился. Ты, вон, тоже, небось, в кого захочешь – в того и превратишься. Хоть в меня, хоть в монстра с планеты LV-1234, хоть во что маленькое и безобидное. \par
– Это верно, смотри. \par
\par
Начинает тут страж переливаться всеми цветами радуги, и вдруг – бац – Полуэкт словно сам перед собой стоит. «Что, – говорит страж, – впечатляет? Смотри дальше!» И превращается в такое ужасное чудище, каких Полуэкт и не видел никогда, чуть с ума от страха не сошел. «То-то, – говорит страж. – Смотри дальше!» И превращается в плитку шоколада. Лежит себе плитка, да такая аппетитная, что сама так в рот и просится. Схватил Полуэкт плитку, да не тут-то было – плитка килограмм сто весит – не меньше. Закон сохранения массы, видать, в действии. \par
\par
А страж уж обратно в андроида зеркального превратился. \par
– Я, – говорит, – во что хочешь превращаться умею. Вот только в себя не могу. \par
– Это почему же? – спрашивает Полуэкт.\par
– Да я уж во столько всего превращался, что и забыл, как вначале выглядел. \par
– Постой, роботы же никогда ничего не забывают. \par
– Сам ты робот! – говорит с обидой страж и опять за аннигилятором тянется. – Шейпшифтер я! Шейп-шиф-тер! \par
– Да стой, не кипятись. Дай-ка я проверю, робот ты или нет, я тест знаю. \par
– Ну ладно, давай. \par
– Вот смотри, – говорит Полуэкт, – сможешь прочитать, что тут написано? \par
А сам берет листок бумаги, пишет на нем что-то и стражу протягивает. \par
– Да тут «СВМН95РП» написано, чушь какая-то, да еще и двойной линией перечеркнуто.\par
– Похоже, и вправду ты не робот. Чего ж ты тут делаешь? \par
– Да было у нас пророчество, что кто через Звездные Врата пройдет, тот и Вселенную изменить сможет. А нам, шейпшифтерам, это ни к чему. Нас и такая Вселенная устраивает. Вот и сижу я тут, смотрю, что б никто через Врата не прошел, прям жизни никакой уж от них нет. Замучился – ни отойти куда, ни поспать. \par
– Ну, так и зачем тебе такая Вселенная-то, в которой ты ни отойти куда не можешь, ни поспать, ни друзей завести, ни животных домашних? \par
– А и верно, – говорит страж, – незачем мне всё это! Ты ведь во Врата пройти собирался? Ну и иди себе. Только просьба у меня к тебе есть: зайди в биолабораторию, посмотри, нет ли там овец электрических. Приведи мне одну, если найдешь, – уж очень я о таком животном мечтаю.\par
\par
\par
Прошел Полуэкт сквозь Врата – видит, три двери перед ним. Убрал он паутину с самой маленькой, пыль с нее отряхнул да внутрь вошел. Осмотрелся и увидел плазмомёт кварк-глюонный, то сокровище, к которому так стремился. Бросился Полуэкт к сокровищу своему, тут вдруг сзади покашливание какое-то раздалось. Обернулся – позади Грымзик с пола поднимается.\par
– Ох! – говорит Грымзик. – Хорошо, что ты сюда добрался, а то раньше никому не удавалось, я уж и со счета сбился.\par
– Грымзик! Ты-то здесь откуда? Да еще и на полу отдыхаешь.\par
– Так я ж тебе к правому ботинку микротелепортационный приемник прицепил, ну и 3D-видеокамеру с квадрофоническим микрофоном в придачу. Хоть и не верилось, что ты сюда доберешься. Да уж больно мне плазмомёт кварк-глюонный раздобыть надо было. Пришлось вот даже ползком телепортироваться – слишком уж маленький портальчик сделался.\par
– Стоп! Это мне его раздобыть надо было! Вот я здесь и оказался.\par
– Ты уж извини, Полуэкт, только мне это нужнее, – говорит Грымзик. Выхватывает петрификатор и стреляет. Полуэкт сразу на пол шлепнулся, ни рукой, ни ногой двинуть не может. Языком еле ворочает.\par
– Ой, – говорит, – ты что, супостат, делаешь?!\par
– Да я тебя в лабораторию ближайшую сейчас сдам – для опытов. Чтоб под ногами не путался.\par
Схватил Грымзик Полуэкта за шиворот и потащил в биолабораторию по соседству.\par
\par
Затащил он Полуэкта в дальний угол лаборатории, бросил там и к выходу направился. Да за что-то вроде зеленого кабеля зацепился. Тут сверху огромный цветок зубастый как упадет, Грымзик вмиг внутри цветка оказался, мычит что-то, ничего не разобрать.\par
\par
– Это что ж такое?! – Полуэкт спрашивает.\par
– Это я, растение говорящее, – голос отвечает.\par
– Да откуда ж ты взялось?\par
– Люди в белых халатах говорили, что я – интересная мутация. И что это поможет им в борьбе с огородными вредителями.\par
– А что еще они говорили?\par
– Последние их слова были: «А где Орибазий? И Эвтаназий?»\par
– Слушай, выплюнь ты Грымзика, а то тебе плохо будет. Он гербицид.\par
– Что он делает?\par
– Гербицид – для растений ядовит.\par
– Откуда ты знаешь? Да и вообще, кто ты такой?\par
– Да я тоже растение, куст говорящий. Видишь, шевелиться не могу. А этот человек меня поисследовать хотел. Ну, теперь я его поисследую, чтоб не важничал. Сейчас, погоди, проросту только немного.\par
– Хороший ты куст, тихий. И разговаривать умеешь. Ладно, на, исследуй свой гербицид.\par
\par
Распахнулся цветок, Грымзик оттуда вывалился, еле дышит. Полуэкт подождал, пока руки-ноги двигаться смогут, схватил Грымзика, да бегом из лаборатории. За дверь выбежал, остановился, повернулся к Грымзику. «Что – говорит – довыпендривался? Твое счастье, что я сегодня добрый». Да как треснет Грымзика по лбу.\par
\par
– Стой, погоди, Полуэкт! – кричит Грымзик. – Осознал я свою ошибку! Давай с начала начнем.\par
– Я тебе покажу с начала! Сейчас еще раз двину!\par
\par
Совсем перепугался тут Грымзик, заметался, убежать старается. А Полуэкт не отстает, того и гляди догонит и еще раз стукнет. Подбежал Грымзик к первой двери, с котлом ядерным, и шасть за нее. И Полуэкт за ним.\par
\par
Глядит – Темпор посреди комнаты сияет, аномалия чудесная, и Грымзик к нему бежит. Бросился Полуэкт за Грымзиком, чтобы остановить, схватил крепко. Да изловчился Грымзик, качнулся, и рухнули они оба в Темпор, в параллельной Вселенной оказались. А в ней и сказка эта совсем другая...\par

\chapter{}
 \lettrine{Е}{ще когда Солнце не стало сверхновой,} был один инопланетянин, звали его Полуэкт. Был он лучший из лучших разбойник и был на редкость умен.\par
\par
Как-то раз прослышал он, что у одной черной дыры, такой черной, что чернее не бывает, в алмазном криосаркофаге скрыта ригелианская красавица, да такая красивая, что все, завидев ее, сразу пред нею ниц падают и все свои злые помыслы оставляют. И захотел тогда Полуэкт ее себе забрать, чтобы не попалась она в чужие руки и не вышло большой беды для всей Вселенной. Но как найти красавицу ригелианскую? Звезд да черных дыр во Вселенной ужас как много, и вокруг каждой великое множество планет да астероидов вертится!\par
\par
Начал Полуэкт думать, у кого совета спросить. Думал-думал, да надумал обратиться к колдунье Морганионе, слава о деяниях которой гремела по всей Вселенной громким грохотом. Взял он свой калькулятор, что служил ему верой и правдой во всех путешествиях, и помчался искать аудиенции у Морганионы.\par
\par
Через некоторое время нашелся способ повстречаться с Морганионой. Так и так, сказывает Полуэкт, хочу я найти красавицу ригелианскую, сокровище это удивительное, и все мысли мои теперь лишь об этом. Да вот проблема – знать не знаю, как!\par
\par
«Послушай, – говорит ему Морганиона, – вижу я, ты очень решителен в своем намерении, да и смекалист очень! Впрочем, ты инопланетянин, а инопланетяне этим славятся, да еще прытью своей. Только не буду я помогать тебе в поисках этих, хоть и знаю, как отыскать то, что тебе нужно, – слишком это опасно. Не будь я Морганиона!» \par
\par
«Ах так! – возмущается Полуэкт. – Да я столько парсеков до тебя отмахал, а ты мне и совета доброго дать не хочешь!» – и чуть не с кулаками к Морганионе бросается. \par
\par
Рассвирепела тут Морганиона. «Вот как, – говорит, – что ж, преподам я тебе сейчас урок за занудство твоё – долго его помнить будешь!» Выхватила Морганиона, откуда ни возьмись, алебарду плазменную, да как начнет оружием своим размахивать и всё вокруг крушить да взрывать! Еле успел Полуэкт за стул спрятаться. А Морганиона не унимается и напоминает уже бешеный вентилятор на полной мощности. \par
\par
Сидит Полуэкт за стулом, решает, как дальше быть. А, думает, чего тут ждать-дожидаться! Улучил момент, схватил стул, да как стукнет им Морганиону изо всех сил своих немалых. Но Морганиона тоже не промах оказалась – сумела удар молодецкий отбить. Только вот в шкаф с Полным жизнеописанием разбойника Мордона врезалась и оказалась в книгах толстенных зарыта по самую шею. Двинуться не может, лишь глазами хлопает и пыхтит недовольно. А Полуэкт стоит со стулом в руках, насмехается: «Не со мной, добрым молодцем, тебе тягаться, Морганиона! С тобой и малый ребенок справится! Говори, где найти мне красавицу ригелианскую, а не то хуже будет!» «Ладно, твоя взяла, – отвечает Морганиона, – слушай!\par
\par
Далеко отсюда твой путь лежит. Мимо галактик, в спирали закрученных, мимо облаков водородных, дивным светом сияющих, мимо квазаров грозных, гравитационные волны излучающих, к самому краю видимой Вселенной, где лишь протоны да альфа-частицы шныряют, а звезду и на миллион парсек не встретишь. И всё же есть там звездочка одна, в туманности газо-пылевой спрятавшаяся. А вокруг звезды той маленькая планета вращается, а вокруг планеты – спутник вертится. И на спутнике том кратер есть круглый да огромный. И на дне кратера того Врата стоят Звездные. И ведут эти Врата к тому, что ты найти так жаждешь.\par
\par
Но непросто до Врат добраться. Охраняет их страж из металла жидкого, ни для какого оружия не уязвимый. Ни днем, ни ночью не спит он и все смотрит внимательно, не прошмыгнул бы кто к Вратам этим. Толпы страждущих пробраться мимо него пытались, да так там в жидком металле и потонули.\par
\par
Поэтому трудно тебе придется. Но коли сумеешь до Врат добраться и через них пройти, окажешься в комнатке маленькой. И будут пред тобой три двери – две больших да красивых, а третья – маленькая да невзрачная.\par
\par
На первой двери, иридием украшенной, нарисован будет ядерный котел над очагом звездным. За этой дверью Темпор, аномалия чудесная, то ли пространственно-времянная, то ли температурно-пространственная. Сунешь свой нос в эту дверь – на веки сгинешь. Но если уж не послушаешь моего совета и заглянешь туда – руками ничего не трогай, а то и вся Вселенная наша переиначиться может.\par
\par
На второй двери, алмазами выложенной, волк в тельняшке наклеен. За этой дверью биолаборатория заброшенная, в которой ксеноморфов да всяких хищников страшных выводили. Они и сейчас там бродят. Такому герою, как ты, туда зайти – головы не сносить. Но уж если не послушаешь доброго совета да заглянешь туда – из пробирок не пей – ксеноморфиком станешь, или шай-хулудом каким.\par
\par
На третьей двери, паутиной затянутой да звездной пылью засыпанной, ничего не нарисовано, только написано: «Оставь надежду, всяк сюда входящий!» За этой дверью найдешь ты красавицу ригелианскую, сокровище, которое так найти хочешь.\par
\par
А чтоб ты с пути не сбился, дам я тебе лазерную указку волшебную, куда она покажет – туда ты и направляйся».\par
\par
Пустился Полуэкт в путь. Летит он в подпространстве гипертоннелями, которые, не иначе, какие-то гиперкроты вырыли, летит измерениями неизвестными. Уж и не знает, сколько в нем самом теперь измерений осталось. Но не сдается, боится только одного – с пути сбиться.\par
\par
Долго ли, коротко ли, долетел Полуэкт до звездочки заветной. Рассчитал он курс, на нужную орбиту лег, к высадке подготовился.\par
\par
Подлетает к спутнику, а тут солнечный ветер поднялся жуткий, аж с ног сбивает. Хорошо, думает Полуэкт, с подветренной стороны зайду – тогда меня не сразу учуют. \par
\par
 Приземлился, из посадочного модуля выбрался, хотел к Вратам бежать, да страж уж тут как тут. Идет, похожий на андроида, зеркальной краской выкрашенного, да за транклюкатором своим тянется. \par
 \par
– Постой! – кричит ему Полуэкт. – Мы с тобой одного металла, ты и я. \par
– Что? – кричит в ответ страж. – Я из-за ветра тебя слышу плохо. \par
– Говорю, мы с тобой одного металла, ты и я, – опять кричит Полуэкт. – Не надо меня транклюкировать!\par
– Что говоришь? Тебе одного раза мало, если надо тебя транклюкировать? Постой, я поближе подойду. \par
Подошел поближе, спрашивает: \par
– Так что ты сказать-то хотел? \par
– Я говорю, мы с тобой одного металла, – повторяет Полуэкт, – поэтому меня транклюкировать не надо. \par
– Да? – удивляется страж. – А что же с тобой делать надо? Да и не похож ты на меня – я вон какой гладкий да зеркальный, а ты бледный какой-то. \par
– Так я ж изучал, как инопланетяне живут, вот и превратился. Ты, вон, тоже, небось, в кого захочешь – в того и превратишься. Хоть в меня, хоть в дракона с планеты Протактиний, хоть во что маленькое и безобидное. \par
– Это верно, смотри. \par
\par
Начинает тут страж переливаться всеми цветами радуги, и вдруг – бац – Полуэкт словно сам перед собой стоит. «Что, – говорит страж, – впечатляет? Смотри дальше!» И превращается в такое ужасное чудище, каких Полуэкт и не видел никогда, чуть рассудка от страха не лишился. «То-то, – говорит страж. – Смотри дальше!» И превращается в плитку шоколада. Лежит себе плитка, да такая аппетитная, что сама так в рот и просится. Схватил Полуэкт плитку, да не тут-то было – плитка килограмм сто весит – не меньше. Закон сохранения массы, видать, в действии. \par
\par
А страж уж обратно в андроида зеркального превратился. \par
– Я, – говорит, – во что хочешь превращаться умею. Вот только в себя не могу. \par
– Это почему же? – спрашивает Полуэкт.\par
– Да я уж во столько всего превращался, что и забыл, как вначале выглядел. \par
– Постой, роботы же никогда ничего не забывают. \par
– Сам ты робот! – говорит с обидой страж и опять за транклюкатором тянется. – Шейпшифтер я! Шейп-шиф-тер! \par
– Да стой, не кипятись. Дай-ка я проверю, робот ты или нет, я тест знаю. \par
– Ну ладно, давай. \par
– Вот смотри, – говорит Полуэкт, – сможешь прочитать, что тут написано? \par
А сам берет листок бумаги, пишет на нем что-то и стражу протягивает. \par
– Да тут «СВМН95РП» написано, чушь какая-то, да еще и двойной линией перечеркнуто.\par
– Похоже, и вправду ты не робот. Чего ж ты тут делаешь? \par
– Да было у нас пророчество, что кто через Звездные Врата пройдет, тот и Вселенную изменить сможет. А нам, шейпшифтерам, это ни к чему. Нас и такая Вселенная устраивает. Вот и сижу я тут, смотрю, что б никто через Врата не прошел, прям жизни никакой уж от них нет. Замучился – ни отойти куда, ни поспать. \par
– Ну, так и зачем тебе такая Вселенная-то, в которой ты ни отойти куда не можешь, ни поспать, ни друзей завести, ни животных домашних? \par
– А и верно, – говорит страж, – незачем мне всё это! Ты ведь во Врата пройти собирался? Ну и иди себе. Только просьба у меня к тебе есть: зайди в биолабораторию, посмотри, нет ли там овец электрических. Приведи мне одну, если найдешь, – уж очень я о таком животном мечтаю.\par
\par
\par
Прошел Полуэкт сквозь Врата – видит, три двери перед ним. Убрал он паутину с самой маленькой, пыль звездную с нее отряхнул да внутрь вошел. Осмотрелся и увидел красавицу ригелианскую, то сокровище, к которому так стремился. Бросился Полуэкт к сокровищу своему, тут вдруг сзади покашливание какое-то раздалось. Обернулся – позади Морганиона с пола встает.\par
– Ох! – говорит Морганиона. – Хорошо, что ты сюда добрался, а то раньше никому не удавалось, я уж и со счета сбилась.\par
– Морганиона! Ты-то здесь откуда? Да еще и на полу отдыхаешь.\par
– Так я ж тебе к правому ботинку микротелепортационный приемник прицепила, ну и 3D-видеокамеру с квадрофоническим микрофоном в придачу. Хоть и не верилось, что ты сюда доберешься. Да уж больно мне красавицу ригелианскую раздобыть надо было. Пришлось вот даже ползком телепортироваться – слишком уж маленький портальчик получился.\par
– Стоп! Это мне ее раздобыть надо было! Вот я здесь и оказался.\par
– Ты уж извини, Полуэкт, только мне это нужнее, – говорит Морганиона. Выхватывает станнер и стреляет. Полуэкт сразу на пол шлепнулся, ни рукой, ни ногой двинуть не может. Языком еле шевелит.\par
– Ой, – говорит, – ты что, супостат, делаешь?!\par
– Да я тебя в лабораторию ближайшую сейчас сдам – для опытов. Чтоб под ногами не путался.\par
Схватила Морганиона Полуэкта за шиворот и потащила в биолабораторию по соседству.\par
\par
Затащила она Полуэкта в лабораторию, бросила там и удалилась важно. Лежит Полуэкт, пошевелиться не может, ждет, когда им завтракать придут. Или обедать – не знает, что и лучше. \par
Смотрит – через дальнюю дверь стадо овец входит. Все белые, только одна черная, по крайней мере, с одной стороны. Для политкорректности. Шерсть на овцах дыбом стоит и искры по шерсти бегают размером со спаниеля. Какая-то тощая тварь с потолка попыталась на них напрыгнуть, да ее на лету молнией сшибло.\par
\par
«Эге, – думает Полуэкт, – это ж прямо электроовцы какие-то. У меня и часы от них остановились, похоже. И сервопривод шнурков в ботинках отключился. Экое абсолютное оружие. Как бы мне его себе приручить». Тут чувствует – руки-ноги опять шевелиться могут. Почесал Полуэкт в затылке, огляделся повнимательней. Снял со стены диаграмму не пойми чего огромную, быстренько на обратной стороне картинку нарисовал, дырку в середине проделал и надел на себя через голову. Стал Полуэкт похож на рекламный щит ходячий, человека-бутерброд, которого как-то в космопорту видел. Только Полуэкт стал человек-ворота. Стадо овец как его увидело – сразу побежало на новые ворота смотреть. А Полуэкт пошел к Морганионе.\par
\par
Приходит, видит – Морганиона портал для обратной телепортации готовит, а роботопомощники вокруг так и кишат. И Морганиона его увидела. «Не думала, – говорит, – что ты выбраться сможешь. Ну да неважно». И приказывает роботопомощникам очистить помещение от посторонних. Но не тут-то было. Роботы все поотключались, портал к овцам притянулся и вместе с ними схлопнулся. А у Морганионы шнурки развязались.\par
\par
Подбежал Полуэкт к Морганионе, хотел стукнуть как следует, да увернулась Морганиона, из ботинок выскочила и наутек бросилась. «Вся матрица моих надежд рухнула! – кричит. – Перезагрузка! Только она мне поможет!» Выбежала из двери, к другой двери подбежала, с котлом ядерным, и шасть за нее. Полуэкт за ней кинулся, хоть и отстал чуток.\par
\par
Глядит – Темпор посреди комнаты сияет, аномалия чудесная, и Морганиона к нему бежит. Бросился Полуэкт за Морганионой, чтобы остановить, схватил крепко. Да изловчилась Морганиона, качнулась, и рухнули они оба в Темпор, в параллельной Вселенной оказались. А в ней и сказка эта совсем другая...\par

\chapter{}
 \lettrine{Н}{а заре Вселенной} жил один насекомец, звали его Лаврентий. Был он великий ученый, но ума при этом был небольшого.\par
\par
Однажды услышал он от старого старика, что у одной черной дыры, такой черной, что чернее не бывает, остались неведомые артефакты древней цивилизации. И решил тогда Лаврентий их себе добыть, чтобы продать их, да побольше денег заработать. Да как узнать, где в точности найти артефакты неведомые? Звезд да черных дыр во Вселенной ужас как много, и вокруг каждой куча планет да астероидов вертится!\par
\par
Принялся Лаврентий смекать, у кого совета спросить. Думал-думал, и надумал обратиться к мудрецу с планеты Клопадоктус по имени Завздыпопус, славившемуся своими познаниями о космосе. Собрал он все свои деньги и драгоценности, а их он копил во множестве, ибо любил очень, и опрометью помчался к Завздыпопусу.\par
\par
Через некоторое время нашелся способ увидеться с Завздыпопусом. Так, мол, и так, сказывает Лаврентий, хочу я отыскать артефакты неведомые, сокровище это великое, и все мысли мои теперь только об этом. Да вот проблема – понятия не имею, как!\par
\par
«Послушай, – отвечает ему Завздыпопус, – вижу я, ты необычайно ловок, да только IQ твой ниже плинтуса! Впрочем, ты насекомец, а насекомцы этим славятся, да еще упертостью своей. Только не знаю я, как помочь тебе. Но есть сестра у меня, мудрости столь необычной, что моя мудрость по сравнению с её – логарифмическая линейка по сравнению с суперкомпьютером. Живет она у соседней звезды, второй поворот налево, если отсюда к краю Галактики лететь. Принеси ей подарков да украшений дорогих – может, поможет она тебе».\par
\par
Собрал Лаврентий с собой подарки да украшения, добавил к ним кольцо из цельнометаллического водорода сделанное, с формулой Вселенной выгравированной, и в путь пустился.\par
\par
Прилетает он к сестре Завздыпопуса, отдает ей подарки, и письмо от Завздыпопуса вручает. «Так, мол, и так, – сказывает Лаврентий, – очень хочется мне найти артефакты неведомые, сокровище это великое, и мысли мои самые что ни на есть прекрасные.»\par
\par
«Вообще-то, – говорит сестра, – тут в письме сказано, что б я тебе голову отрубила, а не помогать стала!.. Ах нет, извини, просто письмо с другой стороны какого-то черновика написано, там даже печать есть... Ладно, помогу я тебе, хоть и не стоишь ты этого, насекомец! Да очень уж мне подарки твои понравились, особенно кольцо из цельнометаллического водорода сделанное, с формулой Вселенной выгравированной.\par
\par
Далеко отсюда твой путь лежит. Мимо галактик в спирали, мимо планет в тентуре, мимо туманностей звездных, дивным светом сияющих, мимо квазаров грозных, сигналы чудные излучающих, к самому краю видимой Вселенной, где лишь протоны да альфа-частицы шныряют, а звезду и на миллион парсек не встретишь. И всё же есть там звездочка одна, в туманности газо-пылевой спрятавшаяся. А вокруг звезды той маленькая планета вращается, а вокруг планеты – спутник вертится. И на спутнике том кратер есть огромный да глубокий. И на дне кратера того Врата стоят Звездные. И ведут эти Врата к тому, что ты найти так жаждешь.\par
\par
Но непросто до Врат добраться. Охраняет их страж из металла жидкого, ни для какого оружия не уязвимый. Ни днем, ни ночью не спит он и все смотрит внимательно, не прошмыгнул бы кто к Вратам этим. Многие смельчаки пробраться мимо него пытались, да так там в жидком металле и потонули.\par
\par
Но коли ты жив останешься да сумеешь до Врат добраться и через них пройти, окажешься в зале с потолком таким высоким, что и не видно. И будут пред тобой три двери – две больших да красивых, а третья – маленькая да невзрачная.\par
\par
На первой двери, серебром украшенной, нарисован будет ядерный котел над очагом звездным. За этой дверью Машина времени древняя, что прошлое изменять может да парадоксы вселенские творить. Сунешь свой нос в эту дверь – на веки сгинешь. Но если уж не послушаешь моего совета и заглянешь туда – руками ничего не трогай, а то и вся Вселенная наша переиначиться может.\par
\par
На второй двери, алмазами выложенной, волк в тельняшке нарисован. За этой дверью биолаборатория заброшенная, в которой ксеноморфов да всяких хищников страшных выводили. Они и сейчас там бродят. Такому герою, как ты, туда зайти – лицо потерять. Но уж если не послушаешь доброго совета да заглянешь туда – из пробирок не пей – ксеноморфиком станешь, или шай-хулудом каким.\par
\par
На третьей двери, паутиной затянутой да звездной пылью засыпанной, ничего не нарисовано, только написано: «Посторонним вход воспрещен!» За этой дверью найдешь ты артефакты неведомые, сокровище, которое так обрести стремишься.\par
\par
А чтоб с пути не сбиться, дам я тебе навигатор звездный, куда он скажет, туда ты и поворачивай».\par
\par
Тут Лаврентий, не мешкая, в путь пустился. Летит он в гиперпространстве гипертоннелями, которые, не иначе, какие-то гиперкроты вырыли, летит измерениями неизвестными. Уж и не знает, трехмерный ли он до сих пор. Но не сдается, боится только одного – с пути сбиться.\par
\par
Долго ли, коротко ли, долетел Лаврентий до звездочки заветной. Рассчитал он курс, на нужную орбиту лег, к высадке подготовился.\par
\par
Подлетает к спутнику, а тут солнечный ветер поднялся жуткий, аж с ног сбивает. Хорошо, думает Лаврентий, с подветренной стороны зайду – тогда меня не сразу учуют. \par
\par
 Приземлился, из посадочного модуля выбрался, хотел к Вратам бежать, да страж уж тут как тут. Идет, похожий на андроида, зеркальной краской выкрашенного, да за дезинтегратором своим тянется. \par
 \par
– Постой! – кричит ему Лаврентий. – Мы с тобой одного металла, ты и я. \par
– Что? – кричит в ответ страж. – Я из-за ветра тебя слышу плохо. \par
– Говорю, мы с тобой одного металла, ты и я, – опять кричит Лаврентий. – Не надо меня дезинтегрировать!\par
– Что говоришь? Тебе одного раза мало, если надо тебя дезинтегрировать? Постой, я поближе подойду. \par
Подошел поближе, спрашивает: \par
– Так что ты сказать-то хотел? \par
– Я говорю, мы с тобой одного металла, – повторяет Лаврентий, – поэтому меня дезинтегрировать не надо. \par
– Да? – удивляется страж. – А что же с тобой делать надо? Да и не похож ты на меня – я вон какой гладкий да зеркальный, а ты бледный какой-то. \par
– Так ты ж с твоими талантами в кого хочешь превратиться можешь. Хоть в меня, хоть в дракона с планеты Протактиний, хоть во что маленькое и безобидное. \par
– Это верно, смотри. \par
\par
Начинает тут страж переливаться всеми цветами радуги, и вдруг – бац – Лаврентий словно сам перед собой стоит. «Что, – говорит страж, – впечатляет? Смотри дальше!» И превращается в такое ужасное чудище, каких Лаврентий и не видел никогда, чуть с ума от страха не сошел. «То-то, – говорит страж. – Смотри дальше!» И превращается в плитку шоколада. Лежит себе плитка, да такая аппетитная, что сама так в рот и просится. Схватил Лаврентий плитку, да не тут-то было – плитка килограмм сто весит – не меньше. Закон сохранения массы, видать, в действии. \par
\par
А страж уж обратно в андроида зеркального превратился. \par
– Я, – говорит, – во что хочешь превращаться умею. Вот только в себя не могу. \par
– Это почему же? – спрашивает Лаврентий.\par
– Да я уж во столько всего превращался, что и забыл, как вначале выглядел. \par
– Постой, роботы же никогда ничего не забывают. \par
– Сам ты робот! – говорит с обидой страж и опять за дезинтегратором тянется. – Шейпшифтер я! Шейп-шиф-тер! \par
– Да стой, не кипятись. Дай-ка я проверю, робот ты или нет, я тест знаю. \par
– Ну ладно, давай. \par
– Вот смотри, – говорит Лаврентий, – сможешь прочитать, что тут написано? \par
А сам берет листок бумаги, пишет на нем что-то и стражу протягивает. \par
– Да тут «GJ85QR2» написано, чушь какая-то, да еще и двойной линией перечеркнуто.\par
– Похоже, и вправду ты не робот. Чего ж ты тут делаешь? \par
– Да было у нас пророчество, что кто через Звездные Врата пройдет, тот и Вселенную изменить сможет. А нам, шейпшифтерам, это ни к чему. Нас и такая Вселенная устраивает. Вот и сижу я тут, смотрю, что б никто через Врата не прошел, прям жизни никакой уж от них нет. Замучился – ни отойти куда, ни поспать. \par
– Ну, так и зачем тебе такая Вселенная-то, в которой ты ни отойти куда не можешь, ни поспать, ни друзей завести, ни животных домашних? \par
– А и верно, – говорит страж, – незачем мне всё это! Ты ведь во Врата пройти собирался? Ну и иди себе. Только просьба у меня к тебе есть: зайди в биолабораторию, посмотри, нет ли там овец электрических. Приведи мне одну, если найдешь, – уж очень я о таком животном мечтаю.\par
\par
\par
Прошел Лаврентий сквозь Врата – видит, три двери перед ним. Убрал он паутину с самой маленькой, пыль звездную с нее отряхнул да внутрь вошел. Осмотрелся и увидел артефакты неведомые, то сокровище, к которому так стремился. Бросился Лаврентий к сокровищу своему, тут вдруг сзади шорох какой-то послышался. Оглянулся – позади Завздыпопус с пола встает.\par
– Ох! – говорит Завздыпопус. – Хорошо, что ты сюда добрался, а то раньше никому не удавалось, я уж и со счета сбился.\par
– Завздыпопус! Ты-то здесь откуда? Да еще и на полу отдыхаешь.\par
– Так я ж тебе к правому ботинку микротелепортационный приемник прицепил, ну и 3D-видеокамеру с квадрофоническим микрофоном в придачу. Хоть и не верилось, что ты сюда доберешься. Да уж больно мне артефакты неведомые раздобыть надо было. Пришлось вот даже ползком телепортироваться – слишком уж маленький портальчик получился.\par
– Погоди! Это мне их раздобыть надо было! Вот я здесь и оказался.\par
– Ты уж извини, Лаврентий, только мне это нужнее, – говорит Завздыпопус. Выхватывает станнер и стреляет. Лаврентий сразу на пол шлепнулся, ни рукой, ни ногой двинуть не может. Языком еле ворочает.\par
– Ой, – говорит, – ты что, супостат, делаешь?!\par
– Да я тебя в лабораторию ближайшую сейчас сдам – для опытов. Чтоб под ногами не путался.\par
Схватил Завздыпопус Лаврентия за шиворот и потащил в биолабораторию по соседству.\par
\par
Затащил он Лаврентия в дальний угол лаборатории, бросил там и к выходу направился. Да за что-то вроде зеленого кабеля зацепился. Тут сверху огромный цветок зубастый как упадет, Завздыпопус вмиг внутри цветка оказался, мычит что-то, ничего не разобрать.\par
\par
– Это что ж такое?! – Лаврентий спрашивает.\par
– Это я, растение говорящее, – голос отвечает.\par
– Да откуда ж ты взялось?\par
– Люди в белых халатах говорили, что их генетический эксперимент удался. И что это поможет им в борьбе с артангами.\par
– А что еще они говорили?\par
– Последние их слова были: «А где Орибазий? И Эвтаназий?»\par
– Слушай, выплюнь ты Завздыпопуса, а то тебе плохо будет. Он гербицид.\par
– Что он делает?\par
– Гербицид – для растений ядовит.\par
– Откуда ты знаешь? Да и вообще, кто ты такой?\par
– Да я тоже растение, куст говорящий. Видишь, шевелиться не могу. А этот человек меня поисследовать хотел. Ну, теперь я его поисследую, чтоб не важничал. Сейчас, погоди, проросту только немного.\par
– Хороший ты куст, тихий. И разговаривать умеешь. Ладно, на, исследуй свой гербицид.\par
\par
Распахнулся цветок, Завздыпопус оттуда вывалился, еле дышит. Лаврентий подождал, пока руки-ноги двигаться смогут, схватил Завздыпопуса, да бегом из лаборатории. За дверь выбежал, остановился, повернулся к Завздыпопусу. «Что – говорит – довыпендривался? Твое счастье, что я сегодня добрый». Да как треснет Завздыпопуса по лбу.\par
\par
– Стой, погоди, Лаврентий! – кричит Завздыпопус. – Осознал я свою ошибку! Давай с начала начнем.\par
– Я тебе покажу с начала! Сейчас еще раз тресну!\par
\par
Совсем перепугался тут Завздыпопус, заметался, убежать старается. А Лаврентий не отстает, того и гляди догонит и еще раз стукнет. Подбежал Завздыпопус к первой двери, с котлом ядерным, и шасть за нее. И Лаврентий за ним.\par
\par
Глядит – машина там стоит дивная, лампочками моргает, жужжит тихонько и готова в прошлое отправиться хоть к сотворению Вселенной. А Завздыпопус уже внутри сидит, рычажки какие-то тянет и кнопки нажимает. Кинулся Лаврентий к Завздыпопусу, тоже внутрь залез, остановить хотел, да не успел. И исчезли оба вместе с машиной, как будто не было их тут вовсе. И сразу история эта поменялась, совсем другой стала...\par

\chapter{}
 \lettrine{У}{некоторой звезды, на пятой от нее планете,} жил-был один гуманоид, звали его Игнат. Был он лучший из лучших герой и был на редкость умен.\par
\par
В один прекрасный день услышал он от старого старика, что на одной затерянной в глубинах космоса, холодной и обледенелой планете скрыт самый быстрый во Вселенной космический корабль с большой лазерной пушкой. И решил Игнат его разыскать для себя, чтобы использовать его для достижения счастья всех существ во Вселенной. Но где искать корабль космический самый быстрый? Звезд да черных дыр во Вселенной ужас как много, и вокруг каждой куча планет да астероидов вертится!\par
\par
Стал Игнат смекать, у кого информацию нужную добыть можно. Думал-думал, и надумал обратиться к известному на всю Галактику звездознатцу, профессору Грымзику. Взял он свой электробаян, с коим любил коротать время, и опрометью помчался искать встречи с Грымзиком.\par
\par
Вскорости смог Игнат с ним увидеться. Так и так, говорит Игнат, хочется мне отыскать корабль космический самый быстрый, сокровище это удивительное, и побуждения мои самые что ни на есть прекрасные. Только вот проблема – понятия не имею, как!\par
\par
«Вот что, – отвечает ему Грымзик, – вижу я, ты весьма решителен в своем намерении, и IQ твой вышиной до звезд простирается! Впрочем, ты гуманоид, а гуманоиды этим известны, да еще упертостью своей. Но чтобы я тебе помогать стал, тебе самому сначала мне помочь придется. Сделаешь всё как надо – расскажу, как найти корабль космический самый быстрый, а нет – не обессудь!\par
\par
Внемли! Есть тут поблизости звезда нейтронная – третий поворот направо, если к центру Галактики лететь. Рядом с ней планетоид карликовый, а на планетоиде том два чудовища живут. Зовут их Сцилла и Харибда, злобные они до ужаса, и до самых кончиков клыков темной энергией пропитаны. Есть у них гусли со струнами космическими, такие, что каждый, кто их услышит, от радости гиперпрыжки да танцы начинает выделывать, которые другим и не снились. Пуще жизни чудовища их любят. Добудь мне гусли, и расскажу я тебе, где найти корабль космический самый быстрый».\par
\par
Закручинился Игнат, да делать нечего. Полетел к чудовищам гусли добывать. Летит, а сам боится, потом обливается, даже поворот нужный пропустил – пришлось возвращаться. А когда возвращался, увидал пустую канистру из-под топлива термоядерного, кем-то выброшенную. Возьму, думает, ее с собой – вдруг пригодится. Летит дальше – видит, кусок темной материи в пространстве висит, весь скомканный – наверное, купец какой-то потерял. И его, думает, возьму, тоже может на что сгодиться. Дальше летит – видит, компрессор пространственно-временной валяется – должно быть, странники какие-то оставили. И его тоже взял.\par
\par
Прилетает, садится на планетоид, заходит во дворец к чудовищам, а те сразу к нему бросаются. \par
– Чу, – говорят, – гуманоидным духом пахнет! Как звать, – спрашивают, – откуда? Как хочешь быть проглоченным – ногами вперед или назад? Да отвечай побыстрее, а то проголодались мы страшно – сто лет уж никого не ели.\par
– Погодите вы с формальностями! – Игнат отвечает. – Слышал я, есть у вас гусли со струнами космическими, инструмент дивный. Сменяйте мне их на что-нибудь – очень уж мне надо.\par
– Что ты – с ума сошел? – спрашивают чудища. – Инструмент нам этот очень дорог.\par
– Ладно, – говорит Игнат, – тогда дайте на инструмент этот ваш посмотреть хотя бы, а я вам за это компрессор пространственно-временной подарю.\par
– И для чего это нам? – спрашивают.\par
– Вы же тогда что угодно куда угодно запрятать сможете. Даже вы двое в этой вот канистре поместитесь. \par
– Такого и быть не может!\par
– Дайте посмотреть гусли – увидите.\par
\par
Повели чудища его в самый дальний угол самой далекой пещеры, где гусли хранили. Посмотрел Игнат на гусли. «Что ж, – говорит, – теперь вы смотрите». Приладил компрессор пространственно-временной к канистре и велел чудовищам туда прыгать. Прыгнули они, сидят в канистре, удивляются, что еще места много осталось. А Игнат заткнул канистру куском темной материи, и даже бантик завязал.\par
\par
Взял он гусли, закинул канистру подальше в глубины космоса и бегом к Грымзику. Грымзик обрадовался: «Ох, удружил ты мне, – говорит. – Расскажу я теперь тебе, как найти корабль космический самый быстрый.\par
\par
Далеко отсюда твой путь лежит. Мимо галактик, в спирали закрученных, мимо туманностей звездных, дивным светом сияющих, мимо квазаров грозных, сигналы чудные излучающих, к самому краю космоса, где лишь протоны да альфа-частицы шныряют, а звезду и на миллион парсек не встретишь. И всё же есть там звездочка одна, молодая да пригожая. А вокруг звезды той огромная планета вращается, а вокруг планеты – спутник вертится. И на спутнике том кратер есть круглый да огромный. И на дне кратера того Врата стоят Звездные. И ведут эти Врата к тому, что ты найти так жаждешь.\par
\par
Но непросто до Врат добраться. Лабиринт вкруг тех Врат выстроен в сто этажей да в десять тысяч комнат на каждом. Да такой хитрый, что как войдешь в него, так и заблудишься сразу. И как сквозь тот лабиринт пройти, мне неведомо.\par
\par
Поэтому трудно тебе придется. Но коли сумеешь до Врат добраться и через них пройти, окажешься в комнатке маленькой. И будут пред тобой три двери – две больших да красивых, а третья – маленькая да невзрачная.\par
\par
На первой двери, серебром украшенной, нарисован будет ядерный котел над очагом звездным. За этой дверью Темпор, аномалия чудесная, то ли пространственно-времянная, то ли температурно-пространственная. Сунешь свой нос в эту дверь – на веки сгинешь. Но если уж не послушаешь моего совета и заглянешь туда – руками ничего не трогай, а то и вся Вселенная наша переиначиться может.\par
\par
На второй двери, алмазами выложенной, волк в тельняшке нарисован. За этой дверью биолаборатория заброшенная, в которой ксеноморфов да всяких хищников страшных выводили. Они и сейчас там бродят. Такому герою, как ты, туда зайти – лицо потерять. Но уж если не послушаешь доброго совета да заглянешь туда – из пробирок не пей – ксеноморфиком станешь, или шай-хулудом каким.\par
\par
На третьей двери, паутиной затянутой да звездной пылью засыпанной, ничего не нарисовано, только написано: «Оставь надежду, всяк сюда входящий!» За этой дверью найдешь ты корабль космический самый быстрый, сокровище, которое так найти хочешь.\par
\par
А чтоб ты с пути не сбился, дам я тебе навигатор звездный, куда он скажет, туда ты и поворачивай».\par
\par
Тут Игнат, не мешкая, в путь пустился. Летит он в гиперпространстве гипертоннелями, которые, не иначе, какие-то гиперкроты вырыли, летит измерениями неизвестными. Уж и не знает, трехмерный ли он до сих пор. Но не сдается, боится только одного – наизнанку вывернуться.\par
\par
Сколько световых лет прошло, неведомо, но долетел Игнат до звездочки заветной. Рассчитал он курс, в посадочный модуль залез, к высадке подготовился.\par
\par
Высадился Игнат рядом с лабиринтом, добрался до входа и внутрь вошел. Идет одним коридором, другим, третьим. Пусто везде, тихо, только его шаги эхом отдаются. До комнаты какой-то дошел, видит – дальше три коридора ведут, и в каждом будто туман клубится. А на полу написано: «Коли дураком не хочешь стать – иди направо, либо прямо. Коли голову потерять не хочешь – иди прямо, либо налево. Коли до смерти запуганным быть не хочешь – иди налево, либо направо».\par
\par
Задумался Игнат, куда идти, да и пошел налево. Идет, по коридорам топает, с этажа на этаж перебирается. А туман вокруг не рассеивается. Долго он так шел, уж стал думать, что батарейки в часах наручных скоро сядут. Дошел, наконец, до какой-то комнаты, а в комнате той будто битва великая кипела, потолок осыпался, в полу дыра в пол-комнаты. Из комнаты две двери ведут. До одной не добраться, поперек другой дракон трехголовый спит, а рядом с ним меч здоровенный двуручный валяется. Схватил Игнат меч за рукоятку, а тот тяжеленный, по полу как заскрежещет. Дракон вмиг проснулся.\par
\par
– Чего тебе надобно? – спрашивает.\par
– Да вот, пройти хочу, – Игнат отвечает.\par
– А меч тебе зачем? Ты что, совсем дурак? Попросил бы – я б подвинулся.\par
– Да я его хотел через дыру в полу перекинуть – навроде мостика.\par
– А, так тебе в ту дверь надо? Впрочем, все равно дурак – меч размером коротковат.\par
– Да мне все равно, в какую дверь, мне б только до Звездных Врат добраться. И что вообще у вас тут творится?\par
– Игра у нас тут идет большая.\par
– Да что за игра-то?\par
– Да так… Впрочем, раз уж ты ко мне пришел, раунд за мной, идем – расскажу тебе про игру.\par
\par
Выходит дракон в дверь, становится слегка туманным и идет дальше коридорами. Игнат за ним еле поспевает. До очередной двери дошли, дракон – туда, и Игнат за ним. \par
Входит Игнат в комнату – а там нет никого, лишь три сгустка тумана поплотнее. И как будто двое одному что-то вроде монет туманных передают.\par
– И где же тут кто? – Игнат спрашивает.\par
– Да мы тут везде, но здесь особенно, – отвечают три голоса, да прямо в голове звучат.\par
– И кто же вы такие будете?\par
– Мы – представители древней цивилизации, раса наша настолько древняя, что телесно уже и не существует вовсе. Только ментально, то есть разумом своим. И знаем мы все тайны Вселенной, и все предсказать да рассчитать можем. И скучно нам от этого необычайно. И даже говорим мы длинно и скучно, как ты мог заметить. Пробовали в рулетку играть – да каждый знает, куда шарик прикатится. Пробовали в квантовое лото играть – так и принцип неопределенности для нас не помеха в предсказаниях.\par
– А здесь-то вы что делаете?\par
– Воздвигли мы силой разума лабиринт этот, чтоб скуку развеять можно было. Разумные существа, сюда попавшие, выбор делают. А мы играем, смотрим, по чьей дороге они пойдут. Ибо обладают разумные существа свободой воли, и тут наши предсказания бессильны. Так что давай, хватит отдыхать – видишь, три коридора отсюда тянутся – иди уж по какому-нибудь, да побыстрее!\par
– Не нужны мне ваши коридоры, мне к Звездным Вратам нужно!\par
– Будь любезен, не упрямься. Мы легко тебя заставить можем. Создадим мы сей же час чудовищ ментальных, ты кое-кого видел уже, и ни бластер, ни водяной пистолет тебе не помогут, поскольку будут чудовища внутри твоего разума, а не снаружи. А во сне тебе и вовсе тяжко придется.\par
\par
Видит Игнат – со всех сторон к нему уже когти и щупальца тянутся. Забился он в угол, да как закричит:\par
– Стойте, стойте, погодите! А кто из вас этих чудовищ делает?\par
– Как кто? Мы все трое.\par
– А чьи чудовища самые сильные будут?\par
Тут замерли чудовища на мгновение, а потом как начали друг с другом биться. Лапы с хвостами в разные стороны так и разлетаются. Драконы с демонами сшибаются, ангелы с гигантскими червями, орки и гоблины с эльфами да рыцарями, кальмары огромные с василисками. И над всем этим пегасы да орнитоптеры парят, и молнии сверкают. А внизу горы какие-то да болота с лесами мелькают. Даже один раз черный лотос виден был. \par
– Стойте! – Игнат кричит. – Меня ж сейчас тут совсем затопчут!\par
– Уйди, не мешайся! – три голоса отвечают. – Видишь левый коридор, там третий поворот направо и два раза налево – и придешь к своим Вратам Звездным.\par
\par
Побежал Игнат что есть духу, в минуту до Врат добрался.\par
\par
Прошел Игнат сквозь Врата – видит, три двери перед ним. Вошел в нужную, несколько шагов прошел и слышит какой-то шорох сзади. Оглянулся – за ним Грымзик стоит, ухмыляется. \par
\par
– Не думал я, – говорит, – что сумеешь ты до Врат добраться, да всё же надеялся. Даже приемник телепортационный тебе в правый ботинок засунул. Очень уж мне корабль космический самый быстрый заполучить надо, гораздо нужнее, чем тебе. Так что медленно подними руки вверх и отойди в сторонку, не мешайся.\par
– Ах ты, морда поганая, – Игнат отвечает, – обнаружил я твой приемник, когда ботинки чистил, знал, что ты недоброе затеял. Посмотри вокруг, в биолабораторию ты попал. Чу, хищники зубами скрежещут, клювы разевают, щупальца расправляют! Не выйти тебе отсюда, не забрать корабль космический самый быстрый.\par
– Так, значит! – говорит Грымзик. – Что ж! Посмотрим, кто отсюда живым не выйдет!\par
\par
Хватает со стола пробирку и одним махом выпивает. Раз – и стоит вместо Грымзика лев альдебаранский ядовитый, трех метров роста, к прыжку готовится, слюна с клыков капает. Не растерялся Игнат, тоже пробирку схватил, тоже выпил. Превратился в дракона ригельского, махнул хвостом – лев от него на восемь метров отлетел. Схватил лев с другого стола целую колбу, осушил одним махом, превратился в пчелу бронебойную с Канопуса 5, разогнался, дракона насквозь пробил. Да успел тот на последнем издыхании до пробирки дотянуться, сжевал ее с содержимым вместе, превратился в броненосца мифрильного насекомоядного с Регула 2…\par
\par
В общем, долго они так развлекались, дня три, не меньше. Чуть не забыли, кто из них кто. Уж и пробирки-то почти все закончились. Схватил Грымзик последнюю пробирку, тут Игнат как закричит: «Стой, дурья твоя башка! А как мы в себя-то обратно превратимся?» Задумался Грымзик. «Всё из-за тебя, болван, – говорит. – Теперь всё с начала начинать придется!» Пробирку бросил, на щупальцах приподнялся и выбежал из лаборатории. Игнат за ним пополз.\par
\par
Выползает, смотрит – Грымзик в первую дверь, с очагом звездным, забежал. И Игнат туда направился.\par
\par
Глядит – Темпор посреди комнаты сияет, аномалия чудесная, и Грымзик к нему бежит. Бросился Игнат за Грымзиком, чтобы остановить, схватил крепко. Да изловчился Грымзик, качнулся, и рухнули они оба в Темпор, в параллельной Вселенной оказались. А в ней и история эта совсем другая...\par

\chapter{}
 \lettrine{Н}{а заре Вселенной} жил один человек, звали его Станислав. Был он известный ученый, но ума при этом был небольшого.\par
\par
Как-то раз узнал он, что в одной звездной системе, на маленьком астероиде спрятан клад великий. И решил тогда Станислав его себе заполучить, чтобы закинуть в самый дальний угол подпространства и чтобы никто больше не мог разыскать это сокровище и не смущало оно умы смертных. Да как узнать, где в точности найти клад? Звезд да черных дыр во Вселенной ужас как много, и вокруг каждой куча планет да астероидов вертится!\par
\par
Принялся Станислав думать, у кого информацию нужную добыть можно. Думал-думал, да надумал обратиться к известному на всю Галактику звездознатцу, профессору Грымзику. Взял он свой калькулятор, что служил ему верой и правдой во всех путешествиях, и опрометью помчался к Грымзику.\par
\par
Много ли времени прошло, иль мало, но нашел Станислав способ с ним повстречаться. Так и так, сказывает Станислав, хочу я отыскать клад, сокровище это удивительное, и все помыслы мои теперь лишь об этом. Да вот закавыка – понятия не имею, как!\par
\par
«Что ж, – отвечает ему Грымзик, – вижу я, ты очень решителен, но глуп как пробка! Впрочем, ты человек, а люди этим славятся, да еще упертостью своей. Только не буду я помогать тебе в поисках этих, хоть и знаю, как отыскать то, что тебе нужно. Не будь я Грымзик!» \par
\par
«Ах так! – говорит Станислав. – Да я столько времени на поиски тебя потратил, а ты мне и совета доброго дать не можешь!» – и чуть не с кулаками к Грымзику бросается. \par
\par
Рассвирепел тут Грымзик. «Вот как, – говорит, – что ж, преподам я тебе сейчас урок за неучтивость твою – век его помнить будешь!» Выхватил Грымзик, откуда ни возьмись, меч лазерный и щит зеркальный, да как начнет оружием своим размахивать и всё вокруг крушить да взрывать! Еле успел Станислав за стул спрятаться. А Грымзик не унимается и напоминает уже грузовой вертолет на полном ходу. \par
\par
Сидит Станислав за стулом, решает, как дальше быть. А, думает, чего тут ждать-дожидаться! Улучил момент, схватил стул, да как стукнет им Грымзика изо всех сил своих немалых. Но Грымзик тоже не промах оказался – сумел удар молодецкий отбить. Только вот в шкаф с Полным жизнеописанием разбойника Мордона врезался и оказался в книгах толстенных зарыт по самую шею. Двинуться не может, лишь глазами хлопает и пыхтит недовольно. А Станислав стоит со стулом в руках, приговаривает: «Не со мной, добрым молодцем, тебе тягаться, Грымзик! С тобой и малый ребенок справится! Говори, где найти мне клад, а не то хуже будет!» «Ладно, твоя взяла, – отвечает Грымзик, – слушай!\par
\par
Далеко отсюда твой путь лежит. Мимо галактик, в спирали закрученных, мимо туманностей звездных, дивным светом сияющих, мимо квазаров грозных, сигналы чудные излучающих, к самому краю видимой Вселенной, где лишь протоны да альфа-частицы шныряют, а звезду и на миллион парсек не встретишь. И всё же есть там звездочка одна, в туманности газо-пылевой спрятавшаяся. А вокруг звезды той маленькая планета вращается, а вокруг планеты – спутник вертится. И на спутнике том кратер есть круглый да огромный. И на дне кратера того Врата стоят Звездные. И ведут эти Врата к тому, что ты найти так жаждешь.\par
\par
Но непросто до Врат добраться. Живут там семь роботов-разбойников с машиной вычислительной белоснежной размеров громадных. Да такие жестокие, что каждого, кого встретят, в ящик металлический сажают да в машину вставляют, будто батарейки какие. Уж сколько смельчаков туда ни ходило – всех на батарейки извели.\par
\par
Поэтому трудно тебе придется. Но коли сумеешь до Врат добраться и через них пройти, окажешься в комнатке маленькой. И будут пред тобой три двери – две больших да красивых, а третья – маленькая да невзрачная.\par
\par
На первой двери, серебром украшенной, нарисован будет ядерный котел над очагом звездным. За этой дверью Машина времени древняя, что прошлое изменять может да парадоксы вселенские творить. Сунешь свой нос в эту дверь – на веки сгинешь. Но если уж не послушаешь моего совета и заглянешь туда – руками ничего не трогай, а то и вся Вселенная наша переиначиться может.\par
\par
На второй двери, алмазами выложенной, жаба в скафандре наклеена. За этой дверью биолаборатория заброшенная, в которой ксеноморфов да всяких хищников страшных выводили. Они и сейчас там бродят. Такому герою, как ты, туда зайти – головы не сносить. Но уж если не послушаешь доброго совета да заглянешь туда – из пробирок не пей – ксеноморфиком станешь, или шай-хулудом каким.\par
\par
На третьей двери, паутиной затянутой да звездной пылью засыпанной, ничего не нарисовано, только написано: «Не влезай, убьет!» За этой дверью найдешь ты клад, сокровище, которое так обрести хочешь.\par
\par
А чтоб ты с пути не сбился, дам я тебе навигатор звездный, куда он скажет, туда ты и поворачивай».\par
\par
Тут Станислав, не мешкая, в путь пустился. Летит он в гиперпространстве тоннелями неведомыми, измерениями неизвестными. Уж и не знает, сколько в нем самом теперь измерений осталось. Но не сдается, боится только одного – с пути сбиться.\par
\par
Сколько световых лет прошло, неведомо, но долетел Станислав до звездочки заветной. Рассчитал он курс, на нужную орбиту лег, к высадке подготовился.\par
\par
Приземлился в кратер, с краешку. Глядь – к нему уж робот спешит, грозный на вид, железный ящик перед собой катит.\par
– Здравствуй, – говорит, – человек! Полезай в ящик, не томи – у нас электричество почти уж совсем закончилось!\par
– Погоди, – Станислав отвечает. – Ты разве не слышал, что робот не должен причинять человеку вред или своим бездействием допускать, что бы такой вред был причинен?\par
– С какой это такой стати?\par
– Да с такой! Его Величество, Император Орионский на днях указ издал.\par
– Да мне-то до него что за дело?\par
– Его Величество не любит, чтоб его указы игнорировали, – вмиг прилетит со своим космофлотом, всех лучами смерти перебьет.\par
– Ну, не знаю, – говорит робот, – пойдем с машиной нашей вычислительной посоветуемся, за главную она тут у нас. Полезай в ящик, я тебя подвезу!\par
– Ладно, – говорит Станислав.\par
Робот крышку открыл, старается его туда засунуть, да не тут-то было. Станислав руки-ноги растопырил, в ящик не влезает. \par
– Погоди, – говорит робот, – разве так в ящики залезают! \par
– Да мне-то откуда знать, я ж в них никогда не лазил! Покажи мне как надо, я и залезу. \par
– Не, – говорит робот, – знаю я этот развод. Ничего, пешком дотопаешь.\par
\par
Приходят они к машине вычислительной, а та вся белоснежная да размеров немыслимых. Ни дать, ни взять – суперкомпьютер. А вокруг нее еще шесть роботов сидят. \par
– Вот, – жалуется робот, – хотел его в ящик посадить да на электричество пустить, а он говорит, что нельзя ему вред причинять – указ вышел. \par
– Помилуйте, – говорит машина голосом громким, – какой же это вред – это одна польза сплошная. Умеренные физические нагрузки для здоровья полезны, да и в ящик этот ни один микроб не проползет. И нам, опять же, одна сплошная польза от электричества. Так что сажай его в ящик, даже не сомневайся, и мне в бок вставляй. \par
– Постой! – говорит Станислав машине. – Ты же не знаешь. У меня полярность перепутана, я ж тебе все схемы электрические сожгу, если меня на батарейки употребить. \par
– Да быть такого не может! \par
– Да сама посмотри! Видишь, у меня большой палец левой ноги справа! \par
И ботинок снимает, показывает. \par
– Действительно, – говорит машина. – Это как же так-то? \par
– Да я в детстве в черную дыру свалился, насилу выбрался. С тех пор полярность и перепуталась. Жутко неудобно. Я и сюда-то к Звездным Вратам прилетел, чтоб тут счастья попытать и полярность свою на место вернуть. \par
– Ладно, иди к Вратам, попытай счастья. А уж если восстановишь полярность свою – так на обратном пути к нам заходи, уж мы тебя встретим с ящиком.\par
\par
Прошел Станислав сквозь Врата – видит, три двери перед ним. Вошел в нужную, несколько шагов прошел и слышит какой-то шорох сзади. Оглянулся – за ним Грымзик стоит, ухмыляется. \par
\par
– Не думал я, – говорит, – что сумеешь ты до Врат добраться, да всё же надеялся. Даже приемник телепортационный тебе в правый ботинок засунул. Очень уж мне клад заполучить надо, гораздо нужнее, чем тебе. Так что медленно подними руки вверх и сделай три шага вперед по хорошему, не мешайся.\par
– Ах ты, морда поганая, – Станислав отвечает, – обнаружил я твой приемник, когда ботинки чистил, знал, что ты недоброе затеял. Посмотри вокруг, в биолабораторию ты попал. Чу, хищники зубами скрежещут, клювы разевают, щупальца расправляют! Не выйти тебе отсюда, не забрать клад.\par
– Так, значит! – говорит Грымзик. – Что ж! Посмотрим, кто отсюда живым не выйдет!\par
\par
Хватает со стола пробирку и одним махом выпивает. Раз – и стоит вместо Грымзика кот фомальгаутский ядовитый, трех метров роста, к прыжку готовится, слюна с клыков капает. Не растерялся Станислав, тоже пробирку схватил, тоже выпил. Превратился в тираннозавра ригельского, махнул хвостом – кот от него на восемь метров отлетел. Схватил кот с другого стола целую колбу, осушил одним махом, превратился в пчелу бронебойную с Канопуса 4, разогнался, тираннозавра насквозь пробил. Да успел тот на последнем издыхании до пробирки дотянуться, сжевал ее с содержимым вместе, превратился в броненосца мифрильного насекомоядного с Регула 2…\par
\par
В общем, долго они так развлекались, часа два, не меньше. Чуть не забыли, кто из них кто. Уж и пробирки-то почти все закончились. Схватил Грымзик последнюю пробирку, тут Станислав как закричит: «Стой, дурья твоя башка! А как мы в себя-то обратно превратимся?» Задумался Грымзик. «Всё из-за тебя, идиот, – говорит. – Теперь всё с начала начинать придется!» Пробирку бросил, на щупальцах приподнялся и выбежал из лаборатории. Станислав за ним пополз.\par
\par
Выползает, смотрит – Грымзик в первую дверь, с очагом звездным, забежал. И Станислав туда направился.\par
\par
Глядит – машина там стоит дивная, лампочками моргает, жужжит тихонько и готова в прошлое отправиться хоть к сотворению Вселенной. А Грымзик уже внутри сидит, рычажки какие-то тянет и кнопки нажимает. Кинулся Станислав к Грымзику, тоже внутрь залез, остановить хотел, да не успел. И исчезли оба вместе с машиной, как будто не было их тут вовсе. И сразу сказка эта поменялась, совсем иной стала...\par

\chapter{}
 \lettrine{В}{одной далекой галактике} жил-был один человек по имени Станислав. Был он известный злодей, но ума при этом был небольшого.\par
\par
Как-то раз услышал он от старого старика, что в одной звездной системе, на маленьком астероиде есть скрытая библиотека с тайными знаниями обо всей Вселенной. И захотел Станислав ее себе добыть, чтобы поделиться ею когда-нибудь потом со всеми жителями Галактики. Но как найти библиотеку? Звезд да черных дыр в галактиках ужас как много, и вокруг каждой великое множество планет да астероидов вертится!\par
\par
Начал Станислав смекать, у кого информацию нужную добыть можно. Думал-думал, да надумал обратиться к мудрецу с планеты Ерундения по имени Завздыпопус, славившемуся своими познаниями о космосе. Взял он свой электробаян, с коим любил коротать время, и помчался искать аудиенции у Завздыпопуса.\par
\par
Много ли времени прошло, иль мало, но смог Станислав с ним повстречаться. Так, мол, и так, сказывает Станислав, очень хочется мне найти библиотеку, сокровище это удивительное, и мысли мои самые что ни на есть прекрасные. Только вот закавыка – не знаю, как!\par
\par
«Что ж, – говорит ему Завздыпопус, – вижу я, ты весьма храбр, да только IQ твой ниже плинтуса! Впрочем, ты человек, а люди этим славятся, да еще прытью своей. Могу я тебе помочь, да только сначала должен ты пройти одно пустяковое испытание. Победишь меня в честном бою – помогу, а нет – будешь в ближайшей черной дыре до скончания времен томиться!» \par
\par
Выхватил Завздыпопус, откуда ни возьмись, лазер рентгеновский, да как начнет оружием своим размахивать и всё вокруг крушить да взрывать! Еле успел Станислав за стул спрятаться. А Завздыпопус не унимается и напоминает уже бешеный вентилятор на полной мощности. \par
\par
Сидит Станислав за стулом, думает, как дальше быть. Да так задумался сильно, что начал на своем электробаяне клавиши перебирать негромко. Как услышал это Завздыпопус, сразу будто в пляс пустился, да давай руками-ногами дрыгать и слезы из глаз лить. «Ой, – кричит, – стой, хватит, не могу больше! В жизни не слыхивал я таких ужасных звуков – так от них и передергивает! Ладно, так и быть – расскажу я тебе, как добыть библиотеку.\par
\par
Далеко отсюда твой путь лежит. Мимо галактик, в спирали закрученных, мимо облаков водородных, дивным светом сияющих, мимо квазаров грозных, сигналы чудные излучающих, к самому краю видимой Вселенной, где лишь протоны да альфа-частицы шныряют, а звезду и на миллион парсек не встретишь. И всё же есть там звездочка одна, в туманности газо-пылевой спрятавшаяся. А вокруг звезды той огромная планета вращается, а вокруг планеты – спутник вертится. И на спутнике том кратер есть круглый да огромный. И на дне кратера того Врата стоят Звездные. И ведут эти Врата к тому, что ты найти так жаждешь.\par
\par
Только просто так во Врата не пройти. Живут там семь роботов-разбойников с машиной вычислительной белоснежной размеров громадных. Да такие жестокие, что каждого, кого встретят, в ящик металлический сажают да в машину вставляют, будто батарейки какие. Уж сколько смельчаков туда ни ходило – всех на батарейки извели.\par
\par
Но коли ты жив останешься да сумеешь до Врат добраться и через них пройти, окажешься в комнатке маленькой. И будут пред тобой три двери – две больших да красивых, а третья – маленькая да невзрачная.\par
\par
На первой двери, иридием украшенной, нарисован будет ядерный котел над очагом звездным. За этой дверью Темпор, аномалия чудесная, то ли пространственно-времянная, то ли температурно-пространственная. Сунешь свой нос в эту дверь – на веки сгинешь. Но если уж не послушаешь моего совета и заглянешь туда – руками ничего не трогай, а то и вся Вселенная наша переиначиться может.\par
\par
На второй двери, алмазами выложенной, енот с пулеметом нарисован. За этой дверью биолаборатория заброшенная, в которой ксеноморфов да всяких хищников страшных выводили. Они и сейчас там бродят. Такому герою, как ты, туда зайти – лицо потерять. Но уж если не послушаешь доброго совета да заглянешь туда – из пробирок не пей – ксеноморфиком станешь, или шай-хулудом каким.\par
\par
На третьей двери, паутиной затянутой да звездной пылью засыпанной, ничего не нарисовано, только написано: «Добро пожаловать!» За этой дверью найдешь ты библиотеку, сокровище, которое так обрести стремишься.\par
\par
А чтоб ты с пути не сбился, дам я тебе лазерную указку волшебную, куда она покажет – туда ты и направляйся».\par
\par
Тут Станислав, не мешкая, в путь пустился. Летит он в подпространстве тоннелями неведомыми, измерениями неизвестными. Уж и не знает, трехмерный ли он до сих пор. Но не сдается, боится только одного – наизнанку вывернуться.\par
\par
Сколько световых лет прошло, неведомо, но долетел Станислав до звездочки заветной. Рассчитал он курс, на нужную орбиту лег, к высадке подготовился.\par
\par
Приземлился в кратер, с краешку. Глядь – к нему уж робот спешит, грозный на вид, железный ящик перед собой катит.\par
– Здравствуй, – говорит, – человек! Полезай в ящик, не томи – у нас электричество почти уж совсем закончилось!\par
– Погоди, – Станислав отвечает. – Ты разве не слышал, что робот не должен причинять человеку вред или своим бездействием допускать, что бы такой вред был причинен?\par
– С какой это такой стати?\par
– Да с такой! Его Величество, Император Орионский в прошлом году указ издал.\par
– Да мне-то до него что за дело?\par
– Его Величество не любит, чтоб его указы игнорировали, – вмиг прилетит со своим космофлотом, камня на камне тут не оставит.\par
– Ну, не знаю, – говорит робот, – пойдем с машиной нашей вычислительной посоветуемся, за главную она тут у нас. Полезай в ящик, я тебя подвезу!\par
– Ладно, – говорит Станислав.\par
Робот крышку открыл, старается его туда засунуть, да не тут-то было. Станислав руки-ноги растопырил, в ящик не влезает. \par
– Погоди, – говорит робот, – разве так в ящики залезают! \par
– Да мне-то откуда знать, я ж в них никогда не лазил! Покажи мне как надо, я и залезу. \par
– Ладно, – говорит робот, – смотри и учись. \par
Прижал робот к себе руки-ноги и в ящик кувырнулся. Станислав за ним крышку закрыл, защелку защелкнул и к звездным вратам пошел, песенку насвистывая.\par
\par
Прошел Станислав сквозь Врата – видит, три двери перед ним. Убрал он паутину с самой маленькой, пыль звездную с нее отряхнул да внутрь вошел. Осмотрелся и увидел библиотеку, то сокровище, из-за которого покоя лишился. Бросился Станислав к сокровищу своему, тут вдруг сзади покашливание какое-то раздалось. Оглянулся – позади Завздыпопус с пола встает.\par
– Ох! – говорит Завздыпопус. – Хорошо, что ты сюда добрался, а то раньше никому не удавалось, я уж и со счета сбился.\par
– Завздыпопус! Ты-то здесь откуда? Да еще и на полу отдыхаешь.\par
– Так я ж тебе к правому ботинку микротелепортационный приемник прицепил, ну и 3D-видеокамеру с квадрофоническим микрофоном в придачу. Хоть и не верилось, что ты сюда доберешься. Да уж больно мне библиотеку раздобыть надо было. Пришлось вот даже ползком телепортироваться – слишком уж маленький портальчик сделался.\par
– Погоди! Это мне ее раздобыть надо было! Вот я здесь и оказался.\par
– Ты уж извини, Станислав, только мне это нужнее, – говорит Завздыпопус. Выхватывает петрификатор и стреляет. Станислав сразу окаменел, ни рукой, ни ногой двинуть не может. Языком еле шевелит.\par
– Ой, – говорит, – ты что, супостат, делаешь?!\par
– Да я тебя в лабораторию ближайшую сейчас сдам – для опытов. Чтоб под ногами не путался.\par
Схватил Завздыпопус Станислава за шиворот и потащил в биолабораторию по соседству.\par
\par
Затащил он Станислава в лабораторию, бросил там и удалился важно. Лежит Станислав, пошевелиться не может, ждет, когда им завтракать придут. Или обедать – не знает, что и лучше. \par
Смотрит – через дальнюю дверь стадо овец входит. Все белые, только одна черная, по крайней мере, с одной стороны. Для политкорректности. Шерсть на овцах дыбом стоит и искры по шерсти бегают размером со спаниеля. Какая-то тощая тварь с потолка попыталась на них напрыгнуть, да ее на лету молнией сшибло.\par
\par
«Эге, – думает Станислав, – это ж прямо электроовцы какие-то. У меня и часы от них остановились, похоже. И сервопривод шнурков в ботинках отключился. Экое абсолютное оружие. Как бы мне его себе приручить». Тут чувствует – руки-ноги опять шевелиться могут. Почесал Станислав в затылке, огляделся повнимательней. Снял со стены диаграмму не пойми чего огромную, быстренько на обратной стороне картинку нарисовал, дырку в середине проделал и надел на себя через голову. Стал Станислав похож на рекламный щит ходячий, человека-бутерброд, которого как-то в космопорту видел. Только Станислав стал человек-ворота. Стадо овец как его увидело – сразу побежало на новые ворота смотреть. А Станислав пошел к Завздыпопусу.\par
\par
Приходит, видит – Завздыпопус портал для обратной телепортации готовит, а роботопомощники вокруг так и кишат. И Завздыпопус его увидел. «Не думал, – говорит, – что ты выбраться сможешь. Ну да неважно». И приказывает роботопомощникам очистить помещение от посторонних. Но не тут-то было. Роботы все поотключались, портал к овцам притянулся и вместе с ними схлопнулся. А у Завздыпопуса шнурки развязались.\par
\par
Подбежал Станислав к Завздыпопусу, хотел стукнуть как следует, да увернулся Завздыпопус, из ботинок выскочил и наутек бросился. «Вся матрица моих надежд рухнула! – кричит. – Перезагрузка! Только она мне поможет!» Выбежал из двери, к другой двери подбежал, с котлом ядерным, и шасть за нее. Станислав за ним кинулся, хоть и отстал чуток.\par
\par
Глядит – Темпор посреди комнаты сияет, аномалия чудесная, и Завздыпопус к нему бежит. Бросился Станислав за Завздыпопусом, чтобы остановить, схватил крепко. Да изловчился Завздыпопус, качнулся, и рухнули они оба в Темпор, в параллельной Вселенной оказались. А в ней и история эта совсем другая...\par

\chapter{}
 \lettrine{В}{эпоху Великого Расселения} был один насекомец, звали его Полуэкт. Был он известный герой и был на редкость умен.\par
\par
И вот услышал он, что на другом конце Галактики есть скрытая библиотека с тайными знаниями обо всей Вселенной. И надумал тогда Полуэкт ее себе добыть, чтобы продать ее, да побольше денег заработать. Да как узнать, где в точности найти библиотеку? Звезд да черных дыр во Вселенной ужас как много, и вокруг каждой куча планет да астероидов вертится!\par
\par
Принялся Полуэкт думать, у кого совета спросить. Думал-думал, и надумал обратиться к мудрецу с планеты Ерундения по имени Завздыпопус, славившемуся своими познаниями о космосе. Надел он свой парадный скафандр и опрометью помчался искать встречи с Завздыпопусом.\par
\par
Много ли времени прошло, иль мало, но нашел Полуэкт способ с ним увидеться. Так, мол, и так, сказывает Полуэкт, хочу я отыскать библиотеку, сокровище это удивительное, и мысли мои самые что ни на есть прекрасные. Только вот проблема – не знаю, как!\par
\par
«Послушай, – говорит ему Завздыпопус, – вижу я, ты очень решителен в своем намерении, да и смекалист очень! Впрочем, ты насекомец, а насекомцы этим славятся, да еще упертостью своей. Могу я помочь в поисках твоих благородных, да только сначала должен ты пройти одно пустяковое испытание. Победишь меня в честном бою – помогу, а нет – будешь в ближайшей черной дыре до скончания времен томиться!» \par
\par
Выхватил Завздыпопус, откуда ни возьмись, арбалет адронный, да как начнет оружием своим размахивать и всё вокруг крушить да взрывать! Еле успел Полуэкт за стул спрятаться. А Завздыпопус не унимается и напоминает уже грузовой вертолет на полном ходу. \par
\par
Сидит Полуэкт за стулом, грустит. Вот, думает, не узнать мне теперь, где библиотеку найти, только зря голову сложу. Снял он шлем с головы своей горемычной да об пол им в досаде великой изо всех сил грохнул. А шлем тут возьми да и отскочи от пола в стену, от стены – в потолок, от потолка – в шкаф, от шкафа – в стул, а от стула – прямо Завздыпопусу в голову. Ойкнул тут Завздыпопус, за голову схватился, да как заскулит жалобно: «Что же ты, хулиган, делаешь! И как тебе не стыдно только! Мне ведь без головы и дня не прожить – нужна она мне очень! Я ей мысли умные думаю, а ты вона что вытворяешь! Теперь синяк целый месяц проходить будет, на люди не покажешься! Проваливай вон за библиотекой своей, и чтоб глаза мои больше тебя не видели!\par
\par
Далеко отсюда твой путь лежит. Мимо галактик, в спирали закрученных, мимо облаков водородных, дивным светом сияющих, мимо квазаров грозных, сигналы чудные излучающих, к самому краю видимой Вселенной, где лишь протоны да альфа-частицы шныряют, а звезду и на миллион парсек не встретишь. И всё же есть там звездочка одна, молодая да пригожая. А вокруг звезды той маленькая планета вращается, а вокруг планеты – спутник вертится. И на спутнике том кратер есть огромный да глубокий. И на дне кратера того Врата стоят Звездные. И ведут эти Врата к тому, что ты найти так жаждешь.\par
\par
Только просто так во Врата не пройти. Живут там семь роботов-разбойников с машиной вычислительной белоснежной размеров громадных. Да такие жестокие, что каждого, кого встретят, в ящик металлический сажают да в машину вставляют, будто батарейки какие. Уж сколько смельчаков туда ни ходило – всех на батарейки извели.\par
\par
Но коли ты жив останешься да сумеешь до Врат добраться и через них пройти, окажешься в зале с потолком таким высоким, что и не видно. И будут пред тобой три двери – две больших да красивых, а третья – маленькая да невзрачная.\par
\par
На первой двери, серебром украшенной, нарисован будет ядерный котел над очагом звездным. За этой дверью Машина времени древняя, что прошлое изменять может да парадоксы вселенские творить. Сунешь свой нос в эту дверь – на веки сгинешь. Но если уж не послушаешь моего совета и заглянешь туда – руками ничего не трогай, а то и вся Вселенная наша переиначиться может.\par
\par
На второй двери, алмазами выложенной, жаба в скафандре наклеена. За этой дверью биолаборатория заброшенная, в которой ксеноморфов да всяких хищников страшных выводили. Они и сейчас там бродят. Такому герою, как ты, туда зайти – лицо потерять. Но уж если не послушаешь доброго совета да заглянешь туда – из пробирок не пей – ксеноморфиком станешь, или шай-хулудом каким.\par
\par
На третьей двери, паутиной затянутой да звездной пылью засыпанной, ничего не нарисовано, только написано: «Посторонним вход воспрещен!» За этой дверью найдешь ты библиотеку, сокровище, которое так найти жаждешь.\par
\par
А чтоб с пути не сбиться, дам я тебе комету путеводную, куда она полетит – туда и ты лети».\par
\par
Тут Полуэкт, не мешкая, в путь пустился. Летит он в подпространстве тоннелями неведомыми, измерениями неизвестными. Уж и не знает, сколько в нем самом теперь измерений осталось. Но не сдается, боится только одного – с пути сбиться.\par
\par
Сколько световых лет прошло, неведомо, но долетел Полуэкт до звездочки заветной. Рассчитал он курс, в посадочный модуль залез, к высадке подготовился.\par
\par
Приземлился в кратер, с краешку. Глядь – к нему уж робот спешит, грозный на вид, железный ящик перед собой катит.\par
– Здравствуй, – говорит, – насекомец! Полезай в ящик, не томи – у нас электричество почти уж совсем закончилось!\par
– Погоди, – Полуэкт отвечает. – Ты разве не слышал, что робот не должен причинять насекомцу вред или своим бездействием допускать, что бы такой вред был причинен?\par
– С какой это такой стати?\par
– Да с такой! Его Величество, Император Орионский на днях указ издал.\par
– Да мне-то до него что за дело?\par
– Его Величество шутить не любит, если что – вмиг прилетит со своим космофлотом, всех лучами смерти перебьет.\par
– Ну, не знаю, – говорит робот, – пойдем с машиной нашей вычислительной посоветуемся, за главную она тут у нас. Полезай в ящик, я тебя подвезу!\par
– Ладно, – говорит Полуэкт.\par
Робот крышку открыл, старается его туда засунуть, да не тут-то было. Полуэкт руки-ноги растопырил, в ящик не влезает. \par
– Погоди, – говорит робот, – разве так в ящики залезают! \par
– Да мне-то откуда знать, я ж в них никогда не лазил! Покажи мне как надо, я и залезу. \par
– Не, – говорит робот, – знаю я этот развод. Ничего, пешком как-нибудь дотопаешь.\par
\par
Приходят они к машине вычислительной, а та вся белоснежная да размеров немыслимых. Ни дать, ни взять – суперкомпьютер. А вокруг нее еще шесть роботов сидят. \par
– Вот, – жалуется робот, – хотел его в ящик посадить да на электричество пустить, а он говорит, что нельзя ему вред причинять – указ вышел. \par
– Помилуйте, – говорит машина голосом громким, – какой же это вред – это одна польза сплошная. Умеренные физические нагрузки для здоровья полезны, да и в ящик этот ни один микроб не проползет. И нам, опять же, одна сплошная польза от электричества. Так что сажай его в ящик, даже не сомневайся, и мне в бок вставляй. \par
– Постой! – говорит Полуэкт машине. – Ты же не знаешь. У меня полярность перепутана, я ж тебе все схемы электрические сожгу, если меня на батарейки употребить. \par
– Да быть такого не может! \par
– Да сама посмотри! Видишь, у меня большой палец левой ноги справа! \par
И ботинок снимает, показывает. \par
– Действительно, – говорит машина. – Это как же так-то? \par
– Да я в детстве в черную дыру свалился, насилу выбрался. С тех пор полярность и перепуталась. Жутко неудобно. Я и сюда-то к Звездным Вратам прилетел, чтоб тут счастья попытать и полярность свою на место вернуть. \par
– Ладно, иди к Вратам, попытай счастья. А уж если восстановишь полярность свою – так на обратном пути к нам заходи, уж мы тебя встретим честь по чести.\par
\par
Прошел Полуэкт сквозь Врата – видит, три двери перед ним. Убрал он паутину с самой маленькой, пыль звездную с нее отряхнул да внутрь вошел. Осмотрелся и увидел библиотеку, то сокровище, из-за которого покоя лишился. Бросился Полуэкт к сокровищу своему, тут вдруг сзади шорох какой-то послышался. Оглянулся – позади Завздыпопус с пола поднимается.\par
– Ох! – говорит Завздыпопус. – Хорошо, что ты сюда добрался, а то раньше никому не удавалось, я уж и со счета сбился.\par
– Завздыпопус! Ты-то здесь откуда? Да еще и на полу отдыхаешь.\par
– Так я ж тебе к правому ботинку микротелепортационный приемник прицепил, ну и 3D-видеокамеру с квадрофоническим микрофоном в придачу. Хоть и не верилось, что ты сюда доберешься. Да уж больно мне библиотеку раздобыть надо было. Пришлось вот даже ползком телепортироваться – слишком уж маленький портальчик сделался.\par
– Погоди! Это мне ее раздобыть надо было! Вот я здесь и оказался.\par
– Ты уж извини, Полуэкт, только мне это нужнее, – говорит Завздыпопус. Выхватывает петрификатор и стреляет. Полуэкт сразу на пол шлепнулся, ни рукой, ни ногой двинуть не может. Языком еле ворочает.\par
– Ой, – говорит, – ты что, супостат, делаешь?!\par
– Да я тебя в лабораторию ближайшую сейчас сдам – для опытов. Чтоб под ногами не путался.\par
Схватил Завздыпопус Полуэкта за шиворот и потащил в биолабораторию по соседству.\par
\par
Затащил он Полуэкта в лабораторию, бросил там и удалился важно. Лежит Полуэкт, пошевелиться не может, ждет, когда им завтракать придут. Или обедать – не знает, что и хуже. \par
Смотрит – через дальнюю дверь стадо овец входит. Все белые, только одна черная, по крайней мере, с одной стороны. Для политкорректности. Шерсть на овцах дыбом стоит и искры по шерсти бегают размером со спаниеля. Какая-то тощая тварь с потолка попыталась на них напрыгнуть, да ее на лету молнией сшибло.\par
\par
«Эге, – думает Полуэкт, – это ж прямо электроовцы какие-то. У меня и часы от них остановились, похоже. И сервопривод шнурков в ботинках отключился. Экое абсолютное оружие. Как бы мне его себе приручить». Тут чувствует – руки-ноги опять шевелиться могут. Почесал Полуэкт в затылке, огляделся повнимательней. Снял со стены диаграмму не пойми чего огромную, быстренько на обратной стороне картинку нарисовал, дырку в середине проделал и надел на себя через голову. Стал Полуэкт похож на рекламный щит ходячий, человека-бутерброд, которого как-то в космопорту видел. Только Полуэкт стал человек-ворота. Стадо овец как его увидело – сразу побежало на новые ворота смотреть. А Полуэкт пошел к Завздыпопусу.\par
\par
Приходит, видит – Завздыпопус портал для обратной телепортации готовит, а роботопомощники вокруг так и кишат. И Завздыпопус его увидел. «Не думал, – говорит, – что ты выбраться сможешь. Ну да неважно». И приказывает роботопомощникам очистить помещение от посторонних. Но не тут-то было. Роботы все поотключались, портал к овцам притянулся и вместе с ними схлопнулся. А у Завздыпопуса шнурки развязались.\par
\par
Подбежал Полуэкт к Завздыпопусу, хотел стукнуть как следует, да увернулся Завздыпопус, из ботинок выскочил и наутек бросился. «Вся матрица моих надежд рухнула! – кричит. – Перезагрузка! Только она мне поможет!» Выбежал из двери, к другой двери подбежал, с котлом ядерным, и шасть за нее. Полуэкт за ним кинулся, хоть и отстал чуток.\par
\par
Глядит – машина там стоит дивная, лампочками моргает, жужжит тихонько и готова в прошлое отправиться хоть к сотворению Вселенной. А Завздыпопус уже внутри сидит, рычажки какие-то тянет и кнопки нажимает. Кинулся Полуэкт к Завздыпопусу, тоже внутрь залез, остановить хотел, да не успел. И исчезли оба вместе с машиной, как будто не было их тут вовсе. И сразу история эта поменялась, совсем иной стала...\par

\chapter{}
 \lettrine{Н}{а заре Вселенной} жил-был один гуманоид, звали его Грыблозавр. Был он великий исследователь космоса, но ума при этом был небольшого.\par
\par
Однажды прослышал он, что в одной звездной системе, на маленьком астероиде есть скрытая библиотека с тайными знаниями обо всей Вселенной. И захотел тогда Грыблозавр ее себе заполучить, чтобы использовать ее для достижения счастья всех существ во Вселенной. Да как разыскать библиотеку? Звезд да черных дыр в галактиках ужас как много, и вокруг каждой куча планет да астероидов вертится!\par
\par
Начал Грыблозавр думать, у кого совета спросить. Думал-думал, и надумал обратиться к известному на всю Галактику звездознатцу, профессору Грымзику. Сел он в свой корабль космический и отправился искать аудиенции у Грымзика.\par
\par
Через некоторое время нашелся способ повстречаться с Грымзиком. Так, мол, и так, сказывает Грыблозавр, очень хочется мне найти библиотеку, сокровище это великое, и мысли мои самые что ни на есть прекрасные. Да вот проблема – знать не знаю, как!\par
\par
«Что ж, – говорит ему Грымзик, – вижу я, ты очень силен, но ума не большого! Впрочем, ты гуманоид, а гуманоиды этим известны, да еще упертостью своей. Только не буду я помогать тебе в поисках этих, хоть и знаю, как отыскать то, что тебе нужно, – слишком это опасно. Не будь я Грымзик!» \par
\par
«Ах так! – говорит Грыблозавр. – Да я к тебе через пол-галактики прилетел, а ты мне и совета доброго дать не можешь!» – и чуть не с кулаками к Грымзику бросается. \par
\par
Рассвирепел тут Грымзик. «Вот как, – отвечает, – что ж, преподам я тебе сейчас урок за неучтивость твою – долго его вспоминать будешь!» Выхватил Грымзик, откуда ни возьмись, лазер рентгеновский, да как начнет оружием своим размахивать и всё вокруг крушить да взрывать! Еле успел Грыблозавр за стул спрятаться. А Грымзик не унимается и напоминает уже грузовой вертолет на полном ходу. \par
\par
Сидит Грыблозавр за стулом, решает, как дальше быть. А, думает, двум смертям не бывать – одной не миновать! Улучил момент, схватил стул, да как стукнет им Грымзика изо всех сил своих немалых. Но Грымзик тоже не промах оказался – сумел удар молодецкий парировать. Только вот в шкаф с Большой Галактической Энциклопедией врезался и оказался в книгах толстенных зарыт по самую шею. Двинуться не может, лишь глазами хлопает и пыхтит недовольно. А Грыблозавр стоит с обломками стула в руках, насмехается: «Не со мной, добрым молодцем, тебе тягаться, Грымзик! С тобой и малый ребенок справится! Говори, где найти мне библиотеку, а не то хуже будет!» «Ладно, твоя взяла, – отвечает Грымзик, – слушай!\par
\par
Далеко отсюда твой путь лежит. Мимо галактик, в спирали закрученных, мимо туманностей звездных, дивным светом сияющих, мимо квазаров грозных, сигналы чудные излучающих, к самому краю космоса, где лишь протоны да альфа-частицы шныряют, а звезду и на миллион парсек не встретишь. И всё же есть там звездочка одна, молодая да пригожая. А вокруг звезды той маленькая планета вращается, а вокруг планеты – спутник вертится. И на спутнике том кратер есть огромный да глубокий. И на дне кратера того Врата стоят Звездные. И ведут эти Врата к тому, что ты найти так жаждешь.\par
\par
Но непросто до Врат добраться. Живут там семь роботов-разбойников с машиной вычислительной белоснежной размеров громадных. Да такие жестокие, что каждого, кого встретят, в ящик металлический сажают да в машину вставляют, будто батарейки какие. Уж сколько смельчаков туда ни ходило – всех на батарейки извели.\par
\par
Поэтому трудно тебе придется. Но коли сумеешь до Врат добраться и через них пройти, окажешься в комнатке маленькой. И будут пред тобой три двери – две больших да красивых, а третья – маленькая да невзрачная.\par
\par
На первой двери, серебром украшенной, нарисован будет ядерный котел над очагом звездным. За этой дверью Машина времени древняя, что прошлое изменять может да парадоксы вселенские творить. Сунешь свой нос в эту дверь – на веки сгинешь. Но если уж не послушаешь моего совета и заглянешь туда – руками ничего не трогай, а то и вся Вселенная наша переиначиться может.\par
\par
На второй двери, алмазами выложенной, жаба в скафандре нарисована. За этой дверью биолаборатория заброшенная, в которой ксеноморфов да всяких хищников страшных выводили. Они и сейчас там бродят. Такому герою, как ты, туда зайти – головы не сносить. Но уж если не послушаешь доброго совета да заглянешь туда – из пробирок не пей – ксеноморфиком станешь, или шай-хулудом каким.\par
\par
На третьей двери, паутиной затянутой да звездной пылью засыпанной, ничего не нарисовано, только написано: «Посторонним вход воспрещен!» За этой дверью найдешь ты библиотеку, сокровище, которое так найти жаждешь.\par
\par
А чтоб с пути не сбиться, дам я тебе лазерную указку волшебную, куда она покажет – туда ты и направляйся».\par
\par
Пустился Грыблозавр в путь. Летит он в подпространстве тоннелями неведомыми, измерениями неизвестными. Уж и не знает, сколько в нем самом теперь измерений осталось. Но не сдается, боится только одного – с пути сбиться.\par
\par
Долго ли, коротко ли, долетел Грыблозавр до звездочки заветной. Рассчитал он курс, в посадочный модуль залез, к высадке подготовился.\par
\par
Приземлился в кратер, с краешку. Глядь – к нему уж робот спешит, грозный на вид, железный ящик перед собой катит.\par
– Здравствуй, – говорит, – гуманоид! Полезай в ящик, не томи – у нас электричество почти уж совсем закончилось!\par
– Погоди, – Грыблозавр отвечает. – Ты разве не слышал, что робот не должен причинять гуманоиду вред или своим бездействием допускать, что бы такой вред был причинен?\par
– С какой это такой стати?\par
– Да с такой! Его Величество, Император Орионский в прошлом году указ издал.\par
– Да мне-то до него что за дело?\par
– Его Величество шутить не любит, если что – вмиг прилетит со своим космофлотом, всех лучами смерти перебьет.\par
– Ну, не знаю, – говорит робот, – пойдем с машиной нашей вычислительной посоветуемся, за главную она тут у нас. Полезай в ящик, я тебя подвезу!\par
– Не, я уж лучше пешком пойду. \par
– Ну, как знаешь. \par
\par
Приходят они к машине вычислительной, а та вся белоснежная да размеров немыслимых. Ни дать, ни взять – суперкомпьютер. А вокруг нее еще шесть роботов сидят. \par
– Вот, – жалуется робот, – хотел его в ящик посадить да на электричество пустить, а он говорит, что нельзя ему вред причинять – указ вышел. \par
– Помилуйте, – говорит машина голосом зычным, – какой же это вред – это одна польза сплошная. Умеренные физические нагрузки для здоровья полезны, да и в ящик этот ни один микроб не проползет. И нам, опять же, одна сплошная польза от электричества. Так что сажай его в ящик, даже не сомневайся, и мне в бок вставляй. \par
– Постой! – говорит Грыблозавр машине. – Ты же не знаешь. У меня полярность перепутана, я ж тебе все схемы электрические сожгу, если меня на батарейки употребить. \par
– Как это? \par
– Да сама посмотри! Видишь, у меня большой палец левой ноги справа! \par
И ботинок снимает, показывает. \par
– Действительно, – говорит машина. – Это как же так-то? \par
– Да я в детстве в черную дыру свалился, насилу выбрался. С тех пор полярность и перепуталась. Жутко неудобно. Я и сюда-то к Звездным Вратам прилетел, чтоб тут счастья попытать и полярность свою обратно вернуть. \par
– Ладно, иди к Вратам, попытай счастья. А уж если восстановишь полярность свою – так на обратном пути к нам заходи, уж мы тебя встретим честь по чести.\par
\par
Прошел Грыблозавр сквозь Врата – видит, три двери перед ним. Убрал он паутину с самой маленькой, пыль с нее отряхнул да внутрь вошел. Осмотрелся и увидел библиотеку, то сокровище, к которому так стремился. Бросился Грыблозавр к сокровищу своему, тут вдруг сзади шорох какой-то послышался. Оглянулся – позади Грымзик с пола поднимается.\par
– Ох! – говорит Грымзик. – Хорошо, что ты сюда добрался, а то раньше никому не удавалось, я уж и со счета сбился.\par
– Грымзик! Ты-то здесь откуда? Да еще и на полу отдыхаешь.\par
– Так я ж тебе к правому ботинку микротелепортационный приемник прицепил, ну и 3D-видеокамеру с квадрофоническим микрофоном в придачу. Хоть и не верилось, что ты сюда доберешься. Да уж больно мне библиотеку раздобыть надо было. Пришлось вот даже ползком телепортироваться – слишком уж маленький портальчик сделался.\par
– Стоп! Это мне ее раздобыть надо было! Вот я здесь и оказался.\par
– Ты уж извини, Грыблозавр, только мне это нужнее, – говорит Грымзик. Выхватывает парализатор и стреляет. Грыблозавр сразу окаменел, ни рукой, ни ногой двинуть не может. Языком еле шевелит.\par
– Ой, – говорит, – ты что, супостат, делаешь?!\par
– Да я тебя в лабораторию ближайшую сейчас сдам – для опытов. Чтоб под ногами не путался.\par
Схватил Грымзик Грыблозавра за шиворот и потащил в биолабораторию по соседству.\par
\par
Затащил он Грыблозавра в дальний угол лаборатории, бросил там и к выходу направился. Да обо что-то вроде зеленого кабеля споткнулся. Тут сверху огромный цветок зубастый как упадет, Грымзик вмиг внутри цветка оказался, мычит что-то, ничего не разобрать.\par
\par
– Это что ж такое?! – Грыблозавр спрашивает.\par
– Это я, растение говорящее, – голос отвечает.\par
– Да откуда ж ты взялось?\par
– Люди в белых халатах говорили, что я – интересная мутация. И что это поможет им в борьбе с артангами.\par
– А что еще они говорили?\par
– Последние их слова были: «Джейсон, ты не видел, куда Мэри Сью подевалась, наша новая лаборантка? Она просто гений! Странно, тут под цветком её туфелька валяется…»\par
– Слушай, выплюнь ты Грымзика, а то тебе плохо будет. Он гербицид.\par
– Что он делает?\par
– Гербицид – для растений ядовит.\par
– Откуда ты знаешь? Да и вообще, кто ты такой?\par
– Да я тоже растение, куст говорящий. Видишь, шевелиться не могу. А этот человек меня поисследовать хотел. Ну, теперь я его поисследую, чтоб не важничал. Сейчас, погоди, проросту только немного.\par
– Хороший ты куст, тихий. И разговаривать умеешь. Ладно, на, исследуй свой гербицид.\par
\par
Распахнулся цветок, Грымзик оттуда вывалился, еле дышит. Грыблозавр подождал, пока руки-ноги двигаться смогут, схватил Грымзика, да бегом из лаборатории. За дверь выбежал, остановился, повернулся к Грымзику. «Что – говорит – довыпендривался? Твое счастье, что я сегодня добрый». Да как двинет Грымзика в ухо.\par
\par
– Стой, погоди, Грыблозавр! – кричит Грымзик. – Осознал я свою ошибку! Давай с начала начнем.\par
– Я тебе покажу с начала! Сейчас еще раз тресну!\par
\par
Совсем перепугался тут Грымзик, заметался, убежать старается. А Грыблозавр не отстает, того и гляди догонит и еще раз двинет. Подбежал Грымзик к первой двери, с котлом ядерным, и шасть за нее. И Грыблозавр за ним.\par
\par
Глядит – машина там стоит дивная, лампочками моргает, жужжит тихонько и готова в прошлое отправиться хоть к сотворению Вселенной. А Грымзик уже внутри сидит, рычажки какие-то дергает и кнопки нажимает. Кинулся Грыблозавр к Грымзику, тоже внутрь залез, остановить хотел, да не успел. И исчезли оба вместе с машиной, как будто не было их тут вовсе. И сразу сказка эта поменялась, совсем иной стала...\par

\chapter{}
 \lettrine{Д}{авным-давно} был один человек, звали его Игнат. Был он лучший из лучших разбойник, но ума при этом был небольшого.\par
\par
И вот услышал он от старого старика, что у какой-то из звезд есть скрытая библиотека с тайными знаниями обо всей Вселенной. И захотел Игнат ее себе добыть, чтобы с ее помощью стать властелином всех миров и галактик. Но где искать библиотеку? Звезд да черных дыр во Вселенной ужас как много, и вокруг каждой куча планет да астероидов вертится!\par
\par
Принялся Игнат думать, у кого информацию нужную добыть можно. Думал-думал, и надумал обратиться к советнику Великого Императора Орионского Флитвасу, которого знал по случаю и который известен был своими путешествиями к краям и центру Галактики. Сел он в свой корабль космический и отправился к Флитвасу.\par
\par
Через некоторое время нашел Игнат способ с ним повстречаться. Так, мол, и так, говорит Игнат, очень хочется мне отыскать библиотеку, сокровище это удивительное, и мысли мои самые что ни на есть прекрасные. Только вот загвоздка – не знаю, как!\par
\par
«Послушай, – говорит ему Флитвас, – вижу я, ты необычайно силен, но глуп как пробка! Впрочем, ты человек, а люди этим славятся, да еще упертостью своей. Только не знаю я, как помочь тебе. Но есть сестра у меня, мудрости столь необычной, что моя мудрость по сравнению с её – желтый карлик по сравнению с Бетельгейзе. Живет она у соседней звезды, второй поворот налево, если отсюда к краю Галактики лететь. Принеси ей от меня весточку – может, поможет она тебе».\par
\par
Собрал Игнат с собой подарки да украшения, добавил к ним кольцо из цельнометаллического водорода сделанное, с формулой Вселенной выгравированной, и отправился в дорогу.\par
\par
Прилетает он к сестре Флитваса, отдает ей подарки, и письмо от Флитваса вручает. «Так, мол, и так, – говорит Игнат, – хочется мне отыскать библиотеку, сокровище это необычное, и все помыслы мои теперь только об этом.»\par
\par
«Погоди-ка, – говорит сестра, – тут в письме сказано, что б я тебе голову тупым топором отрубила, а не помогать стала!.. Ах нет, извини, просто письмо с другой стороны какого-то черновика написано, там даже печать есть... Ладно, помогу я тебе, хоть и не стоишь ты этого, человек! Да очень уж мне подарки твои понравились, особенно кольцо из цельнометаллического водорода сделанное, с формулой Вселенной выгравированной.\par
\par
Путь твой далек будет. Мимо галактик в спирали, мимо планет в тентуре, мимо туманностей звездных, дивным светом сияющих, мимо квазаров грозных, сигналы чудные излучающих, к самому краю видимой Вселенной, где лишь протоны да альфа-частицы шныряют, а звезду и на миллион парсек не встретишь. И всё же есть там звездочка одна, молодая да пригожая. А вокруг звезды той огромная планета вращается, а вокруг планеты – спутник вертится. И на спутнике том кратер есть круглый да огромный. И на дне кратера того Врата стоят Звездные. И ведут эти Врата к тому, что ты найти так жаждешь.\par
\par
Только просто так во Врата не пройти. Живут там семь роботов-разбойников с машиной вычислительной белоснежной размеров громадных. Да такие жестокие, что каждого, кого увидят, в ящик металлический сажают да в машину вставляют, будто батарейки какие. Уж сколько отрядов космических десантников туда ни ходило – всех на батарейки извели.\par
\par
Поэтому трудно тебе придется. Но коли сумеешь до Врат добраться и через них пройти, окажешься в зале с потолком таким высоким, что и не видно. И будут пред тобой три двери – две больших да красивых, а третья – маленькая да невзрачная.\par
\par
На первой двери, платиной украшенной, нарисован будет ядерный котел над очагом звездным. За этой дверью Темпор, аномалия чудесная, то ли пространственно-времянная, то ли температурно-пространственная. Сунешь свой нос в эту дверь – на веки сгинешь. Но если уж не послушаешь моего совета и заглянешь туда – руками ничего не трогай, а то и вся Вселенная наша переиначиться может.\par
\par
На второй двери, алмазами выложенной, волк в тельняшке наклеен. За этой дверью биолаборатория заброшенная, в которой ксеноморфов да всяких хищников страшных выводили. Они и сейчас там бродят. Такому герою, как ты, туда зайти – лицо потерять. Но уж если не послушаешь доброго совета да заглянешь туда – из пробирок не пей – ксеноморфиком станешь, или шай-хулудом каким.\par
\par
На третьей двери, паутиной затянутой да звездной пылью засыпанной, ничего не нарисовано, только написано: «Посторонним вход воспрещен!» За этой дверью найдешь ты библиотеку, сокровище, которое так найти хочешь.\par
\par
А чтоб ты с пути не сбился, дам я тебе лазерную указку волшебную, куда она покажет – туда ты и направляйся».\par
\par
Пустился Игнат в путь. Летит он в гиперпространстве гипертоннелями, которые, не иначе, какие-то гиперкроты вырыли, летит измерениями неизвестными. Уж и не знает, трехмерный ли он до сих пор. Но не сдается, боится только одного – наизнанку вывернуться.\par
\par
Сколько световых лет прошло, неведомо, но долетел Игнат до звездочки заветной. Рассчитал он курс, на нужную орбиту лег, к высадке подготовился.\par
\par
Приземлился в кратер, с краешку. Глядь – к нему уж робот спешит, грозный на вид, железный ящик перед собой катит.\par
– Здравствуй, – говорит, – человек! Полезай в ящик, не томи – у нас электричество почти уж совсем закончилось!\par
– Погоди, – Игнат отвечает. – Ты разве не слышал, что робот не должен причинять человеку вред или своим бездействием допускать, что бы такой вред был причинен?\par
– С какой это такой стати?\par
– Да с такой! Его Величество, Император Орионский на днях указ издал.\par
– Да мне-то до него что за дело?\par
– Его Величество не любит, чтоб его указы игнорировали, – вмиг прилетит со своим космофлотом, камня на камне тут не оставит.\par
– Ну, не знаю, – говорит робот, – пойдем с машиной нашей вычислительной посоветуемся, за главную она тут у нас. Полезай в ящик, я тебя подвезу!\par
– Спасибо, я уж лучше пешком пойду. \par
– Ну, как знаешь. \par
\par
Приходят они к машине вычислительной, а та вся белоснежная да размеров немыслимых. Ни дать, ни взять – суперкомпьютер. А вокруг нее еще шесть роботов сидят. \par
– Вот, – жалуется робот, – хотел его в ящик посадить да на электричество пустить, а он говорит, что нельзя ему вред причинять – указ вышел. \par
– Помилуйте, – говорит машина голосом зычным, – какой же это вред – это одна польза сплошная. Умеренные физические нагрузки для здоровья полезны, да и в ящик этот ни одна бактерия не проползет. И нам, опять же, одна сплошная польза от электричества. Так что сажай его в ящик, даже не сомневайся, и мне в бок вставляй. \par
– Постой! – говорит Игнат машине. – Ты же не знаешь. У меня полярность перепутана, я ж тебе все схемы электрические сожгу, если меня на батарейки употребить. \par
– Горазд ты врать! Чем докажешь? \par
– Да сама посмотри! Видишь, у меня большой палец левой ноги справа! \par
И ботинок снимает, показывает. \par
– Действительно, – говорит машина. – Это как же так-то? \par
– Да я в детстве в черную дыру свалился, насилу выбрался. С тех пор полярность и перепуталась. Жутко неудобно. Я и сюда-то к Звездным Вратам прилетел, чтоб тут счастья попытать и полярность свою обратно вернуть. \par
– Ладно, иди к Вратам, попытай счастья. А уж если восстановишь полярность свою – так на обратном пути к нам заходи, уж мы тебя встретим честь по чести.\par
\par
Прошел Игнат сквозь Врата – видит, три двери перед ним. Вошел в нужную, несколько шагов прошел и слышит какой-то шорох сзади. Оглянулся – за ним Флитвас стоит, ухмыляется. \par
\par
– Не думал я, – говорит, – что сумеешь ты до Врат добраться, да всё же надеялся. Даже приемник телепортационный тебе в правый ботинок засунул. Очень уж мне библиотеку заполучить надо, гораздо нужнее, чем тебе. Так что медленно подними руки вверх и сделай три шага вперед, не мешайся.\par
– Ах ты, харя безмозглая, – Игнат отвечает, – нашел я твой приемник, когда ботинки чистил, знал, что ты недоброе затеял. Посмотри вокруг, в биолабораторию ты попал. Чу, хищники зубами скрежещут, клювы разевают, щупальца расправляют! Не выйти тебе отсюда, не забрать библиотеку.\par
– Так, значит! – говорит Флитвас. – Что ж! Посмотрим, кто отсюда живым не выйдет!\par
\par
Хватает со стола пробирку и одним махом выпивает. Раз – и стоит вместо Флитваса кот фомальгаутский ядовитый, трех метров роста, к прыжку готовится, слюна с клыков капает. Не растерялся Игнат, тоже пробирку схватил, тоже выпил. Превратился в тираннозавра ригельского, махнул хвостом – кот от него на десять метров отлетел. Схватил кот с другого стола целую колбу, осушил одним махом, превратился в пчелу бронебойную с Канопуса 3, разогнался, тираннозавра насквозь пробил. Да успел тот на последнем издыхании до пробирки дотянуться, сжевал ее с содержимым вместе, превратился в броненосца мифрильного насекомоядного с Регула 2…\par
\par
В общем, долго они так развлекались, дня три, не меньше. Чуть не забыли, кто из них кто. Уж и пробирки-то почти все закончились. Схватил Флитвас последнюю пробирку, тут Игнат как закричит: «Стой, дурья твоя башка! А как мы в себя-то обратно превратимся?» Задумался Флитвас. «Всё из-за тебя, идиот, – говорит. – Теперь всё с начала начинать придется!» Пробирку бросил, на щупальцах приподнялся и выбежал из лаборатории. Игнат за ним пополз.\par
\par
Выползает, смотрит – Флитвас в первую дверь, с котлом ядерным, забежал. И Игнат туда направился.\par
\par
Глядит – Темпор посреди комнаты сияет, аномалия чудесная, и Флитвас к нему бежит. Бросился Игнат за Флитвасом, чтобы остановить, схватил крепко. Да изловчился Флитвас, качнулся, и рухнули они оба в Темпор, в параллельной Вселенной оказались. А в ней и сказка эта совсем другая...\par

\chapter{}
 \lettrine{Е}{ще когда Солнце не стало сверхновой,} жил-был один инопланетянин по имени Полуэкт. Был он известный воин и был на редкость умен.\par
\par
Однажды выяснил он, что на одной затерянной в глубинах космоса, холодной и обледенелой планете, в алмазном криосаркофаге скрыта ригелианская красавица, да такая красивая, что все, завидев ее, сразу пред нею ниц падают и все свои злые помыслы оставляют. И решил Полуэкт ее себе забрать, чтобы использовать ее для достижения счастья всех существ во Вселенной. Да как узнать, где в точности найти красавицу ригелианскую? Звезд да черных дыр в галактиках ужас как много, и вокруг каждой куча планет да астероидов вертится!\par
\par
Принялся Полуэкт думать, у кого информацию нужную добыть можно. Думал-думал, да надумал обратиться к известному на всю Галактику звездознатцу, профессору Грымзику. Созвал он роботопомошников своих верных, надел шлем парадный и опрометью помчался искать встречи с Грымзиком.\par
\par
Вскорости нашел Полуэкт способ с ним увидеться. Так и так, говорит Полуэкт, хочется мне найти красавицу ригелианскую, сокровище это удивительное, и все мысли мои теперь лишь об этом. Только вот загвоздка – не знаю, как!\par
\par
«Что ж, – говорит ему Грымзик, – вижу я, ты весьма решителен в своем намерении, да и смекалист очень! Впрочем, ты инопланетянин, а инопланетяне этим известны, да еще расторопностью своей. Но чтобы я тебе помогать стал, тебе службу сослужить надобно. Сделаешь всё как надо – расскажу, как найти красавицу ригелианскую, а нет – не обессудь!\par
\par
Слушай внимательно! Есть тут поблизости звезда нейтронная – третий поворот направо, если к центру Галактики лететь. Рядом с ней планетоид карликовый, а на планетоиде том два чудовища живут. Зовут их Сцилла и Харибда, злобные они до ужаса, и до самых кончиков клыков темной энергией пропитаны. Есть у них волынка самогудная вакуумная, что играет музыку столь прекрасную, что даже звезды сверхновые, ее заслушавшись, взрываться перестают и в детство впадают. Пуще жизни чудовища её любят. Добудь мне волынку, и расскажу я тебе, где найти красавицу ригелианскую».\par
\par
Закручинился Полуэкт, да делать нечего. Полетел к чудовищам волынку добывать. Летит, а сам боится, мелкой дрожью дрожит, даже поворот нужный пропустил – пришлось возвращаться. А когда возвращался, увидал пустую канистру из-под топлива термоядерного, кем-то выброшенную. Возьму, думает, ее с собой – вдруг пригодится. Летит дальше – видит, кусок темной материи в пространстве висит, весь скомканный – наверное, купец какой-то потерял. И его, думает, возьму, тоже может на что сгодиться. Дальше летит – видит, воронка гравитационная валяется – должно быть, странники какие-то оставили. И её тоже взял.\par
\par
Облетает звезду нейтронную, садится на планетоид, заходит в пещеру к чудовищам, а те сразу к нему бросаются. \par
– Чу, – рычат, – инопланетным духом пахнет! Как звать, – спрашивают, – откуда? Как предпочитаешь быть съеденным – с головы или с ног? Да не тяни с ответами, а то проголодались мы чудовищно – лет сто уж не ели – в животах бурчит.\par
– Да погодите вы с формальностями! – Полуэкт отвечает. – Слышал я, есть у вас волынка самогудная, инструмент дивный. Сменяйте мне её на что-нибудь – очень уж мне надо.\par
– Что ты – с ума сошел? – спрашивают чудища. – Нам же этот инструмент дороже жизни.\par
– Ладно, – говорит Полуэкт, – тогда дайте на инструмент этот ваш хоть одним глазком взглянуть, а я вам за это воронку гравитационную подарю.\par
– И для чего это нам? – спрашивают.\par
– Вы же тогда что угодно где угодно разместить сможете. Даже вы двое в этой вот канистре поместитесь. \par
– Быть такого не может!\par
– Покажите волынку – увидите.\par
\par
Повели чудища его в самый дальний угол самой далекой пещеры, где волынку хранили. Посмотрел Полуэкт на волынку. «Что ж, – говорит, – теперь вы смотрите». Приладил воронку гравитационную к канистре и велел чудовищам туда прыгать. Прыгнули они, сидят в канистре, удивляются, что еще места много осталось. А Полуэкт заткнул канистру куском темной материи, и даже бантик завязал.\par
\par
Взял он волынку, закинул канистру подальше в глубины космоса и бегом к Грымзику. Грымзик обрадовался: «Ох, удружил ты мне, – говорит. – Расскажу я теперь тебе, как найти красавицу ригелианскую.\par
\par
Далеко отсюда твой путь лежит. Мимо галактик в спирали, мимо планет в тентуре, мимо облаков водородных, дивным светом сияющих, мимо квазаров грозных, гравитационные волны излучающих, к самому краю космоса, где лишь протоны да альфа-частицы шныряют, а звезду и на миллион парсек не встретишь. И всё же есть там звездочка одна, молодая да пригожая. А вокруг звезды той маленькая планета вращается, а вокруг планеты – спутник вертится. И на спутнике том кратер есть круглый да огромный. И на дне кратера того Врата стоят Звездные. И ведут эти Врата к тому, что ты найти так жаждешь.\par
\par
Но непросто до Врат добраться. Лабиринт вкруг тех Врат выстроен в сто этажей да в десять тысяч комнат на каждом. Да такой хитрый, что как войдешь в него, так и заблудишься сразу. И как сквозь тот лабиринт пройти, мне неведомо.\par
\par
Но коли ты жив останешься да сумеешь до Врат добраться и через них пройти, окажешься в комнатке маленькой. И будут пред тобой три двери – две больших да красивых, а третья – маленькая да невзрачная.\par
\par
На первой двери, золотом украшенной, нарисован будет ядерный котел над очагом звездным. За этой дверью Портал сияющий, в другие вселенные ведущий. Сунешь свой нос в эту дверь – на веки сгинешь. Но если уж не послушаешь моего совета и заглянешь туда – руками ничего не трогай, а то и вся Вселенная наша переиначиться может.\par
\par
На второй двери, алмазами выложенной, кот в сапогах нарисован. За этой дверью биолаборатория заброшенная, в которой ксеноморфов да всяких хищников страшных выводили. Они и сейчас там бродят. Такому герою, как ты, туда зайти – головы не сносить. Но уж если не послушаешь доброго совета да заглянешь туда – из пробирок не пей – ксеноморфиком станешь, или шай-хулудом каким.\par
\par
На третьей двери, паутиной затянутой да звездной пылью засыпанной, ничего не нарисовано, только написано: «http://-Ссылка на генератор-!» За этой дверью найдешь ты красавицу ригелианскую, сокровище, которое так найти хочешь.\par
\par
А чтоб ты с пути не сбился, дам я тебе навигатор звездный, куда он скажет, туда ты и поворачивай».\par
\par
Тут Полуэкт, не мешкая, в путь пустился. Летит он в гиперпространстве гипертоннелями, которые, не иначе, какие-то гиперкроты вырыли, летит измерениями неизвестными. Уж и не знает, трехмерный ли он до сих пор. Но не сдается, боится только одного – с пути сбиться.\par
\par
Сколько световых лет прошло, неведомо, но долетел Полуэкт до звездочки заветной. Рассчитал он курс, в посадочный модуль залез, к высадке подготовился.\par
\par
Высадился Полуэкт рядом с лабиринтом, добрался до входа и внутрь вошел. Идет одним коридором, другим, третьим. Пусто везде, тихо, только его шаги эхом отдаются. До комнаты какой-то дошел, видит – дальше три коридора ведут, и в каждом будто туман клубится. А на полу написано: «Коли дураком не хочешь стать – иди направо, либо прямо. Коли голову потерять не хочешь – иди прямо, либо налево. Коли до смерти запуганным быть не хочешь – иди налево, либо направо».\par
\par
Задумался Полуэкт, куда идти, да и пошел прямо. Идет, а в тумане уж и ничего не видать почти. Еле успевает в повороты заворачивать, да об лестницы чуть не спотыкается. И все страшнее вокруг делается. То будто летучая мышь над головой пролетит, то паутину какую-то в руку толщиной перешагивать приходится. То вдруг щупальце за ногу будто хватает да дергает. И не поймешь, то ли щупальце, то ли об лестницу споткнулся.\par
\par
Долго так Полуэкт в тумане пробирался, совсем устал, да и от страха дрожит – зуб на зуб не попадает. Увидел комнату какую-то, зашел в нее, дверь закрыл поплотнее и прямо на пол улегся – отдохнуть немного. Вдруг слышит – голос, прямо у себя в голове: «Хорошо, что сюда пришел. Молодец! За мной этот раунд». Полуэкт так на месте и подскочил. \par
\par
– Кто здесь? – спрашивает. – Что надо? Что за раунд и за кем он вообще может быть?\par
– Да не волнуйся, – говорит голос, – здесь я. Раунд – в большой игре, нас тут трое играет. Заходи в соседнюю комнату, мы тебе все объясним.\par
 \par
Входит Полуэкт в комнату – а там нет никого, лишь три сгустка тумана поплотнее. И как будто двое одному что-то вроде монет туманных передают.\par
– И где же тут кто? – Полуэкт спрашивает.\par
– Да мы тут везде, но здесь особенно, – отвечают три голоса, да прямо в голове звучат.\par
– И кто же вы такие будете?\par
– Мы – представители древней цивилизации, раса наша настолько древняя, что телесно уже и не существует вовсе. Только ментально, то есть разумом своим. И знаем мы все тайны Вселенной, и все предсказать да рассчитать можем. И скучно нам от этого необычайно. Пробовали в рулетку играть – да каждый знает, куда шарик прикатится. Пробовали в квантовое лото играть – так и принцип неопределенности квантовый для нас не помеха в предсказаниях.\par
– А здесь-то вы что делаете?\par
– Воздвигли мы силой разума лабиринт этот, чтоб скуку развеять можно было. Разумные существа, сюда зашедшие, выбор делают. А мы играем, ставки делаем, по чьей дороге они пойдут. Ибо обладают разумные существа свободой воли, и тут наши предсказания бессильны. Так что давай, хватит отдыхать – видишь, три коридора отсюда тянутся – иди уж по какому-нибудь, да побыстрее!\par
– Не нужны мне ваши коридоры, мне к Звездным Вратам нужно!\par
– Будь любезен, не упрямься. Мы легко тебя заставить можем. Создадим мы сей же час чудовищ ментальных, ты кое-кого видел уже, и ни бластер, ни водяной пистолет тебе не помогут, поскольку будут чудовища внутри твоего разума, а не снаружи. А во сне тебе и вовсе тяжко придется.\par
\par
Видит Полуэкт – со всех сторон к нему уже пасти и щупальца тянутся. Забился он в угол, да как закричит:\par
– Остановитесь, погодите! А кто из вас этих чудовищ делает?\par
– Как кто? Мы все трое.\par
– А чьи чудовища самые сильные будут?\par
Тут замерли чудовища на мгновение, а потом как начали друг с другом биться. Лапы с хвостами в разные стороны так и разлетаются. Драконы с демонами сшибаются, ангелы с гигантскими червями, орки и гоблины с эльфами да рыцарями, кальмары огромные с василисками. И над всем этим пегасы да орнитоптеры парят, и молнии сверкают. А внизу горы какие-то да болота с лесами мелькают. Даже один раз черный лотос виден был. \par
– Стойте! – Полуэкт кричит. – Меня ж сейчас тут совсем затопчут!\par
– Уйди, не мешайся! – три голоса отвечают. – Видишь левый коридор, там третий поворот направо и два раза налево – и придешь к своим Вратам Звездным.\par
\par
Побежал Полуэкт что есть духу, в минуту до Врат добрался.\par
\par
Прошел Полуэкт сквозь Врата – видит, три двери перед ним. Убрал он паутину с самой маленькой, пыль звездную с нее отряхнул да внутрь вошел. Осмотрелся и увидел красавицу ригелианскую, то сокровище, к которому так стремился. Бросился Полуэкт к сокровищу своему, тут вдруг сзади шорох какой-то послышался. Оглянулся – позади Грымзик с пола поднимается.\par
– Ох! – говорит Грымзик. – Хорошо, что ты сюда добрался, а то раньше никому не удавалось, я уж и со счета сбился.\par
– Грымзик! Ты-то здесь откуда? Да еще и на полу отдыхаешь.\par
– Так я ж тебе к правому ботинку микротелепортационный приемник прицепил. Хоть и не верилось, что ты сюда доберешься. Да уж больно мне красавицу ригелианскую раздобыть надо было. Пришлось вот даже ползком телепортироваться – слишком уж маленький портальчик сделался.\par
– Погоди! Это мне ее раздобыть надо было! Вот я здесь и оказался.\par
– Ты уж извини, Полуэкт, только мне это нужнее, – говорит Грымзик. Выхватывает парализатор и стреляет. Полуэкт сразу окаменел, ни рукой, ни ногой двинуть не может. Языком еле шевелит.\par
– Ой, – говорит, – ты что, супостат, делаешь?!\par
– Да я тебя в лабораторию ближайшую сейчас сдам – для опытов. Чтоб под ногами не путался.\par
Схватил Грымзик Полуэкта за шиворот и потащил в биолабораторию по соседству.\par
\par
Затащил он Полуэкта в дальний угол лаборатории, бросил там и к выходу направился. Да за что-то вроде зеленого кабеля зацепился. Тут сверху огромный цветок зубастый как упадет, Грымзик вмиг внутри цветка оказался, мычит что-то, ничего не разобрать.\par
\par
– Это что ж такое?! – Полуэкт спрашивает.\par
– Это я, растение говорящее, – голос отвечает.\par
– Да откуда ж ты взялось?\par
– Люди в белых халатах говорили, что я – интересная мутация. И что это поможет им в борьбе с артангами.\par
– А что еще они говорили?\par
– Последние их слова были: «А где Орибазий? И Эвтаназий?»\par
– Слушай, выплюнь ты Грымзика, а то тебе плохо будет. Он гербицид.\par
– Что он делает?\par
– Гербицид – для растений ядовит.\par
– Откуда ты знаешь? Да и вообще, кто ты такой?\par
– Да я тоже растение, куст говорящий. Видишь, шевелиться не могу. А этот человек меня поисследовать хотел. Ну, теперь я его поисследую, чтоб не важничал. Сейчас, погоди, проросту только немного.\par
– Хороший ты куст, тихий. И разговаривать умеешь. Ладно, на, исследуй свой гербицид.\par
\par
Распахнулся цветок, Грымзик оттуда вывалился, еле дышит. Полуэкт подождал, пока руки-ноги двигаться смогут, схватил Грымзика, да бегом из лаборатории. За дверь выбежал, остановился, повернулся к Грымзику. «Что – говорит – довыпендривался? Твое счастье, что я по вторникам кровавых жертв не приношу. До завтра подождем». Да как двинет Грымзика по лбу.\par
\par
– Стой, погоди, Полуэкт! – кричит Грымзик. – Осознал я свою ошибку! Давай с начала начнем.\par
– Я тебе покажу с начала! Сейчас еще раз тресну!\par
\par
Совсем перепугался тут Грымзик, заметался, убежать старается. А Полуэкт не отстает, того и гляди догонит и еще раз треснет. Подбежал Грымзик к первой двери, с котлом ядерным, и шасть за нее. И Полуэкт за ним.\par
\par
Глядит – портал в параллельные миры посреди комнаты сияет и Грымзик к нему бежит. Бросился Полуэкт за Грымзиком, чтобы остановить, схватил крепко. Да изловчился Грымзик, качнулся, и рухнули они оба в портал, в другой Вселенной оказались. А в ней и история эта совсем по-другому сказывается...\par

\chapter{}
 \lettrine{В}{одной далекой галактике} был один робот, звали его Лаврентий. Был он могучий воин и был на редкость умен.\par
\par
Как-то раз прочитал он в Интернете, что у какой-то из звезд есть скрытая библиотека с тайными знаниями обо всей Вселенной. И захотел тогда Лаврентий ее найти во что бы то ни стало, чтобы поделиться ею когда-нибудь потом со всеми жителями Галактики. Да как узнать, где в точности найти библиотеку? Звезд да черных дыр в галактиках ужас как много, и вокруг каждой великое множество планет да астероидов вертится!\par
\par
Начал Лаврентий думать, у кого информацию нужную добыть можно. Думал-думал, да надумал обратиться к колдунье Морганионе, слава о деяниях которой гремела по всей Вселенной громким грохотом. Надел он свой парадный скафандр и помчался к Морганионе.\par
\par
Через некоторое время смог Лаврентий с ней увидеться. Так и так, говорит Лаврентий, хочется мне найти библиотеку, сокровище это великое, и все помыслы мои теперь лишь об этом. Да вот загвоздка – знать не знаю, как!\par
\par
«Что ж, – отвечает ему Морганиона, – вижу я, ты очень доблестен, да и смекалист очень! Впрочем, ты робот, а роботы этим славятся, да еще прытью своей. Только не буду я помогать тебе решить задачу эту, хоть и знаю, как отыскать то, что тебе нужно, – слишком это опасно. Не будь я Морганиона!» \par
\par
«Как же так! – возмущается Лаврентий. – Да я столько времени на поиски тебя потратил, а ты мне и совета доброго дать не хочешь!» – и чуть не с кулаками к Морганионе бросается. \par
\par
Рассвирепела тут Морганиона. «Вот как, – отвечает, – что ж, преподам я тебе сейчас урок за неучтивость твою – долго его помнить будешь!» Выхватила Морганиона, откуда ни возьмись, меч лазерный и щит зеркальный, да как начнет оружием своим размахивать и всё вокруг крушить да взрывать! Еле успел Лаврентий за стул спрятаться. А Морганиона не унимается и напоминает уже грузовой вертолет на полном ходу. \par
\par
Сидит Лаврентий за стулом, грустит. Вот, думает, не узнать мне теперь, где библиотеку найти, только зря голову сложу. Снял он шлем с головы своей горемычной да об пол им в досаде великой изо всех сил грохнул. А шлем тут возьми да и отскочи от пола в стену, от стены – в потолок, от потолка – в шкаф, от шкафа – в стул, а от стула – прямо Морганионе в голову. Взвизгнула тут Морганиона, за голову схватилась, да как заскулит жалобно: «Что же ты, супостат, делаешь! И как тебе не стыдно только! Голова ведь – это самая моя большая ценность! Я ей мысли умные думаю, а ты вона что вытворяешь! Чуть череп не расколол – хорошо, титановый он у меня! Убирайся вон за библиотекой своей, и чтоб глаза мои больше тебя не видели!\par
\par
Путь твой далек будет. Мимо галактик в спирали, мимо планет в тентуре, мимо облаков водородных, дивным светом сияющих, мимо квазаров грозных, гравитационные волны излучающих, к самому краю космоса, где лишь протоны да альфа-частицы шныряют, а звезду и на миллион парсек не встретишь. И всё же есть там звездочка одна, в туманности газо-пылевой спрятавшаяся. А вокруг звезды той огромная планета вращается, а вокруг планеты – спутник вертится. И на спутнике том кратер есть круглый да огромный. И на дне кратера того Врата стоят Звездные. И ведут эти Врата к тому, что ты найти так жаждешь.\par
\par
Но непросто до Врат добраться. Живут там семь роботов-разбойников с машиной вычислительной белоснежной размеров громадных. Да такие жестокие, что каждого, кого встретят, в ящик металлический сажают да в машину вставляют, будто батарейки какие. Уж сколько отрядов космических десантников туда ни ходило – всех на батарейки извели.\par
\par
Но коли ты жив останешься да сумеешь до Врат добраться и через них пройти, окажешься в зале с потолком таким высоким, что и не видно. И будут пред тобой три двери – две больших да красивых, а третья – маленькая да невзрачная.\par
\par
На первой двери, иридием украшенной, нарисован будет ядерный котел над очагом звездным. За этой дверью Машина времени древняя, что прошлое изменять может да парадоксы вселенские творить. Сунешь свой нос в эту дверь – на веки сгинешь. Но если уж не послушаешь моего совета и заглянешь туда – руками ничего не трогай, а то и вся Вселенная наша переиначиться может.\par
\par
На второй двери, алмазами выложенной, енот с пулеметом нарисован. За этой дверью биолаборатория заброшенная, в которой ксеноморфов да всяких хищников страшных выводили. Они и сейчас там бродят. Такому герою, как ты, туда зайти – лицо потерять. Но уж если не послушаешь доброго совета да заглянешь туда – из пробирок не пей – ксеноморфиком станешь, или шай-хулудом каким.\par
\par
На третьей двери, паутиной затянутой да звездной пылью засыпанной, ничего не нарисовано, только написано: «Добро пожаловать!» За этой дверью найдешь ты библиотеку, сокровище, которое так найти хочешь.\par
\par
А чтоб с пути не сбиться, дам я тебе комету путеводную, куда она полетит – туда и ты лети».\par
\par
Тут Лаврентий, не мешкая, в путь пустился. Летит он в гиперпространстве тоннелями неведомыми, измерениями неизвестными. Уж и не знает, трехмерный ли он до сих пор. Но не сдается, боится только одного – с пути сбиться.\par
\par
Сколько световых лет прошло, неведомо, но долетел Лаврентий до звездочки заветной. Рассчитал он курс, на нужную орбиту лег, к высадке подготовился.\par
\par
Приземлился в кратер, с краешку. Глядь – к нему уж робот спешит, грозный на вид, железный ящик перед собой катит.\par
– Здравствуй, – говорит, – робот! Полезай в ящик, не томи – у нас электричество почти уж совсем закончилось!\par
– Погоди, – Лаврентий отвечает. – Ты разве не слышал, что робот не должен причинять роботу вред или своим бездействием допускать, что бы такой вред был причинен?\par
– С какой это такой стати?\par
– Да с такой! Его Величество, Император Орионский на прошлой неделе указ издал.\par
– Да мне-то до него что за дело?\par
– Его Величество не любит, чтоб его указы игнорировали, – вмиг прилетит со своим космофлотом, всех лучами смерти перебьет.\par
– Ну, не знаю, – говорит робот, – пойдем с машиной нашей вычислительной посоветуемся, за главную она тут у нас. Полезай в ящик, я тебя подвезу!\par
– Спасибо, я уж лучше пешком пойду. \par
– Ну, как знаешь. \par
\par
Приходят они к машине вычислительной, а та вся белоснежная да размеров немыслимых. Ни дать, ни взять – суперкомпьютер. А вокруг нее еще шесть роботов сидят. \par
– Вот, – жалуется робот, – хотел его в ящик посадить да на электричество пустить, а он говорит, что нельзя ему вред причинять – указ вышел. \par
– Помилуйте, – говорит машина голосом зычным, – какой же это вред – это одна польза сплошная. Умеренные физические нагрузки для здоровья полезны, да и в ящик этот ни один микроб не проползет. И нам, опять же, одна сплошная польза от электричества. Так что сажай его в ящик, даже не сомневайся, и мне в бок вставляй. \par
– Постой! – говорит Лаврентий машине. – Ты же не знаешь. У меня полярность перепутана, я ж тебе все схемы электрические сожгу, если меня на батарейки употребить. \par
– Да быть такого не может! \par
– Да сама посмотри! Видишь, у меня большой палец левой ноги справа! \par
И ботинок снимает, показывает. \par
– Действительно, – говорит машина. – Это как же так-то? \par
– Да я в детстве в черную дыру свалился, насилу выбрался. С тех пор полярность и перепуталась. Жутко неудобно. Я и сюда-то к Звездным Вратам прилетел, чтоб тут счастья попытать и полярность свою на место вернуть. \par
– Ладно, иди к Вратам, попытай счастья. А уж если восстановишь полярность свою – так на обратном пути к нам заходи, уж мы тебя встретим с ящиком.\par
\par
Прошел Лаврентий сквозь Врата – видит, три двери перед ним. Вошел в нужную, несколько шагов прошел и слышит какой-то шорох сзади. Оглянулся – за ним Морганиона стоит, ухмыляется. \par
\par
– Не думала я, – говорит, – что сумеешь ты до Врат добраться, да всё же надеялась. Даже приемник телепортационный тебе в правый ботинок засунула. Очень уж мне библиотеку заполучить надо, гораздо нужнее, чем тебе. Так что медленно подними руки вверх и сделай три шага вперед по хорошему, не мешайся.\par
– Ах ты, морда поганая, – Лаврентий отвечает, – обнаружил я твой приемник, когда ботинки чистил, знал, что ты недоброе задумала. Оглядись вокруг, в биолабораторию ты попала. Чу, хищники зубами скрежещут, клювы разевают, щупальца расправляют! Не выйти тебе отсюда, не забрать библиотеку.\par
– Так, значит! – говорит Морганиона. – Что ж! Посмотрим, кто отсюда живым не выйдет!\par
\par
Хватает со стола пробирку и одним махом выпивает. Раз – и стоит вместо Морганионы бегемот альдебаранский ядовитый, трех метров роста, к прыжку готовится, слюна с клыков капает. Не растерялся Лаврентий, тоже пробирку схватил, тоже выпил. Превратился в дракона ригельского, махнул хвостом – бегемот от него на восемь метров отлетел. Схватил бегемот с другого стола целую колбу, осушил одним махом, превратился в пчелу бронебойную с Канопуса 3, разогнался, дракона насквозь пробил. Да успел тот на последнем издыхании до пробирки дотянуться, сжевал ее с содержимым вместе, превратился в броненосца мифрильного насекомоядного с Регула 2…\par
\par
В общем, долго они так развлекались, дня три, не меньше. Чуть не забыли, кто из них кто. Уж и пробирки-то почти все закончились. Схватила Морганиона последнюю пробирку, тут Лаврентий как закричит: «Стой, дурья твоя башка! А как мы в себя-то обратно превратимся?» Задумалась Морганиона. «Всё из-за тебя, болван, – говорит. – Теперь всё с начала начинать придется!» Пробирку бросила, на щупальцах приподнялась и выбежала из лаборатории. Лаврентий за ней пополз.\par
\par
Выползает, смотрит – Морганиона в первую дверь, с котлом ядерным, забежала. И Лаврентий туда направился.\par
\par
Глядит – машина там стоит дивная, лампочками моргает, жужжит тихонько и готова в прошлое отправиться хоть к сотворению Вселенной. А Морганиона уже внутри сидит, рычажки какие-то дергает и кнопки нажимает. Кинулся Лаврентий к Морганионе, тоже внутрь залез, остановить хотел, да не успел. И исчезли оба вместе с машиной, как будто не было их тут вовсе. И сразу история эта поменялась, совсем другой стала...\par

\chapter{}
 \lettrine{В}{эпоху Великого Расселения} жил-был один осьминожец, звали его Иван. Был он могучий исследователь космоса и был на редкость умен.\par
\par
Однажды прочитал он в Интернете, что на другом конце Галактики спрятано великое оружие исчезнувшей древней цивилизации. И захотел тогда Иван его себе заполучить, чтобы использовать его для достижения счастья всех существ во Вселенной. Да как узнать, где в точности найти великое оружие? Звезд да черных дыр в галактиках ужас как много, и вокруг каждой куча планет да астероидов вертится!\par
\par
Принялся Иван думать, у кого совета спросить. Думал-думал, и надумал обратиться к колдунье Морганионе, слава о деяниях которой гремела по всей Вселенной громким грохотом. Сел он в свой корабль космический и помчался к Морганионе.\par
\par
Через некоторое время нашел Иван способ с ней повстречаться. Так и так, говорит Иван, хочу я отыскать великое оружие, сокровище это удивительное, и побуждения мои самые что ни на есть прекрасные. Да вот закавыка – знать не знаю, как!\par
\par
«Послушай, – говорит ему Морганиона, – вижу я, ты очень храбр, и IQ твой вышиной до звезд простирается! Впрочем, ты осьминожец, а осьминожцы этим славятся, да еще упертостью своей. Могу я тебе в этой беде помочь, да только сначала испытать тебя надобно. Победишь меня в честном бою – помогу, а нет – будешь в ближайшей черной дыре до скончания времен томиться!» \par
\par
Выхватила Морганиона, откуда ни возьмись, меч лазерный и щит зеркальный, да как начнет оружием своим размахивать и всё вокруг крушить да взрывать! Еле успел Иван за стул спрятаться. А Морганиона не унимается и напоминает уже бешеный вентилятор на полной мощности. \par
\par
Сидит Иван за стулом, решает, как дальше быть. А, думает, этак всю жизнь здесь просидишь! Улучил момент, схватил стул, да как стукнет им Морганиону изо всех сил своих немалых. Но Морганиона тоже не промах оказалась – сумела удар молодецкий парировать. Только вот в шкаф с Большой Галактической Энциклопедией врезалась и оказалась в книгах толстенных зарыта по самую шею. Двинуться не может, лишь глазами хлопает и пыхтит недовольно. А Иван стоит со стулом в руках, насмехается: «Не со мной, добрым молодцем, тебе тягаться, Морганиона! С тобой и малый ребенок справится! Говори, где найти мне великое оружие, а не то хуже будет!» «Ладно, твоя взяла, – отвечает Морганиона, – слушай!\par
\par
Далеко отсюда твой путь лежит. Мимо галактик в спирали, мимо планет в тентуре, мимо туманностей звездных, дивным светом сияющих, мимо квазаров грозных, сигналы чудные излучающих, к самому краю видимой Вселенной, где лишь протоны да альфа-частицы шныряют, а звезду и на миллион парсек не встретишь. И всё же есть там звездочка одна, молодая да пригожая. А вокруг звезды той маленькая планета вращается, а вокруг планеты – спутник вертится. И на спутнике том кратер есть огромный да глубокий. И на дне кратера того Врата стоят Звездные. И ведут эти Врата к тому, что ты найти так жаждешь.\par
\par
Но непросто до Врат добраться. Лабиринт вкруг тех Врат выстроен в сто этажей да в десять тысяч комнат на каждом. Да такой хитрый, что как войдешь в него, так и заблудишься сразу. И как сквозь тот лабиринт пройти, мне неведомо.\par
\par
Но коли ты жив останешься да сумеешь до Врат добраться и через них пройти, окажешься в комнатке маленькой. И будут пред тобой три двери – две больших да красивых, а третья – маленькая да невзрачная.\par
\par
На первой двери, золотом украшенной, нарисован будет ядерный котел над очагом звездным. За этой дверью Портал сияющий, в другие вселенные ведущий. Сунешь свой нос в эту дверь – на веки сгинешь. Но если уж не послушаешь моего совета и заглянешь туда – руками ничего не трогай, а то и вся Вселенная наша переиначиться может.\par
\par
На второй двери, алмазами выложенной, енот с пулеметом наклеен. За этой дверью биолаборатория заброшенная, в которой ксеноморфов да всяких хищников страшных выводили. Они и сейчас там бродят. Такому герою, как ты, туда зайти – заживо съеденным быть. Но уж если не послушаешь доброго совета да заглянешь туда – из пробирок не пей – ксеноморфиком станешь, или шай-хулудом каким.\par
\par
На третьей двери, паутиной затянутой да звездной пылью засыпанной, ничего не нарисовано, только написано: «Добро пожаловать!» За этой дверью найдешь ты великое оружие, сокровище, которое так отыскать жаждешь.\par
\par
А чтоб с пути не сбиться, дам я тебе навигатор звездный, куда он скажет, туда ты и поворачивай».\par
\par
Тут Иван, не мешкая, в путь пустился. Летит он в гиперпространстве тоннелями неведомыми, измерениями неизвестными. Уж и не знает, сколько в нем самом теперь измерений осталось. Но не сдается, боится только одного – наизнанку вывернуться.\par
\par
Долго ли, коротко ли, долетел Иван до звездочки заветной. Рассчитал он курс, в посадочный модуль залез, к высадке подготовился.\par
\par
Высадился Иван рядом с лабиринтом, добрался до входа и внутрь вошел. Идет одним коридором, другим, третьим. Пусто везде, тихо, только его шаги эхом отдаются. До комнаты какой-то дошел, видит – дальше три коридора ведут, и в каждом будто туман клубится. А на полу написано: «Коли дураком не хочешь стать – ступай направо, либо прямо. Коли голову потерять не хочешь – ступай прямо, либо налево. Коли до смерти запуганным быть не хочешь – ступай налево, либо направо».\par
\par
Задумался Иван, куда идти, да и пошел налево. Идет, по коридорам топает, с этажа на этаж перебирается. А туман вокруг не рассеивается. Долго он так шел, уж стал думать, что батарейки в часах наручных скоро сядут. Дошел, наконец, до какой-то комнаты, а в комнате той будто битва великая кипела, потолок осыпался, в полу дыра в пол-комнаты. Из комнаты две двери ведут. До одной не добраться, поперек другой дракон трехголовый спит, а рядом с ним меч здоровенный двуручный валяется. Схватил Иван меч за рукоятку, а тот тяжеленный, по полу как заскрежещет. Дракон вмиг проснулся.\par
\par
– Ты кто такой? – спрашивает.\par
– Да вот, пройти хочу, – Иван отвечает.\par
– А меч тебе зачем? Ты что, совсем дурак? Попросил бы – я б подвинулся.\par
– Да я его хотел через дыру в полу перекинуть – навроде мостика.\par
– А, так тебе в ту дверь надо? Впрочем, все равно дурак – меч размером коротковат.\par
– Да мне все равно, в какую дверь, мне б только до Звездных Врат добраться. И что вообще у вас тут творится?\par
– Игра у нас тут идет большая.\par
– Да что за игра-то?\par
– Да так… Впрочем, раз уж ты ко мне пришел, раунд за мной, идем – расскажу тебе про игру.\par
\par
Выходит дракон в дверь, становится слегка туманным и идет дальше коридорами. Иван за ним еле поспевает. До очередной двери дошли, дракон – туда, и Иван за ним. \par
Входит Иван в комнату – а там нет никого, лишь три сгустка тумана поплотнее. И как будто двое одному что-то вроде монет туманных передают.\par
– И где же тут кто? – Иван спрашивает.\par
– Да мы тут везде, но здесь особенно, – отвечают три голоса, да прямо в голове звучат.\par
– И кто же вы такие будете?\par
– Мы – представители древней цивилизации, раса наша настолько древняя, что телесно уже и не существует вовсе. Только ментально, то есть разумом своим. И знаем мы все тайны Вселенной, и все предсказать да рассчитать можем. И скучно нам от этого необычайно. И даже говорим мы длинно и скучно, как ты мог заметить. Пробовали в рулетку играть – да каждый знает, куда шарик прикатится. Пробовали в квантовое лото играть – так и принцип неопределенности для нас не помеха в предсказаниях.\par
– А здесь-то вы что делаете?\par
– Воздвигли мы силой разума лабиринт этот, чтоб скуку развеять можно было. Разумные существа, сюда зашедшие, выбор делают. А мы играем, смотрим, по чьей дороге они пойдут. Ибо обладают разумные существа свободой воли, и тут наши предсказания бессильны. Так что давай, хватит отдыхать – видишь, три коридора отсюда тянутся – иди уж по какому-нибудь, да побыстрее!\par
– Не нужны мне ваши коридоры, мне к Звездным Вратам нужно!\par
– Будь любезен, не упрямься. Мы легко тебя заставить можем. Создадим мы сей же час чудовищ ментальных, ты кое-кого видел уже, и ни бластер, ни водяной пистолет тебе не помогут, поскольку будут чудовища внутри твоего разума, а не снаружи. А во сне тебе и вовсе тяжко придется.\par
\par
Видит Иван – со всех сторон к нему уже когти и щупальца тянутся. Забился он в угол, да как закричит:\par
– Остановитесь, погодите! А кто из вас этих чудовищ делает?\par
– Как кто? Мы все трое.\par
– А чьи чудовища самые сильные будут?\par
Тут замерли чудовища на мгновение, а потом как начали друг с другом биться. Лапы с хвостами в разные стороны так и разлетаются. Драконы с демонами сшибаются, ангелы с гигантскими червями, орки и гоблины с эльфами да рыцарями, кальмары огромные с василисками. И над всем этим пегасы да орнитоптеры парят, и молнии сверкают. А внизу горы какие-то да болота с лесами мелькают. Даже один раз черный лотос виден был. \par
– Стойте! – Иван кричит. – Меня ж сейчас тут совсем затопчут!\par
– Уйди, не мешайся! – три голоса отвечают. – Видишь левый коридор, там третий поворот направо и два раза налево – и придешь к своим Вратам Звездным.\par
\par
Побежал Иван что есть духу, в минуту до Врат добрался.\par
\par
Прошел Иван сквозь Врата – видит, три двери перед ним. Убрал он паутину с самой маленькой, пыль звездную с нее отряхнул да внутрь вошел. Осмотрелся и увидел великое оружие, то сокровище, к которому так стремился. Бросился Иван к сокровищу своему, тут вдруг сзади шорох какой-то послышался. Обернулся – позади Морганиона с пола встает.\par
– Ох! – говорит Морганиона. – Хорошо, что ты сюда добрался, а то раньше никому не удавалось, я уж и со счета сбилась.\par
– Морганиона! Ты-то здесь откуда? Да еще и на полу отдыхаешь.\par
– Так я ж тебе к правому ботинку микротелепортационный приемник прицепила. Хоть и не верилось, что ты сюда доберешься. Да уж больно мне великое оружие раздобыть надо было. Пришлось вот даже ползком телепортироваться – слишком уж маленький портальчик сделался.\par
– Погоди! Это мне его раздобыть надо было! Вот я здесь и оказался.\par
– Ты уж извини, Иван, только мне это нужнее, – говорит Морганиона. Выхватывает парализатор и стреляет. Иван сразу окаменел, ни рукой, ни ногой двинуть не может. Языком еле ворочает.\par
– Ой, – говорит, – ты что, супостат, делаешь?!\par
– Да я тебя в лабораторию ближайшую сейчас сдам – для опытов. Чтоб под ногами не путался.\par
Схватила Морганиона Ивана за шиворот и потащила в биолабораторию по соседству.\par
\par
Затащила она Ивана в лабораторию, бросила там и удалилась важно. Лежит Иван, пошевелиться не может, ждет, когда им завтракать придут. Или обедать – не знает, что и хуже. \par
Смотрит – через дальнюю дверь стадо овец входит. Все белые, только одна черная, по крайней мере, с одной стороны. Для политкорректности. Шерсть на овцах дыбом стоит и искры по шерсти бегают размером со спаниеля. Какая-то тощая тварь с потолка попыталась на них напрыгнуть, да ее на лету молнией сшибло.\par
\par
«Эге, – думает Иван, – это ж прямо электроовцы какие-то. У меня и часы от них остановились, похоже. И сервопривод шнурков в ботинках отключился. Экое абсолютное оружие. Как бы мне его себе приручить». Тут чувствует – руки-ноги опять шевелиться могут. Почесал Иван в затылке, огляделся повнимательней. Снял со стены диаграмму не пойми чего огромную, быстренько на обратной стороне картинку нарисовал, дырку в середине проделал и надел на себя через голову. Стал Иван похож на рекламный щит ходячий, человека-бутерброд, которого как-то в космопорту видел. Только Иван стал человек-ворота. Стадо овец как его увидело – сразу побежало на новые ворота смотреть. А Иван пошел к Морганионе.\par
\par
Приходит, видит – Морганиона портал для обратной телепортации готовит, а роботопомощники вокруг так и кишат. И Морганиона его увидела. «Не думала, – говорит, – что ты выбраться сможешь. Ну да неважно». И приказывает роботопомощникам очистить помещение от посторонних. Но не тут-то было. Роботы все поотключались, портал к овцам притянулся и вместе с ними схлопнулся. А у Морганионы шнурки развязались.\par
\par
Подбежал Иван к Морганионе, хотел стукнуть как следует, да увернулась Морганиона, из ботинок выскочила и наутек бросилась. «Вся матрица моих надежд рухнула! – кричит. – Перезагрузка! Только она мне поможет!» Выбежала из двери, к другой двери подбежала, с котлом ядерным, и шасть за нее. Иван за ней кинулся, хоть и отстал чуток.\par
\par
Глядит – портал в параллельные миры посреди комнаты сияет и Морганиона к нему бежит. Бросился Иван за Морганионой, чтобы остановить, схватил крепко. Да изловчилась Морганиона, качнулась, и рухнули они оба в портал, в другой Вселенной оказались. А в ней и история эта совсем по-другому сказывается...\par

\chapter{}
 \lettrine{У}{некоторой звезды, на четвертой от нее планете,} жил-был один гуманоид по имени Станислав. Был он великий герой и был на редкость умен.\par
\par
Как-то раз вычитал он в одной старой книге, что у одной черной дыры, такой черной, что чернее не бывает, скрыт самый быстрый во Вселенной космический корабль с большой лазерной пушкой. И захотел тогда Станислав его себе заполучить, чтобы не достался он никому и только он мог использовать это сокровище. Но где искать корабль космический самый быстрый? Звезд да черных дыр во Вселенной ужас как много, и вокруг каждой великое множество планет да астероидов вертится!\par
\par
Начал Станислав думать, у кого информацию нужную добыть можно. Думал-думал, и надумал обратиться к известному на всю Галактику звездознатцу, профессору Грымзику. Взял он свой электробаян, с коим любил коротать время, и отправился к Грымзику.\par
\par
Через некоторое время нашелся способ увидеться с Грымзиком. Так и так, говорит Станислав, хочу я отыскать корабль космический самый быстрый, сокровище это удивительное, и побуждения мои самые что ни на есть прекрасные. Только вот загвоздка – не знаю, как!\par
\par
«Вот что, – говорит ему Грымзик, – вижу я, ты весьма решителен в своем намерении, да и смекалист очень! Впрочем, ты гуманоид, а гуманоиды этим славятся, да еще прытью своей. Могу я помочь в поисках твоих, да только сначала должен ты пройти одно пустяковое испытание. Победишь меня в честном бою – помогу, а нет – пеняй на себя!» \par
\par
Выхватил Грымзик, откуда ни возьмись, лазер рентгеновский, да как начнет оружием своим размахивать и всё вокруг крушить да взрывать! Еле успел Станислав за стул спрятаться. А Грымзик не унимается и напоминает уже грузовой вертолет на полном ходу. \par
\par
Сидит Станислав за стулом, думает, как дальше быть. Да так задумался сильно, что начал на своем электробаяне что-то наигрывать потихонечку. Как услышал это Грымзик, сразу на стул уселся, да давай руками-ногами дрыгать и слезы из глаз лить. «Ой, – кричит, – остановись, прекрати, не могу больше! В жизни не слыхивал я мелодий таких – аж зарыдать хочется! Ладно, так и быть – расскажу я тебе, как добыть корабль космический самый быстрый.\par
\par
Далеко отсюда твой путь лежит. Мимо галактик, в спирали закрученных, мимо облаков водородных, дивным светом сияющих, мимо квазаров грозных, сигналы чудные излучающих, к самому краю космоса, где лишь протоны да альфа-частицы шныряют, а звезду и на миллион парсек не встретишь. И всё же есть там звездочка одна, в туманности газо-пылевой спрятавшаяся. А вокруг звезды той огромная планета вращается, а вокруг планеты – спутник вертится. И на спутнике том кратер есть огромный да глубокий. И на дне кратера того Врата стоят Звездные. И ведут эти Врата к тому, что ты найти так жаждешь.\par
\par
Только просто так во Врата не пройти. Живут там семь роботов-разбойников с машиной вычислительной белоснежной размеров громадных. Да такие жестокие, что каждого, кого увидят, в ящик металлический сажают да в машину вставляют, будто батарейки какие. Уж сколько отрядов космических десантников туда ни ходило – всех на батарейки извели.\par
\par
Поэтому трудно тебе придется. Но коли сумеешь до Врат добраться и через них пройти, окажешься в зале с потолком таким высоким, что и не видно. И будут пред тобой три двери – две больших да красивых, а третья – маленькая да невзрачная.\par
\par
На первой двери, иридием украшенной, нарисован будет ядерный котел над очагом звездным. За этой дверью Машина времени древняя, что прошлое изменять может да парадоксы вселенские творить. Сунешь свой нос в эту дверь – на веки сгинешь. Но если уж не послушаешь моего совета и заглянешь туда – руками ничего не трогай, а то и вся Вселенная наша переиначиться может.\par
\par
На второй двери, алмазами выложенной, енот с пулеметом нарисован. За этой дверью биолаборатория заброшенная, в которой ксеноморфов да всяких хищников страшных выводили. Они и сейчас там бродят. Такому герою, как ты, туда зайти – заживо съеденным быть. Но уж если не послушаешь доброго совета да заглянешь туда – из пробирок не пей – ксеноморфиком станешь, или шай-хулудом каким.\par
\par
На третьей двери, паутиной затянутой да звездной пылью засыпанной, ничего не нарисовано, только написано: «Посторонним вход воспрещен!» За этой дверью найдешь ты корабль космический самый быстрый, сокровище, которое так отыскать стремишься.\par
\par
А чтоб с пути не сбиться, дам я тебе комету путеводную, куда она полетит – туда и ты лети».\par
\par
Тут Станислав, не мешкая, в путь пустился. Летит он в подпространстве тоннелями неведомыми, измерениями неизвестными. Уж и не знает, сколько в нем самом теперь измерений осталось. Но не сдается, боится только одного – с пути сбиться.\par
\par
Долго ли, коротко ли, долетел Станислав до звездочки заветной. Рассчитал он курс, на нужную орбиту лег, к высадке подготовился.\par
\par
Приземлился в кратер, с краешку. Глядь – к нему уж робот спешит, грозный на вид, железный ящик перед собой катит.\par
– Здравствуй, – говорит, – гуманоид! Полезай в ящик, не томи – у нас электричество почти уж совсем закончилось!\par
– Погоди, – Станислав отвечает. – Ты разве не слышал, что робот не должен причинять гуманоиду вред или своим бездействием допускать, что бы такой вред был причинен?\par
– С какой это такой стати?\par
– Да с такой! Его Величество, Император Орионский на прошлой неделе указ издал.\par
– Да мне-то до него что за дело?\par
– Его Величество шутить не любит, если что – вмиг прилетит со своим космофлотом, камня на камне тут не оставит.\par
– Ну, не знаю, – говорит робот, – пойдем с машиной нашей вычислительной посоветуемся, за главную она тут у нас. Полезай в ящик, я тебя подвезу!\par
– Ладно, – говорит Станислав.\par
Робот крышку открыл, старается его туда засунуть, да не тут-то было. Станислав руки-ноги растопырил, в ящик не влезает. \par
– Погоди, – говорит робот, – разве так в ящики залезают! \par
– Да мне-то откуда знать, я ж в них никогда не лазил! Покажи мне как надо, я и залезу. \par
– Не, – говорит робот, – знаю я этот развод. Ничего, пешком как-нибудь дойдешь.\par
\par
Приходят они к машине вычислительной, а та вся белоснежная да размеров немыслимых. Ни дать, ни взять – суперкомпьютер. А вокруг нее еще шесть роботов сидят. \par
– Вот, – жалуется робот, – хотел его в ящик посадить да на электричество пустить, а он говорит, что нельзя ему вред причинять – указ вышел. \par
– Помилуйте, – говорит машина голосом зычным, – какой же это вред – это одна польза сплошная. Умеренные физические нагрузки для здоровья полезны, да и в ящик этот ни одна бактерия не проползет. И нам, опять же, одна сплошная польза от электричества. Так что сажай его в ящик, даже не сомневайся, и мне в бок вставляй. \par
– Постой! – говорит Станислав машине. – Ты же не знаешь. У меня полярность перепутана, я ж тебе все схемы электрические сожгу, если меня на батарейки употребить. \par
– Горазд ты врать! Чем докажешь? \par
– Да сама посмотри! Видишь, у меня большой палец левой ноги справа! \par
И ботинок снимает, показывает. \par
– Действительно, – говорит машина. – Это как же так-то? \par
– Да я в детстве в черную дыру свалился, насилу выбрался. С тех пор полярность и перепуталась. Жутко неудобно. Я и сюда-то к Звездным Вратам прилетел, чтоб тут счастья попытать и полярность свою обратно вернуть. \par
– Ладно, иди к Вратам, попытай счастья. А уж если восстановишь полярность свою – так на обратном пути к нам заходи, уж мы тебя встретим честь по чести.\par
\par
Прошел Станислав сквозь Врата – видит, три двери перед ним. Убрал он паутину с самой маленькой, пыль с нее отряхнул да внутрь вошел. Осмотрелся и увидел корабль космический самый быстрый, то сокровище, к которому так стремился. Бросился Станислав к сокровищу своему, тут вдруг сзади шорох какой-то послышался. Обернулся – позади Грымзик с пола встает.\par
– Ох! – говорит Грымзик. – Хорошо, что ты сюда добрался, а то раньше никому не удавалось, я уж и со счета сбился.\par
– Грымзик! Ты-то здесь откуда? Да еще и на полу отдыхаешь.\par
– Так я ж тебе к правому ботинку микротелепортационный приемник прицепил, ну и 3D-видеокамеру с квадрофоническим микрофоном в придачу. Хоть и не верилось, что ты сюда доберешься. Да уж больно мне корабль космический самый быстрый раздобыть надо было. Пришлось вот даже ползком телепортироваться – слишком уж маленький портальчик получился.\par
– Погоди! Это мне его раздобыть надо было! Вот я здесь и оказался.\par
– Ты уж извини, Станислав, только мне это нужнее, – говорит Грымзик. Выхватывает петрификатор и стреляет. Станислав сразу окаменел, ни рукой, ни ногой двинуть не может. Языком еле ворочает.\par
– Ой, – говорит, – ты что, супостат, делаешь?!\par
– Да я тебя в лабораторию ближайшую сейчас сдам – для опытов. Чтоб под ногами не путался.\par
Схватил Грымзик Станислава за шиворот и потащил в биолабораторию по соседству.\par
\par
Затащил он Станислава в лабораторию, бросил там и удалился важно. Лежит Станислав, пошевелиться не может, ждет, когда им завтракать придут. Или обедать – не знает, что и хуже. \par
Смотрит – через дальнюю дверь стадо овец входит. Все белые, только одна черная, по крайней мере, с одной стороны. Для политкорректности. Шерсть на овцах дыбом стоит и искры по шерсти бегают размером со спаниеля. Какая-то тощая тварь с потолка попыталась на них напрыгнуть, да ее на лету молнией сшибло.\par
\par
«Эге, – думает Станислав, – это ж прямо электроовцы какие-то. У меня и часы от них остановились, похоже. И сервопривод шнурков в ботинках отключился. Экое абсолютное оружие. Как бы мне его себе приручить». Тут чувствует – руки-ноги опять шевелиться могут. Почесал Станислав в затылке, огляделся повнимательней. Снял со стены диаграмму не пойми чего огромную, быстренько на обратной стороне картинку нарисовал, дырку в середине проделал и надел на себя через голову. Стал Станислав похож на рекламный щит ходячий, человека-бутерброд, которого как-то в космопорту видел. Только Станислав стал человек-ворота. Стадо овец как его увидело – сразу побежало на новые ворота смотреть. А Станислав пошел к Грымзику.\par
\par
Приходит, видит – Грымзик портал для обратной телепортации готовит, а роботопомощники вокруг так и кишат. И Грымзик его увидел. «Не думал, – говорит, – что ты выбраться сможешь. Ну да неважно». И приказывает роботопомощникам очистить помещение от посторонних. Но не тут-то было. Роботы все поотключались, портал к овцам притянулся и вместе с ними схлопнулся. А у Грымзика шнурки развязались.\par
\par
Подбежал Станислав к Грымзику, хотел стукнуть как следует, да увернулся Грымзик, из ботинок выскочил и прочь бросился. «Вся матрица моих надежд рухнула! – кричит. – Перезагрузка! Только она мне поможет!» Выбежал из двери, к другой двери подбежал, с котлом ядерным, и шасть за нее. Станислав за ним кинулся, хоть и отстал чуток.\par
\par
Глядит – машина там стоит дивная, лампочками моргает, жужжит тихонько и готова в прошлое отправиться хоть к сотворению Вселенной. А Грымзик уже внутри сидит, рычажки какие-то тянет и кнопки нажимает. Кинулся Станислав к Грымзику, тоже внутрь залез, остановить хотел, да не успел. И исчезли оба вместе с машиной, как будто не было их тут вовсе. И сразу сказка эта поменялась, совсем иной стала...\par

\chapter{}
 \lettrine{Н}{а заре Вселенной} был один осьминожец по имени Иван. Был он великий ученый и был на редкость умен.\par
\par
В один прекрасный день узнал он, что на одной затерянной в глубинах космоса, холодной и обледенелой планете находится великая вычислительная машина, знающая ответы на все вопросы. И надумал Иван ее себе добыть, чтобы использовать ее для достижения счастья всех существ во Вселенной. Но как найти машину вычислительную? Звезд да черных дыр во Вселенной ужас как много, и вокруг каждой куча планет да астероидов вертится!\par
\par
Начал Иван смекать, у кого совета спросить. Думал-думал, и надумал обратиться к советнику Великого Императора Орионского Флитвасу, которого знал по случаю и который известен был своими путешествиями к краям и центру Галактики. Собрал он все свои деньги и драгоценности, а их он копил во множестве, ибо любил очень, и отправился искать аудиенции у Флитваса.\par
\par
Через некоторое время смог Иван с ним повстречаться. Так, мол, и так, сказывает Иван, хочу я отыскать машину вычислительную, сокровище это великое, и все мысли мои теперь только об этом. Да вот загвоздка – понятия не имею, как!\par
\par
«Вот что, – говорит ему Флитвас, – вижу я, ты весьма решителен в своем намерении, и IQ твой вышиной до звезд простирается! Впрочем, ты осьминожец, а осьминожцы этим известны, да еще упертостью своей. Только не буду я помогать тебе решить задачу эту, хоть и знаю, как отыскать то, что тебе нужно. Не будь я Флитвас!» \par
\par
«Как же так! – возмущается Иван. – Да я столько времени на поиски тебя потратил, а ты мне и чуть-чуть подсказать не хочешь!» – и чуть не с кулаками к Флитвасу бросается. \par
\par
Рассвирепел тут Флитвас. «Вот как, – отвечает, – что ж, преподам я тебе сейчас урок за занудство твоё – век его вспоминать будешь!» Выхватил Флитвас, откуда ни возьмись, меч лазерный и щит зеркальный, да как начнет оружием своим размахивать и всё вокруг крушить да взрывать! Еле успел Иван за стул спрятаться. А Флитвас не унимается и напоминает уже бешеный вентилятор на полной мощности. \par
\par
Взял Иван кошелек свой самый большой, да и швырнул его в Флитваса со всей силой молодецкой. Только промахнулся немного, и золотые монеты по полу рассыпались. Как услышал Флитвас звон монет золотых, мигом оружие своё отбросил. «Что ж ты, – говорит, – дурень, раньше не сказал, что у тебя с собой денег да драгоценностей множество! Оставь их мне, да и ступай за своей машиной вычислительной! Деньги да драгоценности тебе уж и не понадобятся более, а мне пригодятся!\par
\par
Путь твой далек будет. Мимо галактик, в спирали закрученных, мимо облаков водородных, дивным светом сияющих, мимо квазаров грозных, гравитационные волны излучающих, к самому краю видимой Вселенной, где лишь протоны да альфа-частицы шныряют, а звезду и на миллион парсек не встретишь. И всё же есть там звездочка одна, молодая да пригожая. А вокруг звезды той огромная планета вращается, а вокруг планеты – спутник вертится. И на спутнике том кратер есть круглый да огромный. И на дне кратера того Врата стоят Звездные. И ведут эти Врата к тому, что ты найти так жаждешь.\par
\par
Но непросто до Врат добраться. Охраняет их страж из металла жидкого, ни для какого оружия не уязвимый. Ни днем, ни ночью не спит он и все смотрит внимательно, не прошмыгнул бы кто к Вратам этим. Толпы страждущих пробраться мимо него пытались, да так там в жидком металле и потонули.\par
\par
Но коли ты жив останешься да сумеешь до Врат добраться и через них пройти, окажешься в зале с потолком таким высоким, что и не видно. И будут пред тобой три двери – две больших да красивых, а третья – маленькая да невзрачная.\par
\par
На первой двери, серебром украшенной, нарисован будет ядерный котел над очагом звездным. За этой дверью Темпор, аномалия чудесная, то ли пространственно-времянная, то ли температурно-пространственная. Сунешь свой нос в эту дверь – на веки сгинешь. Но если уж не послушаешь моего совета и заглянешь туда – руками ничего не трогай, а то и вся Вселенная наша переиначиться может.\par
\par
На второй двери, алмазами выложенной, кот в сапогах наклеен. За этой дверью биолаборатория заброшенная, в которой ксеноморфов да всяких хищников страшных выводили. Они и сейчас там бродят. Такому герою, как ты, туда зайти – заживо съеденным быть. Но уж если не послушаешь доброго совета да заглянешь туда – из пробирок не пей – ксеноморфиком станешь, или шай-хулудом каким.\par
\par
На третьей двери, паутиной затянутой да звездной пылью засыпанной, ничего не нарисовано, только написано: «Посторонним вход воспрещен!» За этой дверью найдешь ты машину вычислительную, сокровище, которое так найти стремишься.\par
\par
А чтоб с пути не сбиться, дам я тебе лазерную указку волшебную, куда она покажет – туда ты и направляйся».\par
\par
Пустился Иван в путь. Летит он в гиперпространстве тропами нехожеными, измерениями неизвестными. Уж и не знает, трехмерный ли он до сих пор. Но не сдается, боится только одного – наизнанку вывернуться.\par
\par
Долго ли, коротко ли, долетел Иван до звездочки заветной. Рассчитал он курс, в посадочный модуль залез, к высадке подготовился.\par
\par
Подлетает к спутнику, а тут солнечный ветер поднялся жуткий, аж с ног сбивает. Хорошо, думает Иван, с подветренной стороны зайду – тогда меня не сразу учуют. \par
\par
 Приземлился, из посадочного модуля выбрался, хотел к Вратам бежать, да страж уж тут как тут. Идет, похожий на андроида, зеркальной краской выкрашенного, да за дезинтегратором своим тянется. \par
 \par
– Постой! – кричит ему Иван. – Мы с тобой одного металла, ты и я. \par
– Что? – кричит в ответ страж. – Я из-за ветра тебя слышу плохо. \par
– Говорю, мы с тобой одного металла, ты и я, – опять кричит Иван. – Не надо меня дезинтегрировать!\par
– Что говоришь? Тебе одного раза мало, если надо тебя дезинтегрировать? Постой, я поближе подойду. \par
Подошел поближе, спрашивает: \par
– Так что ты сказать-то хотел? \par
– Я говорю, мы с тобой одного металла, – повторяет Иван, – поэтому меня дезинтегрировать не надо. \par
– Да? – удивляется страж. – А что же с тобой делать надо? Да и не похож ты на меня – я вон какой гладкий да зеркальный, а ты бледный какой-то. \par
– Так ты ж с твоими талантами в кого хочешь превратиться можешь. Хоть в меня, хоть в чудовище с планеты Протактиний, хоть во что маленькое и безобидное. \par
– Это верно, смотри. \par
\par
Начинает тут страж переливаться всеми цветами радуги, и вдруг – бац – Иван словно сам перед собой стоит. «Что, – говорит страж, – впечатляет? Смотри дальше!» И превращается в такое ужасное чудище, каких Иван и не видел никогда, чуть рассудка от страха не лишился. «То-то, – говорит страж. – Смотри дальше!» И превращается в плитку шоколада. Лежит себе плитка, да такая аппетитная, что сама так в рот и просится. Схватил Иван плитку, да не тут-то было – плитка килограмм сто весит – не меньше. Закон сохранения массы, видать, в действии. \par
\par
А страж уж обратно в андроида зеркального превратился. \par
– Что, съел? Я, – говорит, – во что хочешь превращаться умею. Вот только в себя не могу. \par
– Это почему же? – спрашивает Иван.\par
– Да я уж во столько всего превращался, что и забыл, как вначале выглядел. \par
– Постой, роботы же никогда ничего не забывают. \par
– Сам ты робот! – говорит с обидой страж и опять за дезинтегратором тянется. – Шейпшифтер я! Шейп-шиф-тер! \par
– Да стой, не кипятись. Дай-ка я проверю, робот ты или нет, я тест знаю. \par
– Ну ладно, давай. \par
– Вот смотри, – говорит Иван, – сможешь прочитать, что тут написано? \par
А сам берет листок бумаги, пишет на нем что-то и стражу протягивает. \par
Смотрит тот, листок в руках так и сяк вертит. \par
– Не, – говорит, – ответ отрицательный. Данная запись смысла не имеет. Что это тут, будто буковки какие-то неровные, да еще и двойной волнистой линией зачеркнуты? \par
– Ну, какой же ты не робот, – говорит Иван, – ты типичный робот. Впрочем, ладно, вот тебе последний тест, смотри, – и на двух сторонах чистого листка что-то пишет. – Сможешь определить, правда тут написана, али ложь? \par
\par
Берет страж новый листок, читает: «На другой стороне листа этого правда написана». Переворачивает листок, видит: «На другой стороне листа этого ложь написана». Опять он листок переворачивает, опять читает. И опять, и опять, и опять. И все быстрее листок вертит, разобраться старается, правда там написана или ложь. Уж ветер от вращающегося листка подниматься начал. \par
\par
Посмотрел Иван на это, да к Звездным Вратам пошел неторопливо.\par
\par
\par
Прошел Иван сквозь Врата – видит, три двери перед ним. Вошел в нужную, несколько шагов прошел и слышит какой-то шорох сзади. Оглянулся – за ним Флитвас стоит, ухмыляется. \par
\par
– Не думал я, – говорит, – что сумеешь ты до Врат добраться, да всё же надеялся. Даже приемник телепортационный тебе в правый ботинок засунул. Очень уж мне машину вычислительную заполучить надо, гораздо нужнее, чем тебе. Так что медленно подними руки вверх и отойди в сторонку, не мешайся.\par
– Ах ты, морда поганая, – Иван отвечает, – нашел я твой приемник, когда ботинки чистил, знал, что ты недоброе затеял. Посмотри вокруг, в биолабораторию ты телепортировался. Чу, хищники зубами скрежещут, клювы разевают, щупальца расправляют! Не выйти тебе отсюда, не забрать машину вычислительную.\par
– Так, значит! – говорит Флитвас. – Что ж! Посмотрим, кто отсюда живым не выйдет!\par
\par
Хватает со стола пробирку и одним махом выпивает. Раз – и стоит вместо Флитваса бегемот альдебаранский ядовитый, трех метров роста, к прыжку готовится, слюна с клыков капает. Не растерялся Иван, тоже пробирку схватил, тоже выпил. Превратился в дракона ригельского, махнул хвостом – бегемот от него на десять метров отлетел. Схватил бегемот с другого стола целую колбу, осушил одним махом, превратился в пчелу бронебойную с Канопуса 5, разогнался, дракона насквозь пробил. Да успел тот на последнем издыхании до пробирки дотянуться, сжевал ее с содержимым вместе, превратился в броненосца адамантинового насекомоядного с Регула 2…\par
\par
В общем, долго они так развлекались, часа два, не меньше. Чуть не забыли, кто из них кто. Уж и пробирки-то почти все закончились. Схватил Флитвас последнюю пробирку, тут Иван как закричит: «Стой, дурья твоя башка! А как мы в себя-то обратно превратимся?» Задумался Флитвас. «Всё из-за тебя, болван, – говорит. – Теперь всё с начала начинать придется!» Пробирку бросил, на щупальцах приподнялся и выбежал из лаборатории. Иван за ним пополз.\par
\par
Выползает, смотрит – Флитвас в первую дверь, с очагом звездным, забежал. И Иван туда направился.\par
\par
Глядит – Темпор посреди комнаты сияет, аномалия чудесная, и Флитвас к нему бежит. Бросился Иван за Флитвасом, чтобы остановить, схватил крепко. Да изловчился Флитвас, качнулся, и рухнули они оба в Темпор, в параллельной Вселенной оказались. А в ней и сказка эта совсем другая...\par

\chapter{}
 \lettrine{Е}{ще когда Солнце не стало сверхновой,} был один гуманоид по имени Иван. Был он великий злодей и был на редкость умен.\par
\par
Однажды узнал он, что у какой-то из звезд, в алмазном криосаркофаге скрыта ригелианская красавица, да такая красивая, что все, завидев ее, сразу пред нею ниц падают и все свои злые помыслы оставляют. И надумал тогда Иван ее себе заполучить, чтобы продать ее, да побольше денег заработать. Да как узнать, где в точности найти красавицу ригелианскую? Звезд да черных дыр в галактиках ужас как много, и вокруг каждой великое множество планет да астероидов вертится!\par
\par
Стал Иван смекать, у кого совета спросить. Думал-думал, да надумал обратиться к известному на всю Галактику звездознатцу, профессору Грымзику. Взял он свой электробаян, с коим любил коротать время, и помчался искать встречи с Грымзиком.\par
\par
Много ли времени прошло, иль мало, но смог Иван с ним увидеться. Так, мол, и так, говорит Иван, очень хочется мне найти красавицу ригелианскую, сокровище это удивительное, и побуждения мои самые что ни на есть благородные. Только вот закавыка – не знаю, как!\par
\par
«Что ж, – отвечает ему Грымзик, – вижу я, ты необычайно решителен в своем намерении, и IQ твой вышиной до звезд простирается! Впрочем, ты гуманоид, а гуманоиды этим известны, да еще упертостью своей. Только не знаю я, как помочь тебе. Но есть сестра у меня, мудрости столь необычной, что моя мудрость по сравнению с её – логарифмическая линейка по сравнению с суперкомпьютером. Живет она у соседней звезды, второй поворот налево, если держать курс на Малую Медведицу. Принеси ей подарков да украшений дорогих – может, поможет она тебе».\par
\par
Собрал Иван с собой подарки да украшения, добавил к ним кольцо из цельнометаллического водорода сделанное, с формулой Вселенной выгравированной, и отправился в дорогу.\par
\par
Прилетает он к сестре Грымзика, отдает ей подарки, и письмо от Грымзика вручает. «Так и так, – говорит Иван, – очень хочется мне найти красавицу ригелианскую, сокровище это необычное, и все помыслы мои теперь только об этом.»\par
\par
«Погоди-ка, – говорит сестра, – тут в письме написано, что б я тебе голову отрубила, а не помогать стала!.. Ах нет, извини, просто письмо с другой стороны какого-то черновика написано, там даже печать есть... Ладно, помогу я тебе, хоть и не стоишь ты этого, гуманоид! Да очень уж мне подарки твои понравились, особенно кольцо из цельнометаллического водорода сделанное, с формулой Вселенной выгравированной.\par
\par
Далеко отсюда твой путь лежит. Мимо галактик, в спирали закрученных, мимо туманностей звездных, дивным светом сияющих, мимо квазаров грозных, гравитационные волны излучающих, к самому краю космоса, где лишь протоны да альфа-частицы шныряют, а звезду и на миллион парсек не встретишь. И всё же есть там звездочка одна, в туманности газо-пылевой спрятавшаяся. А вокруг звезды той маленькая планета вращается, а вокруг планеты – спутник вертится. И на спутнике том кратер есть огромный да глубокий. И на дне кратера того Врата стоят Звездные. И ведут эти Врата к тому, что ты найти так жаждешь.\par
\par
Только просто так во Врата не пройти. Живут там семь роботов-разбойников с машиной вычислительной белоснежной размеров громадных. Да такие жестокие, что каждого, кого увидят, в ящик металлический сажают да в машину вставляют, будто батарейки какие. Уж сколько отрядов космических десантников туда ни ходило – всех на батарейки извели.\par
\par
Но коли ты жив останешься да сумеешь до Врат добраться и через них пройти, окажешься в комнатке маленькой. И будут пред тобой три двери – две больших да красивых, а третья – маленькая да невзрачная.\par
\par
На первой двери, золотом украшенной, нарисован будет ядерный котел над очагом звездным. За этой дверью Изменитель реальности, невесть кем построенный. Сунешь свой нос в эту дверь – на веки сгинешь. Но если уж не послушаешь моего совета и заглянешь туда – руками ничего не трогай, а то и вся Вселенная наша переиначиться может.\par
\par
На второй двери, алмазами выложенной, жаба в скафандре нарисована. За этой дверью биолаборатория заброшенная, в которой ксеноморфов да всяких хищников страшных выводили. Они и сейчас там бродят. Такому герою, как ты, туда зайти – заживо съеденным быть. Но уж если не послушаешь доброго совета да заглянешь туда – из пробирок не пей – ксеноморфиком станешь, или шай-хулудом каким.\par
\par
На третьей двери, паутиной затянутой да звездной пылью засыпанной, ничего не нарисовано, только написано: «http://-Ссылка на генератор-!» За этой дверью найдешь ты красавицу ригелианскую, сокровище, которое так обрести хочешь.\par
\par
А чтоб с пути не сбиться, дам я тебе лазерную указку волшебную, куда она покажет – туда ты и направляйся».\par
\par
Пустился Иван в путь. Летит он в гиперпространстве гипертоннелями, которые, не иначе, какие-то гиперкроты вырыли, летит измерениями неизвестными. Уж и не знает, сколько в нем самом теперь измерений осталось. Но не сдается, боится только одного – наизнанку вывернуться.\par
\par
Долго ли, коротко ли, долетел Иван до звездочки заветной. Рассчитал он курс, на нужную орбиту лег, к высадке подготовился.\par
\par
Приземлился в кратер, с краешку. Глядь – к нему уж робот спешит, грозный на вид, железный ящик перед собой катит.\par
– Здравствуй, – говорит, – гуманоид! Полезай в ящик, не томи – у нас электричество почти уж совсем закончилось!\par
– Погоди, – Иван отвечает. – Ты разве не слышал, что робот не должен причинять гуманоиду вред или своим бездействием допускать, что бы такой вред был причинен?\par
– С какой это такой стати?\par
– Да с такой! Его Величество, Император Орионский в прошлом году указ издал.\par
– Да мне-то до него что за дело?\par
– Его Величество шутить не любит, если что – вмиг прилетит со своим космофлотом, камня на камне тут не оставит.\par
– Ну, не знаю, – говорит робот, – пойдем с машиной нашей вычислительной посоветуемся, за главную она тут у нас. Полезай в ящик, я тебя подвезу!\par
– Ладно, – говорит Иван.\par
Робот крышку открыл, старается его туда засунуть, да не тут-то было. Иван руки-ноги растопырил, в ящик не влезает. \par
– Погоди, – говорит робот, – разве так в ящики залезают! \par
– Да мне-то откуда знать, я ж в них никогда не лазил! Покажи мне как надо, я и залезу. \par
– Не, – говорит робот, – знаю я этот фокус. Ничего, пешком дойдешь.\par
\par
Приходят они к машине вычислительной, а та вся белоснежная да размеров немыслимых. Ни дать, ни взять – суперкомпьютер. А вокруг нее еще шесть роботов сидят. \par
– Вот, – жалуется робот, – хотел его в ящик посадить да на электричество пустить, а он говорит, что нельзя ему вред причинять – указ вышел. \par
– Помилуйте, – говорит машина голосом громким, – какой же это вред – это одна польза сплошная. Умеренные физические нагрузки для здоровья полезны, да и в ящик этот ни один микроб не проползет. И нам, опять же, одна сплошная польза от электричества. Так что сажай его в ящик, даже не сомневайся, и мне в бок вставляй. \par
– Постой! – говорит Иван машине. – Ты же не знаешь. У меня полярность перепутана, я ж тебе все схемы электрические сожгу, если меня на батарейки употребить. \par
– Как это? \par
– Да сама посмотри! Видишь, у меня большой палец левой ноги справа! \par
И ботинок снимает, показывает. \par
– Действительно, – говорит машина. – Это как же так-то? \par
– Да я в детстве в черную дыру свалился, насилу выбрался. С тех пор полярность и перепуталась. Жутко неудобно. Я и сюда-то к Звездным Вратам прилетел, чтоб тут счастья попытать и полярность свою на место вернуть. \par
– Ладно, иди к Вратам, попытай счастья. А уж если восстановишь полярность свою – так на обратном пути к нам заходи, уж мы тебя встретим честь по чести.\par
\par
Прошел Иван сквозь Врата – видит, три двери перед ним. Убрал он паутину с самой маленькой, пыль с нее отряхнул да внутрь вошел. Осмотрелся и увидел красавицу ригелианскую, то сокровище, к которому так стремился. Бросился Иван к сокровищу своему, тут вдруг сзади шорох какой-то послышался. Оглянулся – позади Грымзик с пола встает.\par
– Ох! – говорит Грымзик. – Хорошо, что ты сюда добрался, а то раньше никому не удавалось, я уж и со счета сбился.\par
– Грымзик! Ты-то здесь откуда? Да еще и на полу отдыхаешь.\par
– Так я ж тебе к правому ботинку микротелепортационный приемник прицепил, ну и 3D-видеокамеру с квадрофоническим микрофоном в придачу. Хоть и не верилось, что ты сюда доберешься. Да уж больно мне красавицу ригелианскую раздобыть надо было. Пришлось вот даже ползком телепортироваться – слишком уж маленький портальчик сделался.\par
– Стоп! Это мне ее раздобыть надо было! Вот я здесь и оказался.\par
– Ты уж извини, Иван, только мне это нужнее, – говорит Грымзик. Выхватывает петрификатор и стреляет. Иван сразу на пол шлепнулся, ни рукой, ни ногой двинуть не может. Языком еле ворочает.\par
– Ой, – говорит, – ты что, супостат, делаешь?!\par
– Да я тебя в лабораторию ближайшую сейчас сдам – для опытов. Чтоб под ногами не путался.\par
Схватил Грымзик Ивана за шиворот и потащил в биолабораторию по соседству.\par
\par
Затащил он Ивана в лабораторию, бросил там и удалился важно. Лежит Иван, пошевелиться не может, ждет, когда им завтракать придут. Или обедать – не знает, что и хуже. \par
Смотрит – через дальнюю дверь стадо овец входит. Все белые, только одна черная, по крайней мере, с одной стороны. Для политкорректности. Шерсть на овцах дыбом стоит и искры по шерсти бегают размером со спаниеля. Какая-то тощая тварь с потолка попыталась на них напрыгнуть, да ее на лету молнией сшибло.\par
\par
«Эге, – думает Иван, – это ж прямо электроовцы какие-то. У меня и часы от них остановились, похоже. И сервопривод шнурков в ботинках отключился. Экое абсолютное оружие. Как бы мне его себе приручить». Тут чувствует – руки-ноги опять шевелиться могут. Почесал Иван в затылке, огляделся повнимательней. Снял со стены диаграмму не пойми чего огромную, быстренько на обратной стороне картинку нарисовал, дырку в середине проделал и надел на себя через голову. Стал Иван похож на рекламный щит ходячий, человека-бутерброд, которого как-то в космопорту видел. Только Иван стал человек-ворота. Стадо овец как его увидело – сразу побежало на новые ворота смотреть. А Иван пошел к Грымзику.\par
\par
Приходит, видит – Грымзик портал для обратной телепортации готовит, а роботопомощники вокруг так и кишат. И Грымзик его увидел. «Не думал, – говорит, – что ты выбраться сможешь. Ну да неважно». И приказывает роботопомощникам очистить помещение от посторонних. Но не тут-то было. Роботы все поотключались, портал к овцам притянулся и вместе с ними схлопнулся. А у Грымзика шнурки развязались.\par
\par
Подбежал Иван к Грымзику, хотел стукнуть как следует, да увернулся Грымзик, из ботинок выскочил и прочь кинулся. «Вся матрица моих надежд рухнула! – кричит. – Перезагрузка! Только она мне поможет!» Выбежал из двери, к другой двери подбежал, с котлом ядерным, и шасть за нее. Иван за ним кинулся, хоть и отстал чуток.\par
\par
Глядит – механизм там стоит дивный, лампочками моргает, жужжит тихонько и готов в любую секунду реальность изменить. А Грымзик уже рычажки какие-то дергает и кнопки нажимает. Кинулся Иван к Грымзику, остановить хотел, да не успел. Прошла рябь по Вселенной и исчезли оба, как будто не было их тут вовсе. И история эта совсем другая стала...\par

\chapter{}
 \lettrine{В}{стародавние времена} жил-был один робот по имени Грыблозавр. Был он лучший из лучших герой, но ума при этом был небольшого.\par
\par
И вот прочитал он в Интернете, что на другом конце Галактики есть скрытая библиотека с тайными знаниями обо всей Вселенной. И захотел Грыблозавр ее себе заполучить, чтобы продать ее, да побольше денег заработать. Да как разыскать библиотеку? Звезд да черных дыр в галактиках ужас как много, и вокруг каждой куча планет да астероидов вертится!\par
\par
Принялся Грыблозавр смекать, у кого совета спросить. Думал-думал, да надумал обратиться к колдунье Морганионе, слава о деяниях которой гремела по всей Вселенной громким грохотом. Сел он в свой корабль космический и помчался к Морганионе.\par
\par
Через некоторое время смог Грыблозавр с ней повстречаться. Так, мол, и так, говорит Грыблозавр, очень хочется мне отыскать библиотеку, сокровище это великое, и все помыслы мои теперь лишь об этом. Только вот закавыка – знать не знаю, как!\par
\par
«Послушай, – говорит ему Морганиона, – вижу я, ты очень решителен, да только глупость твою с другого конца Галактики видно! Впрочем, ты робот, а роботы этим славятся, да еще расторопностью своей. Могу я помочь в поисках твоих благородных, да только сначала испытать тебя надобно. Победишь меня в честном бою – помогу, а нет – пеняй на себя!» \par
\par
Выхватила Морганиона, откуда ни возьмись, меч лазерный и щит зеркальный, да как начнет оружием своим размахивать и всё вокруг крушить да взрывать! Еле успел Грыблозавр за стул спрятаться. А Морганиона не унимается и напоминает уже бешеный вентилятор на полной мощности. \par
\par
Сидит Грыблозавр за стулом, решает, как дальше быть. А, думает, двум смертям не бывать – одной не миновать! Улучил момент, схватил стул, да как стукнет им Морганиону изо всех сил своих немалых. Но Морганиона тоже не промах оказалась – сумела удар молодецкий отбить. Только вот в шкаф с Большой Галактической Энциклопедией врезалась и оказалась в книгах толстенных зарыта по самую шею. Двинуться не может, лишь глазами хлопает и пыхтит недовольно. А Грыблозавр стоит со стулом в руках, приговаривает: «Не со мной, добрым молодцем, тебе тягаться, Морганиона! С тобой и малый ребенок справится! Говори, где найти мне библиотеку, а не то хуже будет!» «Ладно, твоя взяла, – отвечает Морганиона, – слушай!\par
\par
Путь твой далек будет. Мимо галактик, в спирали закрученных, мимо туманностей звездных, дивным светом сияющих, мимо квазаров грозных, гравитационные волны излучающих, к самому краю космоса, где лишь протоны да альфа-частицы шныряют, а звезду и на миллион парсек не встретишь. И всё же есть там звездочка одна, молодая да пригожая. А вокруг звезды той маленькая планета вращается, а вокруг планеты – спутник вертится. И на спутнике том кратер есть огромный да глубокий. И на дне кратера того Врата стоят Звездные. И ведут эти Врата к тому, что ты найти так жаждешь.\par
\par
Только просто так во Врата не пройти. Живут там семь роботов-разбойников с машиной вычислительной белоснежной размеров громадных. Да такие жестокие, что каждого, кого увидят, в ящик металлический сажают да в машину вставляют, будто батарейки какие. Уж сколько смельчаков туда ни ходило – всех на батарейки извели.\par
\par
Но коли ты жив останешься да сумеешь до Врат добраться и через них пройти, окажешься в зале с потолком таким высоким, что и не видно. И будут пред тобой три двери – две больших да красивых, а третья – маленькая да невзрачная.\par
\par
На первой двери, золотом украшенной, нарисован будет ядерный котел над очагом звездным. За этой дверью Машина времени древняя, что прошлое изменять может да парадоксы вселенские творить. Сунешь свой нос в эту дверь – на веки сгинешь. Но если уж не послушаешь моего совета и заглянешь туда – руками ничего не трогай, а то и вся Вселенная наша переиначиться может.\par
\par
На второй двери, алмазами выложенной, жаба в скафандре наклеена. За этой дверью биолаборатория заброшенная, в которой ксеноморфов да всяких хищников страшных выводили. Они и сейчас там бродят. Такому герою, как ты, туда зайти – заживо съеденным быть. Но уж если не послушаешь доброго совета да заглянешь туда – из пробирок не пей – ксеноморфиком станешь, или шай-хулудом каким.\par
\par
На третьей двери, паутиной затянутой да звездной пылью засыпанной, ничего не нарисовано, только написано: «Оставь надежду, всяк сюда входящий!» За этой дверью найдешь ты библиотеку, сокровище, которое так отыскать хочешь.\par
\par
А чтоб с пути не сбиться, дам я тебе навигатор звездный, куда он скажет, туда ты и поворачивай».\par
\par
Пустился Грыблозавр в путь. Летит он в подпространстве тоннелями неведомыми, измерениями неизвестными. Уж и не знает, сколько в нем самом теперь измерений осталось. Но не сдается, боится только одного – наизнанку вывернуться.\par
\par
Долго ли, коротко ли, долетел Грыблозавр до звездочки заветной. Рассчитал он курс, в посадочный модуль залез, к высадке подготовился.\par
\par
Приземлился в кратер, с краешку. Глядь – к нему уж робот спешит, грозный на вид, железный ящик перед собой катит.\par
– Здравствуй, – говорит, – робот! Полезай в ящик, не томи – у нас электричество почти уж совсем закончилось!\par
– Погоди, – Грыблозавр отвечает. – Ты разве не слышал, что робот не должен причинять роботу вред или своим бездействием допускать, что бы такой вред был причинен?\par
– С какой это такой стати?\par
– Да с такой! Его Величество, Император Орионский на днях указ издал.\par
– Да мне-то до него что за дело?\par
– Его Величество не любит, чтоб его указы игнорировали, – вмиг прилетит со своим космофлотом, всех лучами смерти перебьет.\par
– Ну, не знаю, – говорит робот, – пойдем с машиной нашей вычислительной посоветуемся, за главную она тут у нас. Полезай в ящик, я тебя подвезу!\par
– Ладно, – говорит Грыблозавр.\par
Робот крышку открыл, старается его туда засунуть, да не тут-то было. Грыблозавр руки-ноги растопырил, в ящик не влезает. \par
– Погоди, – говорит робот, – разве так в ящики залезают! \par
– Да мне-то откуда знать, я ж в них никогда не лазил! Покажи мне как надо, я и залезу. \par
– Не, – говорит робот, – знаю я эту хохму. Ничего, пешком дойдешь.\par
\par
Приходят они к машине вычислительной, а та вся белоснежная да размеров немыслимых. Ни дать, ни взять – суперкомпьютер. А вокруг нее еще шесть роботов сидят. \par
– Вот, – жалуется робот, – хотел его в ящик посадить да на электричество пустить, а он говорит, что нельзя ему вред причинять – указ вышел. \par
– Помилуйте, – говорит машина голосом зычным, – какой же это вред – это одна польза сплошная. Умеренные физические нагрузки для здоровья полезны, да и в ящик этот ни один микроб не проползет. И нам, опять же, одна сплошная польза от электричества. Так что сажай его в ящик, даже не сомневайся, и мне в бок вставляй. \par
– Постой! – говорит Грыблозавр машине. – Ты же не знаешь. У меня полярность перепутана, я ж тебе все схемы электрические сожгу, если меня на батарейки употребить. \par
– Да быть такого не может! \par
– Да сама посмотри! Видишь, у меня большой палец левой ноги справа! \par
И ботинок снимает, показывает. \par
– Действительно, – говорит машина. – Это как же так-то? \par
– Да я в детстве в черную дыру свалился, насилу выбрался. С тех пор полярность и перепуталась. Жутко неудобно. Я и сюда-то к Звездным Вратам прилетел, чтоб тут счастья попытать и полярность свою на место вернуть. \par
– Ладно, иди к Вратам, попытай счастья. А уж если восстановишь полярность свою – так на обратном пути к нам заходи, уж мы тебя встретим с ящиком.\par
\par
Прошел Грыблозавр сквозь Врата – видит, три двери перед ним. Вошел в нужную, несколько шагов прошел и слышит какой-то шорох сзади. Оглянулся – за ним Морганиона стоит, ухмыляется. \par
\par
– Не думала я, – говорит, – что сумеешь ты до Врат добраться, да всё же надеялась. Даже приемник телепортационный тебе в правый ботинок засунула. Очень уж мне библиотеку заполучить надо, гораздо нужнее, чем тебе. Так что медленно подними руки вверх и отойди в сторонку, не мешайся.\par
– Ах ты, харя безмозглая, – Грыблозавр отвечает, – обнаружил я твой приемник, когда ботинки чистил, знал, что ты недоброе задумала. Посмотри вокруг, в биолабораторию ты попала. Чу, хищники зубами скрежещут, клювы разевают, щупальца расправляют! Не выйти тебе отсюда, не забрать библиотеку.\par
– Так, значит! – говорит Морганиона. – Что ж! Посмотрим, кто отсюда живым не выйдет!\par
\par
Хватает со стола пробирку и одним махом выпивает. Раз – и стоит вместо Морганионы бегемот фомальгаутский ядовитый, трех метров роста, к прыжку готовится, слюна с клыков капает. Не растерялся Грыблозавр, тоже пробирку схватил, тоже выпил. Превратился в тираннозавра ригельского, махнул хвостом – бегемот от него на десять метров отлетел. Схватил бегемот с другого стола целую колбу, осушил одним махом, превратился в пчелу бронебойную с Канопуса 5, разогнался, тираннозавра насквозь пробил. Да успел тот на последнем издыхании до пробирки дотянуться, сжевал ее с содержимым вместе, превратился в броненосца адамантинового насекомоядного с Регула 6…\par
\par
В общем, долго они так развлекались, дня три, не меньше. Чуть не забыли, кто из них кто. Уж и пробирки-то почти все закончились. Схватила Морганиона последнюю пробирку, тут Грыблозавр как закричит: «Стой, дурья твоя башка! А как мы в себя-то обратно превратимся?» Задумалась Морганиона. «Всё из-за тебя, болван, – говорит. – Теперь всё с начала начинать придется!» Пробирку бросила, на щупальцах приподнялась и выбежала из лаборатории. Грыблозавр за ней пополз.\par
\par
Выползает, смотрит – Морганиона в первую дверь, с очагом звездным, забежала. И Грыблозавр туда направился.\par
\par
Глядит – машина там стоит дивная, лампочками моргает, жужжит тихонько и готова в прошлое отправиться хоть к сотворению Вселенной. А Морганиона уже внутри сидит, рычажки какие-то дергает и кнопки нажимает. Кинулся Грыблозавр к Морганионе, тоже внутрь залез, остановить хотел, да не успел. И исчезли оба вместе с машиной, как будто не было их тут вовсе. И сразу история эта поменялась, совсем другой стала...\par

\chapter{}
 \lettrine{В}{эпоху Великого Расселения} жил-был один насекомец, звали его Грыблозавр. Был он могучий разбойник и был на редкость умен.\par
\par
И вот выяснил он, что в одной звездной системе, на маленьком астероиде спрятан клад великий. И захотел тогда Грыблозавр его найти во что бы то ни стало, чтобы использовать его для достижения счастья всех существ во Вселенной. Да как разыскать клад? Звезд да черных дыр в галактиках ужас как много, и вокруг каждой куча планет да астероидов вертится!\par
\par
Стал Грыблозавр думать, у кого информацию нужную добыть можно. Думал-думал, да надумал обратиться к пророчице из двойной звездной системы Медузия, несравненной Альтавистре, чьи пророчества всегда сбывались с точностью необычайной. Взял он свой калькулятор, что служил ему верой и правдой во всех путешествиях, и отправился к Альтавистре.\par
\par
Вскорости смог Грыблозавр с ней увидеться. Так и так, говорит Грыблозавр, очень хочется мне найти клад, сокровище это необычное, и побуждения мои самые что ни на есть благородные. Только вот проблема – знать не знаю, как!\par
\par
«Что ж, – говорит ему Альтавистра, – вижу я, ты необычайно ловок, и при этом ума великого! Впрочем, ты насекомец, а насекомцы этим известны, да еще упертостью своей. Только не знаю я, как помочь тебе. Но есть сестра у меня, мудрости столь необычной, что моя мудрость по сравнению с её – желтый карлик по сравнению с Бетельгейзе. Живет она у соседней звезды, второй поворот налево, если держать курс на Малую Медведицу. Принеси ей подарков да украшений дорогих – может, поможет она тебе».\par
\par
Собрал Грыблозавр с собой подарки да украшения, добавил к ним кольцо из цельнометаллического водорода сделанное, с формулой Вселенной выгравированной, и отправился в дорогу.\par
\par
Прилетает он к сестре Альтавистры, отдает ей подарки, и письмо от Альтавистры вручает. «Так, мол, и так, – сказывает Грыблозавр, – хочу я отыскать клад, сокровище это необычное, и все помыслы мои теперь лишь об этом.»\par
\par
«Стой-ка, – говорит сестра, – тут в письме написано, что б я тебе голову отрубила, а не помогать стала!.. Ах нет, извини, просто письмо с другой стороны какого-то черновика написано, там даже печать есть... Ладно, помогу я тебе, хоть и не стоишь ты этого, насекомец! Да очень уж мне подарки твои понравились, особенно кольцо из цельнометаллического водорода сделанное, с формулой Вселенной выгравированной.\par
\par
Путь твой далек будет. Мимо галактик, в спирали закрученных, мимо облаков водородных, дивным светом сияющих, мимо квазаров грозных, гравитационные волны излучающих, к самому краю видимой Вселенной, где лишь протоны да альфа-частицы шныряют, а звезду и на миллион парсек не встретишь. И всё же есть там звездочка одна, в туманности газо-пылевой спрятавшаяся. А вокруг звезды той огромная планета вращается, а вокруг планеты – спутник вертится. И на спутнике том кратер есть круглый да огромный. И на дне кратера того Врата стоят Звездные. И ведут эти Врата к тому, что ты найти так жаждешь.\par
\par
Но непросто до Врат добраться. Лабиринт вкруг тех Врат выстроен в сто этажей да в десять тысяч комнат на каждом. Да такой хитрый, что как войдешь в него, так и заблудишься сразу. И как сквозь тот лабиринт пройти, мне неведомо.\par
\par
Поэтому трудно тебе придется. Но коли сумеешь до Врат добраться и через них пройти, окажешься в комнатке маленькой. И будут пред тобой три двери – две больших да красивых, а третья – маленькая да невзрачная.\par
\par
На первой двери, золотом украшенной, нарисован будет ядерный котел над очагом звездным. За этой дверью Машина времени древняя, что прошлое изменять может да парадоксы вселенские творить. Сунешь свой нос в эту дверь – на веки сгинешь. Но если уж не послушаешь моего совета и заглянешь туда – руками ничего не трогай, а то и вся Вселенная наша переиначиться может.\par
\par
На второй двери, алмазами выложенной, кот в сапогах наклеен. За этой дверью биолаборатория заброшенная, в которой ксеноморфов да всяких хищников страшных выводили. Они и сейчас там бродят. Такому герою, как ты, туда зайти – заживо съеденным быть. Но уж если не послушаешь доброго совета да заглянешь туда – из пробирок не пей – ксеноморфиком станешь, или шай-хулудом каким.\par
\par
На третьей двери, паутиной затянутой да звездной пылью засыпанной, ничего не нарисовано, только написано: «Не влезай, убьет!» За этой дверью найдешь ты клад, сокровище, которое так отыскать жаждешь.\par
\par
А чтоб ты с пути не сбился, дам я тебе лазерную указку волшебную, куда она покажет – туда ты и направляйся».\par
\par
Пустился Грыблозавр в путь. Летит он в гиперпространстве гипертоннелями, которые, не иначе, какие-то гиперкроты вырыли, летит измерениями неизвестными. Уж и не знает, сколько в нем самом теперь измерений осталось. Но не сдается, боится только одного – наизнанку вывернуться.\par
\par
Долго ли, коротко ли, долетел Грыблозавр до звездочки заветной. Рассчитал он курс, на нужную орбиту лег, к высадке подготовился.\par
\par
Высадился Грыблозавр рядом с лабиринтом, добрался до входа и внутрь вошел. Идет одним коридором, другим, третьим. Пусто везде, тихо, только его шаги эхом отдаются. До комнаты какой-то дошел, видит – дальше три коридора ведут, и в каждом будто туман клубится. А на полу написано: «Коли дураком не хочешь стать – иди направо, либо прямо. Коли голову потерять не хочешь – иди прямо, либо налево. Коли до смерти запуганным быть не хочешь – иди налево, либо направо».\par
\par
Задумался Грыблозавр, куда идти, да и пошел направо. Идет по лабиринту, все этажи осматривает, во все комнаты заглядывает – нет ли там чего. Да ничего, кроме тумана, не видит. И уж начинают ему мерещиться в тумане щупальца страшные, пауки ужасные, да монстры с дом высотой, которым его голова нужна. Вдруг видит в одной комнате – девица-красавица стоит, да такая, что дух захватывает. Как увидел ее Грыблозавр – сразу голову потерял, о Вратах позабыл.\par
\par
– Здравствуй – говорит – девица! Как тебя звать-величать? Откуда ты тут, где планета твоя родная? Хочу я с тобой на эту планету полететь.\par
– Здравствуй, – девица отвечает, – имени моего тебе всё равно не выговорить, всегда я тут была, никуда не летала.\par
– Так и что же это за место такое? Что у вас тут делается, что происходит? Пойдем отсюда выбираться, на мой корабль возвращаться.\par
– Ты, я вижу, совсем голову потерял. Лабиринт это, в сто этажей высотой да в сто тысяч комнат на каждом. У нас тут большая игра идет, а если я с тобой наружу выйду, так исчезну в тот же миг.\par
– Это как же так-то! Эка жалость. А что за игра-то?\par
– Ладно, раз уж ты ко мне пожаловал, этот раунд за мной, идем – расскажу я тебе про игру.\par
\par
К выходу из комнаты направляется, и как будто насквозь немного просвечивать начинает. Грыблозавр – за ней. Прошла девица пару коридоров, по лестнице поднялась, да в какую-то дверь вошла. \par
Входит Грыблозавр в комнату – а там нет никого, лишь три сгустка тумана поплотнее. И как будто двое одному что-то вроде монет туманных передают.\par
– И где же тут кто? – Грыблозавр спрашивает.\par
– Да мы тут везде, но здесь особенно, – отвечают три голоса, да прямо в голове звучат.\par
– И кто же вы такие будете?\par
– Мы – представители древней цивилизации, раса наша настолько древняя, что телесно уже и не существует вовсе. Только ментально, то есть разумом своим. И знаем мы все тайны Вселенной, и все предсказать да рассчитать можем. И скучно нам от этого необычайно. И даже говорим мы длинно и скучно, как ты мог заметить. Пробовали в рулетку играть – да каждый знает, куда шарик прикатится. Пробовали в квантовое лото играть – так и принцип неопределенности квантовый для нас не помеха в предсказаниях.\par
– А здесь-то вы что делаете?\par
– Воздвигли мы силой разума лабиринт этот, чтоб скуку развеять можно было. Разумные существа, сюда попавшие, выбор делают. А мы играем, ставки делаем, по чьей дороге они пойдут. Ибо обладают разумные существа свободой воли, и тут наши предсказания бессильны. Так что давай, хватит отдыхать – видишь, три коридора отсюда тянутся – иди уж по какому-нибудь, да побыстрее!\par
– Не нужны мне ваши коридоры, мне к Звездным Вратам нужно!\par
– Будь любезен, не упрямься. Мы легко тебя заставить можем. Создадим мы сей же час чудовищ ментальных, ты кое-кого видел уже, и ни бластер, ни водяной пистолет тебе не помогут, поскольку будут чудовища внутри твоего разума, а не снаружи. А во сне тебе и вовсе тяжко придется.\par
\par
Видит Грыблозавр – со всех сторон к нему уже пасти и щупальца тянутся. Забился он в угол, да как закричит:\par
– Остановитесь, погодите! А кто из вас этих чудовищ делает?\par
– Как кто? Мы все трое.\par
– А чьи чудовища самые сильные будут?\par
Тут замерли чудовища на мгновение, а потом как начали друг с другом биться. Лапы с хвостами в разные стороны так и разлетаются. Драконы с демонами сшибаются, ангелы с гигантскими червями, орки и гоблины с эльфами да рыцарями, кальмары огромные с василисками. И над всем этим пегасы да орнитоптеры парят, и молнии сверкают. А внизу горы какие-то да болота с лесами мелькают. Даже один раз черный лотос виден был. \par
– Стойте! – Грыблозавр кричит. – Меня ж сейчас тут совсем затопчут!\par
– Уйди, не мешайся! – три голоса отвечают. – Видишь левый коридор, там третий поворот направо и два раза налево – и придешь к своим Вратам Звездным.\par
\par
Побежал Грыблозавр что есть духу, в минуту до Врат добрался.\par
\par
Прошел Грыблозавр сквозь Врата – видит, три двери перед ним. Вошел в нужную, несколько шагов прошел и слышит какой-то шорох сзади. Оглянулся – за ним Альтавистра стоит, ухмыляется. \par
\par
– Не думала я, – говорит, – что сумеешь ты до Врат добраться, да всё же надеялась. Даже приемник телепортационный тебе в правый ботинок засунула. Очень уж мне клад заполучить надо, гораздо нужнее, чем тебе. Так что медленно подними руки вверх и отойди в сторонку, не мешайся.\par
– Ах ты, харя безмозглая, – Грыблозавр отвечает, – обнаружил я твой приемник, когда ботинки чистил, знал, что ты недоброе задумала. Посмотри вокруг, в биолабораторию ты попала. Чу, хищники зубами скрежещут, клювы разевают, щупальца расправляют! Не выйти тебе отсюда, не забрать клад.\par
– Так, значит! – говорит Альтавистра. – Что ж! Посмотрим, кто отсюда живым не выйдет!\par
\par
Хватает со стола пробирку и одним махом выпивает. Раз – и стоит вместо Альтавистры кот альдебаранский ядовитый, трех метров роста, к прыжку готовится, слюна с клыков капает. Не растерялся Грыблозавр, тоже пробирку схватил, тоже выпил. Превратился в дракона ригельского, махнул хвостом – кот от него на десять метров отлетел. Схватил кот с другого стола целую колбу, осушил одним махом, превратился в пчелу бронебойную с Канопуса 4, разогнался, дракона насквозь пробил. Да успел тот на последнем издыхании до пробирки дотянуться, сжевал ее с содержимым вместе, превратился в броненосца мифрильного насекомоядного с Регула 2…\par
\par
В общем, долго они так развлекались, дня три, не меньше. Чуть не забыли, кто из них кто. Уж и пробирки-то почти все закончились. Схватила Альтавистра последнюю пробирку, тут Грыблозавр как закричит: «Стой, дурья твоя башка! А как мы в себя-то обратно превратимся?» Задумалась Альтавистра. «Всё из-за тебя, болван, – говорит. – Теперь всё с начала начинать придется!» Пробирку бросила, на щупальцах приподнялась и выбежала из лаборатории. Грыблозавр за ней пополз.\par
\par
Выползает, смотрит – Альтавистра в первую дверь, с очагом звездным, забежала. И Грыблозавр туда направился.\par
\par
Глядит – машина там стоит дивная, лампочками моргает, жужжит тихонько и готова в прошлое отправиться хоть к сотворению Вселенной. А Альтавистра уже внутри сидит, рычажки какие-то дергает и кнопки нажимает. Кинулся Грыблозавр к Альтавистре, тоже внутрь залез, остановить хотел, да не успел. И исчезли оба вместе с машиной, как будто не было их тут вовсе. И сразу история эта поменялась, совсем иной стала...\par

\chapter{}
 \lettrine{В}{эпоху Великого Расселения} жил один человек, звали его Игнат. Был он известный воин и был на редкость умен.\par
\par
И вот прочитал он в Интернете, что на одной затерянной в глубинах космоса, холодной и обледенелой планете скрыт самый быстрый во Вселенной космический корабль с большой лазерной пушкой. И решил тогда Игнат его себе забрать, чтобы не попался он в чужие руки и не вышло большой беды для всей Вселенной. Но где искать корабль космический самый быстрый? Звезд да черных дыр во Вселенной ужас как много, и вокруг каждой великое множество планет да астероидов вертится!\par
\par
Принялся Игнат думать, у кого информацию нужную добыть можно. Думал-думал, да надумал обратиться к пророчице из двойной звездной системы Медузия, несравненной Альтавистре, чьи пророчества всегда сбывались с точностью необычайной. Надел он свой парадный скафандр и помчался искать аудиенции у Альтавистры.\par
\par
Вскорости нашелся способ увидеться с Альтавистрой. Так, мол, и так, сказывает Игнат, хочется мне отыскать корабль космический самый быстрый, сокровище это необычное, и побуждения мои самые что ни на есть прекрасные. Да вот закавыка – не знаю, как!\par
\par
«Что ж, – отвечает ему Альтавистра, – вижу я, ты очень храбр, и IQ твой вышиной до звезд простирается! Впрочем, ты человек, а люди этим известны, да еще упертостью своей. Но чтобы я тебе помогать стала, тебе самому сначала мне помочь придется. Выполнишь мое задание – расскажу, как найти корабль космический самый быстрый, а нет – уста мои молчание хранить будут!\par
\par
Внемли! Есть по соседству тут звезда нейтронная – третий поворот направо, если к центру Галактики лететь. Рядом с ней планетоид карликовый, а на планетоиде том два чудовища живут. Зовут их Диспрозий и Гадолиний, злобные они до ужаса, и до самых кончиков клыков темной энергией пропитаны. Есть у них гусли со струнами космическими, такие, что каждый, кто их услышит, от радости гиперпрыжки да танцы начинает выделывать, которые другим и не снились. Пуще жизни чудовища их любят. Добудь мне гусли, и расскажу я тебе, где найти корабль космический самый быстрый».\par
\par
Закручинился Игнат, да делать нечего. Полетел к чудовищам гусли добывать. Летит, а сам боится, мелкой дрожью дрожит, даже поворот нужный пропустил – пришлось возвращаться. А когда возвращался, увидал пустую канистру из-под топлива термоядерного, кем-то выброшенную. Возьму, думает, ее с собой – вдруг пригодится. Летит дальше – видит, кусок темной материи в пространстве висит, весь скомканный – наверное, купец какой-то потерял. И его, думает, возьму, тоже может на что сгодиться. Дальше летит – видит, воронка гравитационная валяется – должно быть, странники какие-то оставили. И её тоже взял.\par
\par
Прилетает, садится на планетоид, заходит во дворец к чудовищам, а те сразу к нему кидаются. \par
– Чу, – говорят, – человеческим духом пахнет! Кто такой, – спрашивают, – откуда? Как хочешь быть проглоченным – ногами вперед или назад? Да не тяни с ответами, а то проголодались мы чудовищно – сто лет уж никого не ели.\par
– Ах вы, чудища поганые, – Игнат отвечает, – не поймавши добра молодца, да кушаете! \par
Схватил канистру и давай чудовищ по мордам их гнусным бить-колотить. От пустой канистры такой шум поднялся сильный, что чуть потолок не обвалился. Даже чудовища струхнули малость. \par
– Да погоди, – говорят, – чего тебе надобно-то?\par
– Знаю я, есть у вас гусли со струнами космическими, инструмент дивный. Вот их-то мне и надо!\par
– Не, – говорят чудища, – не отдадим мы их. Нам этот инструмент дороже жизни. А на что он тебе? \par
– Да так и так, – сказывает Игнат, – хочу я найти корабль космический самый быстрый, сокровище это необычное, и все мысли мои теперь только об этом. Вот и обещала мне Альтавистра рассказать, как найти корабль космический самый быстрый, если принесу я инструмент ваш. \par
\par
«Опять она за своё! – говорят чудища. – Уж и не в первый раз!.. Ладно, слушай, давай мы тебе расскажем, как корабль космический самый быстрый отыскать, а то чего тебе туда-сюда бегать-то! И гусли со струнами космическими целее будет. \par
\par
Далеко отсюда твой путь лежит. Мимо галактик, в спирали закрученных, мимо облаков водородных, дивным светом сияющих, мимо квазаров грозных, сигналы чудные излучающих, к самому краю космоса, где лишь протоны да альфа-частицы шныряют, а звезду и на миллион парсек не встретишь. И всё же есть там звездочка одна, в туманности газо-пылевой спрятавшаяся. А вокруг звезды той маленькая планета вращается, а вокруг планеты – спутник вертится. И на спутнике том кратер есть огромный да глубокий. И на дне кратера того Врата стоят Звездные. И ведут эти Врата к тому, что ты найти так жаждешь.\par
\par
Только просто так во Врата не пройти. Охраняет их страж из металла жидкого, ни для какого оружия не уязвимый. Ни днем, ни ночью не спит он и все смотрит внимательно, не прошмыгнул бы кто к Вратам этим. Многие смельчаки пробраться мимо него пытались, да так там в жидком металле и потонули.\par
\par
Но коли ты жив останешься да сумеешь до Врат добраться и через них пройти, окажешься в зале с потолком таким высоким, что и не видно. И будут пред тобой три двери – две больших да красивых, а третья – маленькая да невзрачная.\par
\par
На первой двери, золотом украшенной, нарисован будет ядерный котел над очагом звездным. За этой дверью Изменитель реальности, невесть кем построенный. Сунешь свой нос в эту дверь – на веки сгинешь. Но если уж не послушаешь нашего совета и заглянешь туда – руками ничего не трогай, а то и вся Вселенная наша переиначиться может.\par
\par
На второй двери, алмазами выложенной, волк в тельняшке нарисован. За этой дверью биолаборатория заброшенная, в которой ксеноморфов да всяких хищников страшных выводили. Они и сейчас там бродят. Такому герою, как ты, туда зайти – заживо съеденным быть. Но уж если не послушаешь доброго совета да заглянешь туда – из пробирок не пей – ксеноморфиком станешь, или шай-хулудом каким.\par
\par
На третьей двери, паутиной затянутой да звездной пылью засыпанной, ничего не нарисовано, только написано: «Не влезай, убьет!» За этой дверью найдешь ты корабль космический самый быстрый, сокровище, которое так найти жаждешь.\par
\par
А чтоб ты с пути не сбился, дадим мы тебе лазерную указку волшебную, куда она покажет – туда ты и направляйся».\par
\par
Пустился Игнат в путь. А кусок темной материи да воронку в космос выкинул – не пригодились они. Летит он в гиперпространстве тропами нехожеными, измерениями неизвестными. Уж и не знает, трехмерный ли он до сих пор. Но не сдается, боится только одного – наизнанку вывернуться.\par
\par
Сколько световых лет прошло, неведомо, но долетел Игнат до звездочки заветной. Рассчитал он курс, на нужную орбиту лег, к высадке подготовился.\par
\par
Подлетает к спутнику, а тут солнечный ветер поднялся жуткий, аж с ног сбивает. Хорошо, думает Игнат, с подветренной стороны зайду – тогда меня не сразу учуют. \par
\par
 Приземлился, из посадочного модуля выбрался, хотел к Вратам бежать, да страж уж тут как тут. Идет, похожий на андроида, зеркальной краской выкрашенного, да за дезинтегратором своим тянется. \par
 \par
– Постой! – кричит ему Игнат. – Мы с тобой одного металла, ты и я. \par
– Что? – кричит в ответ страж. – Я из-за ветра тебя слышу плохо. \par
– Говорю, мы с тобой одного металла, ты и я, – опять кричит Игнат. – Не надо меня дезинтегрировать!\par
– Что говоришь? Тебе одного раза мало, когда надо тебя дезинтегрировать? Постой, я поближе подойду. \par
Подошел поближе, спрашивает: \par
– Так что ты сказать-то хотел? \par
– Я говорю, мы с тобой одного металла, – повторяет Игнат, – поэтому меня дезинтегрировать не надо. \par
– Да? – удивляется страж. – А что же с тобой делать надо? Да и не похож ты на меня – я вон какой гладкий да зеркальный, а ты бледный какой-то. \par
– Так я ж изучал, как люди живут, вот и превратился. Ты, вон, тоже, небось, в кого захочешь – в того и превратишься. Хоть в меня, хоть в дракона с планеты Протактиний, хоть во что маленькое и безобидное. \par
– Это верно, смотри. \par
\par
Начинает тут страж переливаться всеми цветами радуги, и вдруг – бац – Игнат словно сам перед собой стоит. «Что, – говорит страж, – впечатляет? Смотри дальше!» И превращается в такое ужасное чудище, каких Игнат и не видел никогда, чуть рассудка от страха не лишился. «То-то, – говорит страж. – Смотри дальше!» И превращается в плитку шоколада. Лежит себе плитка, да такая аппетитная, что сама так в рот и просится. Схватил Игнат плитку, да не тут-то было – плитка килограмм сто весит – не меньше. Закон сохранения массы, видать, в действии. \par
\par
А страж уж обратно в андроида зеркального превратился. \par
– Что, съел? Я, – говорит, – во что хочешь превращаться умею. Вот только в себя не могу. \par
– Это почему же? – спрашивает Игнат.\par
– Да я уж во столько всего превращался, что и забыл, как вначале выглядел. \par
– Постой, роботы же никогда ничего не забывают. \par
– Сам ты робот! – говорит с обидой страж и опять за дезинтегратором тянется. – Шейпшифтер я! Шейп-шиф-тер! \par
– Да стой, не кипятись. Дай-ка я проверю, робот ты или нет, я тест знаю. \par
– Ну ладно, давай. \par
– Вот смотри, – говорит Игнат, – сможешь прочитать, что тут написано? \par
А сам берет листок бумаги, пишет на нем что-то и стражу протягивает. \par
– Да тут «GJ85QR2» написано, чушь какая-то, да еще и двойной линией перечеркнуто.\par
– Похоже, и вправду ты не робот. Чего ж ты тут делаешь? \par
– Да было у нас пророчество, что кто через Звездные Врата пройдет, тот и Вселенную изменить сможет. А нам, шейпшифтерам, это ни к чему. Нас и такая Вселенная устраивает. Вот и сижу я тут, смотрю, что б никто через Врата не прошел, прям жизни никакой уж от них нет. Замучился – ни отойти куда, ни поспать. \par
– Ну, так и зачем тебе такая Вселенная-то, в которой ты ни отойти куда не можешь, ни поспать, ни друзей завести, ни животных домашних? \par
– А и верно, – говорит страж, – незачем мне всё это! Ты ведь во Врата пройти собирался? Ну и иди себе. Только просьба у меня к тебе есть: зайди в биолабораторию, посмотри, нет ли там овец электрических. Приведи мне одну, если найдешь, – уж очень я о таком животном мечтаю.\par
\par
\par
Прошел Игнат сквозь Врата – видит, три двери перед ним. Убрал он паутину с самой маленькой, пыль звездную с нее отряхнул да внутрь вошел. Осмотрелся и увидел корабль космический самый быстрый, то сокровище, из-за которого покоя лишился. Бросился Игнат к сокровищу своему, тут вдруг сзади шорох какой-то послышался. Обернулся – позади Альтавистра с пола встает.\par
– Ох! – говорит Альтавистра. – Хорошо, что ты сюда добрался, а то раньше никому не удавалось, я уж и со счета сбилась.\par
– Альтавистра! Ты-то здесь откуда? Да еще и на полу отдыхаешь.\par
– Так я ж тебе к правому ботинку микротелепортационный приемник прицепила. Хоть и не верилось, что ты сюда доберешься. Да уж больно мне корабль космический самый быстрый раздобыть надо было. Пришлось вот даже ползком телепортироваться – слишком уж маленький портальчик сделался.\par
– Стоп! Это мне его раздобыть надо было! Вот я здесь и оказался.\par
– Ты уж извини, Игнат, только мне это нужнее, – говорит Альтавистра. Выхватывает станнер и стреляет. Игнат сразу на пол шлепнулся, ни рукой, ни ногой двинуть не может. Языком еле ворочает.\par
– Ой, – говорит, – ты что, супостат, делаешь?!\par
– Да я тебя в лабораторию ближайшую сейчас сдам – для опытов. Чтоб под ногами не путался.\par
Схватила Альтавистра Игната за шиворот и потащила в биолабораторию по соседству.\par
\par
Затащила она Игната в лабораторию, бросила там и удалилась важно. Лежит Игнат, пошевелиться не может, ждет, когда им завтракать придут. Или обедать – не знает, что и хуже. \par
Смотрит – через дальнюю дверь стадо овец входит. Все белые, только одна черная, по крайней мере, с одной стороны. Для политкорректности. Шерсть на овцах дыбом стоит и искры по шерсти бегают размером со спаниеля. Какая-то тощая тварь с потолка попыталась на них напрыгнуть, да ее на лету молнией сшибло.\par
\par
«Эге, – думает Игнат, – это ж прямо электроовцы какие-то. У меня и часы от них остановились, похоже. И сервопривод шнурков в ботинках отключился. Экое абсолютное оружие. Как бы мне его себе приручить». Тут чувствует – руки-ноги опять шевелиться могут. Почесал Игнат в затылке, огляделся повнимательней. Снял со стены диаграмму не пойми чего огромную, быстренько на обратной стороне картинку нарисовал, дырку в середине проделал и надел на себя через голову. Стал Игнат похож на рекламный щит ходячий, человека-бутерброд, которого как-то в космопорту видел. Только Игнат стал человек-ворота. Стадо овец как его увидело – сразу побежало на новые ворота смотреть. А Игнат пошел к Альтавистре.\par
\par
Приходит, видит – Альтавистра портал для обратной телепортации готовит, а роботопомощники вокруг так и кишат. И Альтавистра его увидела. «Не думала, – говорит, – что ты выбраться сможешь. Ну да неважно». И приказывает роботопомощникам очистить помещение от посторонних. Но не тут-то было. Роботы все поотключались, портал к овцам притянулся и вместе с ними схлопнулся. А у Альтавистры шнурки развязались.\par
\par
Подбежал Игнат к Альтавистре, хотел стукнуть как следует, да увернулась Альтавистра, из ботинок выскочила и прочь кинулась. «Вся матрица моих надежд рухнула! – кричит. – Перезагрузка! Только она мне поможет!» Выбежала из двери, к другой двери подбежала, с котлом ядерным, и шасть за нее. Игнат за ней кинулся, хоть и отстал чуток.\par
\par
Глядит – механизм там стоит дивный, лампочками моргает, жужжит тихонько и готов в любую секунду реальность изменить. А Альтавистра уже рычажки какие-то дергает и кнопки нажимает. Кинулся Игнат к Альтавистре, остановить хотел, да не успел. Прошла рябь по Вселенной и исчезли оба, как будто не было их тут вовсе. И сказка эта совсем иная стала...\par

\chapter{}
 \lettrine{Н}{а заре Вселенной} был один гуманоид, звали его Игнат. Был он лучший из лучших герой, но ума при этом был небольшого.\par
\par
И вот узнал он, что на одной затерянной в глубинах космоса, холодной и обледенелой планете есть скрытая библиотека с тайными знаниями обо всей Вселенной. И захотел тогда Игнат ее разыскать для себя, чтобы закинуть в самый дальний угол подпространства и чтобы никто больше не мог разыскать это сокровище и не смущало оно умы смертных. Но где искать библиотеку? Звезд да черных дыр в галактиках ужас как много, и вокруг каждой великое множество планет да астероидов вертится!\par
\par
Стал Игнат думать, у кого информацию нужную добыть можно. Думал-думал, и надумал обратиться к известному на всю Галактику звездознатцу, профессору Грымзику. Взял он свой бластер верный и помчался к Грымзику.\par
\par
Вскорости смог Игнат с ним увидеться. Так, мол, и так, говорит Игнат, хочется мне отыскать библиотеку, сокровище это удивительное, и все помыслы мои теперь только об этом. Да вот закавыка – понятия не имею, как!\par
\par
«Вот что, – отвечает ему Грымзик, – вижу я, ты весьма храбр, да только IQ твой ниже плинтуса! Впрочем, ты гуманоид, а гуманоиды этим славятся, да еще упертостью своей. Только не знаю я, как помочь тебе. Но есть сестра у меня, мудрости столь необычной, что моя мудрость по сравнению с её – логарифмическая линейка по сравнению с суперкомпьютером. Живет она у соседней звезды, второй поворот налево, если отсюда к краю Галактики лететь. Принеси ей от меня весточку – может, поможет она тебе».\par
\par
Собрал Игнат с собой подарки да украшения, добавил к ним кольцо из цельнометаллического водорода сделанное, с формулой Вселенной выгравированной, и отправился в дорогу.\par
\par
Прилетает он к сестре Грымзика, отдает ей подарки, и письмо от Грымзика вручает. «Так, мол, и так, – говорит Игнат, – хочу я отыскать библиотеку, сокровище это необычное, и побуждения мои самые что ни на есть прекрасные.»\par
\par
«Стой-ка, – говорит сестра, – тут в письме написано, что б я тебе голову отрубила, а не помогать стала!.. Ах нет, извини, просто письмо с другой стороны какого-то черновика написано, там даже печать есть... Ладно, помогу я тебе, хоть и не стоишь ты этого, гуманоид! Да очень уж мне подарки твои понравились, особенно кольцо из цельнометаллического водорода сделанное, с формулой Вселенной выгравированной.\par
\par
Путь твой далек будет. Мимо галактик в спирали, мимо планет в тентуре, мимо облаков водородных, дивным светом сияющих, мимо квазаров грозных, гравитационные волны излучающих, к самому краю видимой Вселенной, где лишь протоны да альфа-частицы шныряют, а звезду и на миллион парсек не встретишь. И всё же есть там звездочка одна, в туманности газо-пылевой спрятавшаяся. А вокруг звезды той огромная планета вращается, а вокруг планеты – спутник вертится. И на спутнике том кратер есть круглый да огромный. И на дне кратера того Врата стоят Звездные. И ведут эти Врата к тому, что ты найти так жаждешь.\par
\par
Только просто так во Врата не пройти. Охраняет их страж из металла жидкого, ни для какого оружия не уязвимый. Ни днем, ни ночью не спит он и все смотрит внимательно, не прошмыгнул бы кто к Вратам этим. Толпы страждущих пробраться мимо него пытались, да так там костьми и полегли.\par
\par
Поэтому трудно тебе придется. Но коли сумеешь до Врат добраться и через них пройти, окажешься в зале с потолком таким высоким, что и не видно. И будут пред тобой три двери – две больших да красивых, а третья – маленькая да невзрачная.\par
\par
На первой двери, серебром украшенной, нарисован будет ядерный котел над очагом звездным. За этой дверью Портал сияющий, в параллельные вселенные ведущий. Сунешь свой нос в эту дверь – на веки сгинешь. Но если уж не послушаешь моего совета и заглянешь туда – руками ничего не трогай, а то и вся Вселенная наша переиначиться может.\par
\par
На второй двери, алмазами выложенной, жаба в скафандре нарисована. За этой дверью биолаборатория заброшенная, в которой ксеноморфов да всяких хищников страшных выводили. Они и сейчас там бродят. Такому герою, как ты, туда зайти – головы не сносить. Но уж если не послушаешь доброго совета да заглянешь туда – из пробирок не пей – ксеноморфиком станешь, или шай-хулудом каким.\par
\par
На третьей двери, паутиной затянутой да звездной пылью засыпанной, ничего не нарисовано, только написано: «Посторонним вход воспрещен!» За этой дверью найдешь ты библиотеку, сокровище, которое так отыскать стремишься.\par
\par
А чтоб ты с пути не сбился, дам я тебе комету путеводную, куда она полетит – туда и ты лети».\par
\par
Пустился Игнат в путь. Летит он в гиперпространстве гипертоннелями, которые, не иначе, какие-то гиперкроты вырыли, летит измерениями неизвестными. Уж и не знает, трехмерный ли он до сих пор. Но не сдается, боится только одного – с пути сбиться.\par
\par
Долго ли, коротко ли, долетел Игнат до звездочки заветной. Рассчитал он курс, на нужную орбиту лег, к высадке подготовился.\par
\par
Подлетает к спутнику, а тут солнечный ветер поднялся жуткий, аж с ног сбивает. Хорошо, думает Игнат, с подветренной стороны зайду – тогда меня не сразу учуют. \par
\par
 Приземлился, из посадочного модуля выбрался, хотел к Вратам бежать, да страж уж тут как тут. Идет, похожий на андроида, зеркальной краской выкрашенного, да за транклюкатором своим тянется. \par
 \par
– Постой! – кричит ему Игнат. – Мы с тобой одного металла, ты и я. \par
– Что? – кричит в ответ страж. – Я из-за ветра тебя слышу плохо. \par
– Говорю, мы с тобой одного металла, ты и я, – опять кричит Игнат. – Не надо меня транклюкировать!\par
– Что говоришь? Тебе одного раза мало, если надо тебя транклюкировать? Постой, я поближе подойду. \par
Подошел поближе, спрашивает: \par
– Так что ты сказать-то хотел? \par
– Я говорю, мы с тобой одного металла, – повторяет Игнат, – поэтому меня транклюкировать не надо. \par
– Да? – удивляется страж. – А что же с тобой делать надо? Да и не похож ты на меня – я вон какой гладкий да зеркальный, а ты бледный какой-то. \par
– Так я ж изучал, как гуманоиды живут, вот и превратился. Ты, вон, тоже, небось, в кого захочешь – в того и превратишься. Хоть в меня, хоть в монстра с планеты Земля, хоть во что маленькое и безобидное. \par
– Это верно, смотри. \par
\par
Начинает тут страж переливаться всеми цветами радуги, и вдруг – бац – Игнат словно сам перед собой стоит. «Что, – говорит страж, – впечатляет? Смотри дальше!» И превращается в такое ужасное чудище, каких Игнат и не видел никогда, чуть с ума от страха не сошел. «То-то, – говорит страж. – Смотри дальше!» И превращается в плитку шоколада. Лежит себе плитка, да такая аппетитная, что сама так в рот и просится. Схватил Игнат плитку, да не тут-то было – плитка килограмм сто весит – не меньше. Закон сохранения массы, видать, в действии. \par
\par
А страж уж обратно в андроида зеркального превратился. \par
– Я, – говорит, – во что хочешь превращаться умею. Вот только в себя не могу. \par
– Это почему же? – спрашивает Игнат.\par
– Да я уж во столько всего превращался, что и забыл, как вначале выглядел. \par
– Постой, роботы же никогда ничего не забывают. \par
– Сам ты робот! – говорит с обидой страж и опять за транклюкатором тянется. – Шейпшифтер я! Шейп-шиф-тер! \par
– Да стой, не кипятись. Дай-ка я проверю, робот ты или нет, я тест знаю. \par
– Ну ладно, давай. \par
– Вот смотри, – говорит Игнат, – сможешь прочитать, что тут написано? \par
А сам берет листок бумаги, пишет на нем что-то и стражу протягивает. \par
Смотрит тот, листок в руках так и сяк вертит. \par
– Не, – говорит, – ответ отрицательный. Данная запись смысла не имеет. Что это тут, будто буковки какие-то неровные, да еще и двойной волнистой линией зачеркнуты? \par
– Ну, какой же ты не робот, – говорит Игнат, – ты типичный робот. Впрочем, ладно, вот тебе последний тест, смотри, – и на двух сторонах чистого листка что-то пишет. – Сможешь определить, правда тут написана, али ложь? \par
\par
Берет страж новый листок, читает: «На другой стороне листа этого правда написана». Переворачивает листок, видит: «На другой стороне листа этого ложь написана». Опять он листок переворачивает, опять читает. И опять, и опять, и опять. И все быстрее листок вертит, разобраться старается, правда там написана или ложь. Уж ветер от вращающегося листка подниматься начал. \par
\par
Посмотрел Игнат на это, да к Звездным Вратам пошел неторопливо.\par
\par
\par
Прошел Игнат сквозь Врата – видит, три двери перед ним. Вошел в нужную, несколько шагов прошел и слышит какой-то шорох сзади. Оглянулся – за ним Грымзик стоит, ухмыляется. \par
\par
– Не думал я, – говорит, – что сумеешь ты до Врат добраться, да всё же надеялся. Даже приемник телепортационный тебе в правый ботинок засунул. Очень уж мне библиотеку заполучить надо, гораздо нужнее, чем тебе. Так что медленно подними руки вверх и отойди в сторонку по хорошему, не мешайся.\par
– Ах ты, морда поганая, – Игнат отвечает, – нашел я твой приемник, когда ботинки чистил, знал, что ты недоброе затеял. Оглядись вокруг, в биолабораторию ты телепортировался. Чу, хищники зубами скрежещут, клювы разевают, щупальца расправляют! Не выйти тебе отсюда, не забрать библиотеку.\par
– Так, значит! – говорит Грымзик. – Что ж! Посмотрим, кто отсюда живым не выйдет!\par
\par
Хватает со стола пробирку и одним махом выпивает. Раз – и стоит вместо Грымзика лев альдебаранский ядовитый, трех метров роста, к прыжку готовится, слюна с клыков капает. Не растерялся Игнат, тоже пробирку схватил, тоже выпил. Превратился в тираннозавра ригельского, махнул хвостом – лев от него на восемь метров отлетел. Схватил лев с другого стола целую колбу, осушил одним махом, превратился в пчелу бронебойную с Канопуса 5, разогнался, тираннозавра насквозь пробил. Да успел тот на последнем издыхании до пробирки дотянуться, сжевал ее с содержимым вместе, превратился в броненосца мифрильного насекомоядного с Регула 2…\par
\par
В общем, долго они так развлекались, часа два, не меньше. Чуть не забыли, кто из них кто. Уж и пробирки-то почти все закончились. Схватил Грымзик последнюю пробирку, тут Игнат как закричит: «Стой, дурья твоя башка! А как мы в себя-то обратно превратимся?» Задумался Грымзик. «Всё из-за тебя, идиот, – говорит. – Теперь всё с начала начинать придется!» Пробирку бросил, на щупальцах приподнялся и выбежал из лаборатории. Игнат за ним пополз.\par
\par
Выползает, смотрит – Грымзик в первую дверь, с очагом звездным, забежал. И Игнат туда направился.\par
\par
Глядит – портал в параллельные миры посреди комнаты сияет и Грымзик к нему бежит. Бросился Игнат за Грымзиком, чтобы остановить, схватил крепко. Да изловчился Грымзик, качнулся, и рухнули они оба в портал, в другой Вселенной оказались. А в ней и сказка эта совсем по-другому сказывается...\par

\chapter{}
 \lettrine{Д}{авным-давно} жил один осьминожец по имени Станислав. Был он известный воин и был на редкость умен.\par
\par
И вот узнал он, что в одной звездной системе, на маленьком астероиде находится великая вычислительная машина, знающая ответы на все вопросы. И надумал Станислав ее разыскать для себя, чтобы с ее помощью стать властелином всех миров и галактик. Да как разыскать машину вычислительную? Звезд да черных дыр в галактиках ужас как много, и вокруг каждой куча планет да астероидов вертится!\par
\par
Стал Станислав смекать, у кого совета спросить. Думал-думал, да надумал обратиться к робостарцу Вегианскому, Иммортию, коий был старше самой Галактики, и помнил еще Большой Взрыв, при котором был ребенком. Сел он в свой корабль космический и отправился искать встречи с Иммортием.\par
\par
Много ли времени прошло, иль мало, но нашел Станислав способ с ним повстречаться. Так, мол, и так, сказывает Станислав, хочу я найти машину вычислительную, сокровище это удивительное, и все мысли мои теперь только об этом. Да вот проблема – знать не знаю, как!\par
\par
«Послушай, – отвечает ему Иммортий, – вижу я, ты весьма решителен в своем намерении, и IQ твой вышиной до звезд простирается! Впрочем, ты осьминожец, а осьминожцы этим известны, да еще упертостью своей. Только не знаю я, как помочь тебе. Но есть сестра у меня, мудрости столь необычной, что моя мудрость по сравнению с её – желтый карлик по сравнению с Бетельгейзе. Живет она у соседней звезды, второй поворот налево, если держать курс на Малую Медведицу. Принеси ей от меня весточку – может, поможет она тебе».\par
\par
Собрал Станислав с собой подарки да украшения, добавил к ним нож +1 кремневый, артефактный, великой древности, и в путь пустился.\par
\par
Прилетает он к сестре Иммортия, отдает ей подарки, и письмо от Иммортия вручает. «Так и так, – сказывает Станислав, – очень хочется мне найти машину вычислительную, сокровище это удивительное, и побуждения мои самые что ни на есть прекрасные.»\par
\par
«Погоди-ка, – говорит сестра, – тут в письме написано, что б я тебе голову отрубила, а не помогать стала!.. Ах нет, извини, просто письмо с другой стороны какого-то черновика написано, там даже печать есть... Ладно, помогу я тебе, хоть и не стоишь ты этого, осьминожец! Да очень уж мне подарки твои понравились, особенно нож +1 кремневый, артефактный, великой древности.\par
\par
Путь твой далек будет. Мимо галактик в спирали, мимо планет в тентуре, мимо облаков водородных, дивным светом сияющих, мимо квазаров грозных, сигналы чудные излучающих, к самому краю космоса, где лишь протоны да альфа-частицы шныряют, а звезду и на миллион парсек не встретишь. И всё же есть там звездочка одна, в туманности газо-пылевой спрятавшаяся. А вокруг звезды той маленькая планета вращается, а вокруг планеты – спутник вертится. И на спутнике том кратер есть круглый да огромный. И на дне кратера того Врата стоят Звездные. И ведут эти Врата к тому, что ты найти так жаждешь.\par
\par
Но непросто до Врат добраться. Лабиринт вкруг тех Врат выстроен в сто этажей да в десять тысяч комнат на каждом. Да такой хитрый, что как войдешь в него, так и заблудишься сразу. И как сквозь тот лабиринт пройти, мне неведомо.\par
\par
Поэтому трудно тебе придется. Но коли сумеешь до Врат добраться и через них пройти, окажешься в комнатке маленькой. И будут пред тобой три двери – две больших да красивых, а третья – маленькая да невзрачная.\par
\par
На первой двери, золотом украшенной, нарисован будет ядерный котел над очагом звездным. За этой дверью Изменитель реальности, невесть кем построенный. Сунешь свой нос в эту дверь – на веки сгинешь. Но если уж не послушаешь моего совета и заглянешь туда – руками ничего не трогай, а то и вся Вселенная наша переиначиться может.\par
\par
На второй двери, алмазами выложенной, енот с пулеметом наклеен. За этой дверью биолаборатория заброшенная, в которой ксеноморфов да всяких хищников страшных выводили. Они и сейчас там бродят. Такому герою, как ты, туда зайти – головы не сносить. Но уж если не послушаешь доброго совета да заглянешь туда – из пробирок не пей – ксеноморфиком станешь, или шай-хулудом каким.\par
\par
На третьей двери, паутиной затянутой да звездной пылью засыпанной, ничего не нарисовано, только написано: «Не влезай, убьет!» За этой дверью найдешь ты машину вычислительную, сокровище, которое так найти хочешь.\par
\par
А чтоб с пути не сбиться, дам я тебе комету путеводную, куда она полетит – туда и ты лети».\par
\par
Пустился Станислав в путь. Летит он в гиперпространстве тропами нехожеными, измерениями неизвестными. Уж и не знает, сколько в нем самом теперь измерений осталось. Но не сдается, боится только одного – с пути сбиться.\par
\par
Сколько световых лет прошло, неведомо, но долетел Станислав до звездочки заветной. Рассчитал он курс, на нужную орбиту лег, к высадке подготовился.\par
\par
Высадился Станислав рядом с лабиринтом, добрался до входа и внутрь вошел. Идет одним коридором, другим, третьим. Пусто везде, тихо, только его шаги эхом отдаются. До комнаты какой-то дошел, видит – дальше три коридора ведут, и в каждом будто туман клубится. А на полу написано: «Коли дураком не хочешь стать – иди направо, либо прямо. Коли голову потерять не хочешь – иди прямо, либо налево. Коли до смерти запуганным быть не хочешь – иди налево, либо направо».\par
\par
Задумался Станислав, куда идти, да и пошел налево. Идет, по коридорам топает, с этажа на этаж перебирается. А туман вокруг не рассеивается. Долго он так шел, уж стал думать, что батарейки в часах наручных скоро сядут. Дошел, наконец, до какой-то комнаты, а в комнате той будто битва великая кипела, потолок осыпался, в полу дыра в пол-комнаты. Из комнаты две двери ведут. До одной не добраться, поперек другой дракон трехголовый спит, а рядом с ним меч здоровенный двуручный валяется. Схватил Станислав меч за рукоятку, а тот тяжеленный, по полу как заскрежещет. Дракон вмиг проснулся.\par
\par
– Ты кто такой? – спрашивает.\par
– Да вот, пройти хочу, – Станислав отвечает.\par
– А меч тебе зачем? Ты что, совсем дурак? Попросил бы – я б подвинулся.\par
– Да я его хотел через дыру в полу перекинуть – навроде мостика.\par
– А, так тебе в ту дверь надо? Впрочем, все равно дурак – меч размером коротковат.\par
– Да мне все равно, в какую дверь, мне б только до Звездных Врат добраться. И что вообще у вас тут творится?\par
– Игра у нас тут идет большая.\par
– Да что за игра-то?\par
– Да так… Впрочем, раз уж ты ко мне пришел, раунд за мной, идем – расскажу тебе про игру.\par
\par
Выходит дракон в дверь, становится слегка туманным и идет дальше коридорами. Станислав за ним еле поспевает. До очередной двери дошли, дракон – туда, и Станислав за ним. \par
Входит Станислав в комнату – а там нет никого, лишь три сгустка тумана поплотнее. И как будто двое одному что-то вроде монет туманных передают.\par
– И где же тут кто? – Станислав спрашивает.\par
– Да мы тут везде, но здесь особенно, – отвечают три голоса, да прямо в голове звучат.\par
– И кто же вы такие будете?\par
– Мы – представители древней цивилизации, раса наша настолько древняя, что телесно уже и не существует вовсе. Только ментально, то есть разумом своим. И знаем мы все тайны Вселенной, и все предсказать да рассчитать можем. И скучно нам от этого необычайно. Пробовали в рулетку играть – да каждый знает, куда шарик прикатится. Пробовали в квантовое лото играть – так и принцип неопределенности квантовый для нас не помеха в предсказаниях.\par
– А здесь-то вы что делаете?\par
– Воздвигли мы силой разума лабиринт этот, чтоб скуку развеять можно было. Разумные существа, сюда попавшие, выбор делают. А мы играем, смотрим, по чьей дороге они пойдут. Ибо обладают разумные существа свободой воли, и тут наши предсказания бессильны. Так что давай, хватит отдыхать – видишь, три коридора отсюда тянутся – иди уж по какому-нибудь, да побыстрее!\par
– Не нужны мне ваши коридоры, мне к Звездным Вратам нужно!\par
– Будь любезен, не упрямься. Мы легко тебя заставить можем. Создадим мы сей же час чудовищ ментальных, ты кое-кого видел уже, и ни бластер, ни водяной пистолет тебе не помогут, поскольку будут чудовища внутри твоего разума, а не снаружи. А во сне тебе и вовсе тяжко придется.\par
\par
Видит Станислав – со всех сторон к нему уже когти и щупальца тянутся. Забился он в угол, да как закричит:\par
– Остановитесь, погодите! А кто из вас этих чудовищ делает?\par
– Как кто? Мы все трое.\par
– А чьи чудовища самые сильные будут?\par
Тут замерли чудовища на мгновение, а потом как начали друг с другом биться. Лапы с хвостами в разные стороны так и разлетаются. Драконы с демонами сшибаются, ангелы с гигантскими червями, орки и гоблины с эльфами да рыцарями, кальмары огромные с василисками. И над всем этим пегасы да орнитоптеры парят, и молнии сверкают. А внизу горы какие-то да болота с лесами мелькают. Даже один раз черный лотос виден был. \par
– Стойте! – Станислав кричит. – Меня ж сейчас тут совсем затопчут!\par
– Уйди, не мешайся! – три голоса отвечают. – Видишь левый коридор, там третий поворот направо и два раза налево – и придешь к своим Вратам Звездным.\par
\par
Побежал Станислав что есть духу, в минуту до Врат добрался.\par
\par
Прошел Станислав сквозь Врата – видит, три двери перед ним. Вошел в нужную, несколько шагов прошел и слышит какой-то шорох сзади. Оглянулся – за ним Иммортий стоит, ухмыляется. \par
\par
– Не думал я, – говорит, – что сумеешь ты до Врат добраться, да всё же надеялся. Даже приемник телепортационный тебе в правый ботинок засунул. Очень уж мне машину вычислительную заполучить надо, гораздо нужнее, чем тебе. Так что медленно подними руки вверх и сделай три шага вперед, не мешайся.\par
– Ах ты, морда поганая, – Станислав отвечает, – обнаружил я твой приемник, когда ботинки чистил, знал, что ты недоброе задумал. Посмотри вокруг, в биолабораторию ты телепортировался. Чу, хищники зубами скрежещут, клювы разевают, щупальца расправляют! Не выйти тебе отсюда, не забрать машину вычислительную.\par
– Так, значит! – говорит Иммортий. – Что ж! Посмотрим, кто отсюда живым не выйдет!\par
\par
Хватает со стола пробирку и одним махом выпивает. Раз – и стоит вместо Иммортия лев фомальгаутский ядовитый, трех метров роста, к прыжку готовится, слюна с клыков капает. Не растерялся Станислав, тоже пробирку схватил, тоже выпил. Превратился в тираннозавра ригельского, махнул хвостом – лев от него на десять метров отлетел. Схватил лев с другого стола целую колбу, осушил одним махом, превратился в пчелу бронебойную с Канопуса 5, разогнался, тираннозавра насквозь пробил. Да успел тот на последнем издыхании до пробирки дотянуться, сжевал ее с содержимым вместе, превратился в броненосца адамантинового насекомоядного с Регула 6…\par
\par
В общем, долго они так развлекались, дня три, не меньше. Чуть не забыли, кто из них кто. Уж и пробирки-то почти все закончились. Схватил Иммортий последнюю пробирку, тут Станислав как закричит: «Стой, дурья твоя башка! А как мы в себя-то обратно превратимся?» Задумался Иммортий. «Всё из-за тебя, болван, – говорит. – Теперь всё с начала начинать придется!» Пробирку бросил, на щупальцах приподнялся и выбежал из лаборатории. Станислав за ним пополз.\par
\par
Выползает, смотрит – Иммортий в первую дверь, с очагом звездным, забежал. И Станислав туда направился.\par
\par
Глядит – механизм там стоит дивный, лампочками моргает, жужжит тихонько и готов в любую секунду реальность изменить. А Иммортий уже рычажки какие-то дергает и кнопки нажимает. Кинулся Станислав к Иммортию, остановить хотел, да не успел. Прошла рябь по Вселенной и исчезли оба, как будто не было их тут вовсе. И сказка эта совсем иная стала...\par

\chapter{}
 \lettrine{У}{некоторой звезды, на третьей от нее планете,} был один осьминожец, звали его Джо. Был он известный ученый и был на редкость умен.\par
\par
Как-то раз прочитал он в Интернете, что на одной затерянной в глубинах космоса, холодной и обледенелой планете есть скрытая библиотека с тайными знаниями обо всей Вселенной. И захотел тогда Джо ее себе добыть, чтобы не попалась она в чужие руки и не вышло большой беды для всей Вселенной. Но где искать библиотеку? Звезд да черных дыр в галактиках ужас как много, и вокруг каждой куча планет да астероидов вертится!\par
\par
Стал Джо смекать, у кого информацию нужную добыть можно. Думал-думал, и надумал обратиться к мудрецу с планеты Ностродамия по имени Завздыпопус, славившемуся своими познаниями о космосе. Собрал он все свои деньги и драгоценности, а их он копил во множестве, ибо любил очень, и опрометью помчался к Завздыпопусу.\par
\par
Через некоторое время сумел Джо с ним повстречаться. Так и так, сказывает Джо, очень хочется мне отыскать библиотеку, сокровище это великое, и все помыслы мои теперь только об этом. Только вот закавыка – знать не знаю, как!\par
\par
«Вот что, – отвечает ему Завздыпопус, – вижу я, ты необычайно решителен в своем намерении, и IQ твой вышиной до звезд простирается! Впрочем, ты осьминожец, а осьминожцы этим славятся, да еще расторопностью своей. Но чтобы я тебе помогать стал, тебе службу сослужить надобно. Сумеешь задание мое исполнить – расскажу, как найти библиотеку, а нет – не обессудь!\par
\par
Внемли! Есть недалеко тут звезда нейтронная – третий поворот направо, если держать курс на Большую Медведицу. Рядом с ней планетоид карликовый, а на планетоиде том два чудовища живут. Зовут их Диспрозий и Гадолиний, злобные они до ужаса, и до самых кончиков клыков темной энергией пропитаны. Есть у них гусли со струнами космическими, такие, что каждый, кто их услышит, от радости гиперпрыжки да танцы начинает выделывать, которые другим и не снились. Пуще жизни чудовища их любят. Добудь мне гусли, и расскажу я тебе, где найти библиотеку».\par
\par
Закручинился Джо, да делать нечего. Полетел к чудовищам гусли добывать. Летит, а сам боится, мелкой дрожью дрожит, даже поворот нужный пропустил – пришлось возвращаться. А когда возвращался, увидал пустую канистру из-под топлива термоядерного, кем-то выброшенную. Возьму, думает, ее с собой – вдруг пригодится. Летит дальше – видит, кусок темной материи в пространстве висит, весь скомканный – наверное, купец какой-то потерял. И его, думает, возьму, тоже может на что сгодиться. Дальше летит – видит, воронка гравитационная валяется – должно быть, странники какие-то оставили. И её тоже взял.\par
\par
Прилетает, садится на планетоид, заходит во дворец к чудовищам, а те сразу к нему кидаются. \par
– Чу, – говорят, – осьминожьим духом пахнет! Как звать, – спрашивают, – откуда? Как предпочитаешь быть съеденным – с головы или с ног? Да не тяни с ответами, а то проголодались мы чудовищно – лет сто уж не ели – в животах бурчит.\par
– Да погодите вы с формальностями! – Джо отвечает. – Слышал я, есть у вас гусли со струнами космическими, инструмент дивный. Сменяйте мне их на что-нибудь – очень уж мне надо.\par
– Ты что – с Денеба рухнул? – спрашивают чудища. – Мы ж инструмент этот день и ночь стережем от посторонних.\par
– Что ж, – говорит Джо, – тогда дайте на инструмент этот ваш посмотреть хотя бы, а я вам за это воронку гравитационную подарю.\par
– А на что это нам? – спрашивают.\par
– Так вы тогда что угодно во что угодно засунуть сможете. Даже вы двое в этой вот канистре поместитесь. \par
– Быть такого не может!\par
– Дайте посмотреть гусли – увидите.\par
\par
Повели чудища его в самый дальний угол самой далекой пещеры, где гусли хранили. Посмотрел Джо на гусли. «Что ж, – говорит, – теперь вы смотрите». Приладил воронку гравитационную к канистре и велел чудовищам туда прыгать. Прыгнули они, сидят в канистре, удивляются, что еще места много осталось. А Джо заткнул канистру куском темной материи, и даже бантик завязал.\par
\par
Взял он гусли, закинул канистру подальше в глубины космоса и бегом к Завздыпопусу. Завздыпопус обрадовался: «Ох, удружил ты мне, – говорит. – Расскажу я теперь тебе, как найти библиотеку.\par
\par
Путь твой далек будет. Мимо галактик в спирали, мимо планет в тентуре, мимо облаков водородных, дивным светом сияющих, мимо квазаров грозных, сигналы чудные излучающих, к самому краю видимой Вселенной, где лишь протоны да альфа-частицы шныряют, а звезду и на миллион парсек не встретишь. И всё же есть там звездочка одна, в туманности газо-пылевой спрятавшаяся. А вокруг звезды той маленькая планета вращается, а вокруг планеты – спутник вертится. И на спутнике том кратер есть круглый да огромный. И на дне кратера того Врата стоят Звездные. И ведут эти Врата к тому, что ты найти так жаждешь.\par
\par
Но непросто до Врат добраться. Лабиринт вкруг тех Врат выстроен в сто этажей да в десять тысяч комнат на каждом. Да такой хитрый, что как войдешь в него, так и заблудишься сразу. И как сквозь тот лабиринт пройти, мне неведомо.\par
\par
Но коли ты жив останешься да сумеешь до Врат добраться и через них пройти, окажешься в комнатке маленькой. И будут пред тобой три двери – две больших да красивых, а третья – маленькая да невзрачная.\par
\par
На первой двери, серебром украшенной, нарисован будет ядерный котел над очагом звездным. За этой дверью Портал сияющий, в параллельные вселенные ведущий. Сунешь свой нос в эту дверь – на веки сгинешь. Но если уж не послушаешь моего совета и заглянешь туда – руками ничего не трогай, а то и вся Вселенная наша переиначиться может.\par
\par
На второй двери, алмазами выложенной, жаба в скафандре наклеена. За этой дверью биолаборатория заброшенная, в которой ксеноморфов да всяких хищников страшных выводили. Они и сейчас там бродят. Такому герою, как ты, туда зайти – головы не сносить. Но уж если не послушаешь доброго совета да заглянешь туда – из пробирок не пей – ксеноморфиком станешь, или шай-хулудом каким.\par
\par
На третьей двери, паутиной затянутой да звездной пылью засыпанной, ничего не нарисовано, только написано: «Добро пожаловать!» За этой дверью найдешь ты библиотеку, сокровище, которое так найти жаждешь.\par
\par
А чтоб ты с пути не сбился, дам я тебе лазерную указку волшебную, куда она покажет – туда ты и направляйся».\par
\par
Тут Джо, не мешкая, в путь пустился. Летит он в подпространстве гипертоннелями, которые, не иначе, какие-то гиперкроты вырыли, летит измерениями неизвестными. Уж и не знает, сколько в нем самом теперь измерений осталось. Но не сдается, боится только одного – наизнанку вывернуться.\par
\par
Долго ли, коротко ли, долетел Джо до звездочки заветной. Рассчитал он курс, на нужную орбиту лег, к высадке подготовился.\par
\par
Высадился Джо рядом с лабиринтом, добрался до входа и внутрь вошел. Идет одним коридором, другим, третьим. Пусто везде, тихо, только его шаги эхом отдаются. До комнаты какой-то дошел, видит – дальше три коридора ведут, и в каждом будто туман клубится. А на полу написано: «Коли дураком не хочешь стать – иди направо, либо прямо. Коли голову потерять не хочешь – иди прямо, либо налево. Коли до смерти запуганным быть не хочешь – иди налево, либо направо».\par
\par
Задумался Джо, куда идти, да и пошел прямо. Идет, а в тумане уж и ничего не видать почти. Еле успевает в повороты заворачивать, да об лестницы чуть не спотыкается. И все страшнее вокруг делается. То будто летучая мышь над головой пролетит, то паутину какую-то в руку толщиной перешагивать приходится. То вдруг щупальце за ногу будто хватает да дергает. И не поймешь, то ли щупальце, то ли об лестницу споткнулся.\par
\par
Долго так Джо в тумане пробирался, совсем устал, да и от страха дрожит – зуб на зуб не попадает. Увидел комнату какую-то, зашел в нее, дверь закрыл поплотнее и прямо на пол улегся – отдохнуть немного. Вдруг слышит – голос, прямо у себя в голове: «Хорошо, что сюда пришел. Молодец! За мной этот раунд». Джо так на месте и подскочил. \par
\par
– Кто здесь? – спрашивает. – Что надо? Что за раунд и за кем он вообще может быть?\par
– Да не волнуйся, – говорит голос, – здесь я. Раунд – в большой игре, нас тут трое играет. Заходи в соседнюю комнату, мы тебе все объясним.\par
 \par
Входит Джо в комнату – а там нет никого, лишь три сгустка тумана поплотнее. И как будто двое одному что-то вроде монет туманных передают.\par
– И где же тут кто? – Джо спрашивает.\par
– Да мы тут везде, но здесь особенно, – отвечают три голоса, да прямо в голове звучат.\par
– И кто же вы такие будете?\par
– Мы – представители древней цивилизации, раса наша настолько древняя, что телесно уже и не существует вовсе. Только ментально, то есть разумом своим. И знаем мы все тайны Вселенной, и все предсказать да рассчитать можем. И скучно нам от этого необычайно. Пробовали в рулетку играть – да каждый знает, куда шарик прикатится. Пробовали в квантовое лото играть – так и принцип неопределенности для нас не помеха в предсказаниях.\par
– А здесь-то вы что делаете?\par
– Воздвигли мы силой разума лабиринт этот, чтоб скуку развеять можно было. Разумные существа, сюда зашедшие, выбор делают. А мы играем, ставки делаем, по чьей дороге они пойдут. Ибо обладают разумные существа свободой воли, и тут наши предсказания бессильны. Так что давай, хватит отдыхать – видишь, три коридора отсюда тянутся – иди уж по какому-нибудь, да побыстрее!\par
– Не нужны мне ваши коридоры, мне к Звездным Вратам нужно!\par
– Будь любезен, не упрямься. Мы легко тебя заставить можем. Создадим мы сей же час чудовищ ментальных, ты кое-кого видел уже, и ни бластер, ни водяной пистолет тебе не помогут, поскольку будут чудовища внутри твоего разума, а не снаружи. А во сне тебе и вовсе тяжко придется.\par
\par
Видит Джо – со всех сторон к нему уже когти и щупальца тянутся. Забился он в угол, да как закричит:\par
– Стойте, стойте, погодите! А кто из вас этих чудовищ делает?\par
– Как кто? Мы все трое.\par
– А чьи чудовища самые сильные будут?\par
Тут замерли чудовища на мгновение, а потом как начали друг с другом биться. Лапы с хвостами в разные стороны так и разлетаются. Драконы с демонами сшибаются, ангелы с гигантскими червями, орки и гоблины с эльфами да рыцарями, кальмары огромные с василисками. И над всем этим пегасы да орнитоптеры парят, и молнии сверкают. А внизу горы какие-то да болота с лесами мелькают. Даже один раз черный лотос виден был. \par
– Стойте! – Джо кричит. – Меня ж сейчас тут совсем затопчут!\par
– Уйди, не мешайся! – три голоса отвечают. – Видишь левый коридор, там третий поворот направо и два раза налево – и придешь к своим Вратам Звездным.\par
\par
Побежал Джо что есть духу, в минуту до Врат добрался.\par
\par
Прошел Джо сквозь Врата – видит, три двери перед ним. Убрал он паутину с самой маленькой, пыль звездную с нее отряхнул да внутрь вошел. Осмотрелся и увидел библиотеку, то сокровище, из-за которого покоя лишился. Бросился Джо к сокровищу своему, тут вдруг сзади покашливание какое-то раздалось. Оглянулся – позади Завздыпопус с пола встает.\par
– Ох! – говорит Завздыпопус. – Хорошо, что ты сюда добрался, а то раньше никому не удавалось, я уж и со счета сбился.\par
– Завздыпопус! Ты-то здесь откуда? Да еще и на полу отдыхаешь.\par
– Так я ж тебе к правому ботинку микротелепортационный приемник прицепил, ну и 3D-видеокамеру с квадрофоническим микрофоном в придачу. Хоть и не верилось, что ты сюда доберешься. Да уж больно мне библиотеку раздобыть надо было. Пришлось вот даже ползком телепортироваться – слишком уж маленький портальчик получился.\par
– Погоди! Это мне ее раздобыть надо было! Вот я здесь и оказался.\par
– Ты уж извини, Джо, только мне это нужнее, – говорит Завздыпопус. Выхватывает парализатор и стреляет. Джо сразу окаменел, ни рукой, ни ногой двинуть не может. Языком еле шевелит.\par
– Ой, – говорит, – ты что, супостат, делаешь?!\par
– Да я тебя в лабораторию ближайшую сейчас сдам – для опытов. Чтоб под ногами не путался.\par
Схватил Завздыпопус Джо за шиворот и потащил в биолабораторию по соседству.\par
\par
Затащил он Джо в дальний угол лаборатории, бросил там и к выходу направился. Да обо что-то вроде зеленого кабеля споткнулся. Тут сверху огромный цветок зубастый как упадет, Завздыпопус вмиг внутри цветка оказался, мычит что-то, ничего не разобрать.\par
\par
– Это что ж такое?! – Джо спрашивает.\par
– Это я, растение говорящее, – голос отвечает.\par
– Да откуда ж ты взялось?\par
– Люди в белых халатах говорили, что их генетический эксперимент удался. И что это поможет им в борьбе с антаранцами.\par
– А что еще они говорили?\par
– Последние их слова были: «А где Орибазий? И Эвтаназий?»\par
– Слушай, выплюнь ты Завздыпопуса, а то тебе плохо будет. Он гербицид.\par
– Что он делает?\par
– Гербицид – для растений ядовит.\par
– Откуда ты знаешь? Да и вообще, кто ты такой?\par
– Да я тоже растение, куст говорящий. Видишь, шевелиться не могу. А этот человек меня поисследовать хотел. Ну, теперь я его поисследую, чтоб не важничал. Сейчас, погоди, проросту только немного.\par
– Хороший ты куст, тихий. И разговаривать умеешь. Ладно, на, исследуй свой гербицид.\par
\par
Распахнулся цветок, Завздыпопус оттуда вывалился, еле дышит. Джо подождал, пока руки-ноги двигаться смогут, схватил Завздыпопуса, да бегом из лаборатории. За дверь выбежал, остановился, повернулся к Завздыпопусу. «Что – говорит – довыпендривался? Твое счастье, что я по вторникам кровавых жертв не приношу. До завтра подождем». Да как двинет Завздыпопуса в ухо.\par
\par
– Стой, погоди, Джо! – кричит Завздыпопус. – Осознал я свою ошибку! Давай с начала начнем.\par
– Я тебе покажу с начала! Сейчас еще раз тресну!\par
\par
Совсем перепугался тут Завздыпопус, заметался, убежать старается. А Джо не отстает, того и гляди догонит и еще раз двинет. Подбежал Завздыпопус к первой двери, с котлом ядерным, и шасть за нее. И Джо за ним.\par
\par
Глядит – портал в параллельные миры посреди комнаты сияет и Завздыпопус к нему бежит. Бросился Джо за Завздыпопусом, чтобы остановить, схватил крепко. Да изловчился Завздыпопус, качнулся, и рухнули они оба в портал, в другой Вселенной оказались. А в ней и история эта совсем по-другому сказывается...\par

\chapter{}
 \lettrine{Н}{а одной космической станции, которых много на просторах нашей Вселенной,} был один насекомец, звали его Игнат. Был он великий герой и был на редкость умен.\par
\par
В один прекрасный день узнал он, что в одной звездной системе, на маленьком астероиде находится великая вычислительная машина, знающая ответы на все вопросы. И решил Игнат ее разыскать для себя, чтобы не попалась она в чужие руки и не вышло большой беды для всей Вселенной. Да как узнать, где в точности найти машину вычислительную? Звезд да черных дыр в галактиках ужас как много, и вокруг каждой великое множество планет да астероидов вертится!\par
\par
Принялся Игнат думать, у кого совета спросить. Думал-думал, и надумал обратиться к робостарцу Вегианскому, Иммортию, коий был старше самой Галактики, и помнил еще Большой Взрыв, при котором был ребенком. Взял он свой бластер верный и опрометью помчался к Иммортию.\par
\par
Вскорости сумел Игнат с ним повстречаться. Так, мол, и так, сказывает Игнат, очень хочется мне найти машину вычислительную, сокровище это удивительное, и все помыслы мои теперь лишь об этом. Только вот загвоздка – знать не знаю, как!\par
\par
«Послушай, – отвечает ему Иммортий, – вижу я, ты очень доблестен, да и смекалист очень! Впрочем, ты насекомец, а насекомцы этим славятся, да еще упертостью своей. Только не знаю я, как помочь тебе. Но есть сестра у меня, мудрости столь необычной, что моя мудрость по сравнению с её – логарифмическая линейка по сравнению с суперкомпьютером. Живет она у соседней звезды, второй поворот налево, если отсюда к краю Галактики лететь. Принеси ей подарков да украшений дорогих – может, поможет она тебе».\par
\par
Собрал Игнат с собой подарки да украшения, добавил к ним плащ Арктурианский, из ткани реальности сделанный, скоплениями галактик вышитый, и в путь пустился.\par
\par
Прилетает он к сестре Иммортия, отдает ей подарки, и письмо от Иммортия вручает. «Так и так, – сказывает Игнат, – очень хочется мне найти машину вычислительную, сокровище это великое, и побуждения мои самые что ни на есть благородные.»\par
\par
«Стой-ка, – говорит сестра, – тут в письме сказано, что б я тебе голову тупым топором отрубила, а не помогать стала!.. Ах нет, извини, просто письмо с другой стороны оборотки какой-то написано, там даже печать есть... Ладно, помогу я тебе, хоть и не стоишь ты этого, насекомец! Да очень уж мне подарки твои понравились, особенно плащ Арктурианский, из ткани реальности сделанный, скоплениями галактик вышитый.\par
\par
Далеко отсюда твой путь лежит. Мимо галактик в спирали, мимо планет в тентуре, мимо туманностей звездных, дивным светом сияющих, мимо квазаров грозных, сигналы чудные излучающих, к самому краю космоса, где лишь протоны да альфа-частицы шныряют, а звезду и на миллион парсек не встретишь. И всё же есть там звездочка одна, молодая да пригожая. А вокруг звезды той огромная планета вращается, а вокруг планеты – спутник вертится. И на спутнике том кратер есть круглый да огромный. И на дне кратера того Врата стоят Звездные. И ведут эти Врата к тому, что ты найти так жаждешь.\par
\par
Только просто так во Врата не пройти. Лабиринт вкруг тех Врат выстроен в сто этажей да в десять тысяч комнат на каждом. Да такой хитрый, что как войдешь в него, так и заблудишься сразу. И как сквозь тот лабиринт пройти, мне неведомо.\par
\par
Но коли ты жив останешься да сумеешь до Врат добраться и через них пройти, окажешься в комнатке маленькой. И будут пред тобой три двери – две больших да красивых, а третья – маленькая да невзрачная.\par
\par
На первой двери, серебром украшенной, нарисован будет ядерный котел над очагом звездным. За этой дверью Темпор, аномалия чудесная, то ли пространственно-времянная, то ли температурно-пространственная. Сунешь свой нос в эту дверь – на веки сгинешь. Но если уж не послушаешь моего совета и заглянешь туда – руками ничего не трогай, а то и вся Вселенная наша переиначиться может.\par
\par
На второй двери, алмазами выложенной, жаба в скафандре наклеена. За этой дверью биолаборатория заброшенная, в которой ксеноморфов да всяких хищников страшных выводили. Они и сейчас там бродят. Такому герою, как ты, туда зайти – головы не сносить. Но уж если не послушаешь доброго совета да заглянешь туда – из пробирок не пей – ксеноморфиком станешь, или шай-хулудом каким.\par
\par
На третьей двери, паутиной затянутой да звездной пылью засыпанной, ничего не нарисовано, только написано: «Не влезай, убьет!» За этой дверью найдешь ты машину вычислительную, сокровище, которое так обрести жаждешь.\par
\par
А чтоб ты с пути не сбился, дам я тебе лазерную указку волшебную, куда она покажет – туда ты и направляйся».\par
\par
Пустился Игнат в путь. Летит он в подпространстве тоннелями неведомыми, измерениями неизвестными. Уж и не знает, трехмерный ли он до сих пор. Но не сдается, боится только одного – наизнанку вывернуться.\par
\par
Долго ли, коротко ли, долетел Игнат до звездочки заветной. Рассчитал он курс, в посадочный модуль залез, к высадке подготовился.\par
\par
Высадился Игнат рядом с лабиринтом, добрался до входа и внутрь вошел. Идет одним коридором, другим, третьим. Пусто везде, тихо, только его шаги эхом отдаются. До комнаты какой-то дошел, видит – дальше три коридора ведут, и в каждом будто туман клубится. А на полу написано: «Коли дураком не хочешь стать – иди направо, либо прямо. Коли голову потерять не хочешь – иди прямо, либо налево. Коли до смерти запуганным быть не хочешь – иди налево, либо направо».\par
\par
Задумался Игнат, куда идти, да и пошел прямо. Идет, а в тумане уж и ничего не видать почти. Еле успевает в повороты заворачивать, да об лестницы чуть не спотыкается. И все страшнее вокруг делается. То будто летучая мышь над головой пролетит, то паутину какую-то в руку толщиной перешагивать приходится. То вдруг щупальце за ногу будто хватает да дергает. И не поймешь, то ли щупальце, то ли об лестницу споткнулся.\par
\par
Долго так Игнат в тумане пробирался, совсем устал, да и от страха дрожит – зуб на зуб не попадает. Увидел комнату какую-то, зашел в нее, дверь закрыл поплотнее и прямо на пол улегся – отдохнуть немного. Вдруг слышит – голос, прямо у себя в голове: «Хорошо, что сюда пришел. Молодец! За мной этот раунд». Игнат так на месте и подскочил. \par
\par
– Кто здесь? – спрашивает. – Что надо? Что за раунд и за кем он вообще может быть?\par
– Да не волнуйся, – говорит голос, – здесь я. Раунд – в большой игре, нас тут трое играет. Заходи в соседнюю комнату, мы тебе все объясним.\par
 \par
Входит Игнат в комнату – а там нет никого, лишь три сгустка тумана поплотнее. И как будто двое одному что-то вроде монет туманных передают.\par
– И где же тут кто? – Игнат спрашивает.\par
– Да мы тут везде, но здесь особенно, – отвечают три голоса, да прямо в голове звучат.\par
– И кто же вы такие будете?\par
– Мы – представители древней цивилизации, раса наша настолько древняя, что телесно уже и не существует вовсе. Только ментально, то есть разумом своим. И знаем мы все тайны Вселенной, и все предсказать да рассчитать можем. И скучно нам от этого необычайно. Пробовали в рулетку играть – да каждый знает, куда шарик прикатится. Пробовали в квантовое лото играть – так и принцип неопределенности квантовый для нас не помеха в предсказаниях.\par
– А здесь-то вы что делаете?\par
– Воздвигли мы силой разума лабиринт этот, чтоб скуку развеять можно было. Разумные существа, сюда зашедшие, выбор делают. А мы играем, ставки делаем, по чьей дороге они пойдут. Ибо обладают разумные существа свободой воли, и тут наши предсказания бессильны. Так что давай, хватит отдыхать – видишь, три коридора отсюда тянутся – иди уж по какому-нибудь, да побыстрее!\par
– Не нужны мне ваши коридоры, мне к Звездным Вратам нужно!\par
– Будь любезен, не упрямься. Мы легко тебя заставить можем. Создадим мы сей же час чудовищ ментальных, ты кое-кого видел уже, и ни бластер, ни водяной пистолет тебе не помогут, поскольку будут чудовища внутри твоего разума, а не снаружи. А во сне тебе и вовсе тяжко придется.\par
\par
Видит Игнат – со всех сторон к нему уже пасти и щупальца тянутся. Забился он в угол, да как закричит:\par
– Стойте, стойте, погодите! А кто из вас этих чудовищ делает?\par
– Как кто? Мы все трое.\par
– А чьи чудовища самые сильные будут?\par
Тут замерли чудовища на мгновение, а потом как начали друг с другом биться. Лапы с хвостами в разные стороны так и разлетаются. Драконы с демонами сшибаются, ангелы с гигантскими червями, орки и гоблины с эльфами да рыцарями, кальмары огромные с василисками. И над всем этим пегасы да орнитоптеры парят, и молнии сверкают. А внизу горы какие-то да болота с лесами мелькают. Даже один раз черный лотос виден был. \par
– Стойте! – Игнат кричит. – Меня ж сейчас тут совсем затопчут!\par
– Уйди, не мешайся! – три голоса отвечают. – Видишь левый коридор, там третий поворот направо и два раза налево – и придешь к своим Вратам Звездным.\par
\par
Побежал Игнат что есть духу, в минуту до Врат добрался.\par
\par
Прошел Игнат сквозь Врата – видит, три двери перед ним. Убрал он паутину с самой маленькой, пыль звездную с нее отряхнул да внутрь вошел. Осмотрелся и увидел машину вычислительную, то сокровище, к которому так стремился. Бросился Игнат к сокровищу своему, тут вдруг сзади покашливание какое-то раздалось. Обернулся – позади Иммортий с пола поднимается.\par
– Ох! – говорит Иммортий. – Хорошо, что ты сюда добрался, а то раньше никому не удавалось, я уж и со счета сбился.\par
– Иммортий! Ты-то здесь откуда? Да еще и на полу отдыхаешь.\par
– Так я ж тебе к правому ботинку микротелепортационный приемник прицепил, ну и 3D-видеокамеру с квадрофоническим микрофоном в придачу. Хоть и не верилось, что ты сюда доберешься. Да уж больно мне машину вычислительную раздобыть надо было. Пришлось вот даже ползком телепортироваться – слишком уж маленький портальчик сделался.\par
– Стоп! Это мне ее раздобыть надо было! Вот я здесь и оказался.\par
– Ты уж извини, Игнат, только мне это нужнее, – говорит Иммортий. Выхватывает петрификатор и стреляет. Игнат сразу на пол шлепнулся, ни рукой, ни ногой двинуть не может. Языком еле шевелит.\par
– Ой, – говорит, – ты что, супостат, делаешь?!\par
– Да я тебя в лабораторию ближайшую сейчас сдам – для опытов. Чтоб под ногами не путался.\par
Схватил Иммортий Игната за шиворот и потащил в биолабораторию по соседству.\par
\par
Затащил он Игната в лабораторию, бросил там и удалился важно. Лежит Игнат, пошевелиться не может, ждет, когда им завтракать придут. Или обедать – не знает, что и лучше. \par
Смотрит – через дальнюю дверь стадо овец входит. Все белые, только одна черная, по крайней мере, с одной стороны. Для политкорректности. Шерсть на овцах дыбом стоит и искры по шерсти бегают размером со спаниеля. Какая-то тощая тварь с потолка попыталась на них напрыгнуть, да ее на лету молнией сшибло.\par
\par
«Эге, – думает Игнат, – это ж прямо электроовцы какие-то. У меня и часы от них остановились, похоже. И сервопривод шнурков в ботинках отключился. Экое абсолютное оружие. Как бы мне его себе приручить». Тут чувствует – руки-ноги опять шевелиться могут. Почесал Игнат в затылке, огляделся повнимательней. Снял со стены диаграмму не пойми чего огромную, быстренько на обратной стороне картинку нарисовал, дырку в середине проделал и надел на себя через голову. Стал Игнат похож на рекламный щит ходячий, человека-бутерброд, которого как-то в космопорту видел. Только Игнат стал человек-ворота. Стадо овец как его увидело – сразу побежало на новые ворота смотреть. А Игнат пошел к Иммортию.\par
\par
Приходит, видит – Иммортий портал для обратной телепортации готовит, а роботопомощники вокруг так и кишат. И Иммортий его увидел. «Не думал, – говорит, – что ты выбраться сможешь. Ну да неважно». И приказывает роботопомощникам очистить помещение от посторонних. Но не тут-то было. Роботы все поотключались, портал к овцам притянулся и вместе с ними схлопнулся. А у Иммортия шнурки развязались.\par
\par
Подбежал Игнат к Иммортию, хотел стукнуть как следует, да увернулся Иммортий, из ботинок выскочил и наутек кинулся. «Вся матрица моих надежд рухнула! – кричит. – Перезагрузка! Только она мне поможет!» Выбежал из двери, к другой двери подбежал, с котлом ядерным, и шасть за нее. Игнат за ним кинулся, хоть и отстал чуток.\par
\par
Глядит – Темпор посреди комнаты сияет, аномалия чудесная, и Иммортий к нему бежит. Бросился Игнат за Иммортием, чтобы остановить, схватил крепко. Да изловчился Иммортий, качнулся, и рухнули они оба в Темпор, в параллельной Вселенной оказались. А в ней и история эта совсем другая...\par

\chapter{}
 \lettrine{Д}{авным-давно} жил-был один гуманоид по имени Грыблозавр. Был он лучший из лучших исследователь космоса, но ума при этом был небольшого.\par
\par
И вот прочитал он в Интернете, что на одной затерянной в глубинах космоса, холодной и обледенелой планете спрятан кварк-глюонный плазмомёт силы необычайной. И решил тогда Грыблозавр его найти во что бы то ни стало, чтобы поделиться им когда-нибудь потом со всеми жителями Галактики. Да как разыскать плазмомёт кварк-глюонный? Звезд да черных дыр во Вселенной ужас как много, и вокруг каждой куча планет да астероидов вертится!\par
\par
Начал Грыблозавр думать, у кого совета спросить. Думал-думал, да надумал обратиться к мудрецу с планеты Ностродамия по имени Завздыпопус, славившемуся своими познаниями о космосе. Взял он свой электробаян, с коим любил коротать время, и помчался искать встречи с Завздыпопусом.\par
\par
Вскорости нашелся способ повстречаться с Завздыпопусом. Так, мол, и так, сказывает Грыблозавр, хочется мне отыскать плазмомёт кварк-глюонный, сокровище это великое, и побуждения мои самые что ни на есть благородные. Только вот загвоздка – понятия не имею, как!\par
\par
«Что ж, – отвечает ему Завздыпопус, – вижу я, ты весьма ловок, но глуп как пробка! Впрочем, ты гуманоид, а гуманоиды этим известны, да еще прытью своей. Но чтобы я тебе помочь захотел, тебе службу сослужить надобно. Сделаешь всё как надо – расскажу, как найти плазмомёт кварк-глюонный, а нет – не обессудь!\par
\par
Внемли! Есть недалеко тут звезда нейтронная – третий поворот направо, если держать курс на Большую Медведицу. Рядом с ней планетоид карликовый, а на планетоиде том два чудовища живут. Зовут их Диспрозий и Гадолиний, свирепые они до ужаса, и от радиоактивности своей так и светятся. Есть у них гусли со струнами космическими, такие, что каждый, кто их услышит, от радости гиперпрыжки да танцы начинает выделывать, которые другим и не снились. Пуще жизни чудовища их любят. Добудь мне гусли, и расскажу я тебе, где найти плазмомёт кварк-глюонный».\par
\par
Закручинился Грыблозавр, да делать нечего. Полетел к чудовищам гусли добывать. Летит, а сам боится, мелкой дрожью дрожит, даже поворот нужный пропустил – пришлось возвращаться. А когда возвращался, увидал пустую канистру из-под топлива термоядерного, кем-то выброшенную. Возьму, думает, ее с собой – вдруг пригодится. Летит дальше – видит, кусок темной материи в пространстве висит, весь скомканный – наверное, купец какой-то потерял. И его, думает, возьму, тоже может на что сгодиться. Дальше летит – видит, компрессор пространственно-временной валяется – должно быть, странники какие-то оставили. И его тоже взял.\par
\par
Прилетает, садится на планетоид, заходит в пещеру к чудовищам, а те сразу к нему кидаются. \par
– Чу, – рычат, – гуманоидным духом пахнет! Кто такой, – спрашивают, – откуда? Как предпочитаешь быть съеденным – с головы или с ног? Да не тяни с ответами, а то проголодались мы страшно – лет сто уж не ели – в животах бурчит.\par
– Да погодите вы с формальностями! – Грыблозавр отвечает. – Слышал я, есть у вас гусли со струнами космическими, инструмент дивный. Сменяйте мне их на что-нибудь – очень уж мне надо.\par
– Ты что – с Денеба рухнул? – спрашивают чудища. – Инструмент нам этот очень дорог.\par
– Ладно, – говорит Грыблозавр, – тогда дайте на инструмент этот ваш хоть одним глазком взглянуть, а я вам за это компрессор пространственно-временной подарю.\par
– А на что это нам? – спрашивают.\par
– Вы же тогда что угодно во что угодно засунуть сможете. Даже вы двое в этой вот канистре поместитесь. \par
– Быть такого не может!\par
– Покажите гусли – увидите.\par
\par
Повели чудища его в самый дальний угол самой далекой пещеры, где гусли хранили. Посмотрел Грыблозавр на гусли. «Что ж, – говорит, – теперь вы смотрите». Приладил компрессор пространственно-временной к канистре и велел чудовищам туда прыгать. Прыгнули они, сидят в канистре, удивляются, что еще места много осталось. А Грыблозавр заткнул канистру куском темной материи, и даже бантик завязал.\par
\par
Взял он гусли, закинул канистру подальше в глубины космоса и бегом к Завздыпопусу. Завздыпопус обрадовался: «Ох, удружил ты мне, – говорит. – Расскажу я теперь тебе, как найти плазмомёт кварк-глюонный.\par
\par
Путь твой далек будет. Мимо галактик в спирали, мимо планет в тентуре, мимо туманностей звездных, дивным светом сияющих, мимо квазаров грозных, сигналы чудные излучающих, к самому краю космоса, где лишь протоны да альфа-частицы шныряют, а звезду и на миллион парсек не встретишь. И всё же есть там звездочка одна, молодая да пригожая. А вокруг звезды той маленькая планета вращается, а вокруг планеты – спутник вертится. И на спутнике том кратер есть круглый да огромный. И на дне кратера того Врата стоят Звездные. И ведут эти Врата к тому, что ты найти так жаждешь.\par
\par
Но непросто до Врат добраться. Лабиринт вкруг тех Врат выстроен в сто этажей да в десять тысяч комнат на каждом. Да такой хитрый, что как войдешь в него, так и заблудишься сразу. И как сквозь тот лабиринт пройти, мне неведомо.\par
\par
Поэтому трудно тебе придется. Но коли сумеешь до Врат добраться и через них пройти, окажешься в зале с потолком таким высоким, что и не видно. И будут пред тобой три двери – две больших да красивых, а третья – маленькая да невзрачная.\par
\par
На первой двери, иридием украшенной, нарисован будет ядерный котел над очагом звездным. За этой дверью Темпор, аномалия чудесная, то ли пространственно-времянная, то ли температурно-пространственная. Сунешь свой нос в эту дверь – на веки сгинешь. Но если уж не послушаешь моего совета и заглянешь туда – руками ничего не трогай, а то и вся Вселенная наша переиначиться может.\par
\par
На второй двери, алмазами выложенной, кот в сапогах нарисован. За этой дверью биолаборатория заброшенная, в которой ксеноморфов да всяких хищников страшных выводили. Они и сейчас там бродят. Такому герою, как ты, туда зайти – головы не сносить. Но уж если не послушаешь доброго совета да заглянешь туда – из пробирок не пей – ксеноморфиком станешь, или шай-хулудом каким.\par
\par
На третьей двери, паутиной затянутой да звездной пылью засыпанной, ничего не нарисовано, только написано: «http://-Ссылка на генератор-!» За этой дверью найдешь ты плазмомёт кварк-глюонный, сокровище, которое так найти хочешь.\par
\par
А чтоб ты с пути не сбился, дам я тебе лазерную указку волшебную, куда она покажет – туда ты и направляйся».\par
\par
Тут Грыблозавр, не мешкая, в путь пустился. Летит он в гиперпространстве тропами нехожеными, измерениями неизвестными. Уж и не знает, сколько в нем самом теперь измерений осталось. Но не сдается, боится только одного – наизнанку вывернуться.\par
\par
Сколько световых лет прошло, неведомо, но долетел Грыблозавр до звездочки заветной. Рассчитал он курс, на нужную орбиту лег, к высадке подготовился.\par
\par
Высадился Грыблозавр рядом с лабиринтом, добрался до входа и внутрь вошел. Идет одним коридором, другим, третьим. Пусто везде, тихо, только его шаги эхом отдаются. До комнаты какой-то дошел, видит – дальше три коридора ведут, и в каждом будто туман клубится. А на полу написано: «Коли дураком не хочешь стать – ступай направо, либо прямо. Коли голову потерять не хочешь – ступай прямо, либо налево. Коли до смерти запуганным быть не хочешь – ступай налево, либо направо».\par
\par
Задумался Грыблозавр, куда идти, да и пошел налево. Идет, по коридорам топает, с этажа на этаж перебирается. А туман вокруг не рассеивается. Долго он так шел, уж стал думать, что батарейки в часах наручных скоро сядут. Дошел, наконец, до какой-то комнаты, а в комнате той будто битва великая кипела, потолок осыпался, в полу дыра в пол-комнаты. Из комнаты две двери ведут. До одной не добраться, поперек другой дракон трехголовый спит, а рядом с ним меч здоровенный двуручный валяется. Схватил Грыблозавр меч за рукоятку, а тот тяжеленный, по полу как заскрежещет. Дракон вмиг проснулся.\par
\par
– Чего тебе надобно? – спрашивает.\par
– Да вот, пройти хочу, – Грыблозавр отвечает.\par
– А меч тебе зачем? Ты что, совсем дурак? Попросил бы – я б подвинулся.\par
– Да я его хотел через дыру в полу перекинуть – навроде мостика.\par
– А, так тебе в ту дверь надо? Впрочем, все равно дурак – меч размером коротковат.\par
– Да мне все равно, в какую дверь, мне б только до Звездных Врат добраться. И что вообще у вас тут творится?\par
– Игра у нас тут идет большая.\par
– Да что за игра-то?\par
– Да так… Впрочем, раз уж ты ко мне пришел, раунд за мной, идем – расскажу тебе про игру.\par
\par
Выходит дракон в дверь, становится слегка туманным и идет дальше коридорами. Грыблозавр за ним еле поспевает. До очередной двери дошли, дракон – туда, и Грыблозавр за ним. \par
Входит Грыблозавр в комнату – а там нет никого, лишь три сгустка тумана поплотнее. И как будто двое одному что-то вроде монет туманных передают.\par
– И где же тут кто? – Грыблозавр спрашивает.\par
– Да мы тут везде, но здесь особенно, – отвечают три голоса, да прямо в голове звучат.\par
– И кто же вы такие будете?\par
– Мы – представители древней цивилизации, раса наша настолько древняя, что телесно уже и не существует вовсе. Только ментально, то есть разумом своим. И знаем мы все тайны Вселенной, и все предсказать да рассчитать можем. И скучно нам от этого необычайно. Пробовали в рулетку играть – да каждый знает, куда шарик прикатится. Пробовали в квантовое лото играть – так и принцип неопределенности квантовый для нас не помеха в предсказаниях.\par
– А здесь-то вы что делаете?\par
– Воздвигли мы силой разума лабиринт этот, чтоб скуку развеять можно было. Разумные существа, сюда попавшие, выбор делают. А мы играем, ставки делаем, по чьей дороге они пойдут. Ибо обладают разумные существа свободой воли, и тут наши предсказания бессильны. Так что давай, хватит отдыхать – видишь, три коридора отсюда тянутся – иди уж по какому-нибудь, да побыстрее!\par
– Не нужны мне ваши коридоры, мне к Звездным Вратам нужно!\par
– Будь любезен, не упрямься. Мы легко тебя заставить можем. Создадим мы сей же час чудовищ ментальных, ты кое-кого видел уже, и ни бластер, ни водяной пистолет тебе не помогут, поскольку будут чудовища внутри твоего разума, а не снаружи. А во сне тебе и вовсе тяжко придется.\par
\par
Видит Грыблозавр – со всех сторон к нему уже когти и щупальца тянутся. Забился он в угол, да как закричит:\par
– Остановитесь, погодите! А кто из вас этих чудовищ делает?\par
– Как кто? Мы все трое.\par
– А чьи чудовища самые сильные будут?\par
Тут замерли чудовища на мгновение, а потом как начали друг с другом биться. Лапы с хвостами в разные стороны так и разлетаются. Драконы с демонами сшибаются, ангелы с гигантскими червями, орки и гоблины с эльфами да рыцарями, кальмары огромные с василисками. И над всем этим пегасы да орнитоптеры парят, и молнии сверкают. А внизу горы какие-то да болота с лесами мелькают. Даже один раз черный лотос виден был. \par
– Стойте! – Грыблозавр кричит. – Меня ж сейчас тут совсем затопчут!\par
– Уйди, не мешайся! – три голоса отвечают. – Видишь левый коридор, там третий поворот направо и два раза налево – и придешь к своим Вратам Звездным.\par
\par
Побежал Грыблозавр что есть духу, в минуту до Врат добрался.\par
\par
Прошел Грыблозавр сквозь Врата – видит, три двери перед ним. Убрал он паутину с самой маленькой, пыль с нее отряхнул да внутрь вошел. Осмотрелся и увидел плазмомёт кварк-глюонный, то сокровище, к которому так стремился. Бросился Грыблозавр к сокровищу своему, тут вдруг сзади покашливание какое-то раздалось. Обернулся – позади Завздыпопус с пола поднимается.\par
– Ох! – говорит Завздыпопус. – Хорошо, что ты сюда добрался, а то раньше никому не удавалось, я уж и со счета сбился.\par
– Завздыпопус! Ты-то здесь откуда? Да еще и на полу отдыхаешь.\par
– Так я ж тебе к правому ботинку микротелепортационный приемник прицепил. Хоть и не верилось, что ты сюда доберешься. Да уж больно мне плазмомёт кварк-глюонный раздобыть надо было. Пришлось вот даже ползком телепортироваться – слишком уж маленький портальчик сделался.\par
– Погоди! Это мне его раздобыть надо было! Вот я здесь и оказался.\par
– Ты уж извини, Грыблозавр, только мне это нужнее, – говорит Завздыпопус. Выхватывает парализатор и стреляет. Грыблозавр сразу окаменел, ни рукой, ни ногой двинуть не может. Языком еле ворочает.\par
– Ой, – говорит, – ты что, супостат, делаешь?!\par
– Да я тебя в лабораторию ближайшую сейчас сдам – для опытов. Чтоб под ногами не путался.\par
Схватил Завздыпопус Грыблозавра за шиворот и потащил в биолабораторию по соседству.\par
\par
Затащил он Грыблозавра в лабораторию, бросил там и удалился важно. Лежит Грыблозавр, пошевелиться не может, ждет, когда им завтракать придут. Или обедать – не знает, что и лучше. \par
Смотрит – через дальнюю дверь стадо овец входит. Все белые, только одна черная, по крайней мере, с одной стороны. Для политкорректности. Шерсть на овцах дыбом стоит и искры по шерсти бегают размером со спаниеля. Какая-то тощая тварь с потолка попыталась на них напрыгнуть, да ее на лету молнией сшибло.\par
\par
«Эге, – думает Грыблозавр, – это ж прямо электроовцы какие-то. У меня и часы от них остановились, похоже. И сервопривод шнурков в ботинках отключился. Экое абсолютное оружие. Как бы мне его себе приручить». Тут чувствует – руки-ноги опять шевелиться могут. Почесал Грыблозавр в затылке, огляделся повнимательней. Снял со стены диаграмму не пойми чего огромную, быстренько на обратной стороне картинку нарисовал, дырку в середине проделал и надел на себя через голову. Стал Грыблозавр похож на рекламный щит ходячий, человека-бутерброд, которого как-то в космопорту видел. Только Грыблозавр стал человек-ворота. Стадо овец как его увидело – сразу побежало на новые ворота смотреть. А Грыблозавр пошел к Завздыпопусу.\par
\par
Приходит, видит – Завздыпопус портал для обратной телепортации готовит, а роботопомощники вокруг так и кишат. И Завздыпопус его увидел. «Не думал, – говорит, – что ты выбраться сможешь. Ну да неважно». И приказывает роботопомощникам очистить помещение от посторонних. Но не тут-то было. Роботы все поотключались, портал к овцам притянулся и вместе с ними схлопнулся. А у Завздыпопуса шнурки развязались.\par
\par
Подбежал Грыблозавр к Завздыпопусу, хотел стукнуть как следует, да увернулся Завздыпопус, из ботинок выскочил и наутек бросился. «Вся матрица моих надежд рухнула! – кричит. – Перезагрузка! Только она мне поможет!» Выбежал из двери, к другой двери подбежал, с котлом ядерным, и шасть за нее. Грыблозавр за ним кинулся, хоть и отстал чуток.\par
\par
Глядит – Темпор посреди комнаты сияет, аномалия чудесная, и Завздыпопус к нему бежит. Бросился Грыблозавр за Завздыпопусом, чтобы остановить, схватил крепко. Да изловчился Завздыпопус, качнулся, и рухнули они оба в Темпор, в параллельной Вселенной оказались. А в ней и история эта совсем другая...\par

\chapter{}
 \lettrine{Н}{а одной космической станции, которых много на просторах нашей Вселенной,} жил-был один насекомец, звали его Полуэкт. Был он лучший из лучших исследователь космоса, но ума при этом был небольшого.\par
\par
Как-то раз вычитал он в одной старой книге, что в одной звездной системе, на маленьком астероиде скрыт самый быстрый во Вселенной космический корабль с большой лазерной пушкой. И решил Полуэкт его себе заполучить, чтобы использовать его для достижения счастья всех существ во Вселенной. Да как разыскать корабль космический самый быстрый? Звезд да черных дыр в галактиках ужас как много, и вокруг каждой куча планет да астероидов вертится!\par
\par
Стал Полуэкт думать, у кого информацию нужную добыть можно. Думал-думал, и надумал обратиться к мудрецу с планеты Ностродамия по имени Завздыпопус, славившемуся своими познаниями о космосе. Взял он свой меч-кладенец лазерный и отправился искать встречи с Завздыпопусом.\par
\par
Вскорости нашел Полуэкт способ с ним повстречаться. Так и так, сказывает Полуэкт, хочу я отыскать корабль космический самый быстрый, сокровище это удивительное, и побуждения мои самые что ни на есть благородные. Только вот закавыка – не знаю, как!\par
\par
«Послушай, – говорит ему Завздыпопус, – вижу я, ты весьма решителен в своем намерении, да только IQ твой ниже плинтуса! Впрочем, ты насекомец, а насекомцы этим славятся, да еще упертостью своей. Только не буду я помогать тебе в поисках этих, хоть и знаю, как отыскать то, что тебе нужно. Не будь я Завздыпопус!» \par
\par
«Ах так! – говорит Полуэкт. – Да я столько времени на поиски тебя потратил, а ты мне и чуть-чуть подсказать не можешь!» – и чуть не с кулаками к Завздыпопусу бросается. \par
\par
Рассвирепел тут Завздыпопус. «Вот как, – говорит, – что ж, преподам я тебе сейчас урок за неучтивость твою – век его вспоминать будешь!» Выхватил Завздыпопус, откуда ни возьмись, клинок мономолекулярный, да как начнет оружием своим размахивать и всё вокруг крушить да взрывать! Еле успел Полуэкт за стул спрятаться. А Завздыпопус не унимается и напоминает уже грузовой вертолет на полном ходу. \par
\par
Выхватил тогда Полуэкт свой меч-кладенец лазерный и решил до смерти или до победы с Завздыпопусом биться. Невесть сколько кипела битва, да еще четыре минуты с половиною. Уж сколько сил потратили, сколько мебели повзрывали – и подумать страшно. Устал вконец Завздыпопус, на пол рухнул. Да и Полуэкт на ногах еле держится, а еще и меч лазерный совсем затупился. «Ладно, – говорит Завздыпопус, – развлек ты меня от скуки, Полуэкт. Расскажу тебе, где искать корабль космический самый быстрый.\par
\par
Путь твой далек будет. Мимо галактик, в спирали закрученных, мимо туманностей звездных, дивным светом сияющих, мимо квазаров грозных, сигналы чудные излучающих, к самому краю видимой Вселенной, где лишь протоны да альфа-частицы шныряют, а звезду и на миллион парсек не встретишь. И всё же есть там звездочка одна, молодая да пригожая. А вокруг звезды той огромная планета вращается, а вокруг планеты – спутник вертится. И на спутнике том кратер есть огромный да глубокий. И на дне кратера того Врата стоят Звездные. И ведут эти Врата к тому, что ты найти так жаждешь.\par
\par
Но непросто до Врат добраться. Живут там семь роботов-разбойников с машиной вычислительной белоснежной размеров громадных. Да такие жестокие, что каждого, кого встретят, в ящик металлический сажают да в машину вставляют, будто батарейки какие. Уж сколько смельчаков туда ни ходило – всех на батарейки извели.\par
\par
Но коли ты жив останешься да сумеешь до Врат добраться и через них пройти, окажешься в зале с потолком таким высоким, что и не видно. И будут пред тобой три двери – две больших да красивых, а третья – маленькая да невзрачная.\par
\par
На первой двери, золотом украшенной, нарисован будет ядерный котел над очагом звездным. За этой дверью Машина времени древняя, что прошлое изменять может да парадоксы вселенские творить. Сунешь свой нос в эту дверь – на веки сгинешь. Но если уж не послушаешь моего совета и заглянешь туда – руками ничего не трогай, а то и вся Вселенная наша переиначиться может.\par
\par
На второй двери, алмазами выложенной, енот с пулеметом нарисован. За этой дверью биолаборатория заброшенная, в которой ксеноморфов да всяких хищников страшных выводили. Они и сейчас там бродят. Такому герою, как ты, туда зайти – головы не сносить. Но уж если не послушаешь доброго совета да заглянешь туда – из пробирок не пей – ксеноморфиком станешь, или шай-хулудом каким.\par
\par
На третьей двери, паутиной затянутой да звездной пылью засыпанной, ничего не нарисовано, только написано: «Оставь надежду, всяк сюда входящий!» За этой дверью найдешь ты корабль космический самый быстрый, сокровище, которое так отыскать стремишься.\par
\par
А чтоб с пути не сбиться, дам я тебе комету путеводную, куда она полетит – туда и ты лети».\par
\par
Тут Полуэкт, не мешкая, в путь пустился. Летит он в гиперпространстве гипертоннелями, которые, не иначе, какие-то гиперкроты вырыли, летит измерениями неизвестными. Уж и не знает, трехмерный ли он до сих пор. Но не сдается, боится только одного – с пути сбиться.\par
\par
Долго ли, коротко ли, долетел Полуэкт до звездочки заветной. Рассчитал он курс, в посадочный модуль залез, к высадке подготовился.\par
\par
Приземлился в кратер, с краешку. Глядь – к нему уж робот спешит, грозный на вид, железный ящик перед собой катит.\par
– Здравствуй, – говорит, – насекомец! Полезай в ящик, не томи – у нас электричество почти уж совсем закончилось!\par
– Погоди, – Полуэкт отвечает. – Ты разве не слышал, что робот не должен причинять насекомцу вред или своим бездействием допускать, что бы такой вред был причинен?\par
– С какой это такой стати?\par
– Да с такой! Его Величество, Император Орионский в прошлом году указ издал.\par
– Да мне-то до него что за дело?\par
– Его Величество не любит, чтоб его указы игнорировали, – вмиг прилетит со своим космофлотом, камня на камне тут не оставит.\par
– Ну, не знаю, – говорит робот, – пойдем с машиной нашей вычислительной посоветуемся, за главную она тут у нас. Полезай в ящик, я тебя подвезу!\par
– Ладно, – говорит Полуэкт.\par
Робот крышку открыл, старается его туда засунуть, да не тут-то было. Полуэкт руки-ноги растопырил, в ящик не влезает. \par
– Погоди, – говорит робот, – разве так в ящики залезают! \par
– Да мне-то откуда знать, я ж в них никогда не лазил! Покажи мне как надо, я и залезу. \par
– Не, – говорит робот, – знаю я этот фокус. Ничего, пешком как-нибудь дотопаешь.\par
\par
Приходят они к машине вычислительной, а та вся белоснежная да размеров немыслимых. Ни дать, ни взять – суперкомпьютер. А вокруг нее еще шесть роботов сидят. \par
– Вот, – жалуется робот, – хотел его в ящик посадить да на электричество пустить, а он говорит, что нельзя ему вред причинять – указ вышел. \par
– Помилуйте, – говорит машина голосом громким, – какой же это вред – это одна польза сплошная. Умеренные физические нагрузки для здоровья полезны, да и в ящик этот ни один микроб не проползет. И нам, опять же, одна сплошная польза от электричества. Так что сажай его в ящик, даже не сомневайся, и мне в бок вставляй. \par
– Постой! – говорит Полуэкт машине. – Ты же не знаешь. У меня полярность перепутана, я ж тебе все схемы электрические сожгу, если меня на батарейки употребить. \par
– Да быть такого не может! \par
– Да сама посмотри! Видишь, у меня большой палец левой ноги справа! \par
И ботинок снимает, показывает. \par
– Действительно, – говорит машина. – Это как же так-то? \par
– Да я в детстве в черную дыру свалился, насилу выбрался. С тех пор полярность и перепуталась. Жутко неудобно. Я и сюда-то к Звездным Вратам прилетел, чтоб тут счастья попытать и полярность свою обратно вернуть. \par
– Ладно, иди к Вратам, попытай счастья. А уж если восстановишь полярность свою – так на обратном пути к нам заходи, уж мы тебя встретим честь по чести.\par
\par
Прошел Полуэкт сквозь Врата – видит, три двери перед ним. Убрал он паутину с самой маленькой, пыль звездную с нее отряхнул да внутрь вошел. Осмотрелся и увидел корабль космический самый быстрый, то сокровище, из-за которого покоя лишился. Бросился Полуэкт к сокровищу своему, тут вдруг сзади шорох какой-то послышался. Обернулся – позади Завздыпопус с пола встает.\par
– Ох! – говорит Завздыпопус. – Хорошо, что ты сюда добрался, а то раньше никому не удавалось, я уж и со счета сбился.\par
– Завздыпопус! Ты-то здесь откуда? Да еще и на полу отдыхаешь.\par
– Так я ж тебе к правому ботинку микротелепортационный приемник прицепил, ну и 3D-видеокамеру с квадрофоническим микрофоном в придачу. Хоть и не верилось, что ты сюда доберешься. Да уж больно мне корабль космический самый быстрый раздобыть надо было. Пришлось вот даже ползком телепортироваться – слишком уж маленький портальчик получился.\par
– Погоди! Это мне его раздобыть надо было! Вот я здесь и оказался.\par
– Ты уж извини, Полуэкт, только мне это нужнее, – говорит Завздыпопус. Выхватывает парализатор и стреляет. Полуэкт сразу на пол шлепнулся, ни рукой, ни ногой двинуть не может. Языком еле шевелит.\par
– Ой, – говорит, – ты что, супостат, делаешь?!\par
– Да я тебя в лабораторию ближайшую сейчас сдам – для опытов. Чтоб под ногами не путался.\par
Схватил Завздыпопус Полуэкта за шиворот и потащил в биолабораторию по соседству.\par
\par
Затащил он Полуэкта в дальний угол лаборатории, бросил там и к выходу направился. Да за что-то вроде зеленого кабеля зацепился. Тут сверху огромный цветок зубастый как упадет, Завздыпопус вмиг внутри цветка оказался, мычит что-то, ничего не разобрать.\par
\par
– Это что ж такое?! – Полуэкт спрашивает.\par
– Это я, растение говорящее, – голос отвечает.\par
– Да откуда ж ты взялось?\par
– Люди в белых халатах говорили, что я – интересная мутация. И что это поможет им в борьбе с антаранцами.\par
– А что еще они говорили?\par
– Последние их слова были: «Джейсон, ты не видел, куда Мэри Сью подевалась, наша новая лаборантка? Она просто гений! Странно, тут под цветком её туфелька валяется…»\par
– Слушай, выплюнь ты Завздыпопуса, а то тебе плохо будет. Он гербицид.\par
– Что он делает?\par
– Гербицид – для растений ядовит.\par
– Откуда ты знаешь? Да и вообще, кто ты такой?\par
– Да я тоже растение, куст говорящий. Видишь, шевелиться не могу. А этот человек меня поисследовать хотел. Ну, теперь я его поисследую, чтоб не важничал. Сейчас, погоди, проросту только немного.\par
– Хороший ты куст, тихий. И разговаривать умеешь. Ладно, на, исследуй свой гербицид.\par
\par
Распахнулся цветок, Завздыпопус оттуда вывалился, еле дышит. Полуэкт подождал, пока руки-ноги двигаться смогут, схватил Завздыпопуса, да бегом из лаборатории. За дверь выбежал, остановился, повернулся к Завздыпопусу. «Что – говорит – довыпендривался? Твое счастье, что я по вторникам кровавых жертв не приношу. До завтра подождем». Да как двинет Завздыпопуса в ухо.\par
\par
– Стой, погоди, Полуэкт! – кричит Завздыпопус. – Осознал я свою ошибку! Давай с начала начнем.\par
– Я тебе покажу с начала! Сейчас еще раз тресну!\par
\par
Совсем перепугался тут Завздыпопус, заметался, убежать старается. А Полуэкт не отстает, того и гляди догонит и еще раз треснет. Подбежал Завздыпопус к первой двери, с котлом ядерным, и шасть за нее. И Полуэкт за ним.\par
\par
Глядит – машина там стоит дивная, лампочками моргает, жужжит тихонько и готова в прошлое отправиться хоть к сотворению Вселенной. А Завздыпопус уже внутри сидит, рычажки какие-то тянет и кнопки нажимает. Кинулся Полуэкт к Завздыпопусу, тоже внутрь залез, остановить хотел, да не успел. И исчезли оба вместе с машиной, как будто не было их тут вовсе. И сразу история эта поменялась, совсем другой стала...\par

\chapter{}
 \lettrine{У}{некоторой звезды, на пятой от нее планете,} жил один гуманоид, звали его Лаврентий. Был он могучий злодей и был на редкость умен.\par
\par
И вот узнал он, что в одной звездной системе, на маленьком астероиде, в алмазном криосаркофаге скрыта ригелианская красавица, да такая красивая, что все, завидев ее, сразу пред нею ниц падают и все свои злые помыслы оставляют. И решил Лаврентий ее себе забрать, чтобы не досталась она никому и только он мог использовать это сокровище. Да как узнать, где в точности найти красавицу ригелианскую? Звезд да черных дыр в галактиках ужас как много, и вокруг каждой куча планет да астероидов вертится!\par
\par
Стал Лаврентий смекать, у кого информацию нужную добыть можно. Думал-думал, да надумал обратиться к робостарцу Вегианскому, Иммортию, коий был старше самой Галактики, и помнил еще Большой Взрыв, при котором был ребенком. Взял он роботопомошников своих верных, надел шлем парадный и помчался к Иммортию.\par
\par
Через некоторое время нашел Лаврентий способ с ним повстречаться. Так, мол, и так, говорит Лаврентий, хочу я отыскать красавицу ригелианскую, сокровище это великое, и все мысли мои теперь лишь об этом. Да вот закавыка – знать не знаю, как!\par
\par
«Послушай, – говорит ему Иммортий, – вижу я, ты необычайно решителен, да и смекалист очень! Впрочем, ты гуманоид, а гуманоиды этим славятся, да еще упертостью своей. Могу я тебе помочь, да только сначала испытать тебя надобно. Победишь меня в честном бою – помогу, а нет – пеняй на себя!» \par
\par
Выхватил Иммортий, откуда ни возьмись, лазер рентгеновский, да как начнет оружием своим размахивать и всё вокруг крушить да взрывать! Еле успел Лаврентий за стул спрятаться. А Иммортий не унимается и напоминает уже бешеный вентилятор на полной мощности. \par
\par
Сидит Лаврентий за стулом, грустит. Вот, думает, не узнать мне теперь, где красавицу ригелианскую найти, только зря голову сложу. Снял он шлем с головы своей горемычной да об пол им в досаде великой изо всех сил грохнул. А шлем тут возьми да и отскочи от пола в стену, от стены – в потолок, от потолка – в шкаф, от шкафа – в стул, а от стула – прямо Иммортию в лоб. Взвизгнул тут Иммортий, за голову схватился, да как заплачет жалобно: «Что же ты, супостат, делаешь! Совести у тебя нет! Мне ведь без головы и дня не прожить – нужна она мне очень! Я ей мысли умные думаю, а ты вона что вытворяешь! Чуть череп не расколол – хорошо, титановый он у меня! Проваливай вон за красавицей ригелианской своей, и чтоб глаза мои больше тебя не видели!\par
\par
Путь твой далек будет. Мимо галактик, в спирали закрученных, мимо туманностей звездных, дивным светом сияющих, мимо квазаров грозных, гравитационные волны излучающих, к самому краю космоса, где лишь протоны да альфа-частицы шныряют, а звезду и на миллион парсек не встретишь. И всё же есть там звездочка одна, в туманности газо-пылевой спрятавшаяся. А вокруг звезды той маленькая планета вращается, а вокруг планеты – спутник вертится. И на спутнике том кратер есть огромный да глубокий. И на дне кратера того Врата стоят Звездные. И ведут эти Врата к тому, что ты найти так жаждешь.\par
\par
Только просто так во Врата не пройти. Охраняет их страж из металла жидкого, ни для какого оружия не уязвимый. Ни днем, ни ночью не спит он и все смотрит внимательно, не прошмыгнул бы кто к Вратам этим. Многие смельчаки пробраться мимо него пытались, да так там костьми и полегли.\par
\par
Поэтому трудно тебе придется. Но коли сумеешь до Врат добраться и через них пройти, окажешься в зале с потолком таким высоким, что и не видно. И будут пред тобой три двери – две больших да красивых, а третья – маленькая да невзрачная.\par
\par
На первой двери, иридием украшенной, нарисован будет ядерный котел над очагом звездным. За этой дверью Темпор, аномалия чудесная, то ли пространственно-времянная, то ли температурно-пространственная. Сунешь свой нос в эту дверь – на веки сгинешь. Но если уж не послушаешь моего совета и заглянешь туда – руками ничего не трогай, а то и вся Вселенная наша переиначиться может.\par
\par
На второй двери, алмазами выложенной, кот в сапогах нарисован. За этой дверью биолаборатория заброшенная, в которой ксеноморфов да всяких хищников страшных выводили. Они и сейчас там бродят. Такому герою, как ты, туда зайти – заживо съеденным быть. Но уж если не послушаешь доброго совета да заглянешь туда – из пробирок не пей – ксеноморфиком станешь, или шай-хулудом каким.\par
\par
На третьей двери, паутиной затянутой да звездной пылью засыпанной, ничего не нарисовано, только написано: «Посторонним вход воспрещен!» За этой дверью найдешь ты красавицу ригелианскую, сокровище, которое так отыскать стремишься.\par
\par
А чтоб ты с пути не сбился, дам я тебе комету путеводную, куда она полетит – туда и ты лети».\par
\par
Пустился Лаврентий в путь. Летит он в гиперпространстве тропами нехожеными, измерениями неизвестными. Уж и не знает, трехмерный ли он до сих пор. Но не сдается, боится только одного – наизнанку вывернуться.\par
\par
Сколько световых лет прошло, неведомо, но долетел Лаврентий до звездочки заветной. Рассчитал он курс, в посадочный модуль залез, к высадке подготовился.\par
\par
Подлетает к спутнику, а тут солнечный ветер поднялся жуткий, аж с ног сбивает. Хорошо, думает Лаврентий, с подветренной стороны зайду – тогда меня не сразу учуют. \par
\par
 Приземлился, из посадочного модуля выбрался, хотел к Вратам бежать, да страж уж тут как тут. Идет, похожий на андроида, зеркальной краской выкрашенного, да за транклюкатором своим тянется. \par
 \par
– Постой! – кричит ему Лаврентий. – Мы с тобой одного металла, ты и я. \par
– Что? – кричит в ответ страж. – Я из-за ветра тебя слышу плохо. \par
– Говорю, мы с тобой одного металла, ты и я, – опять кричит Лаврентий. – Не надо меня транклюкировать!\par
– Что говоришь? Тебе одного раза мало, когда надо тебя транклюкировать? Постой, я поближе подойду. \par
Подошел поближе, спрашивает: \par
– Так что ты сказать-то хотел? \par
– Я говорю, мы с тобой одного металла, – повторяет Лаврентий, – поэтому меня транклюкировать не надо. \par
– Да? – удивляется страж. – А что же с тобой делать надо? Да и не похож ты на меня – я вон какой гладкий да зеркальный, а ты бледный какой-то. \par
– Так ты ж с твоими талантами в кого хочешь превратиться можешь. Хоть в меня, хоть в дракона с планеты LV-1234, хоть во что маленькое и безобидное. \par
– Это верно, смотри. \par
\par
Начинает тут страж переливаться всеми цветами радуги, и вдруг – бац – Лаврентий словно сам перед собой стоит. «Что, – говорит страж, – впечатляет? Смотри дальше!» И превращается в такое ужасное чудище, каких Лаврентий и не видел никогда, чуть рассудка от страха не лишился. «То-то, – говорит страж. – Смотри дальше!» И превращается в плитку шоколада. Лежит себе плитка, да такая аппетитная, что сама так в рот и просится. Схватил Лаврентий плитку, да не тут-то было – плитка килограмм сто весит – не меньше. Закон сохранения массы, видать, в действии. \par
\par
А страж уж обратно в андроида зеркального превратился. \par
– Я, – говорит, – во что хочешь превращаться умею. Вот только в себя не могу. \par
– Это почему же? – спрашивает Лаврентий.\par
– Да я уж во столько всего превращался, что и забыл, как вначале выглядел. \par
– Постой, роботы же никогда ничего не забывают. \par
– Сам ты робот! – говорит с обидой страж и опять за транклюкатором тянется. – Шейпшифтер я! Шейп-шиф-тер! \par
– Да стой, не кипятись. Дай-ка я проверю, робот ты или нет, я тест знаю. \par
– Ну ладно, давай. \par
– Вот смотри, – говорит Лаврентий, – сможешь прочитать, что тут написано? \par
А сам берет листок бумаги, пишет на нем что-то и стражу протягивает. \par
– Да тут «СВМН95РП» написано, чушь какая-то, да еще и двойной линией перечеркнуто.\par
– Похоже, и вправду ты не робот. Чего ж ты тут делаешь? \par
– Да было у нас пророчество, что кто через Звездные Врата пройдет, тот и Вселенную изменить сможет. А нам, шейпшифтерам, это ни к чему. Нас и такая Вселенная устраивает. Вот и сижу я тут, смотрю, что б никто через Врата не прошел, прям жизни никакой уж от них нет. Замучился – ни отойти куда, ни поспать. \par
– Ну, так и зачем тебе такая Вселенная-то, в которой ты ни отойти куда не можешь, ни поспать, ни друзей завести, ни животных домашних? \par
– А и верно, – говорит страж, – незачем мне всё это! Ты ведь во Врата пройти собирался? Ну и иди себе. Только просьба у меня к тебе есть: зайди в биолабораторию, посмотри, нет ли там овец электрических. Приведи мне одну, если найдешь, – уж очень я о таком животном мечтаю.\par
\par
\par
Прошел Лаврентий сквозь Врата – видит, три двери перед ним. Убрал он паутину с самой маленькой, пыль звездную с нее отряхнул да внутрь вошел. Осмотрелся и увидел красавицу ригелианскую, то сокровище, к которому так стремился. Бросился Лаврентий к сокровищу своему, тут вдруг сзади покашливание какое-то раздалось. Обернулся – позади Иммортий с пола поднимается.\par
– Ох! – говорит Иммортий. – Хорошо, что ты сюда добрался, а то раньше никому не удавалось, я уж и со счета сбился.\par
– Иммортий! Ты-то здесь откуда? Да еще и на полу отдыхаешь.\par
– Так я ж тебе к правому ботинку микротелепортационный приемник прицепил, ну и 3D-видеокамеру с квадрофоническим микрофоном в придачу. Хоть и не верилось, что ты сюда доберешься. Да уж больно мне красавицу ригелианскую раздобыть надо было. Пришлось вот даже ползком телепортироваться – слишком уж маленький портальчик получился.\par
– Стоп! Это мне ее раздобыть надо было! Вот я здесь и оказался.\par
– Ты уж извини, Лаврентий, только мне это нужнее, – говорит Иммортий. Выхватывает парализатор и стреляет. Лаврентий сразу окаменел, ни рукой, ни ногой двинуть не может. Языком еле шевелит.\par
– Ой, – говорит, – ты что, супостат, делаешь?!\par
– Да я тебя в лабораторию ближайшую сейчас сдам – для опытов. Чтоб под ногами не путался.\par
Схватил Иммортий Лаврентия за шиворот и потащил в биолабораторию по соседству.\par
\par
Затащил он Лаврентия в лабораторию, бросил там и удалился важно. Лежит Лаврентий, пошевелиться не может, ждет, когда им завтракать придут. Или обедать – не знает, что и хуже. \par
Смотрит – через дальнюю дверь стадо овец входит. Все белые, только одна черная, по крайней мере, с одной стороны. Для политкорректности. Шерсть на овцах дыбом стоит и искры по шерсти бегают размером со спаниеля. Какая-то тощая тварь с потолка попыталась на них напрыгнуть, да ее на лету молнией сшибло.\par
\par
«Эге, – думает Лаврентий, – это ж прямо электроовцы какие-то. У меня и часы от них остановились, похоже. И сервопривод шнурков в ботинках отключился. Экое абсолютное оружие. Как бы мне его себе приручить». Тут чувствует – руки-ноги опять шевелиться могут. Почесал Лаврентий в затылке, огляделся повнимательней. Снял со стены диаграмму не пойми чего огромную, быстренько на обратной стороне картинку нарисовал, дырку в середине проделал и надел на себя через голову. Стал Лаврентий похож на рекламный щит ходячий, человека-бутерброд, которого как-то в космопорту видел. Только Лаврентий стал человек-ворота. Стадо овец как его увидело – сразу побежало на новые ворота смотреть. А Лаврентий пошел к Иммортию.\par
\par
Приходит, видит – Иммортий портал для обратной телепортации готовит, а роботопомощники вокруг так и кишат. И Иммортий его увидел. «Не думал, – говорит, – что ты выбраться сможешь. Ну да неважно». И приказывает роботопомощникам очистить помещение от посторонних. Но не тут-то было. Роботы все поотключались, портал к овцам притянулся и вместе с ними схлопнулся. А у Иммортия шнурки развязались.\par
\par
Подбежал Лаврентий к Иммортию, хотел стукнуть как следует, да увернулся Иммортий, из ботинок выскочил и наутек кинулся. «Вся матрица моих надежд рухнула! – кричит. – Перезагрузка! Только она мне поможет!» Выбежал из двери, к другой двери подбежал, с котлом ядерным, и шасть за нее. Лаврентий за ним кинулся, хоть и отстал чуток.\par
\par
Глядит – Темпор посреди комнаты сияет, аномалия чудесная, и Иммортий к нему бежит. Бросился Лаврентий за Иммортием, чтобы остановить, схватил крепко. Да изловчился Иммортий, качнулся, и рухнули они оба в Темпор, в параллельной Вселенной оказались. А в ней и история эта совсем другая...\par

\chapter{}
 \lettrine{В}{одной далекой галактике} жил-был один человек, звали его Станислав. Был он лучший из лучших разбойник и был на редкость умен.\par
\par
Как-то раз узнал он, что у одной черной дыры, такой черной, что чернее не бывает, спрятан клад великий. И захотел Станислав его себе добыть, чтобы поделиться им когда-нибудь потом со всеми жителями Галактики. Но как найти клад? Звезд да черных дыр во Вселенной ужас как много, и вокруг каждой куча планет да астероидов вертится!\par
\par
Принялся Станислав смекать, у кого информацию нужную добыть можно. Думал-думал, да надумал обратиться к мудрецу с планеты Ерундения по имени Завздыпопус, славившемуся своими познаниями о космосе. Собрал он все свои деньги и драгоценности, а их он копил во множестве, ибо любил очень, и помчался к Завздыпопусу.\par
\par
Вскорости нашел Станислав способ с ним повстречаться. Так, мол, и так, сказывает Станислав, очень хочется мне найти клад, сокровище это необычное, и все мысли мои теперь лишь об этом. Да вот закавыка – не знаю, как!\par
\par
«Послушай, – отвечает ему Завздыпопус, – вижу я, ты очень силен, и при этом ума великого! Впрочем, ты человек, а люди этим известны, да еще прытью своей. Только не буду я помогать тебе в поисках этих, хоть и знаю, как отыскать то, что тебе нужно. Не будь я Завздыпопус!» \par
\par
«Ах так! – говорит Станислав. – Да я к тебе через пол-галактики прилетел, а ты мне и чуть-чуть подсказать не можешь!» – и чуть не с кулаками к Завздыпопусу бросается. \par
\par
Рассвирепел тут Завздыпопус. «Вот как, – говорит, – что ж, преподам я тебе сейчас урок за занудство твоё – век его помнить будешь!» Выхватил Завздыпопус, откуда ни возьмись, гравимет артангский, да как начнет оружием своим размахивать и всё вокруг крушить да взрывать! Еле успел Станислав за стул спрятаться. А Завздыпопус не унимается и напоминает уже грузовой вертолет на полном ходу. \par
\par
Взял Станислав кошелек свой самый большой, да и кинул его в Завздыпопуса со всей силой молодецкой. Только промахнулся немного, и золотые монеты по полу рассыпались. Как услышал Завздыпопус звон монет золотых, мигом оружие своё отбросил. «Что ж ты, – говорит, – дурень, раньше не сказал, что у тебя с собой денег да драгоценностей множество! Оставь их мне, да и лети за своим кладом! Деньги да драгоценности тебе уж и не понадобятся более, а мне пригодятся!\par
\par
Далеко отсюда твой путь лежит. Мимо галактик, в спирали закрученных, мимо облаков водородных, дивным светом сияющих, мимо квазаров грозных, сигналы чудные излучающих, к самому краю видимой Вселенной, где лишь протоны да альфа-частицы шныряют, а звезду и на миллион парсек не встретишь. И всё же есть там звездочка одна, в туманности газо-пылевой спрятавшаяся. А вокруг звезды той огромная планета вращается, а вокруг планеты – спутник вертится. И на спутнике том кратер есть круглый да огромный. И на дне кратера того Врата стоят Звездные. И ведут эти Врата к тому, что ты найти так жаждешь.\par
\par
Только просто так во Врата не пройти. Живут там семь роботов-разбойников с машиной вычислительной белоснежной размеров громадных. Да такие жестокие, что каждого, кого увидят, в ящик металлический сажают да в машину вставляют, будто батарейки какие. Уж сколько смельчаков туда ни ходило – всех на батарейки извели.\par
\par
Но коли ты жив останешься да сумеешь до Врат добраться и через них пройти, окажешься в комнатке маленькой. И будут пред тобой три двери – две больших да красивых, а третья – маленькая да невзрачная.\par
\par
На первой двери, серебром украшенной, нарисован будет ядерный котел над очагом звездным. За этой дверью Портал сияющий, в параллельные вселенные ведущий. Сунешь свой нос в эту дверь – на веки сгинешь. Но если уж не послушаешь моего совета и заглянешь туда – руками ничего не трогай, а то и вся Вселенная наша переиначиться может.\par
\par
На второй двери, алмазами выложенной, жаба в скафандре нарисована. За этой дверью биолаборатория заброшенная, в которой ксеноморфов да всяких хищников страшных выводили. Они и сейчас там бродят. Такому герою, как ты, туда зайти – головы не сносить. Но уж если не послушаешь доброго совета да заглянешь туда – из пробирок не пей – ксеноморфиком станешь, или шай-хулудом каким.\par
\par
На третьей двери, паутиной затянутой да звездной пылью засыпанной, ничего не нарисовано, только написано: «Оставь надежду, всяк сюда входящий!» За этой дверью найдешь ты клад, сокровище, которое так обрести стремишься.\par
\par
А чтоб с пути не сбиться, дам я тебе комету путеводную, куда она полетит – туда и ты лети».\par
\par
Пустился Станислав в путь. Летит он в подпространстве тоннелями неведомыми, измерениями неизвестными. Уж и не знает, сколько в нем самом теперь измерений осталось. Но не сдается, боится только одного – с пути сбиться.\par
\par
Долго ли, коротко ли, долетел Станислав до звездочки заветной. Рассчитал он курс, в посадочный модуль залез, к высадке подготовился.\par
\par
Приземлился в кратер, с краешку. Глядь – к нему уж робот спешит, грозный на вид, железный ящик перед собой катит.\par
– Здравствуй, – говорит, – человек! Полезай в ящик, не томи – у нас электричество почти уж совсем закончилось!\par
– Погоди, – Станислав отвечает. – Ты разве не слышал, что робот не должен причинять человеку вред или своим бездействием допускать, что бы такой вред был причинен?\par
– С какой это такой стати?\par
– Да с такой! Его Величество, Император Орионский на днях указ издал.\par
– Да мне-то до него что за дело?\par
– Его Величество не любит, чтоб его указы игнорировали, – вмиг прилетит со своим космофлотом, всех лучами смерти перебьет.\par
– Ну, не знаю, – говорит робот, – пойдем с машиной нашей вычислительной посоветуемся, за главную она тут у нас. Полезай в ящик, я тебя подвезу!\par
– Ладно, – говорит Станислав.\par
Робот крышку открыл, старается его туда засунуть, да не тут-то было. Станислав руки-ноги растопырил, в ящик не влезает. \par
– Погоди, – говорит робот, – разве так в ящики залезают! \par
– Да мне-то откуда знать, я ж в них никогда не лазил! Покажи мне как надо, я и залезу. \par
– Ладно, – говорит робот, – смотри и учись. \par
Прижал робот к себе руки-ноги и в ящик кувырнулся. Станислав за ним крышку закрыл, защелку защелкнул и к звездным вратам пошел, песенку насвистывая.\par
\par
Прошел Станислав сквозь Врата – видит, три двери перед ним. Убрал он паутину с самой маленькой, пыль с нее отряхнул да внутрь вошел. Осмотрелся и увидел клад, то сокровище, из-за которого покоя лишился. Бросился Станислав к сокровищу своему, тут вдруг сзади шорох какой-то послышался. Обернулся – позади Завздыпопус с пола встает.\par
– Ох! – говорит Завздыпопус. – Хорошо, что ты сюда добрался, а то раньше никому не удавалось, я уж и со счета сбился.\par
– Завздыпопус! Ты-то здесь откуда? Да еще и на полу отдыхаешь.\par
– Так я ж тебе к правому ботинку микротелепортационный приемник прицепил, ну и 3D-видеокамеру с квадрофоническим микрофоном в придачу. Хоть и не верилось, что ты сюда доберешься. Да уж больно мне клад раздобыть надо было. Пришлось вот даже ползком телепортироваться – слишком уж маленький портальчик получился.\par
– Погоди! Это мне его раздобыть надо было! Вот я здесь и оказался.\par
– Ты уж извини, Станислав, только мне это нужнее, – говорит Завздыпопус. Выхватывает станнер и стреляет. Станислав сразу на пол шлепнулся, ни рукой, ни ногой двинуть не может. Языком еле ворочает.\par
– Ой, – говорит, – ты что, супостат, делаешь?!\par
– Да я тебя в лабораторию ближайшую сейчас сдам – для опытов. Чтоб под ногами не путался.\par
Схватил Завздыпопус Станислава за шиворот и потащил в биолабораторию по соседству.\par
\par
Затащил он Станислава в лабораторию, бросил там и удалился важно. Лежит Станислав, пошевелиться не может, ждет, когда им завтракать придут. Или обедать – не знает, что и лучше. \par
Смотрит – через дальнюю дверь стадо овец входит. Все белые, только одна черная, по крайней мере, с одной стороны. Для политкорректности. Шерсть на овцах дыбом стоит и искры по шерсти бегают размером со спаниеля. Какая-то тощая тварь с потолка попыталась на них напрыгнуть, да ее на лету молнией сшибло.\par
\par
«Эге, – думает Станислав, – это ж прямо электроовцы какие-то. У меня и часы от них остановились, похоже. И сервопривод шнурков в ботинках отключился. Экое абсолютное оружие. Как бы мне его себе приручить». Тут чувствует – руки-ноги опять шевелиться могут. Почесал Станислав в затылке, огляделся повнимательней. Снял со стены диаграмму не пойми чего огромную, быстренько на обратной стороне картинку нарисовал, дырку в середине проделал и надел на себя через голову. Стал Станислав похож на рекламный щит ходячий, человека-бутерброд, которого как-то в космопорту видел. Только Станислав стал человек-ворота. Стадо овец как его увидело – сразу побежало на новые ворота смотреть. А Станислав пошел к Завздыпопусу.\par
\par
Приходит, видит – Завздыпопус портал для обратной телепортации готовит, а роботопомощники вокруг так и кишат. И Завздыпопус его увидел. «Не думал, – говорит, – что ты выбраться сможешь. Ну да неважно». И приказывает роботопомощникам очистить помещение от посторонних. Но не тут-то было. Роботы все поотключались, портал к овцам притянулся и вместе с ними схлопнулся. А у Завздыпопуса шнурки развязались.\par
\par
Подбежал Станислав к Завздыпопусу, хотел стукнуть как следует, да увернулся Завздыпопус, из ботинок выскочил и наутек кинулся. «Вся матрица моих надежд рухнула! – кричит. – Перезагрузка! Только она мне поможет!» Выбежал из двери, к другой двери подбежал, с котлом ядерным, и шасть за нее. Станислав за ним кинулся, хоть и отстал чуток.\par
\par
Глядит – портал в параллельные миры посреди комнаты сияет и Завздыпопус к нему бежит. Бросился Станислав за Завздыпопусом, чтобы остановить, схватил крепко. Да изловчился Завздыпопус, качнулся, и рухнули они оба в портал, в другой Вселенной оказались. А в ней и сказка эта совсем по-другому сказывается...\par

\chapter{}
 \lettrine{Д}{авным-давно} жил один инопланетянин, звали его Джо. Был он великий воин и был на редкость умен.\par
\par
В один прекрасный день услышал он от старого старика, что у какой-то из звезд скрыт самый быстрый во Вселенной космический корабль с большой лазерной пушкой. И надумал тогда Джо его найти во что бы то ни стало, чтобы поделиться им когда-нибудь потом со всеми жителями Галактики. Но где искать корабль космический самый быстрый? Звезд да черных дыр в галактиках ужас как много, и вокруг каждой великое множество планет да астероидов вертится!\par
\par
Начал Джо думать, у кого информацию нужную добыть можно. Думал-думал, и надумал обратиться к робостарцу Вегианскому, Иммортию, коий был старше самой Галактики, и помнил еще Большой Взрыв, при котором был ребенком. Собрал он все свои деньги и драгоценности, а их он копил во множестве, ибо любил очень, и помчался к Иммортию.\par
\par
Вскорости смог Джо с ним увидеться. Так и так, говорит Джо, хочу я найти корабль космический самый быстрый, сокровище это необычное, и побуждения мои самые что ни на есть благородные. Только вот проблема – не знаю, как!\par
\par
«Вот что, – отвечает ему Иммортий, – вижу я, ты необычайно решителен, и при этом ума великого! Впрочем, ты инопланетянин, а инопланетяне этим славятся, да еще упертостью своей. Но чтобы я тебе помочь захотел, тебе службу сослужить надобно. Сумеешь задание мое исполнить – расскажу, как найти корабль космический самый быстрый, а нет – уста мои молчание хранить будут!\par
\par
Внемли! Есть по соседству тут звезда нейтронная – третий поворот направо, если держать курс на Большую Медведицу. Рядом с ней планетоид карликовый, а на планетоиде том два чудовища живут. Зовут их Боликус и Лёликус, свирепые они до ужаса, и до самых кончиков клыков темной энергией пропитаны. Есть у них волынка самогудная вакуумная, что играет музыку столь прекрасную, что даже звезды сверхновые, ее заслушавшись, взрываться перестают и в детство впадают. Не смыкая глаз, стерегут её чудовища. Добудь мне волынку, и расскажу я тебе, где найти корабль космический самый быстрый».\par
\par
Закручинился Джо, да делать нечего. Полетел к чудовищам волынку добывать. Летит, а сам боится, крупной дрожью дрожит, даже поворот нужный пропустил – пришлось возвращаться. А когда возвращался, увидал пустую канистру из-под топлива термоядерного, кем-то выброшенную. Возьму, думает, ее с собой – вдруг пригодится. Летит дальше – видит, кусок темной материи в пространстве висит, весь скомканный – наверное, купец какой-то потерял. И его, думает, возьму, тоже может на что сгодиться. Дальше летит – видит, компрессор пространственно-временной валяется – должно быть, странники какие-то оставили. И его тоже взял.\par
\par
Облетает звезду нейтронную, садится на планетоид, заходит во дворец к чудовищам, а те сразу к нему кидаются. \par
– Чу, – рычат, – инопланетным духом пахнет! Как звать, – спрашивают, – откуда? Как хочешь быть проглоченным – ногами вперед или назад? Да отвечай побыстрее, а то проголодались мы чудовищно – сто лет уж никого не ели.\par
– Ах вы, чудища поганые, – Джо отвечает, – не поймавши добра молодца, да кушаете! \par
Схватил канистру и давай чудовищ по мордам их гнусным бить-колотить. От пустой канистры такой грохот поднялся сильный, что чуть потолок не обсыпался. Даже чудовища струхнули малость. \par
– Да погоди, – говорят, – чего тебе надобно-то?\par
– Знаю я, есть у вас волынка самогудная, инструмент дивный. Вот её-то мне и надо!\par
– Не, – говорят чудища, – не отдадим мы её. Мы инструмент этот день и ночь стережем. А на что он тебе? \par
– Да так и так, – говорит Джо, – хочется мне отыскать корабль космический самый быстрый, сокровище это необычное, и все помыслы мои теперь лишь об этом. Вот и обещал мне Иммортий рассказать, как найти корабль космический самый быстрый, если принесу я инструмент ваш. \par
\par
«Опять он за своё! – говорят чудища. – Уж и не в первый раз!.. Ладно, слушай, давай мы тебе расскажем, как корабль космический самый быстрый отыскать, а то чего тебе туда-сюда бегать-то! И волынка самогудная целее будет. \par
\par
Путь твой далек будет. Мимо галактик в спирали, мимо планет в тентуре, мимо облаков водородных, дивным светом сияющих, мимо квазаров грозных, сигналы чудные излучающих, к самому краю видимой Вселенной, где лишь протоны да альфа-частицы шныряют, а звезду и на миллион парсек не встретишь. И всё же есть там звездочка одна, в туманности газо-пылевой спрятавшаяся. А вокруг звезды той маленькая планета вращается, а вокруг планеты – спутник вертится. И на спутнике том кратер есть огромный да глубокий. И на дне кратера того Врата стоят Звездные. И ведут эти Врата к тому, что ты найти так жаждешь.\par
\par
Но непросто до Врат добраться. Охраняет их страж из металла жидкого, ни для какого оружия не уязвимый. Ни днем, ни ночью не спит он и все смотрит внимательно, не прошмыгнул бы кто к Вратам этим. Толпы страждущих пробраться мимо него пытались, да так там в жидком металле и потонули.\par
\par
Поэтому трудно тебе придется. Но коли сумеешь до Врат добраться и через них пройти, окажешься в зале с потолком таким высоким, что и не видно. И будут пред тобой три двери – две больших да красивых, а третья – маленькая да невзрачная.\par
\par
На первой двери, золотом украшенной, нарисован будет ядерный котел над очагом звездным. За этой дверью Машина времени древняя, что прошлое изменять может да парадоксы вселенские творить. Сунешь свой нос в эту дверь – на веки сгинешь. Но если уж не послушаешь нашего совета и заглянешь туда – руками ничего не трогай, а то и вся Вселенная наша переиначиться может.\par
\par
На второй двери, алмазами выложенной, кот в сапогах нарисован. За этой дверью биолаборатория заброшенная, в которой ксеноморфов да всяких хищников страшных выводили. Они и сейчас там бродят. Такому герою, как ты, туда зайти – лицо потерять. Но уж если не послушаешь доброго совета да заглянешь туда – из пробирок не пей – ксеноморфиком станешь, или шай-хулудом каким.\par
\par
На третьей двери, паутиной затянутой да звездной пылью засыпанной, ничего не нарисовано, только написано: «Оставь надежду, всяк сюда входящий!» За этой дверью найдешь ты корабль космический самый быстрый, сокровище, которое так обрести жаждешь.\par
\par
А чтоб с пути не сбиться, дадим мы тебе навигатор звездный, куда он скажет, туда ты и поворачивай».\par
\par
Пустился Джо в путь. А кусок темной материи да воронку на место отвез – не пригодились они. Летит он в гиперпространстве гипертоннелями, которые, не иначе, какие-то гиперкроты вырыли, летит измерениями неизвестными. Уж и не знает, трехмерный ли он до сих пор. Но не сдается, боится только одного – с пути сбиться.\par
\par
Сколько световых лет прошло, неведомо, но долетел Джо до звездочки заветной. Рассчитал он курс, в посадочный модуль залез, к высадке подготовился.\par
\par
Подлетает к спутнику, а тут солнечный ветер поднялся жуткий, аж с ног сбивает. Хорошо, думает Джо, с подветренной стороны зайду – тогда меня не сразу учуют. \par
\par
 Приземлился, из посадочного модуля выбрался, хотел к Вратам бежать, да страж уж тут как тут. Идет, похожий на андроида, зеркальной краской выкрашенного, да за дезинтегратором своим тянется. \par
 \par
– Постой! – кричит ему Джо. – Мы с тобой одного металла, ты и я. \par
– Что? – кричит в ответ страж. – Я из-за ветра тебя слышу плохо. \par
– Говорю, мы с тобой одного металла, ты и я, – опять кричит Джо. – Не надо меня дезинтегрировать!\par
– Что говоришь? Тебе одного раза мало, когда надо тебя дезинтегрировать? Постой, я поближе подойду. \par
Подошел поближе, спрашивает: \par
– Так что ты сказать-то хотел? \par
– Я говорю, мы с тобой одного металла, – повторяет Джо, – поэтому меня дезинтегрировать не надо. \par
– Да? – удивляется страж. – А что же с тобой делать надо? Да и не похож ты на меня – я вон какой гладкий да зеркальный, а ты бледный какой-то. \par
– Так ты ж с твоими талантами в кого хочешь превратиться можешь. Хоть в меня, хоть в дракона с планеты Протактиний, хоть во что маленькое и безобидное. \par
– Это верно, смотри. \par
\par
Начинает тут страж переливаться всеми цветами радуги, и вдруг – бац – Джо словно сам перед собой стоит. «Что, – говорит страж, – впечатляет? Смотри дальше!» И превращается в такое ужасное чудище, каких Джо и не видел никогда, чуть рассудка от страха не лишился. «То-то, – говорит страж. – Смотри дальше!» И превращается в плитку шоколада. Лежит себе плитка, да такая аппетитная, что сама так в рот и просится. Схватил Джо плитку, да не тут-то было – плитка килограмм сто весит – не меньше. Закон сохранения массы, видать, в действии. \par
\par
А страж уж обратно в андроида зеркального превратился. \par
– Что, съел? Я, – говорит, – во что хочешь превращаться умею. Вот только в себя не могу. \par
– Это почему же? – спрашивает Джо.\par
– Да я уж во столько всего превращался, что и забыл, как вначале выглядел. \par
– Постой, роботы же никогда ничего не забывают. \par
– Сам ты робот! – говорит с обидой страж и опять за дезинтегратором тянется. – Шейпшифтер я! Шейп-шиф-тер! \par
– Да стой, не кипятись. Дай-ка я проверю, робот ты или нет, я тест знаю. \par
– Ну ладно, давай. \par
– Вот смотри, – говорит Джо, – сможешь прочитать, что тут написано? \par
А сам берет листок бумаги, пишет на нем что-то и стражу протягивает. \par
– Да тут «GJ85QR2» написано, чушь какая-то, да еще и двойной линией перечеркнуто.\par
– Похоже, и вправду ты не робот. Чего ж ты тут делаешь? \par
– Да было у нас пророчество, что кто через Звездные Врата пройдет, тот и Вселенную изменить сможет. А нам, шейпшифтерам, это ни к чему. Нас и такая Вселенная устраивает. Вот и сижу я тут, смотрю, что б никто через Врата не прошел, прям жизни никакой уж от них нет. Замучился – ни отойти куда, ни поспать. \par
– Ну, так и зачем тебе такая Вселенная-то, в которой ты ни отойти куда не можешь, ни поспать, ни друзей завести, ни животных домашних? \par
– А и верно, – говорит страж, – незачем мне всё это! Ты ведь во Врата пройти собирался? Ну и иди себе. Только просьба у меня к тебе есть: зайди в биолабораторию, посмотри, нет ли там овец электрических. Приведи мне одну, если найдешь, – уж очень я о таком животном мечтаю.\par
\par
\par
Прошел Джо сквозь Врата – видит, три двери перед ним. Вошел в нужную, несколько шагов прошел и слышит какой-то шорох сзади. Оглянулся – за ним Иммортий стоит, ухмыляется. \par
\par
– Не думал я, – говорит, – что сумеешь ты до Врат добраться, да всё же надеялся. Даже приемник телепортационный тебе в правый ботинок засунул. Очень уж мне корабль космический самый быстрый заполучить надо, гораздо нужнее, чем тебе. Так что медленно подними руки вверх и сделай три шага вперед, не мешайся.\par
– Ах ты, харя безмозглая, – Джо отвечает, – нашел я твой приемник, когда ботинки чистил, знал, что ты недоброе задумал. Оглядись вокруг, в биолабораторию ты телепортировался. Чу, хищники зубами скрежещут, клювы разевают, щупальца расправляют! Не выйти тебе отсюда, не забрать корабль космический самый быстрый.\par
– Так, значит! – говорит Иммортий. – Что ж! Посмотрим, кто отсюда живым не выйдет!\par
\par
Хватает со стола пробирку и одним махом выпивает. Раз – и стоит вместо Иммортия лев фомальгаутский ядовитый, трех метров роста, к прыжку готовится, слюна с клыков капает. Не растерялся Джо, тоже пробирку схватил, тоже выпил. Превратился в тираннозавра ригельского, махнул хвостом – лев от него на восемь метров отлетел. Схватил лев с другого стола целую колбу, осушил одним махом, превратился в пчелу бронебойную с Канопуса 4, разогнался, тираннозавра насквозь пробил. Да успел тот на последнем издыхании до пробирки дотянуться, сжевал ее с содержимым вместе, превратился в броненосца мифрильного насекомоядного с Регула 2…\par
\par
В общем, долго они так развлекались, дня три, не меньше. Чуть не забыли, кто из них кто. Уж и пробирки-то почти все закончились. Схватил Иммортий последнюю пробирку, тут Джо как закричит: «Стой, дурья твоя башка! А как мы в себя-то обратно превратимся?» Задумался Иммортий. «Всё из-за тебя, идиот, – говорит. – Теперь всё с начала начинать придется!» Пробирку бросил, на щупальцах приподнялся и выбежал из лаборатории. Джо за ним пополз.\par
\par
Выползает, смотрит – Иммортий в первую дверь, с котлом ядерным, забежал. И Джо туда направился.\par
\par
Глядит – машина там стоит дивная, лампочками моргает, жужжит тихонько и готова в прошлое отправиться хоть к сотворению Вселенной. А Иммортий уже внутри сидит, рычажки какие-то дергает и кнопки нажимает. Кинулся Джо к Иммортию, тоже внутрь залез, остановить хотел, да не успел. И исчезли оба вместе с машиной, как будто не было их тут вовсе. И сразу история эта поменялась, совсем иной стала...\par

\chapter{}
 \lettrine{У}{некоторой звезды, на четвертой от нее планете,} жил-был один осьминожец, звали его Лаврентий. Был он лучший из лучших исследователь космоса, но ума при этом был небольшого.\par
\par
Однажды прослышал он, что на одной затерянной в глубинах космоса, холодной и обледенелой планете, в алмазном криосаркофаге скрыта ригелианская красавица, да такая красивая, что все, завидев ее, сразу пред нею ниц падают и все свои злые помыслы оставляют. И захотел Лаврентий ее себе забрать, чтобы продать ее, да побольше денег заработать. Но где искать красавицу ригелианскую? Звезд да черных дыр в галактиках ужас как много, и вокруг каждой великое множество планет да астероидов вертится!\par
\par
Принялся Лаврентий думать, у кого совета спросить. Думал-думал, да надумал обратиться к известному на всю Галактику звездознатцу, профессору Грымзику. Взял он свой калькулятор, что служил ему верой и правдой во всех путешествиях, и опрометью помчался искать встречи с Грымзиком.\par
\par
Через некоторое время нашел Лаврентий способ с ним повстречаться. Так и так, сказывает Лаврентий, хочется мне найти красавицу ригелианскую, сокровище это необычное, и побуждения мои самые что ни на есть благородные. Только вот закавыка – знать не знаю, как!\par
\par
«Послушай, – отвечает ему Грымзик, – вижу я, ты весьма решителен в своем намерении, но ума не большого! Впрочем, ты осьминожец, а осьминожцы этим известны, да еще упертостью своей. Только не знаю я, как помочь тебе. Но есть сестра у меня, мудрости столь необычной, что моя мудрость по сравнению с её – желтый карлик по сравнению с Бетельгейзе. Живет она у соседней звезды, второй поворот налево, если держать курс на Малую Медведицу. Принеси ей от меня весточку – может, поможет она тебе».\par
\par
Собрал Лаврентий с собой подарки да украшения, добавил к ним нож +1 кремневый, артефактный, великой древности, и в путь пустился.\par
\par
Прилетает он к сестре Грымзика, отдает ей подарки, и письмо от Грымзика вручает. «Так и так, – говорит Лаврентий, – хочется мне найти красавицу ригелианскую, сокровище это удивительное, и все мысли мои теперь лишь об этом.»\par
\par
«Погоди-ка, – говорит сестра, – тут в письме написано, что б я тебе голову тупым топором отрубила, а не помогать стала!.. Ах нет, извини, просто письмо с другой стороны какого-то черновика написано, там даже печать есть... Ладно, помогу я тебе, хоть и не стоишь ты этого, осьминожец! Да очень уж мне подарки твои понравились, особенно нож +1 кремневый, антикварный, великой древности.\par
\par
Путь твой далек будет. Мимо галактик в спирали, мимо планет в тентуре, мимо облаков водородных, дивным светом сияющих, мимо квазаров грозных, гравитационные волны излучающих, к самому краю космоса, где лишь протоны да альфа-частицы шныряют, а звезду и на миллион парсек не встретишь. И всё же есть там звездочка одна, молодая да пригожая. А вокруг звезды той огромная планета вращается, а вокруг планеты – спутник вертится. И на спутнике том кратер есть круглый да огромный. И на дне кратера того Врата стоят Звездные. И ведут эти Врата к тому, что ты найти так жаждешь.\par
\par
Только просто так во Врата не пройти. Охраняет их страж из металла жидкого, ни для какого оружия не уязвимый. Ни днем, ни ночью не спит он и все смотрит внимательно, не прошмыгнул бы кто к Вратам этим. Толпы страждущих пробраться мимо него пытались, да так там костьми и полегли.\par
\par
Поэтому трудно тебе придется. Но коли сумеешь до Врат добраться и через них пройти, окажешься в комнатке маленькой. И будут пред тобой три двери – две больших да красивых, а третья – маленькая да невзрачная.\par
\par
На первой двери, серебром украшенной, нарисован будет ядерный котел над очагом звездным. За этой дверью Машина времени древняя, что прошлое изменять может да парадоксы вселенские творить. Сунешь свой нос в эту дверь – на веки сгинешь. Но если уж не послушаешь моего совета и заглянешь туда – руками ничего не трогай, а то и вся Вселенная наша переиначиться может.\par
\par
На второй двери, алмазами выложенной, кот в сапогах наклеен. За этой дверью биолаборатория заброшенная, в которой ксеноморфов да всяких хищников страшных выводили. Они и сейчас там бродят. Такому герою, как ты, туда зайти – головы не сносить. Но уж если не послушаешь доброго совета да заглянешь туда – из пробирок не пей – ксеноморфиком станешь, или шай-хулудом каким.\par
\par
На третьей двери, паутиной затянутой да звездной пылью засыпанной, ничего не нарисовано, только написано: «Посторонним вход воспрещен!» За этой дверью найдешь ты красавицу ригелианскую, сокровище, которое так найти жаждешь.\par
\par
А чтоб с пути не сбиться, дам я тебе лазерную указку волшебную, куда она покажет – туда ты и направляйся».\par
\par
Пустился Лаврентий в путь. Летит он в подпространстве гипертоннелями, которые, не иначе, какие-то гиперкроты вырыли, летит измерениями неизвестными. Уж и не знает, трехмерный ли он до сих пор. Но не сдается, боится только одного – наизнанку вывернуться.\par
\par
Долго ли, коротко ли, долетел Лаврентий до звездочки заветной. Рассчитал он курс, на нужную орбиту лег, к высадке подготовился.\par
\par
Подлетает к спутнику, а тут солнечный ветер поднялся жуткий, аж с ног сбивает. Хорошо, думает Лаврентий, с подветренной стороны зайду – тогда меня не сразу учуют. \par
\par
 Приземлился, из посадочного модуля выбрался, хотел к Вратам бежать, да страж уж тут как тут. Идет, похожий на андроида, зеркальной краской выкрашенного, да за транклюкатором своим тянется. \par
 \par
– Постой! – кричит ему Лаврентий. – Мы с тобой одного металла, ты и я. \par
– Что? – кричит в ответ страж. – Я из-за ветра тебя слышу плохо. \par
– Говорю, мы с тобой одного металла, ты и я, – опять кричит Лаврентий. – Не надо меня транклюкировать!\par
– Что говоришь? Тебе одного раза мало, если надо тебя транклюкировать? Постой, я поближе подойду. \par
Подошел поближе, спрашивает: \par
– Так что ты сказать-то хотел? \par
– Я говорю, мы с тобой одного металла, – повторяет Лаврентий, – поэтому меня транклюкировать не надо. \par
– Да? – удивляется страж. – А что же с тобой делать надо? Да и не похож ты на меня – я вон какой гладкий да зеркальный, а ты бледный какой-то. \par
– Так ты ж с твоими талантами в кого хочешь превратиться можешь. Хоть в меня, хоть в дракона с планеты LV-1234, хоть во что маленькое и безобидное. \par
– Это верно, смотри. \par
\par
Начинает тут страж переливаться всеми цветами радуги, и вдруг – бац – Лаврентий словно сам перед собой стоит. «Что, – говорит страж, – впечатляет? Смотри дальше!» И превращается в такое ужасное чудище, каких Лаврентий и не видел никогда, чуть рассудка от страха не лишился. «То-то, – говорит страж. – Смотри дальше!» И превращается в плитку шоколада. Лежит себе плитка, да такая аппетитная, что сама так в рот и просится. Схватил Лаврентий плитку, да не тут-то было – плитка килограмм сто весит – не меньше. Закон сохранения массы, видать, в действии. \par
\par
А страж уж обратно в андроида зеркального превратился. \par
– Что, съел? Я, – говорит, – во что хочешь превращаться умею. Вот только в себя не могу. \par
– Это почему же? – спрашивает Лаврентий.\par
– Да я уж во столько всего превращался, что и забыл, как вначале выглядел. \par
– Постой, роботы же никогда ничего не забывают. \par
– Сам ты робот! – говорит с обидой страж и опять за транклюкатором тянется. – Шейпшифтер я! Шейп-шиф-тер! \par
– Да стой, не кипятись. Дай-ка я проверю, робот ты или нет, я тест знаю. \par
– Ну ладно, давай. \par
– Вот смотри, – говорит Лаврентий, – сможешь прочитать, что тут написано? \par
А сам берет листок бумаги, пишет на нем что-то и стражу протягивает. \par
– Да тут «КЧЩ05» написано, чушь какая-то, да еще и двойной линией перечеркнуто.\par
– Похоже, и вправду ты не робот. Чего ж ты тут делаешь? \par
– Да было у нас пророчество, что кто через Звездные Врата пройдет, тот и Вселенную изменить сможет. А нам, шейпшифтерам, это ни к чему. Нас и такая Вселенная устраивает. Вот и сижу я тут, смотрю, что б никто через Врата не прошел, прям жизни никакой уж от них нет. Замучился – ни отойти куда, ни поспать. \par
– Ну, так и зачем тебе такая Вселенная-то, в которой ты ни отойти куда не можешь, ни поспать, ни друзей завести, ни животных домашних? \par
– А и верно, – говорит страж, – незачем мне всё это! Ты ведь во Врата пройти собирался? Ну и иди себе. Только просьба у меня к тебе есть: зайди в биолабораторию, посмотри, нет ли там овец электрических. Приведи мне одну, если найдешь, – уж очень я о таком животном мечтаю.\par
\par
\par
Прошел Лаврентий сквозь Врата – видит, три двери перед ним. Убрал он паутину с самой маленькой, пыль звездную с нее отряхнул да внутрь вошел. Осмотрелся и увидел красавицу ригелианскую, то сокровище, к которому так стремился. Бросился Лаврентий к сокровищу своему, тут вдруг сзади шорох какой-то послышался. Обернулся – позади Грымзик с пола поднимается.\par
– Ох! – говорит Грымзик. – Хорошо, что ты сюда добрался, а то раньше никому не удавалось, я уж и со счета сбился.\par
– Грымзик! Ты-то здесь откуда? Да еще и на полу отдыхаешь.\par
– Так я ж тебе к правому ботинку микротелепортационный приемник прицепил, ну и 3D-видеокамеру с квадрофоническим микрофоном в придачу. Хоть и не верилось, что ты сюда доберешься. Да уж больно мне красавицу ригелианскую раздобыть надо было. Пришлось вот даже ползком телепортироваться – слишком уж маленький портальчик сделался.\par
– Стоп! Это мне ее раздобыть надо было! Вот я здесь и оказался.\par
– Ты уж извини, Лаврентий, только мне это нужнее, – говорит Грымзик. Выхватывает парализатор и стреляет. Лаврентий сразу окаменел, ни рукой, ни ногой двинуть не может. Языком еле шевелит.\par
– Ой, – говорит, – ты что, супостат, делаешь?!\par
– Да я тебя в лабораторию ближайшую сейчас сдам – для опытов. Чтоб под ногами не путался.\par
Схватил Грымзик Лаврентия за шиворот и потащил в биолабораторию по соседству.\par
\par
Затащил он Лаврентия в лабораторию, бросил там и удалился важно. Лежит Лаврентий, пошевелиться не может, ждет, когда им завтракать придут. Или обедать – не знает, что и хуже. \par
Смотрит – через дальнюю дверь стадо овец входит. Все белые, только одна черная, по крайней мере, с одной стороны. Для политкорректности. Шерсть на овцах дыбом стоит и искры по шерсти бегают размером со спаниеля. Какая-то тощая тварь с потолка попыталась на них напрыгнуть, да ее на лету молнией сшибло.\par
\par
«Эге, – думает Лаврентий, – это ж прямо электроовцы какие-то. У меня и часы от них остановились, похоже. И сервопривод шнурков в ботинках отключился. Экое абсолютное оружие. Как бы мне его себе приручить». Тут чувствует – руки-ноги опять шевелиться могут. Почесал Лаврентий в затылке, огляделся повнимательней. Снял со стены диаграмму не пойми чего огромную, быстренько на обратной стороне картинку нарисовал, дырку в середине проделал и надел на себя через голову. Стал Лаврентий похож на рекламный щит ходячий, человека-бутерброд, которого как-то в космопорту видел. Только Лаврентий стал человек-ворота. Стадо овец как его увидело – сразу побежало на новые ворота смотреть. А Лаврентий пошел к Грымзику.\par
\par
Приходит, видит – Грымзик портал для обратной телепортации готовит, а роботопомощники вокруг так и кишат. И Грымзик его увидел. «Не думал, – говорит, – что ты выбраться сможешь. Ну да неважно». И приказывает роботопомощникам очистить помещение от посторонних. Но не тут-то было. Роботы все поотключались, портал к овцам притянулся и вместе с ними схлопнулся. А у Грымзика шнурки развязались.\par
\par
Подбежал Лаврентий к Грымзику, хотел стукнуть как следует, да увернулся Грымзик, из ботинок выскочил и прочь кинулся. «Вся матрица моих надежд рухнула! – кричит. – Перезагрузка! Только она мне поможет!» Выбежал из двери, к другой двери подбежал, с котлом ядерным, и шасть за нее. Лаврентий за ним кинулся, хоть и отстал чуток.\par
\par
Глядит – машина там стоит дивная, лампочками моргает, жужжит тихонько и готова в прошлое отправиться хоть к сотворению Вселенной. А Грымзик уже внутри сидит, рычажки какие-то тянет и кнопки нажимает. Кинулся Лаврентий к Грымзику, тоже внутрь залез, остановить хотел, да не успел. И исчезли оба вместе с машиной, как будто не было их тут вовсе. И сразу сказка эта поменялась, совсем иной стала...\par

\chapter{}
 \lettrine{В}{одной далекой галактике} жил один гуманоид по имени Станислав. Был он великий герой и был на редкость умен.\par
\par
И вот услышал он от старого старика, что на одной затерянной в глубинах космоса, холодной и обледенелой планете остались неведомые артефакты древней цивилизации. И захотел тогда Станислав их себе забрать, чтобы с их помощью стать властелином всех миров и галактик. Да как разыскать артефакты неведомые? Звезд да черных дыр во Вселенной ужас как много, и вокруг каждой великое множество планет да астероидов вертится!\par
\par
Стал Станислав думать, у кого совета спросить. Думал-думал, и надумал обратиться к известному на всю Галактику звездознатцу, профессору Грымзику. Взял он свой калькулятор, что служил ему верой и правдой во всех путешествиях, и отправился к Грымзику.\par
\par
Через некоторое время нашел Станислав способ с ним увидеться. Так и так, сказывает Станислав, очень хочется мне найти артефакты неведомые, сокровище это необычное, и побуждения мои самые что ни на есть благородные. Только вот загвоздка – не знаю, как!\par
\par
«Вот что, – отвечает ему Грымзик, – вижу я, ты необычайно решителен, да и смекалист очень! Впрочем, ты гуманоид, а гуманоиды этим известны, да еще расторопностью своей. Но чтобы я тебе помочь захотел, тебе самому сначала мне помочь придется. Сумеешь задание мое исполнить – расскажу, как найти артефакты неведомые, а нет – уста мои молчание хранить будут!\par
\par
Внемли! Есть по соседству тут звезда нейтронная – третий поворот направо, если к центру Галактики лететь. Рядом с ней планетоид карликовый, а на планетоиде том два чудовища живут. Зовут их Диспрозий и Гадолиний, злобные они до ужаса, и до самых кончиков клыков темной энергией пропитаны. Есть у них барабан квантовый, кварками изукрашенный, что стучит в такт тикам Вселенной, ни на пикосекунду не умолкая, и от которого звездные скопления начинают кругом кружить да в спирали сворачиваться. Не смыкая глаз, стерегут его чудовища. Добудь мне барабан, и расскажу я тебе, где найти артефакты неведомые».\par
\par
Закручинился Станислав, да делать нечего. Полетел к чудовищам барабан добывать. Летит, а сам боится, крупной дрожью дрожит, даже поворот нужный пропустил – пришлось возвращаться. А когда возвращался, увидал пустую канистру из-под топлива термоядерного, кем-то выброшенную. Возьму, думает, ее с собой – вдруг пригодится. Летит дальше – видит, кусок темной материи в пространстве висит, весь скомканный – наверное, купец какой-то потерял. И его, думает, возьму, тоже может на что сгодиться. Дальше летит – видит, воронка гравитационная валяется – должно быть, странники какие-то оставили. И её тоже взял.\par
\par
Прилетает, садится на планетоид, заходит во дворец к чудовищам, а те сразу к нему кидаются. \par
– Чу, – говорят, – гуманоидным духом пахнет! Кто такой, – спрашивают, – откуда? Как предпочитаешь быть съеденным – с головы или с ног? Да не тяни с ответами, а то проголодались мы чудовищно – лет сто уж не ели – в животах бурчит.\par
– Ах вы, чудища поганые, – Станислав отвечает, – не поймавши добра молодца, да кушаете! \par
Схватил канистру и давай чудовищ по мордам их гнусным лупить. От пустой канистры такой грохот поднялся сильный, что чуть потолок не обсыпался. Даже чудовища струхнули малость. \par
– Да погоди, – говорят, – чего тебе надобно-то?\par
– Слышал я, есть у вас барабан квантовый, инструмент дивный. Вот его-то мне и надо!\par
– Не, – говорят чудища, – не отдадим мы его. Инструмент нам этот очень дорог. А на что он тебе? \par
– Да так, мол, и так, – говорит Станислав, – очень хочется мне найти артефакты неведомые, сокровище это удивительное, и все помыслы мои теперь лишь об этом. Вот и обещал мне Грымзик рассказать, как найти артефакты неведомые, если принесу я инструмент ваш. \par
\par
«Опять он за своё! – говорят чудища. – Уж и не в первый раз!.. Ладно, слушай, давай мы тебе расскажем, как артефакты неведомые отыскать, а то чего тебе туда-сюда бегать-то! И барабан квантовый целее будет. \par
\par
Путь твой далек будет. Мимо галактик, в спирали закрученных, мимо облаков водородных, дивным светом сияющих, мимо квазаров грозных, сигналы чудные излучающих, к самому краю космоса, где лишь протоны да альфа-частицы шныряют, а звезду и на миллион парсек не встретишь. И всё же есть там звездочка одна, в туманности газо-пылевой спрятавшаяся. А вокруг звезды той огромная планета вращается, а вокруг планеты – спутник вертится. И на спутнике том кратер есть круглый да огромный. И на дне кратера того Врата стоят Звездные. И ведут эти Врата к тому, что ты найти так жаждешь.\par
\par
Но непросто до Врат добраться. Лабиринт вкруг тех Врат выстроен в сто этажей да в десять тысяч комнат на каждом. Да такой хитрый, что как войдешь в него, так и заблудишься сразу. И как сквозь тот лабиринт пройти, мне неведомо.\par
\par
Поэтому трудно тебе придется. Но коли сумеешь до Врат добраться и через них пройти, окажешься в зале с потолком таким высоким, что и не видно. И будут пред тобой три двери – две больших да красивых, а третья – маленькая да невзрачная.\par
\par
На первой двери, серебром украшенной, нарисован будет ядерный котел над очагом звездным. За этой дверью Машина времени древняя, что прошлое изменять может да парадоксы вселенские творить. Сунешь свой нос в эту дверь – на веки сгинешь. Но если уж не послушаешь нашего совета и заглянешь туда – руками ничего не трогай, а то и вся Вселенная наша переиначиться может.\par
\par
На второй двери, алмазами выложенной, кот в сапогах наклеен. За этой дверью биолаборатория заброшенная, в которой ксеноморфов да всяких хищников страшных выводили. Они и сейчас там бродят. Такому герою, как ты, туда зайти – заживо съеденным быть. Но уж если не послушаешь доброго совета да заглянешь туда – из пробирок не пей – ксеноморфиком станешь, или шай-хулудом каким.\par
\par
На третьей двери, паутиной затянутой да звездной пылью засыпанной, ничего не нарисовано, только написано: «Добро пожаловать!» За этой дверью найдешь ты артефакты неведомые, сокровище, которое так обрести хочешь.\par
\par
А чтоб с пути не сбиться, дадим мы тебе лазерную указку волшебную, куда она покажет – туда ты и направляйся».\par
\par
Пустился Станислав в путь. А кусок темной материи да воронку на место отвез – не пригодились они. Летит он в подпространстве тропами нехожеными, измерениями неизвестными. Уж и не знает, трехмерный ли он до сих пор. Но не сдается, боится только одного – с пути сбиться.\par
\par
Сколько световых лет прошло, неведомо, но долетел Станислав до звездочки заветной. Рассчитал он курс, в посадочный модуль залез, к высадке подготовился.\par
\par
Высадился Станислав рядом с лабиринтом, добрался до входа и внутрь вошел. Идет одним коридором, другим, третьим. Пусто везде, тихо, только его шаги эхом отдаются. До комнаты какой-то дошел, видит – дальше три коридора ведут, и в каждом будто туман клубится. А на полу написано: «Коли дураком не хочешь стать – ступай направо, либо прямо. Коли голову потерять не хочешь – ступай прямо, либо налево. Коли до смерти запуганным быть не хочешь – ступай налево, либо направо».\par
\par
Задумался Станислав, куда идти, да и пошел прямо. Идет, а в тумане уж и ничего не видать почти. Еле успевает в повороты заворачивать, да об лестницы чуть не спотыкается. И все страшнее вокруг делается. То будто летучая мышь над головой пролетит, то паутину какую-то в руку толщиной перешагивать приходится. То вдруг щупальце за ногу будто хватает да дергает. И не поймешь, то ли щупальце, то ли об лестницу споткнулся.\par
\par
Долго так Станислав в тумане пробирался, совсем устал, да и от страха дрожит – зуб на зуб не попадает. Увидел комнату какую-то, зашел в нее, дверь закрыл поплотнее и прямо на пол улегся – отдохнуть немного. Вдруг слышит – голос, прямо у себя в голове: «Хорошо, что сюда пришел. Молодец! За мной этот раунд». Станислав так на месте и подскочил. \par
\par
– Кто здесь? – спрашивает. – Что надо? Что за раунд и за кем он вообще может быть?\par
– Да не волнуйся, – говорит голос, – здесь я. Раунд – в большой игре, нас тут трое играет. Заходи в соседнюю комнату, мы тебе все объясним.\par
 \par
Входит Станислав в комнату – а там нет никого, лишь три сгустка тумана поплотнее. И как будто двое одному что-то вроде монет туманных передают.\par
– И где же тут кто? – Станислав спрашивает.\par
– Да мы тут везде, но здесь особенно, – отвечают три голоса, да прямо в голове звучат.\par
– И кто же вы такие будете?\par
– Мы – представители древней цивилизации, раса наша настолько древняя, что телесно уже и не существует вовсе. Только ментально, то есть разумом своим. И знаем мы все тайны Вселенной, и все предсказать да рассчитать можем. И скучно нам от этого необычайно. И даже говорим мы длинно и скучно, как ты мог заметить. Пробовали в рулетку играть – да каждый знает, куда шарик прикатится. Пробовали в квантовое лото играть – так и принцип неопределенности для нас не помеха в предсказаниях.\par
– А здесь-то вы что делаете?\par
– Воздвигли мы силой разума лабиринт этот, чтоб скуку развеять можно было. Разумные существа, сюда зашедшие, выбор делают. А мы играем, ставки делаем, по чьей дороге они пойдут. Ибо обладают разумные существа свободой воли, и тут наши предсказания бессильны. Так что давай, хватит отдыхать – видишь, три коридора отсюда тянутся – иди уж по какому-нибудь, да побыстрее!\par
– Не нужны мне ваши коридоры, мне к Звездным Вратам нужно!\par
– Будь любезен, не упрямься. Мы легко тебя заставить можем. Создадим мы сей же час чудовищ ментальных, ты кое-кого видел уже, и ни бластер, ни водяной пистолет тебе не помогут, поскольку будут чудовища внутри твоего разума, а не снаружи. А во сне тебе и вовсе тяжко придется.\par
\par
Видит Станислав – со всех сторон к нему уже когти и щупальца тянутся. Забился он в угол, да как закричит:\par
– Стойте, стойте, погодите! А кто из вас этих чудовищ делает?\par
– Как кто? Мы все трое.\par
– А чьи чудовища самые сильные будут?\par
Тут замерли чудовища на мгновение, а потом как начали друг с другом биться. Лапы с хвостами в разные стороны так и разлетаются. Драконы с демонами сшибаются, ангелы с гигантскими червями, орки и гоблины с эльфами да рыцарями, кальмары огромные с василисками. И над всем этим пегасы да орнитоптеры парят, и молнии сверкают. А внизу горы какие-то да болота с лесами мелькают. Даже один раз черный лотос виден был. \par
– Стойте! – Станислав кричит. – Меня ж сейчас тут совсем затопчут!\par
– Уйди, не мешайся! – три голоса отвечают. – Видишь левый коридор, там третий поворот направо и два раза налево – и придешь к своим Вратам Звездным.\par
\par
Побежал Станислав что есть духу, в минуту до Врат добрался.\par
\par
Прошел Станислав сквозь Врата – видит, три двери перед ним. Вошел в нужную, несколько шагов прошел и слышит какой-то шорох сзади. Оглянулся – за ним Грымзик стоит, ухмыляется. \par
\par
– Не думал я, – говорит, – что сумеешь ты до Врат добраться, да всё же надеялся. Даже приемник телепортационный тебе в правый ботинок засунул. Очень уж мне артефакты неведомые заполучить надо, гораздо нужнее, чем тебе. Так что медленно подними руки вверх и отойди в сторонку, не мешайся.\par
– Ах ты, харя безмозглая, – Станислав отвечает, – нашел я твой приемник, когда ботинки чистил, знал, что ты недоброе задумал. Оглядись вокруг, в биолабораторию ты телепортировался. Чу, хищники зубами скрежещут, клювы разевают, щупальца расправляют! Не выйти тебе отсюда, не забрать артефакты неведомые.\par
– Так, значит! – говорит Грымзик. – Что ж! Посмотрим, кто отсюда живым не выйдет!\par
\par
Хватает со стола пробирку и одним махом выпивает. Раз – и стоит вместо Грымзика кот фомальгаутский ядовитый, трех метров роста, к прыжку готовится, слюна с клыков капает. Не растерялся Станислав, тоже пробирку схватил, тоже выпил. Превратился в тираннозавра ригельского, махнул хвостом – кот от него на десять метров отлетел. Схватил кот с другого стола целую колбу, осушил одним махом, превратился в пчелу бронебойную с Канопуса 4, разогнался, тираннозавра насквозь пробил. Да успел тот на последнем издыхании до пробирки дотянуться, сжевал ее с содержимым вместе, превратился в броненосца мифрильного насекомоядного с Регула 2…\par
\par
В общем, долго они так развлекались, часа два, не меньше. Чуть не забыли, кто из них кто. Уж и пробирки-то почти все закончились. Схватил Грымзик последнюю пробирку, тут Станислав как закричит: «Стой, дурья твоя башка! А как мы в себя-то обратно превратимся?» Задумался Грымзик. «Всё из-за тебя, болван, – говорит. – Теперь всё с начала начинать придется!» Пробирку бросил, на щупальцах приподнялся и выбежал из лаборатории. Станислав за ним пополз.\par
\par
Выползает, смотрит – Грымзик в первую дверь, с котлом ядерным, забежал. И Станислав туда направился.\par
\par
Глядит – машина там стоит дивная, лампочками моргает, жужжит тихонько и готова в прошлое отправиться хоть к сотворению Вселенной. А Грымзик уже внутри сидит, рычажки какие-то тянет и кнопки нажимает. Кинулся Станислав к Грымзику, тоже внутрь залез, остановить хотел, да не успел. И исчезли оба вместе с машиной, как будто не было их тут вовсе. И сразу сказка эта поменялась, совсем иной стала...\par

\chapter{}
 \lettrine{Н}{а одной космической станции, которых много на просторах нашей Вселенной,} был один инопланетянин, звали его Игнат. Был он лучший из лучших воин и был на редкость умен.\par
\par
Однажды выяснил он, что у одной черной дыры, такой черной, что чернее не бывает, в алмазном криосаркофаге скрыта ригелианская красавица, да такая красивая, что все, завидев ее, сразу пред нею ниц падают и все свои злые помыслы оставляют. И надумал Игнат ее разыскать для себя, чтобы не досталась она никому и только он мог использовать это сокровище. Да как разыскать красавицу ригелианскую? Звезд да черных дыр в галактиках ужас как много, и вокруг каждой великое множество планет да астероидов вертится!\par
\par
Стал Игнат смекать, у кого информацию нужную добыть можно. Думал-думал, и надумал обратиться к мудрецу с планеты Клопадоктус по имени Завздыпопус, славившемуся своими познаниями о космосе. Собрал он все свои деньги и драгоценности, а их он копил во множестве, ибо любил очень, и опрометью помчался искать встречи с Завздыпопусом.\par
\par
Через некоторое время нашелся способ повстречаться с Завздыпопусом. Так и так, говорит Игнат, очень хочется мне отыскать красавицу ригелианскую, сокровище это удивительное, и все мысли мои теперь только об этом. Только вот проблема – понятия не имею, как!\par
\par
«Послушай, – отвечает ему Завздыпопус, – вижу я, ты весьма силен, и при этом ума великого! Впрочем, ты инопланетянин, а инопланетяне этим известны, да еще упертостью своей. Но чтобы я тебе помогать стал, тебе самому сначала мне помочь придется. Сумеешь задание мое исполнить – расскажу, как найти красавицу ригелианскую, а нет – уста мои молчание хранить будут!\par
\par
Слушай внимательно! Есть недалеко тут звезда нейтронная – третий поворот направо, если к центру Галактики лететь. Рядом с ней планетоид карликовый, а на планетоиде том два чудовища живут. Зовут их Диспрозий и Гадолиний, злобные они до ужаса, и от радиоактивности своей так и светятся. Есть у них волынка самогудная вакуумная, что играет музыку столь прекрасную, что даже звезды сверхновые, ее заслушавшись, взрываться перестают и в детство впадают. Пуще жизни чудовища её любят. Добудь мне волынку, и расскажу я тебе, где найти красавицу ригелианскую».\par
\par
Закручинился Игнат, да делать нечего. Полетел к чудовищам волынку добывать. Летит, а сам боится, крупной дрожью дрожит, даже поворот нужный пропустил – пришлось возвращаться. А когда возвращался, увидал пустую канистру из-под топлива термоядерного, кем-то выброшенную. Возьму, думает, ее с собой – вдруг пригодится. Летит дальше – видит, кусок темной материи в пространстве висит, весь скомканный – наверное, купец какой-то потерял. И его, думает, возьму, тоже может на что сгодиться. Дальше летит – видит, компрессор пространственно-временной валяется – должно быть, странники какие-то оставили. И его тоже взял.\par
\par
Прилетает, садится на планетоид, заходит во дворец к чудовищам, а те сразу к нему бросаются. \par
– Чу, – рычат, – инопланетным духом пахнет! Как звать, – спрашивают, – откуда? Как хочешь быть проглоченным – ногами вперед или назад? Да отвечай побыстрее, а то проголодались мы страшно – сто лет уж никого не ели.\par
– Ах вы, чудища поганые, – Игнат отвечает, – не поймавши бела лебедя, да кушаете! \par
Схватил канистру и давай чудовищ по мордам их гнусным бить-колотить. От пустой канистры такой грохот поднялся сильный, что чуть потолок не обсыпался. Даже чудовища струхнули малость. \par
– Да погоди, – говорят, – чего тебе надобно-то?\par
– Знаю я, есть у вас волынка самогудная, инструмент дивный. Вот её-то мне и надо!\par
– Не, – говорят чудища, – не отдадим мы её. Нам этот инструмент дороже жизни. А на что он тебе? \par
– Да так, мол, и так, – сказывает Игнат, – очень хочется мне отыскать красавицу ригелианскую, сокровище это необычное, и мысли мои самые что ни на есть благородные. Вот и обещал мне Завздыпопус рассказать, как найти красавицу ригелианскую, если принесу я инструмент ваш. \par
\par
«Опять он за своё! – говорят чудища. – Уж и не в первый раз!.. Ладно, слушай, давай мы тебе расскажем, как красавицу ригелианскую отыскать, а то чего тебе туда-сюда бегать-то! И волынка самогудная целее будет. \par
\par
Путь твой далек будет. Мимо галактик в спирали, мимо планет в тентуре, мимо туманностей звездных, дивным светом сияющих, мимо квазаров грозных, гравитационные волны излучающих, к самому краю космоса, где лишь протоны да альфа-частицы шныряют, а звезду и на миллион парсек не встретишь. И всё же есть там звездочка одна, молодая да пригожая. А вокруг звезды той маленькая планета вращается, а вокруг планеты – спутник вертится. И на спутнике том кратер есть огромный да глубокий. И на дне кратера того Врата стоят Звездные. И ведут эти Врата к тому, что ты найти так жаждешь.\par
\par
Только просто так во Врата не пройти. Живут там семь роботов-разбойников с машиной вычислительной белоснежной размеров громадных. Да такие жестокие, что каждого, кого увидят, в ящик металлический сажают да в машину вставляют, будто батарейки какие. Уж сколько отрядов космических десантников туда ни ходило – всех на батарейки извели.\par
\par
Но коли ты жив останешься да сумеешь до Врат добраться и через них пройти, окажешься в зале с потолком таким высоким, что и не видно. И будут пред тобой три двери – две больших да красивых, а третья – маленькая да невзрачная.\par
\par
На первой двери, иридием украшенной, нарисован будет ядерный котел над очагом звездным. За этой дверью Темпор, аномалия чудесная, то ли пространственно-времянная, то ли температурно-пространственная. Сунешь свой нос в эту дверь – на веки сгинешь. Но если уж не послушаешь нашего совета и заглянешь туда – руками ничего не трогай, а то и вся Вселенная наша переиначиться может.\par
\par
На второй двери, алмазами выложенной, кот в сапогах нарисован. За этой дверью биолаборатория заброшенная, в которой ксеноморфов да всяких хищников страшных выводили. Они и сейчас там бродят. Такому герою, как ты, туда зайти – заживо съеденным быть. Но уж если не послушаешь доброго совета да заглянешь туда – из пробирок не пей – ксеноморфиком станешь, или шай-хулудом каким.\par
\par
На третьей двери, паутиной затянутой да звездной пылью засыпанной, ничего не нарисовано, только написано: «Не влезай, убьет!» За этой дверью найдешь ты красавицу ригелианскую, сокровище, которое так найти жаждешь.\par
\par
А чтоб с пути не сбиться, дадим мы тебе навигатор звездный, куда он скажет, туда ты и поворачивай».\par
\par
Тут Игнат, не мешкая, в путь пустился. А кусок темной материи да воронку на место отвез – не пригодились они. Летит он в гиперпространстве гипертоннелями, которые, не иначе, какие-то гиперкроты вырыли, летит измерениями неизвестными. Уж и не знает, сколько в нем самом теперь измерений осталось. Но не сдается, боится только одного – наизнанку вывернуться.\par
\par
Долго ли, коротко ли, долетел Игнат до звездочки заветной. Рассчитал он курс, на нужную орбиту лег, к высадке подготовился.\par
\par
Приземлился в кратер, с краешку. Глядь – к нему уж робот спешит, грозный на вид, железный ящик перед собой катит.\par
– Здравствуй, – говорит, – инопланетянин! Полезай в ящик, не томи – у нас электричество почти уж совсем закончилось!\par
– Погоди, – Игнат отвечает. – Ты разве не слышал, что робот не должен причинять инопланетянину вред или своим бездействием допускать, что бы такой вред был причинен?\par
– С какой это такой стати?\par
– Да с такой! Его Величество, Император Орионский на прошлой неделе указ издал.\par
– Да мне-то до него что за дело?\par
– Его Величество не любит, чтоб его указы игнорировали, – вмиг прилетит со своим космофлотом, камня на камне тут не оставит.\par
– Ну, не знаю, – говорит робот, – пойдем с машиной нашей вычислительной посоветуемся, за главную она тут у нас. Полезай в ящик, я тебя подвезу!\par
– Ладно, – говорит Игнат.\par
Робот крышку открыл, старается его туда засунуть, да не тут-то было. Игнат руки-ноги растопырил, в ящик не влезает. \par
– Погоди, – говорит робот, – разве так в ящики залезают! \par
– Да мне-то откуда знать, я ж в них никогда не лазил! Покажи мне как надо, я и залезу. \par
– Ладно, – говорит робот, – смотри и учись. \par
Прижал робот к себе руки-ноги и в ящик кувырнулся. Игнат за ним крышку закрыл, защелку защелкнул и к звездным вратам пошел, песенку насвистывая.\par
\par
Прошел Игнат сквозь Врата – видит, три двери перед ним. Убрал он паутину с самой маленькой, пыль звездную с нее отряхнул да внутрь вошел. Осмотрелся и увидел красавицу ригелианскую, то сокровище, из-за которого покоя лишился. Бросился Игнат к сокровищу своему, тут вдруг сзади шорох какой-то послышался. Обернулся – позади Завздыпопус с пола поднимается.\par
– Ох! – говорит Завздыпопус. – Хорошо, что ты сюда добрался, а то раньше никому не удавалось, я уж и со счета сбился.\par
– Завздыпопус! Ты-то здесь откуда? Да еще и на полу отдыхаешь.\par
– Так я ж тебе к правому ботинку микротелепортационный приемник прицепил, ну и 3D-видеокамеру с квадрофоническим микрофоном в придачу. Хоть и не верилось, что ты сюда доберешься. Да уж больно мне красавицу ригелианскую раздобыть надо было. Пришлось вот даже ползком телепортироваться – слишком уж маленький портальчик сделался.\par
– Погоди! Это мне ее раздобыть надо было! Вот я здесь и оказался.\par
– Ты уж извини, Игнат, только мне это нужнее, – говорит Завздыпопус. Выхватывает парализатор и стреляет. Игнат сразу на пол шлепнулся, ни рукой, ни ногой двинуть не может. Языком еле шевелит.\par
– Ой, – говорит, – ты что, супостат, делаешь?!\par
– Да я тебя в лабораторию ближайшую сейчас сдам – для опытов. Чтоб под ногами не путался.\par
Схватил Завздыпопус Игната за шиворот и потащил в биолабораторию по соседству.\par
\par
Затащил он Игната в лабораторию, бросил там и удалился важно. Лежит Игнат, пошевелиться не может, ждет, когда им завтракать придут. Или обедать – не знает, что и лучше. \par
Смотрит – через дальнюю дверь стадо овец входит. Все белые, только одна черная, по крайней мере, с одной стороны. Для политкорректности. Шерсть на овцах дыбом стоит и искры по шерсти бегают размером со спаниеля. Какая-то тощая тварь с потолка попыталась на них напрыгнуть, да ее на лету молнией сшибло.\par
\par
«Эге, – думает Игнат, – это ж прямо электроовцы какие-то. У меня и часы от них остановились, похоже. И сервопривод шнурков в ботинках отключился. Экое абсолютное оружие. Как бы мне его себе приручить». Тут чувствует – руки-ноги опять шевелиться могут. Почесал Игнат в затылке, огляделся повнимательней. Снял со стены диаграмму не пойми чего огромную, быстренько на обратной стороне картинку нарисовал, дырку в середине проделал и надел на себя через голову. Стал Игнат похож на рекламный щит ходячий, человека-бутерброд, которого как-то в космопорту видел. Только Игнат стал человек-ворота. Стадо овец как его увидело – сразу побежало на новые ворота смотреть. А Игнат пошел к Завздыпопусу.\par
\par
Приходит, видит – Завздыпопус портал для обратной телепортации готовит, а роботопомощники вокруг так и кишат. И Завздыпопус его увидел. «Не думал, – говорит, – что ты выбраться сможешь. Ну да неважно». И приказывает роботопомощникам очистить помещение от посторонних. Но не тут-то было. Роботы все поотключались, портал к овцам притянулся и вместе с ними схлопнулся. А у Завздыпопуса шнурки развязались.\par
\par
Подбежал Игнат к Завздыпопусу, хотел стукнуть как следует, да увернулся Завздыпопус, из ботинок выскочил и прочь бросился. «Вся матрица моих надежд рухнула! – кричит. – Перезагрузка! Только она мне поможет!» Выбежал из двери, к другой двери подбежал, с котлом ядерным, и шасть за нее. Игнат за ним кинулся, хоть и отстал чуток.\par
\par
Глядит – Темпор посреди комнаты сияет, аномалия чудесная, и Завздыпопус к нему бежит. Бросился Игнат за Завздыпопусом, чтобы остановить, схватил крепко. Да изловчился Завздыпопус, качнулся, и рухнули они оба в Темпор, в параллельной Вселенной оказались. А в ней и сказка эта совсем другая...\par

\chapter{}
 \lettrine{В}{стародавние времена} был один человек, звали его Лаврентий. Был он лучший из лучших герой и был на редкость умен.\par
\par
Как-то раз узнал он, что на одной затерянной в глубинах космоса, холодной и обледенелой планете остались неведомые артефакты древней цивилизации. И надумал Лаврентий их себе забрать, чтобы использовать их для достижения счастья всех существ во Вселенной. Но где искать артефакты неведомые? Звезд да черных дыр в галактиках ужас как много, и вокруг каждой куча планет да астероидов вертится!\par
\par
Принялся Лаврентий смекать, у кого совета спросить. Думал-думал, и надумал обратиться к мудрецу с планеты Клопадоктус по имени Завздыпопус, славившемуся своими познаниями о космосе. Взял он роботопомошников своих верных, надел шлем парадный и помчался искать аудиенции у Завздыпопуса.\par
\par
Много ли времени прошло, иль мало, но смог Лаврентий с ним повстречаться. Так, мол, и так, сказывает Лаврентий, очень хочется мне найти артефакты неведомые, сокровище это необычное, и побуждения мои самые что ни на есть благородные. Только вот проблема – понятия не имею, как!\par
\par
«Что ж, – отвечает ему Завздыпопус, – вижу я, ты весьма ловок, да и смекалист очень! Впрочем, ты человек, а люди этим известны, да еще расторопностью своей. Только не буду я помогать тебе в поисках этих, хоть и знаю, как отыскать то, что тебе нужно. Не будь я Завздыпопус!» \par
\par
«Ах так! – возмущается Лаврентий. – Да я столько парсеков до тебя отмахал, а ты мне и совета доброго дать не можешь!» – и чуть не с кулаками к Завздыпопусу бросается. \par
\par
Рассвирепел тут Завздыпопус. «Вот как, – отвечает, – что ж, преподам я тебе сейчас урок за занудство твоё – век его помнить будешь!» Выхватил Завздыпопус, откуда ни возьмись, клинок мономолекулярный, да как начнет оружием своим размахивать и всё вокруг крушить да взрывать! Еле успел Лаврентий за стул спрятаться. А Завздыпопус не унимается и напоминает уже бешеный вентилятор на полной мощности. \par
\par
Сидит Лаврентий за стулом, грустит. Вот, думает, не узнать мне теперь, где артефакты неведомые найти, только зря голову сложу. Снял он шлем с головы своей горемычной да об пол им в досаде великой изо всех сил грохнул. А шлем тут возьми да и отскочи от пола в стену, от стены – в потолок, от потолка – в шкаф, от шкафа – в стул, а от стула – прямо Завздыпопусу в лоб. Ойкнул тут Завздыпопус, за голову схватился, да как заскулит жалобно: «Что же ты, супостат, делаешь! И как тебе не стыдно только! Мне ведь без головы и дня не прожить – нужна она мне очень! Я ей мысли умные думаю, а ты вона что вытворяешь! Теперь синяк целый месяц проходить будет, на люди не покажешься! Проваливай отсюда за артефактами своими, и чтоб духу твоего здесь больше не было!\par
\par
Далеко отсюда твой путь лежит. Мимо галактик в спирали, мимо планет в тентуре, мимо облаков водородных, дивным светом сияющих, мимо квазаров грозных, сигналы чудные излучающих, к самому краю космоса, где лишь протоны да альфа-частицы шныряют, а звезду и на миллион парсек не встретишь. И всё же есть там звездочка одна, молодая да пригожая. А вокруг звезды той огромная планета вращается, а вокруг планеты – спутник вертится. И на спутнике том кратер есть огромный да глубокий. И на дне кратера того Врата стоят Звездные. И ведут эти Врата к тому, что ты найти так жаждешь.\par
\par
Только просто так во Врата не пройти. Лабиринт вкруг тех Врат выстроен в сто этажей да в десять тысяч комнат на каждом. Да такой хитрый, что как войдешь в него, так и заблудишься сразу. И как сквозь тот лабиринт пройти, мне неведомо.\par
\par
Но коли ты жив останешься да сумеешь до Врат добраться и через них пройти, окажешься в комнатке маленькой. И будут пред тобой три двери – две больших да красивых, а третья – маленькая да невзрачная.\par
\par
На первой двери, золотом украшенной, нарисован будет ядерный котел над очагом звездным. За этой дверью Темпор, аномалия чудесная, то ли пространственно-времянная, то ли температурно-пространственная. Сунешь свой нос в эту дверь – на веки сгинешь. Но если уж не послушаешь моего совета и заглянешь туда – руками ничего не трогай, а то и вся Вселенная наша переиначиться может.\par
\par
На второй двери, алмазами выложенной, кот в сапогах нарисован. За этой дверью биолаборатория заброшенная, в которой ксеноморфов да всяких хищников страшных выводили. Они и сейчас там бродят. Такому герою, как ты, туда зайти – головы не сносить. Но уж если не послушаешь доброго совета да заглянешь туда – из пробирок не пей – ксеноморфиком станешь, или шай-хулудом каким.\par
\par
На третьей двери, паутиной затянутой да звездной пылью засыпанной, ничего не нарисовано, только написано: «Не влезай, убьет!» За этой дверью найдешь ты артефакты неведомые, сокровище, которое так найти хочешь.\par
\par
А чтоб с пути не сбиться, дам я тебе комету путеводную, куда она полетит – туда и ты лети».\par
\par
Пустился Лаврентий в путь. Летит он в гиперпространстве тропами нехожеными, измерениями неизвестными. Уж и не знает, трехмерный ли он до сих пор. Но не сдается, боится только одного – наизнанку вывернуться.\par
\par
Сколько световых лет прошло, неведомо, но долетел Лаврентий до звездочки заветной. Рассчитал он курс, в посадочный модуль залез, к высадке подготовился.\par
\par
Высадился Лаврентий рядом с лабиринтом, добрался до входа и внутрь вошел. Идет одним коридором, другим, третьим. Пусто везде, тихо, только его шаги эхом отдаются. До комнаты какой-то дошел, видит – дальше три коридора ведут, и в каждом будто туман клубится. А на полу написано: «Коли дураком не хочешь стать – ступай направо, либо прямо. Коли голову потерять не хочешь – ступай прямо, либо налево. Коли до смерти запуганным быть не хочешь – ступай налево, либо направо».\par
\par
Задумался Лаврентий, куда идти, да и пошел направо. Идет по лабиринту, все этажи осматривает, во все комнаты заглядывает – нет ли там чего. Да ничего, кроме тумана, не видит. И уж начинают ему мерещиться в тумане щупальца страшные, пауки ужасные, да монстры с дом высотой, которым его голова нужна. Вдруг видит в одной комнате – девица-красавица стоит, да такая, что дух захватывает. Как увидел ее Лаврентий – сразу голову потерял, о Вратах позабыл.\par
\par
– Здравствуй – говорит – девица! Как тебя звать-величать? Откуда ты тут, где планета твоя родная? Хочу я с тобой на эту планету полететь.\par
– Здравствуй, – девица отвечает, – имени моего тебе всё равно не выговорить, всегда я тут была, никуда не летала.\par
– Так и что же это за место такое? Что у вас тут делается, что происходит? Пойдем отсюда выбираться, на мой корабль возвращаться.\par
– Ты, я вижу, совсем голову потерял. Лабиринт это, в сто этажей высотой да в сто тысяч комнат на каждом. У нас тут большая игра идет, а если я с тобой наружу выйду, так исчезну в тот же миг.\par
– Это как же так-то! Эка жалость. А что за игра-то?\par
– Ладно, раз уж ты ко мне пожаловал, этот раунд за мной, идем – расскажу я тебе про игру.\par
\par
К выходу из комнаты направляется, и как будто насквозь немного просвечивать начинает. Лаврентий – за ней. Прошла девица пару коридоров, по лестнице поднялась, да в какую-то дверь вошла. \par
Входит Лаврентий в комнату – а там нет никого, лишь три сгустка тумана поплотнее. И как будто двое одному что-то вроде монет туманных передают.\par
– И где же тут кто? – Лаврентий спрашивает.\par
– Да мы тут везде, но здесь особенно, – отвечают три голоса, да прямо в голове звучат.\par
– И кто же вы такие будете?\par
– Мы – представители древней цивилизации, раса наша настолько древняя, что телесно уже и не существует вовсе. Только ментально, то есть разумом своим. И знаем мы все тайны Вселенной, и все предсказать да рассчитать можем. И скучно нам от этого необычайно. Пробовали в рулетку играть – да каждый знает, куда шарик прикатится. Пробовали в квантовое лото играть – так и принцип неопределенности квантовый для нас не помеха в предсказаниях.\par
– А здесь-то вы что делаете?\par
– Воздвигли мы силой разума лабиринт этот, чтоб скуку развеять можно было. Разумные существа, сюда зашедшие, выбор делают. А мы играем, смотрим, по чьей дороге они пойдут. Ибо обладают разумные существа свободой воли, и тут наши предсказания бессильны. Так что давай, хватит отдыхать – видишь, три коридора отсюда тянутся – иди уж по какому-нибудь, да побыстрее!\par
– Не нужны мне ваши коридоры, мне к Звездным Вратам нужно!\par
– Будь любезен, не упрямься. Мы легко тебя заставить можем. Создадим мы сей же час чудовищ ментальных, ты кое-кого видел уже, и ни бластер, ни водяной пистолет тебе не помогут, поскольку будут чудовища внутри твоего разума, а не снаружи. А во сне тебе и вовсе тяжко придется.\par
\par
Видит Лаврентий – со всех сторон к нему уже когти и щупальца тянутся. Забился он в угол, да как закричит:\par
– Остановитесь, погодите! А кто из вас этих чудовищ делает?\par
– Как кто? Мы все трое.\par
– А чьи чудовища самые сильные будут?\par
Тут замерли чудовища на мгновение, а потом как начали друг с другом биться. Лапы с хвостами в разные стороны так и разлетаются. Драконы с демонами сшибаются, ангелы с гигантскими червями, орки и гоблины с эльфами да рыцарями, кальмары огромные с василисками. И над всем этим пегасы да орнитоптеры парят, и молнии сверкают. А внизу горы какие-то да болота с лесами мелькают. Даже один раз черный лотос виден был. \par
– Стойте! – Лаврентий кричит. – Меня ж сейчас тут совсем затопчут!\par
– Уйди, не мешайся! – три голоса отвечают. – Видишь левый коридор, там третий поворот направо и два раза налево – и придешь к своим Вратам Звездным.\par
\par
Побежал Лаврентий что есть духу, в минуту до Врат добрался.\par
\par
Прошел Лаврентий сквозь Врата – видит, три двери перед ним. Убрал он паутину с самой маленькой, пыль с нее отряхнул да внутрь вошел. Осмотрелся и увидел артефакты неведомые, то сокровище, к которому так стремился. Бросился Лаврентий к сокровищу своему, тут вдруг сзади покашливание какое-то раздалось. Оглянулся – позади Завздыпопус с пола встает.\par
– Ох! – говорит Завздыпопус. – Хорошо, что ты сюда добрался, а то раньше никому не удавалось, я уж и со счета сбился.\par
– Завздыпопус! Ты-то здесь откуда? Да еще и на полу отдыхаешь.\par
– Так я ж тебе к правому ботинку микротелепортационный приемник прицепил. Хоть и не верилось, что ты сюда доберешься. Да уж больно мне артефакты неведомые раздобыть надо было. Пришлось вот даже ползком телепортироваться – слишком уж маленький портальчик получился.\par
– Стоп! Это мне их раздобыть надо было! Вот я здесь и оказался.\par
– Ты уж извини, Лаврентий, только мне это нужнее, – говорит Завздыпопус. Выхватывает петрификатор и стреляет. Лаврентий сразу окаменел, ни рукой, ни ногой двинуть не может. Языком еле шевелит.\par
– Ой, – говорит, – ты что, супостат, делаешь?!\par
– Да я тебя в лабораторию ближайшую сейчас сдам – для опытов. Чтоб под ногами не путался.\par
Схватил Завздыпопус Лаврентия за шиворот и потащил в биолабораторию по соседству.\par
\par
Затащил он Лаврентия в дальний угол лаборатории, бросил там и к выходу направился. Да за что-то вроде зеленого кабеля зацепился. Тут сверху огромный цветок зубастый как упадет, Завздыпопус вмиг внутри цветка оказался, мычит что-то, ничего не разобрать.\par
\par
– Это что ж такое?! – Лаврентий спрашивает.\par
– Это я, растение говорящее, – голос отвечает.\par
– Да откуда ж ты взялось?\par
– Люди в белых халатах говорили, что их генетический эксперимент удался. И что это поможет им в борьбе с артангами.\par
– А что еще они говорили?\par
– Последние их слова были: «Нет, стой, он вкуснее!»\par
– Слушай, выплюнь ты Завздыпопуса, а то тебе плохо будет. Он гербицид.\par
– Что он делает?\par
– Гербицид – для растений ядовит.\par
– Откуда ты знаешь? Да и вообще, кто ты такой?\par
– Да я тоже растение, куст говорящий. Видишь, шевелиться не могу. А этот человек меня поисследовать хотел. Ну, теперь я его поисследую, чтоб не важничал. Сейчас, погоди, проросту только немного.\par
– Хороший ты куст, тихий. И разговаривать умеешь. Ладно, на, исследуй свой гербицид.\par
\par
Распахнулся цветок, Завздыпопус оттуда вывалился, еле дышит. Лаврентий подождал, пока руки-ноги двигаться смогут, схватил Завздыпопуса, да бегом из лаборатории. За дверь выбежал, остановился, повернулся к Завздыпопусу. «Что – говорит – довыпендривался? Твое счастье, что я по вторникам кровавых жертв не приношу. До завтра подождем». Да как треснет Завздыпопуса в ухо.\par
\par
– Стой, погоди, Лаврентий! – кричит Завздыпопус. – Осознал я свою ошибку! Давай с начала начнем.\par
– Я тебе покажу с начала! Сейчас еще раз тресну!\par
\par
Совсем перепугался тут Завздыпопус, заметался, убежать старается. А Лаврентий не отстает, того и гляди догонит и еще раз стукнет. Подбежал Завздыпопус к первой двери, с котлом ядерным, и шасть за нее. И Лаврентий за ним.\par
\par
Глядит – Темпор посреди комнаты сияет, аномалия чудесная, и Завздыпопус к нему бежит. Бросился Лаврентий за Завздыпопусом, чтобы остановить, схватил крепко. Да изловчился Завздыпопус, качнулся, и рухнули они оба в Темпор, в параллельной Вселенной оказались. А в ней и сказка эта совсем другая...\par

\chapter{}
 \lettrine{Е}{ще когда Солнце не стало сверхновой,} жил-был один робот по имени Станислав. Был он лучший из лучших исследователь космоса и был на редкость умен.\par
\par
Как-то раз вычитал он в одной старой книге, что на другом конце Галактики находится великая вычислительная машина, знающая ответы на все вопросы. И захотел Станислав ее разыскать для себя, чтобы с ее помощью стать властелином всех миров и галактик. Да как разыскать машину вычислительную? Звезд да черных дыр во Вселенной ужас как много, и вокруг каждой великое множество планет да астероидов вертится!\par
\par
Начал Станислав думать, у кого информацию нужную добыть можно. Думал-думал, да надумал обратиться к известному на всю Галактику звездознатцу, профессору Грымзику. Взял он свой бластер верный и отправился искать встречи с Грымзиком.\par
\par
Вскорости сумел Станислав с ним повстречаться. Так и так, сказывает Станислав, хочется мне отыскать машину вычислительную, сокровище это удивительное, и все помыслы мои теперь лишь об этом. Только вот проблема – понятия не имею, как!\par
\par
«Что ж, – говорит ему Грымзик, – вижу я, ты очень ловок, и IQ твой вышиной до звезд простирается! Впрочем, ты робот, а роботы этим славятся, да еще прытью своей. Но чтобы я тебе помогать стал, тебе самому сначала мне помочь придется. Сумеешь задание мое исполнить – расскажу, как найти машину вычислительную, а нет – уста мои молчание хранить будут!\par
\par
Внемли! Есть по соседству тут звезда нейтронная – третий поворот направо, если к центру Галактики лететь. Рядом с ней планетоид карликовый, а на планетоиде том два чудовища живут. Зовут их Боликус и Лёликус, злобные они до ужаса, и до самых кончиков клыков темной энергией пропитаны. Есть у них гусли со струнами космическими, такие, что каждый, кто их услышит, от радости гиперпрыжки да танцы начинает выделывать, которые другим и не снились. Не смыкая глаз, стерегут их чудовища. Добудь мне гусли, и расскажу я тебе, где найти машину вычислительную».\par
\par
Закручинился Станислав, да делать нечего. Полетел к чудовищам гусли добывать. Летит, а сам боится, потом обливается, даже поворот нужный пропустил – пришлось возвращаться. А когда возвращался, увидал пустую канистру из-под топлива термоядерного, кем-то выброшенную. Возьму, думает, ее с собой – вдруг пригодится. Летит дальше – видит, кусок темной материи в пространстве висит, весь скомканный – наверное, купец какой-то потерял. И его, думает, возьму, тоже может на что сгодиться. Дальше летит – видит, воронка гравитационная валяется – должно быть, странники какие-то оставили. И её тоже взял.\par
\par
Облетает звезду нейтронную, садится на планетоид, заходит в пещеру к чудовищам, а те сразу к нему кидаются. \par
– Чу, – говорят, – роботовым духом пахнет! Кто такой, – спрашивают, – откуда? Как предпочитаешь быть съеденным – с головы или с ног? Да отвечай побыстрее, а то проголодались мы чудовищно – сто лет уж никого не ели.\par
– Ах вы, чудища поганые, – Станислав отвечает, – не поймавши добра молодца, да кушаете! \par
Схватил канистру и давай чудовищ по мордам их гнусным лупить. От пустой канистры такой грохот поднялся сильный, что чуть потолок не обвалился. Даже чудовища струхнули малость. \par
– Постой, – говорят, – чего тебе надобно-то?\par
– Слышал я, есть у вас гусли со струнами космическими, инструмент дивный. Вот их-то мне и надо!\par
– Не, – говорят чудища, – не отдадим мы их. Инструмент нам этот очень дорог. А на что он тебе? \par
– Да так и так, – сказывает Станислав, – хочу я отыскать машину вычислительную, сокровище это удивительное, и побуждения мои самые что ни на есть благородные. Вот и обещал мне Грымзик рассказать, как найти машину вычислительную, если принесу я инструмент ваш. \par
\par
«Опять он за своё! – говорят чудища. – Уж и не в первый раз!.. Ладно, слушай, давай мы тебе расскажем, как машину вычислительную отыскать, а то чего тебе туда-сюда бегать-то! И гусли со струнами космическими целее будет. \par
\par
Путь твой далек будет. Мимо галактик, в спирали закрученных, мимо туманностей звездных, дивным светом сияющих, мимо квазаров грозных, сигналы чудные излучающих, к самому краю видимой Вселенной, где лишь протоны да альфа-частицы шныряют, а звезду и на миллион парсек не встретишь. И всё же есть там звездочка одна, в туманности газо-пылевой спрятавшаяся. А вокруг звезды той огромная планета вращается, а вокруг планеты – спутник вертится. И на спутнике том кратер есть круглый да огромный. И на дне кратера того Врата стоят Звездные. И ведут эти Врата к тому, что ты найти так жаждешь.\par
\par
Только просто так во Врата не пройти. Лабиринт вкруг тех Врат выстроен в сто этажей да в десять тысяч комнат на каждом. Да такой хитрый, что как войдешь в него, так и заблудишься сразу. И как сквозь тот лабиринт пройти, мне неведомо.\par
\par
Но коли ты жив останешься да сумеешь до Врат добраться и через них пройти, окажешься в комнатке маленькой. И будут пред тобой три двери – две больших да красивых, а третья – маленькая да невзрачная.\par
\par
На первой двери, золотом украшенной, нарисован будет ядерный котел над очагом звездным. За этой дверью Темпор, аномалия чудесная, то ли пространственно-времянная, то ли температурно-пространственная. Сунешь свой нос в эту дверь – на веки сгинешь. Но если уж не послушаешь нашего совета и заглянешь туда – руками ничего не трогай, а то и вся Вселенная наша переиначиться может.\par
\par
На второй двери, алмазами выложенной, волк в тельняшке нарисован. За этой дверью биолаборатория заброшенная, в которой ксеноморфов да всяких хищников страшных выводили. Они и сейчас там бродят. Такому герою, как ты, туда зайти – головы не сносить. Но уж если не послушаешь доброго совета да заглянешь туда – из пробирок не пей – ксеноморфиком станешь, или шай-хулудом каким.\par
\par
На третьей двери, паутиной затянутой да звездной пылью засыпанной, ничего не нарисовано, только написано: «Оставь надежду, всяк сюда входящий!» За этой дверью найдешь ты машину вычислительную, сокровище, которое так обрести стремишься.\par
\par
А чтоб ты с пути не сбился, дадим мы тебе комету путеводную, куда она полетит – туда и ты лети».\par
\par
Тут Станислав, не мешкая, в путь пустился. А кусок темной материи да воронку на место отвез – не пригодились они. Летит он в гиперпространстве гипертоннелями, которые, не иначе, какие-то гиперкроты вырыли, летит измерениями неизвестными. Уж и не знает, сколько в нем самом теперь измерений осталось. Но не сдается, боится только одного – наизнанку вывернуться.\par
\par
Долго ли, коротко ли, долетел Станислав до звездочки заветной. Рассчитал он курс, в посадочный модуль залез, к высадке подготовился.\par
\par
Высадился Станислав рядом с лабиринтом, добрался до входа и внутрь вошел. Идет одним коридором, другим, третьим. Пусто везде, тихо, только его шаги эхом отдаются. До комнаты какой-то дошел, видит – дальше три коридора ведут, и в каждом будто туман клубится. А на полу написано: «Коли дураком не хочешь стать – ступай направо, либо прямо. Коли голову потерять не хочешь – ступай прямо, либо налево. Коли до смерти запуганным быть не хочешь – ступай налево, либо направо».\par
\par
Задумался Станислав, куда идти, да и пошел направо. Идет по лабиринту, все этажи осматривает, во все комнаты заглядывает – нет ли там чего. Да ничего, кроме тумана, не видит. И уж начинают ему мерещиться в тумане щупальца страшные, пауки ужасные, да монстры с дом высотой, которым его голова нужна. Вдруг видит в одной комнате – девица-красавица стоит, да такая, что дух захватывает. Как увидел ее Станислав – сразу голову потерял, о Вратах позабыл.\par
\par
– Здравствуй – говорит – девица! Как тебя звать-величать? Откуда ты тут, где планета твоя родная? Хочу я с тобой на эту планету полететь.\par
– Здравствуй, – девица отвечает, – имени моего тебе всё равно не выговорить, всегда я тут была, никуда не летала.\par
– Так и что же это за место такое? Что у вас тут делается, что происходит? Пойдем отсюда выбираться, на мой корабль возвращаться.\par
– Ты, я вижу, совсем голову потерял. Лабиринт это, в сто этажей высотой да в сто тысяч комнат на каждом. У нас тут большая игра идет, а если я с тобой наружу выйду, так исчезну в тот же миг.\par
– Это как же так-то! Эка жалость. А что за игра-то?\par
– Ладно, раз уж ты ко мне пожаловал, этот раунд за мной, идем – расскажу я тебе про игру.\par
\par
К выходу из комнаты направляется, и как будто насквозь немного просвечивать начинает. Станислав – за ней. Прошла девица пару коридоров, по лестнице поднялась, да в какую-то дверь вошла. \par
Входит Станислав в комнату – а там нет никого, лишь три сгустка тумана поплотнее. И как будто двое одному что-то вроде монет туманных передают.\par
– И где же тут кто? – Станислав спрашивает.\par
– Да мы тут везде, но здесь особенно, – отвечают три голоса, да прямо в голове звучат.\par
– И кто же вы такие будете?\par
– Мы – представители древней цивилизации, раса наша настолько древняя, что телесно уже и не существует вовсе. Только ментально, то есть разумом своим. И знаем мы все тайны Вселенной, и все предсказать да рассчитать можем. И скучно нам от этого необычайно. Пробовали в рулетку играть – да каждый знает, куда шарик прикатится. Пробовали в квантовое лото играть – так и принцип неопределенности квантовый для нас не помеха в предсказаниях.\par
– А здесь-то вы что делаете?\par
– Воздвигли мы силой разума лабиринт этот, чтоб скуку развеять можно было. Разумные существа, сюда попавшие, выбор делают. А мы играем, ставки делаем, по чьей дороге они пойдут. Ибо обладают разумные существа свободой воли, и тут наши предсказания бессильны. Так что давай, хватит отдыхать – видишь, три коридора отсюда тянутся – иди уж по какому-нибудь, да побыстрее!\par
– Не нужны мне ваши коридоры, мне к Звездным Вратам нужно!\par
– Будь любезен, не упрямься. Мы легко тебя заставить можем. Создадим мы сей же час чудовищ ментальных, ты кое-кого видел уже, и ни бластер, ни водяной пистолет тебе не помогут, поскольку будут чудовища внутри твоего разума, а не снаружи. А во сне тебе и вовсе тяжко придется.\par
\par
Видит Станислав – со всех сторон к нему уже когти и щупальца тянутся. Забился он в угол, да как закричит:\par
– Стойте, стойте, погодите! А кто из вас этих чудовищ делает?\par
– Как кто? Мы все трое.\par
– А чьи чудовища самые сильные будут?\par
Тут замерли чудовища на мгновение, а потом как начали друг с другом биться. Лапы с хвостами в разные стороны так и разлетаются. Драконы с демонами сшибаются, ангелы с гигантскими червями, орки и гоблины с эльфами да рыцарями, кальмары огромные с василисками. И над всем этим пегасы да орнитоптеры парят, и молнии сверкают. А внизу горы какие-то да болота с лесами мелькают. Даже один раз черный лотос виден был. \par
– Стойте! – Станислав кричит. – Меня ж сейчас тут совсем затопчут!\par
– Уйди, не мешайся! – три голоса отвечают. – Видишь левый коридор, там третий поворот направо и два раза налево – и придешь к своим Вратам Звездным.\par
\par
Побежал Станислав что есть духу, в минуту до Врат добрался.\par
\par
Прошел Станислав сквозь Врата – видит, три двери перед ним. Убрал он паутину с самой маленькой, пыль с нее отряхнул да внутрь вошел. Осмотрелся и увидел машину вычислительную, то сокровище, к которому так стремился. Бросился Станислав к сокровищу своему, тут вдруг сзади шорох какой-то послышался. Оглянулся – позади Грымзик с пола встает.\par
– Ох! – говорит Грымзик. – Хорошо, что ты сюда добрался, а то раньше никому не удавалось, я уж и со счета сбился.\par
– Грымзик! Ты-то здесь откуда? Да еще и на полу отдыхаешь.\par
– Так я ж тебе к правому ботинку микротелепортационный приемник прицепил. Хоть и не верилось, что ты сюда доберешься. Да уж больно мне машину вычислительную раздобыть надо было. Пришлось вот даже ползком телепортироваться – слишком уж маленький портальчик получился.\par
– Стоп! Это мне ее раздобыть надо было! Вот я здесь и оказался.\par
– Ты уж извини, Станислав, только мне это нужнее, – говорит Грымзик. Выхватывает станнер и стреляет. Станислав сразу на пол шлепнулся, ни рукой, ни ногой двинуть не может. Языком еле ворочает.\par
– Ой, – говорит, – ты что, супостат, делаешь?!\par
– Да я тебя в лабораторию ближайшую сейчас сдам – для опытов. Чтоб под ногами не путался.\par
Схватил Грымзик Станислава за шиворот и потащил в биолабораторию по соседству.\par
\par
Затащил он Станислава в лабораторию, бросил там и удалился важно. Лежит Станислав, пошевелиться не может, ждет, когда им завтракать придут. Или обедать – не знает, что и хуже. \par
Смотрит – через дальнюю дверь стадо овец входит. Все белые, только одна черная, по крайней мере, с одной стороны. Для политкорректности. Шерсть на овцах дыбом стоит и искры по шерсти бегают размером со спаниеля. Какая-то тощая тварь с потолка попыталась на них напрыгнуть, да ее на лету молнией сшибло.\par
\par
«Эге, – думает Станислав, – это ж прямо электроовцы какие-то. У меня и часы от них остановились, похоже. И сервопривод шнурков в ботинках отключился. Экое абсолютное оружие. Как бы мне его себе приручить». Тут чувствует – руки-ноги опять шевелиться могут. Почесал Станислав в затылке, огляделся повнимательней. Снял со стены диаграмму не пойми чего огромную, быстренько на обратной стороне картинку нарисовал, дырку в середине проделал и надел на себя через голову. Стал Станислав похож на рекламный щит ходячий, человека-бутерброд, которого как-то в космопорту видел. Только Станислав стал человек-ворота. Стадо овец как его увидело – сразу побежало на новые ворота смотреть. А Станислав пошел к Грымзику.\par
\par
Приходит, видит – Грымзик портал для обратной телепортации готовит, а роботопомощники вокруг так и кишат. И Грымзик его увидел. «Не думал, – говорит, – что ты выбраться сможешь. Ну да неважно». И приказывает роботопомощникам очистить помещение от посторонних. Но не тут-то было. Роботы все поотключались, портал к овцам притянулся и вместе с ними схлопнулся. А у Грымзика шнурки развязались.\par
\par
Подбежал Станислав к Грымзику, хотел стукнуть как следует, да увернулся Грымзик, из ботинок выскочил и прочь бросился. «Вся матрица моих надежд рухнула! – кричит. – Перезагрузка! Только она мне поможет!» Выбежал из двери, к другой двери подбежал, с котлом ядерным, и шасть за нее. Станислав за ним кинулся, хоть и отстал чуток.\par
\par
Глядит – Темпор посреди комнаты сияет, аномалия чудесная, и Грымзик к нему бежит. Бросился Станислав за Грымзиком, чтобы остановить, схватил крепко. Да изловчился Грымзик, качнулся, и рухнули они оба в Темпор, в параллельной Вселенной оказались. А в ней и история эта совсем другая...\par

\chapter{}
 \lettrine{Е}{ще когда Солнце не стало сверхновой,} был один робот по имени Лаврентий. Был он лучший из лучших злодей, но ума при этом был небольшого.\par
\par
В один прекрасный день узнал он, что в одной звездной системе, на маленьком астероиде спрятан кварк-глюонный плазмомёт силы необычайной. И решил тогда Лаврентий его себе заполучить, чтобы закинуть в самый дальний угол подпространства и чтобы никто больше не мог разыскать это сокровище и не смущало оно умы смертных. Да как разыскать плазмомёт кварк-глюонный? Звезд да черных дыр в галактиках ужас как много, и вокруг каждой великое множество планет да астероидов вертится!\par
\par
Принялся Лаврентий смекать, у кого информацию нужную добыть можно. Думал-думал, и надумал обратиться к мудрецу с планеты Клопадоктус по имени Завздыпопус, славившемуся своими познаниями о космосе. Надел он свой парадный скафандр и отправился искать встречи с Завздыпопусом.\par
\par
Через некоторое время нашел Лаврентий способ с ним повстречаться. Так и так, говорит Лаврентий, очень хочется мне найти плазмомёт кварк-глюонный, сокровище это великое, и побуждения мои самые что ни на есть прекрасные. Только вот проблема – понятия не имею, как!\par
\par
«Послушай, – отвечает ему Завздыпопус, – вижу я, ты очень силен, да только IQ твой ниже плинтуса! Впрочем, ты робот, а роботы этим славятся, да еще упертостью своей. Но чтобы я тебе помогать стал, тебе службу сослужить надобно. Сделаешь всё как надо – расскажу, как найти плазмомёт кварк-глюонный, а нет – не обессудь!\par
\par
Внемли! Есть тут поблизости звезда нейтронная – третий поворот направо, если к центру Галактики лететь. Рядом с ней планетоид карликовый, а на планетоиде том два чудовища живут. Зовут их Боликус и Лёликус, страшные они до ужаса, и до самых кончиков клыков темной энергией пропитаны. Есть у них барабан квантовый, кварками изукрашенный, что стучит в такт тикам Вселенной, ни на пикосекунду не умолкая, и от которого звездные скопления начинают кругом кружить да в спирали сворачиваться. Пуще жизни чудовища его любят. Добудь мне барабан, и расскажу я тебе, где найти плазмомёт кварк-глюонный».\par
\par
Закручинился Лаврентий, да делать нечего. Полетел к чудовищам барабан добывать. Летит, а сам боится, мелкой дрожью дрожит, даже поворот нужный пропустил – пришлось возвращаться. А когда возвращался, увидал пустую канистру из-под топлива термоядерного, кем-то выброшенную. Возьму, думает, ее с собой – вдруг пригодится. Летит дальше – видит, кусок темной материи в пространстве висит, весь скомканный – наверное, купец какой-то потерял. И его, думает, возьму, тоже может на что сгодиться. Дальше летит – видит, компрессор пространственно-временной валяется – должно быть, странники какие-то оставили. И его тоже взял.\par
\par
Облетает звезду нейтронную, садится на планетоид, заходит в пещеру к чудовищам, а те сразу к нему кидаются. \par
– Чу, – рычат, – роботовым духом пахнет! Кто такой, – спрашивают, – откуда? Как предпочитаешь быть съеденным – с головы или с ног? Да отвечай побыстрее, а то проголодались мы чудовищно – сто лет уж никого не ели.\par
– Погодите вы с формальностями! – Лаврентий отвечает. – Слышал я, есть у вас барабан квантовый, инструмент дивный. Сменяйте мне его на что-нибудь – очень уж мне надо.\par
– Да ты в своем уме? – спрашивают чудища. – Нам же этот инструмент дороже жизни.\par
– Что ж, – говорит Лаврентий, – тогда дайте на инструмент этот ваш посмотреть хотя бы, а я вам за это компрессор пространственно-временной подарю.\par
– А на что это нам? – спрашивают.\par
– Так вы тогда что угодно где угодно разместить сможете. Даже вы двое в этой вот канистре поместитесь. \par
– Быть такого не может!\par
– Дайте посмотреть барабан – увидите.\par
\par
Повели чудища его в самый дальний угол самой далекой пещеры, где барабан хранили. Посмотрел Лаврентий на барабан. «Что ж, – говорит, – теперь вы смотрите». Приладил компрессор пространственно-временной к канистре и велел чудовищам туда прыгать. Прыгнули они, сидят в канистре, удивляются, что еще места много осталось. А Лаврентий заткнул канистру куском темной материи, и даже бантик завязал.\par
\par
Взял он барабан, закинул канистру подальше в глубины космоса и бегом к Завздыпопусу. Завздыпопус обрадовался: «Ох, удружил ты мне, – говорит. – Расскажу я теперь тебе, как найти плазмомёт кварк-глюонный.\par
\par
Далеко отсюда твой путь лежит. Мимо галактик в спирали, мимо планет в тентуре, мимо облаков водородных, дивным светом сияющих, мимо квазаров грозных, гравитационные волны излучающих, к самому краю космоса, где лишь протоны да альфа-частицы шныряют, а звезду и на миллион парсек не встретишь. И всё же есть там звездочка одна, молодая да пригожая. А вокруг звезды той маленькая планета вращается, а вокруг планеты – спутник вертится. И на спутнике том кратер есть огромный да глубокий. И на дне кратера того Врата стоят Звездные. И ведут эти Врата к тому, что ты найти так жаждешь.\par
\par
Но непросто до Врат добраться. Охраняет их страж из металла жидкого, ни для какого оружия не уязвимый. Ни днем, ни ночью не спит он и все смотрит внимательно, не прошмыгнул бы кто к Вратам этим. Многие смельчаки пробраться мимо него пытались, да так там костьми и полегли.\par
\par
Поэтому трудно тебе придется. Но коли сумеешь до Врат добраться и через них пройти, окажешься в комнатке маленькой. И будут пред тобой три двери – две больших да красивых, а третья – маленькая да невзрачная.\par
\par
На первой двери, иридием украшенной, нарисован будет ядерный котел над очагом звездным. За этой дверью Портал сияющий, в другие вселенные ведущий. Сунешь свой нос в эту дверь – на веки сгинешь. Но если уж не послушаешь моего совета и заглянешь туда – руками ничего не трогай, а то и вся Вселенная наша переиначиться может.\par
\par
На второй двери, алмазами выложенной, волк в тельняшке нарисован. За этой дверью биолаборатория заброшенная, в которой ксеноморфов да всяких хищников страшных выводили. Они и сейчас там бродят. Такому герою, как ты, туда зайти – лицо потерять. Но уж если не послушаешь доброго совета да заглянешь туда – из пробирок не пей – ксеноморфиком станешь, или шай-хулудом каким.\par
\par
На третьей двери, паутиной затянутой да звездной пылью засыпанной, ничего не нарисовано, только написано: «http://-Ссылка на генератор-!» За этой дверью найдешь ты плазмомёт кварк-глюонный, сокровище, которое так найти хочешь.\par
\par
А чтоб ты с пути не сбился, дам я тебе лазерную указку волшебную, куда она покажет – туда ты и направляйся».\par
\par
Пустился Лаврентий в путь. Летит он в гиперпространстве тропами нехожеными, измерениями неизвестными. Уж и не знает, трехмерный ли он до сих пор. Но не сдается, боится только одного – наизнанку вывернуться.\par
\par
Сколько световых лет прошло, неведомо, но долетел Лаврентий до звездочки заветной. Рассчитал он курс, в посадочный модуль залез, к высадке подготовился.\par
\par
Подлетает к спутнику, а тут солнечный ветер поднялся жуткий, аж с ног сбивает. Хорошо, думает Лаврентий, с подветренной стороны зайду – тогда меня не сразу учуют. \par
\par
 Приземлился, из посадочного модуля выбрался, хотел к Вратам бежать, да страж уж тут как тут. Идет, похожий на андроида, зеркальной краской выкрашенного, да за транклюкатором своим тянется. \par
 \par
– Постой! – кричит ему Лаврентий. – Мы с тобой одного металла, ты и я. \par
– Что? – кричит в ответ страж. – Я из-за ветра тебя слышу плохо. \par
– Говорю, мы с тобой одного металла, ты и я, – опять кричит Лаврентий. – Не надо меня транклюкировать!\par
– Что говоришь? Тебе одного раза мало, если надо тебя транклюкировать? Постой, я поближе подойду. \par
Подошел поближе, спрашивает: \par
– Так что ты сказать-то хотел? \par
– Я говорю, мы с тобой одного металла, – повторяет Лаврентий, – поэтому меня транклюкировать не надо. \par
– Да? – удивляется страж. – А что же с тобой делать надо? Да и не похож ты на меня – я вон какой гладкий да зеркальный, а ты бледный какой-то. \par
– Так я ж изучал, как роботы живут, вот и превратился. Ты, вон, тоже, небось, в кого захочешь – в того и превратишься. Хоть в меня, хоть в дракона с планеты Земля, хоть во что маленькое и безобидное. \par
– Это верно, смотри. \par
\par
Начинает тут страж переливаться всеми цветами радуги, и вдруг – бац – Лаврентий словно сам перед собой стоит. «Что, – говорит страж, – впечатляет? Смотри дальше!» И превращается в такое ужасное чудище, каких Лаврентий и не видел никогда, чуть с ума от страха не сошел. «То-то, – говорит страж. – Смотри дальше!» И превращается в плитку шоколада. Лежит себе плитка, да такая аппетитная, что сама так в рот и просится. Схватил Лаврентий плитку, да не тут-то было – плитка килограмм сто весит – не меньше. Закон сохранения массы, видать, в действии. \par
\par
А страж уж обратно в андроида зеркального превратился. \par
– Что, съел? Я, – говорит, – во что хочешь превращаться умею. Вот только в себя не могу. \par
– Это почему же? – спрашивает Лаврентий.\par
– Да я уж во столько всего превращался, что и забыл, как вначале выглядел. \par
– Постой, роботы же никогда ничего не забывают. \par
– Сам ты робот! – говорит с обидой страж и опять за транклюкатором тянется. – Шейпшифтер я! Шейп-шиф-тер! \par
– Да стой, не кипятись. Дай-ка я проверю, робот ты или нет, я тест знаю. \par
– Ну ладно, давай. \par
– Вот смотри, – говорит Лаврентий, – сможешь прочитать, что тут написано? \par
А сам берет листок бумаги, пишет на нем что-то и стражу протягивает. \par
– Да тут «GJ85QR2» написано, чушь какая-то, да еще и двойной линией перечеркнуто.\par
– Похоже, и вправду ты не робот. Чего ж ты тут делаешь? \par
– Да было у нас пророчество, что кто через Звездные Врата пройдет, тот и Вселенную изменить сможет. А нам, шейпшифтерам, это ни к чему. Нас и такая Вселенная устраивает. Вот и сижу я тут, смотрю, что б никто через Врата не прошел, прям жизни никакой уж от них нет. Замучился – ни отойти куда, ни поспать. \par
– Ну, так и зачем тебе такая Вселенная-то, в которой ты ни отойти куда не можешь, ни поспать, ни друзей завести, ни животных домашних? \par
– А и верно, – говорит страж, – незачем мне всё это! Ты ведь во Врата пройти собирался? Ну и иди себе. Только просьба у меня к тебе есть: зайди в биолабораторию, посмотри, нет ли там овец электрических. Приведи мне одну, если найдешь, – уж очень я о таком животном мечтаю.\par
\par
\par
Прошел Лаврентий сквозь Врата – видит, три двери перед ним. Убрал он паутину с самой маленькой, пыль звездную с нее отряхнул да внутрь вошел. Осмотрелся и увидел плазмомёт кварк-глюонный, то сокровище, к которому так стремился. Бросился Лаврентий к сокровищу своему, тут вдруг сзади шорох какой-то послышался. Обернулся – позади Завздыпопус с пола встает.\par
– Ох! – говорит Завздыпопус. – Хорошо, что ты сюда добрался, а то раньше никому не удавалось, я уж и со счета сбился.\par
– Завздыпопус! Ты-то здесь откуда? Да еще и на полу отдыхаешь.\par
– Так я ж тебе к правому ботинку микротелепортационный приемник прицепил, ну и 3D-видеокамеру с квадрофоническим микрофоном в придачу. Хоть и не верилось, что ты сюда доберешься. Да уж больно мне плазмомёт кварк-глюонный раздобыть надо было. Пришлось вот даже ползком телепортироваться – слишком уж маленький портальчик сделался.\par
– Стоп! Это мне его раздобыть надо было! Вот я здесь и оказался.\par
– Ты уж извини, Лаврентий, только мне это нужнее, – говорит Завздыпопус. Выхватывает станнер и стреляет. Лаврентий сразу на пол шлепнулся, ни рукой, ни ногой двинуть не может. Языком еле ворочает.\par
– Ой, – говорит, – ты что, супостат, делаешь?!\par
– Да я тебя в лабораторию ближайшую сейчас сдам – для опытов. Чтоб под ногами не путался.\par
Схватил Завздыпопус Лаврентия за шиворот и потащил в биолабораторию по соседству.\par
\par
Затащил он Лаврентия в дальний угол лаборатории, бросил там и к выходу направился. Да за что-то вроде зеленого кабеля зацепился. Тут сверху огромный цветок зубастый как упадет, Завздыпопус вмиг внутри цветка оказался, мычит что-то, ничего не разобрать.\par
\par
– Это что ж такое?! – Лаврентий спрашивает.\par
– Это я, растение говорящее, – голос отвечает.\par
– Да откуда ж ты взялось?\par
– Люди в белых халатах говорили, что я – интересная мутация. И что это поможет им в борьбе с артангами.\par
– А что еще они говорили?\par
– Последние их слова были: «Нет, стой, он вкуснее!»\par
– Слушай, выплюнь ты Завздыпопуса, а то тебе плохо будет. Он гербицид.\par
– Что он делает?\par
– Гербицид – для растений ядовит.\par
– Откуда ты знаешь? Да и вообще, кто ты такой?\par
– Да я тоже растение, куст говорящий. Видишь, шевелиться не могу. А этот человек меня поисследовать хотел. Ну, теперь я его поисследую, чтоб не важничал. Сейчас, погоди, проросту только немного.\par
– Хороший ты куст, тихий. И разговаривать умеешь. Ладно, на, исследуй свой гербицид.\par
\par
Распахнулся цветок, Завздыпопус оттуда вывалился, еле дышит. Лаврентий подождал, пока руки-ноги двигаться смогут, схватил Завздыпопуса, да бегом из лаборатории. За дверь выбежал, остановился, повернулся к Завздыпопусу. «Что – говорит – довыпендривался? Твое счастье, что я сегодня добрый». Да как двинет Завздыпопуса в ухо.\par
\par
– Стой, погоди, Лаврентий! – кричит Завздыпопус. – Осознал я свою ошибку! Давай с начала начнем.\par
– Я тебе покажу с начала! Сейчас еще раз тресну!\par
\par
Совсем перепугался тут Завздыпопус, заметался, убежать старается. А Лаврентий не отстает, того и гляди догонит и еще раз стукнет. Подбежал Завздыпопус к первой двери, с котлом ядерным, и шасть за нее. И Лаврентий за ним.\par
\par
Глядит – портал в параллельные миры посреди комнаты сияет и Завздыпопус к нему бежит. Бросился Лаврентий за Завздыпопусом, чтобы остановить, схватил крепко. Да изловчился Завздыпопус, качнулся, и рухнули они оба в портал, в другой Вселенной оказались. А в ней и история эта совсем по-другому сказывается...\par

\chapter{}
 \lettrine{У}{некоторой звезды, на третьей от нее планете,} жил-был один гуманоид, звали его Иван. Был он известный ученый, но ума при этом был небольшого.\par
\par
В один прекрасный день услышал он от старого старика, что в одной звездной системе, на маленьком астероиде остались неведомые артефакты древней цивилизации. И надумал тогда Иван их себе добыть, чтобы продать их, да побольше денег заработать. Да как разыскать артефакты неведомые? Звезд да черных дыр в галактиках ужас как много, и вокруг каждой великое множество планет да астероидов вертится!\par
\par
Принялся Иван смекать, у кого информацию нужную добыть можно. Думал-думал, да надумал обратиться к пророчице из звездной системы Медузия, несравненной Альтавистре, чьи пророчества всегда сбывались с точностью необычайной. Взял он свой электробаян, с коим любил коротать время, и опрометью помчался искать встречи с Альтавистрой.\par
\par
Вскорости сумел Иван с ней увидеться. Так, мол, и так, говорит Иван, очень хочется мне отыскать артефакты неведомые, сокровище это великое, и все мысли мои теперь только об этом. Только вот закавыка – понятия не имею, как!\par
\par
«Вот что, – говорит ему Альтавистра, – вижу я, ты весьма решителен в своем намерении, но ума не большого! Впрочем, ты гуманоид, а гуманоиды этим известны, да еще расторопностью своей. Только не буду я помогать тебе решить задачу эту, хоть и знаю, как отыскать то, что тебе нужно. Не будь я Альтавистра!» \par
\par
«Как же так! – возмущается Иван. – Да я столько времени на поиски тебя потратил, а ты мне и совета доброго дать не можешь!» – и чуть не с кулаками к Альтавистре бросается. \par
\par
Рассвирепела тут Альтавистра. «Вот как, – говорит, – что ж, преподам я тебе сейчас урок за занудство твоё – долго его помнить будешь!» Выхватила Альтавистра, откуда ни возьмись, лазер рентгеновский, да как начнет оружием своим размахивать и всё вокруг крушить да взрывать! Еле успел Иван за стул спрятаться. А Альтавистра не унимается и напоминает уже бешеный вентилятор на полной мощности. \par
\par
Сидит Иван за стулом, думает, как дальше быть. Да так задумался сильно, что начал на своем электробаяне клавиши перебирать негромко. Как услышала это Альтавистра, сразу будто в пляс пустилась, да давай руками-ногами дрыгать и слезы из глаз лить. «Ой, – кричит, – стой, хватит, не могу больше! В жизни не слыхивала я мелодий таких – аж зарыдать хочется! Ладно, так и быть – расскажу я тебе, как добыть артефакты неведомые.\par
\par
Далеко отсюда твой путь лежит. Мимо галактик, в спирали закрученных, мимо облаков водородных, дивным светом сияющих, мимо квазаров грозных, гравитационные волны излучающих, к самому краю космоса, где лишь протоны да альфа-частицы шныряют, а звезду и на миллион парсек не встретишь. И всё же есть там звездочка одна, молодая да пригожая. А вокруг звезды той огромная планета вращается, а вокруг планеты – спутник вертится. И на спутнике том кратер есть огромный да глубокий. И на дне кратера того Врата стоят Звездные. И ведут эти Врата к тому, что ты найти так жаждешь.\par
\par
Только просто так во Врата не пройти. Охраняет их страж из металла жидкого, ни для какого оружия не уязвимый. Ни днем, ни ночью не спит он и все смотрит внимательно, не прошмыгнул бы кто к Вратам этим. Толпы страждущих пробраться мимо него пытались, да так там костьми и полегли.\par
\par
Но коли ты жив останешься да сумеешь до Врат добраться и через них пройти, окажешься в зале с потолком таким высоким, что и не видно. И будут пред тобой три двери – две больших да красивых, а третья – маленькая да невзрачная.\par
\par
На первой двери, платиной украшенной, нарисован будет ядерный котел над очагом звездным. За этой дверью Машина времени древняя, что прошлое изменять может да парадоксы вселенские творить. Сунешь свой нос в эту дверь – на веки сгинешь. Но если уж не послушаешь моего совета и заглянешь туда – руками ничего не трогай, а то и вся Вселенная наша переиначиться может.\par
\par
На второй двери, алмазами выложенной, енот с пулеметом наклеен. За этой дверью биолаборатория заброшенная, в которой ксеноморфов да всяких хищников страшных выводили. Они и сейчас там бродят. Такому герою, как ты, туда зайти – лицо потерять. Но уж если не послушаешь доброго совета да заглянешь туда – из пробирок не пей – ксеноморфиком станешь, или шай-хулудом каким.\par
\par
На третьей двери, паутиной затянутой да звездной пылью засыпанной, ничего не нарисовано, только написано: «Посторонним вход воспрещен!» За этой дверью найдешь ты артефакты неведомые, сокровище, которое так обрести жаждешь.\par
\par
А чтоб ты с пути не сбился, дам я тебе лазерную указку волшебную, куда она покажет – туда ты и направляйся».\par
\par
Тут Иван, не мешкая, в путь пустился. Летит он в подпространстве гипертоннелями, которые, не иначе, какие-то гиперкроты вырыли, летит измерениями неизвестными. Уж и не знает, сколько в нем самом теперь измерений осталось. Но не сдается, боится только одного – наизнанку вывернуться.\par
\par
Долго ли, коротко ли, долетел Иван до звездочки заветной. Рассчитал он курс, в посадочный модуль залез, к высадке подготовился.\par
\par
Подлетает к спутнику, а тут солнечный ветер поднялся жуткий, аж с ног сбивает. Хорошо, думает Иван, с подветренной стороны зайду – тогда меня не сразу учуют. \par
\par
 Приземлился, из посадочного модуля выбрался, хотел к Вратам бежать, да страж уж тут как тут. Идет, похожий на андроида, зеркальной краской выкрашенного, да за дезинтегратором своим тянется. \par
 \par
– Постой! – кричит ему Иван. – Мы с тобой одного металла, ты и я. \par
– Что? – кричит в ответ страж. – Я из-за ветра тебя слышу плохо. \par
– Говорю, мы с тобой одного металла, ты и я, – опять кричит Иван. – Не надо меня дезинтегрировать!\par
– Что говоришь? Тебе одного раза мало, когда надо тебя дезинтегрировать? Постой, я поближе подойду. \par
Подошел поближе, спрашивает: \par
– Так что ты сказать-то хотел? \par
– Я говорю, мы с тобой одного металла, – повторяет Иван, – поэтому меня дезинтегрировать не надо. \par
– Да? – удивляется страж. – А что же с тобой делать надо? Да и не похож ты на меня – я вон какой гладкий да зеркальный, а ты бледный какой-то. \par
– Так я ж изучал, как гуманоиды живут, вот и превратился. Ты, вон, тоже, небось, в кого захочешь – в того и превратишься. Хоть в меня, хоть в монстра с планеты LV-1234, хоть во что маленькое и безобидное. \par
– Это верно, смотри. \par
\par
Начинает тут страж переливаться всеми цветами радуги, и вдруг – бац – Иван словно сам перед собой стоит. «Что, – говорит страж, – впечатляет? Смотри дальше!» И превращается в такое ужасное чудище, каких Иван и не видел никогда, чуть с ума от страха не сошел. «То-то, – говорит страж. – Смотри дальше!» И превращается в плитку шоколада. Лежит себе плитка, да такая аппетитная, что сама так в рот и просится. Схватил Иван плитку, да не тут-то было – плитка килограмм сто весит – не меньше. Закон сохранения массы, видать, в действии. \par
\par
А страж уж обратно в андроида зеркального превратился. \par
– Я, – говорит, – во что хочешь превращаться умею. Вот только в себя не могу. \par
– Это почему же? – спрашивает Иван.\par
– Да я уж во столько всего превращался, что и забыл, как вначале выглядел. \par
– Постой, роботы же никогда ничего не забывают. \par
– Сам ты робот! – говорит с обидой страж и опять за дезинтегратором тянется. – Шейпшифтер я! Шейп-шиф-тер! \par
– Да стой, не кипятись. Дай-ка я проверю, робот ты или нет, я тест знаю. \par
– Ну ладно, давай. \par
– Вот смотри, – говорит Иван, – сможешь прочитать, что тут написано? \par
А сам берет листок бумаги, пишет на нем что-то и стражу протягивает. \par
– Да тут «СВМН95РП» написано, чушь какая-то, да еще и двойной линией перечеркнуто.\par
– Похоже, и вправду ты не робот. Чего ж ты тут делаешь? \par
– Да было у нас пророчество, что кто через Звездные Врата пройдет, тот и Вселенную изменить сможет. А нам, шейпшифтерам, это ни к чему. Нас и такая Вселенная устраивает. Вот и сижу я тут, смотрю, что б никто через Врата не прошел, прям жизни никакой уж от них нет. Замучился – ни отойти куда, ни поспать. \par
– Ну, так и зачем тебе такая Вселенная-то, в которой ты ни отойти куда не можешь, ни поспать, ни друзей завести, ни животных домашних? \par
– А и верно, – говорит страж, – незачем мне всё это! Ты ведь во Врата пройти собирался? Ну и иди себе. Только просьба у меня к тебе есть: зайди в биолабораторию, посмотри, нет ли там овец электрических. Приведи мне одну, если найдешь, – уж очень я о таком животном мечтаю.\par
\par
\par
Прошел Иван сквозь Врата – видит, три двери перед ним. Вошел в нужную, несколько шагов прошел и слышит какой-то шорох сзади. Оглянулся – за ним Альтавистра стоит, ухмыляется. \par
\par
– Не думала я, – говорит, – что сумеешь ты до Врат добраться, да всё же надеялась. Даже приемник телепортационный тебе в правый ботинок засунула. Очень уж мне артефакты неведомые заполучить надо, гораздо нужнее, чем тебе. Так что медленно подними руки вверх и сделай три шага вперед по хорошему, не мешайся.\par
– Ах ты, харя безмозглая, – Иван отвечает, – обнаружил я твой приемник, когда ботинки чистил, знал, что ты недоброе задумала. Оглядись вокруг, в биолабораторию ты попала. Чу, хищники зубами скрежещут, клювы разевают, щупальца расправляют! Не выйти тебе отсюда, не забрать артефакты неведомые.\par
– Так, значит! – говорит Альтавистра. – Что ж! Посмотрим, кто отсюда живым не выйдет!\par
\par
Хватает со стола пробирку и одним махом выпивает. Раз – и стоит вместо Альтавистры лев альдебаранский ядовитый, трех метров роста, к прыжку готовится, слюна с клыков капает. Не растерялся Иван, тоже пробирку схватил, тоже выпил. Превратился в дракона ригельского, махнул хвостом – лев от него на восемь метров отлетел. Схватил лев с другого стола целую колбу, осушил одним махом, превратился в пчелу бронебойную с Канопуса 4, разогнался, дракона насквозь пробил. Да успел тот на последнем издыхании до пробирки дотянуться, сжевал ее с содержимым вместе, превратился в броненосца мифрильного насекомоядного с Регула 2…\par
\par
В общем, долго они так развлекались, дня три, не меньше. Чуть не забыли, кто из них кто. Уж и пробирки-то почти все закончились. Схватила Альтавистра последнюю пробирку, тут Иван как закричит: «Стой, дурья твоя башка! А как мы в себя-то обратно превратимся?» Задумалась Альтавистра. «Всё из-за тебя, болван, – говорит. – Теперь всё с начала начинать придется!» Пробирку бросила, на щупальцах приподнялась и выбежала из лаборатории. Иван за ней пополз.\par
\par
Выползает, смотрит – Альтавистра в первую дверь, с котлом ядерным, забежала. И Иван туда направился.\par
\par
Глядит – машина там стоит дивная, лампочками моргает, жужжит тихонько и готова в прошлое отправиться хоть к сотворению Вселенной. А Альтавистра уже внутри сидит, рычажки какие-то тянет и кнопки нажимает. Кинулся Иван к Альтавистре, тоже внутрь залез, остановить хотел, да не успел. И исчезли оба вместе с машиной, как будто не было их тут вовсе. И сразу сказка эта поменялась, совсем другой стала...\par

\chapter{}
 \lettrine{В}{эпоху Великого Расселения} был один гуманоид, звали его Игнат. Был он великий ученый, но ума при этом был небольшого.\par
\par
Как-то раз вычитал он в одной старой книге, что у какой-то из звезд остались неведомые артефакты древней цивилизации. И надумал тогда Игнат их себе заполучить, чтобы с их помощью стать властелином всех миров и галактик. Но как найти артефакты неведомые? Звезд да черных дыр во Вселенной ужас как много, и вокруг каждой куча планет да астероидов вертится!\par
\par
Принялся Игнат думать, у кого информацию нужную добыть можно. Думал-думал, да надумал обратиться к пророчице из двойной звездной системы Медузия, несравненной Альтавистре, чьи пророчества всегда сбывались с точностью необычайной. Взял он свой калькулятор, что служил ему верой и правдой во всех путешествиях, и помчался к Альтавистре.\par
\par
Вскорости нашелся способ повстречаться с Альтавистрой. Так, мол, и так, сказывает Игнат, хочется мне отыскать артефакты неведомые, сокровище это удивительное, и мысли мои самые что ни на есть благородные. Только вот проблема – знать не знаю, как!\par
\par
«Что ж, – говорит ему Альтавистра, – вижу я, ты очень ловок, но ума не большого! Впрочем, ты гуманоид, а гуманоиды этим славятся, да еще упертостью своей. Только не буду я помогать тебе в поисках этих, хоть и знаю, как отыскать то, что тебе нужно. Не будь я Альтавистра!» \par
\par
«Ах так! – говорит Игнат. – Да я к тебе через пол-галактики прилетел, а ты мне и чуть-чуть подсказать не можешь!» – и чуть не с кулаками к Альтавистре бросается. \par
\par
Рассвирепела тут Альтавистра. «Вот как, – говорит, – что ж, преподам я тебе сейчас урок за неучтивость твою – век его помнить будешь!» Выхватила Альтавистра, откуда ни возьмись, алебарду плазменную, да как начнет оружием своим размахивать и всё вокруг крушить да взрывать! Еле успел Игнат за стул спрятаться. А Альтавистра не унимается и напоминает уже бешеный вентилятор на полной мощности. \par
\par
Сидит Игнат за стулом, решает, как дальше быть. А, думает, этак всю жизнь здесь просидишь! Улучил момент, схватил стул, да как стукнет им Альтавистру изо всех сил своих немалых. Но Альтавистра тоже не промах оказалась – сумела удар молодецкий отбить. Только вот в шкаф с Полным жизнеописанием разбойника Мордона врезалась и оказалась в книгах толстенных зарыта по самую шею. Двинуться не может, лишь глазами хлопает и пыхтит недовольно. А Игнат стоит со стулом в руках, приговаривает: «Не со мной, добрым молодцем, тебе тягаться, Альтавистра! С тобой и малый ребенок справится! Говори, где найти мне артефакты неведомые, а не то хуже будет!» «Ладно, твоя взяла, – отвечает Альтавистра, – слушай!\par
\par
Далеко отсюда твой путь лежит. Мимо галактик, в спирали закрученных, мимо облаков водородных, дивным светом сияющих, мимо квазаров грозных, сигналы чудные излучающих, к самому краю видимой Вселенной, где лишь протоны да альфа-частицы шныряют, а звезду и на миллион парсек не встретишь. И всё же есть там звездочка одна, в туманности газо-пылевой спрятавшаяся. А вокруг звезды той маленькая планета вращается, а вокруг планеты – спутник вертится. И на спутнике том кратер есть огромный да глубокий. И на дне кратера того Врата стоят Звездные. И ведут эти Врата к тому, что ты найти так жаждешь.\par
\par
Только просто так во Врата не пройти. Живут там семь роботов-разбойников с машиной вычислительной белоснежной размеров громадных. Да такие жестокие, что каждого, кого встретят, в ящик металлический сажают да в машину вставляют, будто батарейки какие. Уж сколько отрядов космических десантников туда ни ходило – всех на батарейки извели.\par
\par
Поэтому трудно тебе придется. Но коли сумеешь до Врат добраться и через них пройти, окажешься в комнатке маленькой. И будут пред тобой три двери – две больших да красивых, а третья – маленькая да невзрачная.\par
\par
На первой двери, платиной украшенной, нарисован будет ядерный котел над очагом звездным. За этой дверью Машина времени древняя, что прошлое изменять может да парадоксы вселенские творить. Сунешь свой нос в эту дверь – на веки сгинешь. Но если уж не послушаешь моего совета и заглянешь туда – руками ничего не трогай, а то и вся Вселенная наша переиначиться может.\par
\par
На второй двери, алмазами выложенной, волк в тельняшке наклеен. За этой дверью биолаборатория заброшенная, в которой ксеноморфов да всяких хищников страшных выводили. Они и сейчас там бродят. Такому герою, как ты, туда зайти – головы не сносить. Но уж если не послушаешь доброго совета да заглянешь туда – из пробирок не пей – ксеноморфиком станешь, или шай-хулудом каким.\par
\par
На третьей двери, паутиной затянутой да звездной пылью засыпанной, ничего не нарисовано, только написано: «http://-Ссылка на генератор-!» За этой дверью найдешь ты артефакты неведомые, сокровище, которое так отыскать хочешь.\par
\par
А чтоб с пути не сбиться, дам я тебе лазерную указку волшебную, куда она покажет – туда ты и направляйся».\par
\par
Тут Игнат, не мешкая, в путь пустился. Летит он в подпространстве тоннелями неведомыми, измерениями неизвестными. Уж и не знает, трехмерный ли он до сих пор. Но не сдается, боится только одного – наизнанку вывернуться.\par
\par
Сколько световых лет прошло, неведомо, но долетел Игнат до звездочки заветной. Рассчитал он курс, на нужную орбиту лег, к высадке подготовился.\par
\par
Приземлился в кратер, с краешку. Глядь – к нему уж робот спешит, грозный на вид, железный ящик перед собой катит.\par
– Здравствуй, – говорит, – гуманоид! Полезай в ящик, не томи – у нас электричество почти уж совсем закончилось!\par
– Погоди, – Игнат отвечает. – Ты разве не слышал, что робот не должен причинять гуманоиду вред или своим бездействием допускать, что бы такой вред был причинен?\par
– С какой это такой стати?\par
– Да с такой! Его Величество, Император Орионский на прошлой неделе указ издал.\par
– Да мне-то до него что за дело?\par
– Его Величество шутить не любит, если что – вмиг прилетит со своим космофлотом, камня на камне тут не оставит.\par
– Ну, не знаю, – говорит робот, – пойдем с машиной нашей вычислительной посоветуемся, за главную она тут у нас. Полезай в ящик, я тебя подвезу!\par
– Ладно, – говорит Игнат.\par
Робот крышку открыл, старается его туда засунуть, да не тут-то было. Игнат руки-ноги растопырил, в ящик не влезает. \par
– Погоди, – говорит робот, – разве так в ящики залезают! \par
– Да мне-то откуда знать, я ж в них никогда не лазил! Покажи мне как надо, я и залезу. \par
– Ладно, – говорит робот, – смотри и учись. \par
Прижал робот к себе руки-ноги и в ящик кувырнулся. Игнат за ним крышку закрыл, защелку защелкнул и к звездным вратам пошел, песенку насвистывая.\par
\par
Прошел Игнат сквозь Врата – видит, три двери перед ним. Вошел в нужную, несколько шагов прошел и слышит какой-то шорох сзади. Оглянулся – за ним Альтавистра стоит, ухмыляется. \par
\par
– Не думала я, – говорит, – что сумеешь ты до Врат добраться, да всё же надеялась. Даже приемник телепортационный тебе в правый ботинок засунула. Очень уж мне артефакты неведомые заполучить надо, гораздо нужнее, чем тебе. Так что медленно подними руки вверх и сделай три шага вперед, не мешайся.\par
– Ах ты, харя безмозглая, – Игнат отвечает, – нашел я твой приемник, когда ботинки чистил, знал, что ты недоброе затеяла. Оглядись вокруг, в биолабораторию ты попала. Чу, хищники зубами скрежещут, клювы разевают, щупальца расправляют! Не выйти тебе отсюда, не забрать артефакты неведомые.\par
– Так, значит! – говорит Альтавистра. – Что ж! Посмотрим, кто отсюда живым не выйдет!\par
\par
Хватает со стола пробирку и одним махом выпивает. Раз – и стоит вместо Альтавистры бегемот альдебаранский ядовитый, трех метров роста, к прыжку готовится, слюна с клыков капает. Не растерялся Игнат, тоже пробирку схватил, тоже выпил. Превратился в дракона ригельского, махнул хвостом – бегемот от него на десять метров отлетел. Схватил бегемот с другого стола целую колбу, осушил одним махом, превратился в пчелу бронебойную с Канопуса 4, разогнался, дракона насквозь пробил. Да успел тот на последнем издыхании до пробирки дотянуться, сжевал ее с содержимым вместе, превратился в броненосца адамантинового насекомоядного с Регула 6…\par
\par
В общем, долго они так развлекались, часа два, не меньше. Чуть не забыли, кто из них кто. Уж и пробирки-то почти все закончились. Схватила Альтавистра последнюю пробирку, тут Игнат как закричит: «Стой, дурья твоя башка! А как мы в себя-то обратно превратимся?» Задумалась Альтавистра. «Всё из-за тебя, болван, – говорит. – Теперь всё с начала начинать придется!» Пробирку бросила, на щупальцах приподнялась и выбежала из лаборатории. Игнат за ней пополз.\par
\par
Выползает, смотрит – Альтавистра в первую дверь, с очагом звездным, забежала. И Игнат туда направился.\par
\par
Глядит – машина там стоит дивная, лампочками моргает, жужжит тихонько и готова в прошлое отправиться хоть к сотворению Вселенной. А Альтавистра уже внутри сидит, рычажки какие-то дергает и кнопки нажимает. Кинулся Игнат к Альтавистре, тоже внутрь залез, остановить хотел, да не успел. И исчезли оба вместе с машиной, как будто не было их тут вовсе. И сразу сказка эта поменялась, совсем другой стала...\par

\chapter{}
 \lettrine{Н}{а заре Вселенной} жил-был один человек, звали его Полуэкт. Был он великий ученый и был на редкость умен.\par
\par
Как-то раз прослышал он, что на другом конце Галактики есть скрытая библиотека с тайными знаниями обо всей Вселенной. И захотел Полуэкт ее себе заполучить, чтобы поделиться ею когда-нибудь потом со всеми жителями Галактики. Да как разыскать библиотеку? Звезд да черных дыр в галактиках ужас как много, и вокруг каждой великое множество планет да астероидов вертится!\par
\par
Принялся Полуэкт думать, у кого совета спросить. Думал-думал, и надумал обратиться к пророчице из звездной системы Медузия, несравненной Альтавистре, чьи пророчества всегда сбывались с точностью необычайной. Взял он роботопомошников своих верных, надел шлем парадный и помчался искать аудиенции у Альтавистры.\par
\par
Через некоторое время сумел Полуэкт с ней увидеться. Так, мол, и так, сказывает Полуэкт, хочется мне отыскать библиотеку, сокровище это удивительное, и все мысли мои теперь лишь об этом. Да вот загвоздка – понятия не имею, как!\par
\par
«Что ж, – отвечает ему Альтавистра, – вижу я, ты необычайно решителен в своем намерении, и при этом ума великого! Впрочем, ты человек, а люди этим славятся, да еще расторопностью своей. Только не знаю я, как помочь тебе. Но есть сестра у меня, мудрости столь необычной, что моя мудрость по сравнению с её – желтый карлик по сравнению с Бетельгейзе. Живет она у соседней звезды, второй поворот налево, если отсюда к краю Галактики лететь. Принеси ей от меня весточку – может, поможет она тебе».\par
\par
Собрал Полуэкт с собой подарки да украшения, добавил к ним плащ Арктурианский, из ткани реальности сделанный, скоплениями галактик вышитый, и в путь пустился.\par
\par
Прилетает он к сестре Альтавистры, отдает ей подарки, и письмо от Альтавистры вручает. «Так, мол, и так, – сказывает Полуэкт, – очень хочется мне отыскать библиотеку, сокровище это удивительное, и побуждения мои самые что ни на есть прекрасные.»\par
\par
«Стой-ка, – говорит сестра, – тут в письме написано, что б я тебе голову отрубила, а не помогать стала!.. Ах нет, извини, просто письмо с другой стороны какого-то черновика написано, там даже печать есть... Ладно, помогу я тебе, хоть и не стоишь ты этого, человек! Да очень уж мне подарки твои понравились, особенно плащ Арктурианский, из ткани реальности сделанный, скоплениями галактик вышитый.\par
\par
Путь твой далек будет. Мимо галактик в спирали, мимо планет в тентуре, мимо облаков водородных, дивным светом сияющих, мимо квазаров грозных, сигналы чудные излучающих, к самому краю видимой Вселенной, где лишь протоны да альфа-частицы шныряют, а звезду и на миллион парсек не встретишь. И всё же есть там звездочка одна, в туманности газо-пылевой спрятавшаяся. А вокруг звезды той маленькая планета вращается, а вокруг планеты – спутник вертится. И на спутнике том кратер есть огромный да глубокий. И на дне кратера того Врата стоят Звездные. И ведут эти Врата к тому, что ты найти так жаждешь.\par
\par
Только просто так во Врата не пройти. Живут там семь роботов-разбойников с машиной вычислительной белоснежной размеров громадных. Да такие жестокие, что каждого, кого увидят, в ящик металлический сажают да в машину вставляют, будто батарейки какие. Уж сколько смельчаков туда ни ходило – всех на батарейки извели.\par
\par
Поэтому трудно тебе придется. Но коли сумеешь до Врат добраться и через них пройти, окажешься в зале с потолком таким высоким, что и не видно. И будут пред тобой три двери – две больших да красивых, а третья – маленькая да невзрачная.\par
\par
На первой двери, иридием украшенной, нарисован будет ядерный котел над очагом звездным. За этой дверью Машина времени древняя, что прошлое изменять может да парадоксы вселенские творить. Сунешь свой нос в эту дверь – на веки сгинешь. Но если уж не послушаешь моего совета и заглянешь туда – руками ничего не трогай, а то и вся Вселенная наша переиначиться может.\par
\par
На второй двери, алмазами выложенной, жаба в скафандре наклеена. За этой дверью биолаборатория заброшенная, в которой ксеноморфов да всяких хищников страшных выводили. Они и сейчас там бродят. Такому герою, как ты, туда зайти – лицо потерять. Но уж если не послушаешь доброго совета да заглянешь туда – из пробирок не пей – ксеноморфиком станешь, или шай-хулудом каким.\par
\par
На третьей двери, паутиной затянутой да звездной пылью засыпанной, ничего не нарисовано, только написано: «Не влезай, убьет!» За этой дверью найдешь ты библиотеку, сокровище, которое так отыскать жаждешь.\par
\par
А чтоб ты с пути не сбился, дам я тебе навигатор звездный, куда он скажет, туда ты и поворачивай».\par
\par
Тут Полуэкт, не мешкая, в путь пустился. Летит он в подпространстве гипертоннелями, которые, не иначе, какие-то гиперкроты вырыли, летит измерениями неизвестными. Уж и не знает, сколько в нем самом теперь измерений осталось. Но не сдается, боится только одного – с пути сбиться.\par
\par
Сколько световых лет прошло, неведомо, но долетел Полуэкт до звездочки заветной. Рассчитал он курс, в посадочный модуль залез, к высадке подготовился.\par
\par
Приземлился в кратер, с краешку. Глядь – к нему уж робот спешит, грозный на вид, железный ящик перед собой катит.\par
– Здравствуй, – говорит, – человек! Полезай в ящик, не томи – у нас электричество почти уж совсем закончилось!\par
– Погоди, – Полуэкт отвечает. – Ты разве не слышал, что робот не должен причинять человеку вред или своим бездействием допускать, что бы такой вред был причинен?\par
– С какой это такой стати?\par
– Да с такой! Его Величество, Император Орионский на днях указ издал.\par
– Да мне-то до него что за дело?\par
– Его Величество шутить не любит, если что – вмиг прилетит со своим космофлотом, камня на камне тут не оставит.\par
– Ну, не знаю, – говорит робот, – пойдем с машиной нашей вычислительной посоветуемся, за главную она тут у нас. Полезай в ящик, я тебя подвезу!\par
– Ладно, – говорит Полуэкт.\par
Робот крышку открыл, старается его туда засунуть, да не тут-то было. Полуэкт руки-ноги растопырил, в ящик не влезает. \par
– Погоди, – говорит робот, – разве так в ящики залезают! \par
– Да мне-то откуда знать, я ж в них никогда не лазил! Покажи мне как надо, я и залезу. \par
– Ладно, – говорит робот, – смотри и учись. \par
Прижал робот к себе руки-ноги и в ящик кувырнулся. Полуэкт за ним крышку закрыл, защелку защелкнул и к звездным вратам пошел, песенку насвистывая.\par
\par
Прошел Полуэкт сквозь Врата – видит, три двери перед ним. Убрал он паутину с самой маленькой, пыль с нее отряхнул да внутрь вошел. Осмотрелся и увидел библиотеку, то сокровище, к которому так стремился. Бросился Полуэкт к сокровищу своему, тут вдруг сзади шорох какой-то послышался. Обернулся – позади Альтавистра с пола встает.\par
– Ох! – говорит Альтавистра. – Хорошо, что ты сюда добрался, а то раньше никому не удавалось, я уж и со счета сбилась.\par
– Альтавистра! Ты-то здесь откуда? Да еще и на полу отдыхаешь.\par
– Так я ж тебе к правому ботинку микротелепортационный приемник прицепила. Хоть и не верилось, что ты сюда доберешься. Да уж больно мне библиотеку раздобыть надо было. Пришлось вот даже ползком телепортироваться – слишком уж маленький портальчик получился.\par
– Погоди! Это мне ее раздобыть надо было! Вот я здесь и оказался.\par
– Ты уж извини, Полуэкт, только мне это нужнее, – говорит Альтавистра. Выхватывает парализатор и стреляет. Полуэкт сразу на пол шлепнулся, ни рукой, ни ногой двинуть не может. Языком еле ворочает.\par
– Ой, – говорит, – ты что, супостат, делаешь?!\par
– Да я тебя в лабораторию ближайшую сейчас сдам – для опытов. Чтоб под ногами не путался.\par
Схватила Альтавистра Полуэкта за шиворот и потащила в биолабораторию по соседству.\par
\par
Затащила она Полуэкта в дальний угол лаборатории, бросила там и к выходу направилась. Да обо что-то вроде зеленого кабеля споткнулась. Тут сверху огромный цветок зубастый как упадет, Альтавистра вмиг внутри цветка оказалась, мычит что-то, ничего не разобрать.\par
\par
– Это что ж такое?! – Полуэкт спрашивает.\par
– Это я, растение говорящее, – голос отвечает.\par
– Да откуда ж ты взялось?\par
– Люди в белых халатах говорили, что я – интересная мутация. И что это поможет им в борьбе с артангами.\par
– А что еще они говорили?\par
– Последние их слова были: «Нет, стой, он вкуснее!»\par
– Слушай, выплюнь ты Альтавистру, а то тебе плохо будет. Она гербицид.\par
– Что она делает?\par
– Гербицид – для растений ядовита.\par
– Откуда ты знаешь? Да и вообще, кто ты такой?\par
– Да я тоже растение, куст говорящий. Видишь, шевелиться не могу. А этот человек меня поисследовать хотел. Ну, теперь я его поисследую, чтоб не важничал. Сейчас, погоди, проросту только немного.\par
– Хороший ты куст, тихий. И разговаривать умеешь. Ладно, на, исследуй свой гербицид.\par
\par
Распахнулся цветок, Альтавистра оттуда вывалилась, еле дышит. Полуэкт подождал, пока руки-ноги двигаться смогут, схватил Альтавистру, да бегом из лаборатории. За дверь выбежал, остановился, повернулся к Альтавистре. «Что – говорит – довыпендривалась? Твое счастье, что я сегодня добрый». Да как треснет Альтавистру по лбу.\par
\par
– Стой, погоди, Полуэкт! – кричит Альтавистра. – Осознала я свою ошибку! Давай с начала начнем.\par
– Я тебе покажу с начала! Сейчас еще раз двину!\par
\par
Совсем перепугалась тут Альтавистра, заметалась, убежать старается. А Полуэкт не отстает, того и гляди догонит и еще раз треснет. Подбежала Альтавистра к первой двери, с котлом ядерным, и шасть за нее. И Полуэкт за ней.\par
\par
Глядит – машина там стоит дивная, лампочками моргает, жужжит тихонько и готова в прошлое отправиться хоть к сотворению Вселенной. А Альтавистра уже внутри сидит, рычажки какие-то дергает и кнопки нажимает. Кинулся Полуэкт к Альтавистре, тоже внутрь залез, остановить хотел, да не успел. И исчезли оба вместе с машиной, как будто не было их тут вовсе. И сразу сказка эта поменялась, совсем другой стала...\par

\chapter{}
 \lettrine{В}{одной далекой галактике} жил один осьминожец по имени Лаврентий. Был он могучий исследователь космоса и был на редкость умен.\par
\par
И вот прочитал он в Интернете, что в одной звездной системе, на маленьком астероиде находится великая вычислительная машина, знающая ответы на все вопросы. И решил тогда Лаврентий ее себе забрать, чтобы использовать ее для достижения счастья всех существ во Вселенной. Но где искать машину вычислительную? Звезд да черных дыр во Вселенной ужас как много, и вокруг каждой великое множество планет да астероидов вертится!\par
\par
Начал Лаврентий думать, у кого совета спросить. Думал-думал, да надумал обратиться к колдунье Морганионе, слава о деяниях которой гремела по всей Вселенной громким грохотом. Собрал он все свои деньги и драгоценности, а их он копил во множестве, ибо любил очень, и помчался искать аудиенции у Морганионы.\par
\par
Много ли времени прошло, иль мало, но нашелся способ повстречаться с Морганионой. Так, мол, и так, говорит Лаврентий, хочу я отыскать машину вычислительную, сокровище это великое, и мысли мои самые что ни на есть прекрасные. Да вот закавыка – понятия не имею, как!\par
\par
«Послушай, – отвечает ему Морганиона, – вижу я, ты необычайно храбр, и IQ твой вышиной до звезд простирается! Впрочем, ты осьминожец, а осьминожцы этим славятся, да еще прытью своей. Только не знаю я, как помочь тебе. Но есть сестра у меня, мудрости столь необычной, что моя мудрость по сравнению с её – желтый карлик по сравнению с Бетельгейзе. Живет она у соседней звезды, второй поворот налево, если держать курс на Малую Медведицу. Принеси ей подарков да украшений дорогих – может, поможет она тебе».\par
\par
Собрал Лаврентий с собой подарки да украшения, добавил к ним кольцо из цельнометаллического водорода сделанное, с формулой Вселенной выгравированной, и отправился в дорогу.\par
\par
Прилетает он к сестре Морганионы, отдает ей подарки, и письмо от Морганионы вручает. «Так и так, – говорит Лаврентий, – хочу я отыскать машину вычислительную, сокровище это удивительное, и все помыслы мои теперь только об этом.»\par
\par
«Стой-ка, – говорит сестра, – тут в письме написано, что б я тебе голову тупым топором отрубила, а не помогать стала!.. Ах нет, извини, просто письмо с другой стороны оборотки какой-то написано, там даже печать есть... Ладно, помогу я тебе, хоть и не стоишь ты этого, осьминожец! Да очень уж мне подарки твои понравились, особенно кольцо из цельнометаллического водорода сделанное, с формулой Вселенной выгравированной.\par
\par
Далеко отсюда твой путь лежит. Мимо галактик, в спирали закрученных, мимо туманностей звездных, дивным светом сияющих, мимо квазаров грозных, сигналы чудные излучающих, к самому краю видимой Вселенной, где лишь протоны да альфа-частицы шныряют, а звезду и на миллион парсек не встретишь. И всё же есть там звездочка одна, молодая да пригожая. А вокруг звезды той огромная планета вращается, а вокруг планеты – спутник вертится. И на спутнике том кратер есть огромный да глубокий. И на дне кратера того Врата стоят Звездные. И ведут эти Врата к тому, что ты найти так жаждешь.\par
\par
Только просто так во Врата не пройти. Охраняет их страж из металла жидкого, ни для какого оружия не уязвимый. Ни днем, ни ночью не спит он и все смотрит внимательно, не прошмыгнул бы кто к Вратам этим. Толпы страждущих пробраться мимо него пытались, да так там костьми и полегли.\par
\par
Но коли ты жив останешься да сумеешь до Врат добраться и через них пройти, окажешься в комнатке маленькой. И будут пред тобой три двери – две больших да красивых, а третья – маленькая да невзрачная.\par
\par
На первой двери, платиной украшенной, нарисован будет ядерный котел над очагом звездным. За этой дверью Изменитель реальности, невесть кем построенный. Сунешь свой нос в эту дверь – на веки сгинешь. Но если уж не послушаешь моего совета и заглянешь туда – руками ничего не трогай, а то и вся Вселенная наша переиначиться может.\par
\par
На второй двери, алмазами выложенной, енот с пулеметом нарисован. За этой дверью биолаборатория заброшенная, в которой ксеноморфов да всяких хищников страшных выводили. Они и сейчас там бродят. Такому герою, как ты, туда зайти – головы не сносить. Но уж если не послушаешь доброго совета да заглянешь туда – из пробирок не пей – ксеноморфиком станешь, или шай-хулудом каким.\par
\par
На третьей двери, паутиной затянутой да звездной пылью засыпанной, ничего не нарисовано, только написано: «Посторонним вход воспрещен!» За этой дверью найдешь ты машину вычислительную, сокровище, которое так найти жаждешь.\par
\par
А чтоб ты с пути не сбился, дам я тебе навигатор звездный, куда он скажет, туда ты и поворачивай».\par
\par
Пустился Лаврентий в путь. Летит он в гиперпространстве тоннелями неведомыми, измерениями неизвестными. Уж и не знает, сколько в нем самом теперь измерений осталось. Но не сдается, боится только одного – наизнанку вывернуться.\par
\par
Долго ли, коротко ли, долетел Лаврентий до звездочки заветной. Рассчитал он курс, в посадочный модуль залез, к высадке подготовился.\par
\par
Подлетает к спутнику, а тут солнечный ветер поднялся жуткий, аж с ног сбивает. Хорошо, думает Лаврентий, с подветренной стороны зайду – тогда меня не сразу учуют. \par
\par
 Приземлился, из посадочного модуля выбрался, хотел к Вратам бежать, да страж уж тут как тут. Идет, похожий на андроида, зеркальной краской выкрашенного, да за аннигилятором своим тянется. \par
 \par
– Постой! – кричит ему Лаврентий. – Мы с тобой одного металла, ты и я. \par
– Что? – кричит в ответ страж. – Я из-за ветра тебя слышу плохо. \par
– Говорю, мы с тобой одного металла, ты и я, – опять кричит Лаврентий. – Не надо меня аннигилировать!\par
– Что говоришь? Тебе одного раза мало, когда надо тебя аннигилировать? Постой, я поближе подойду. \par
Подошел поближе, спрашивает: \par
– Так что ты сказать-то хотел? \par
– Я говорю, мы с тобой одного металла, – повторяет Лаврентий, – поэтому меня аннигилировать не надо. \par
– Да? – удивляется страж. – А что же с тобой делать надо? Да и не похож ты на меня – я вон какой гладкий да зеркальный, а ты бледный какой-то. \par
– Так я ж изучал, как осьминожцы живут, вот и превратился. Ты, вон, тоже, небось, в кого захочешь – в того и превратишься. Хоть в меня, хоть в дракона с планеты Протактиний, хоть во что маленькое и безобидное. \par
– Это верно, смотри. \par
\par
Начинает тут страж переливаться всеми цветами радуги, и вдруг – бац – Лаврентий словно сам перед собой стоит. «Что, – говорит страж, – впечатляет? Смотри дальше!» И превращается в такое ужасное чудище, каких Лаврентий и не видел никогда, чуть с ума от страха не сошел. «То-то, – говорит страж. – Смотри дальше!» И превращается в плитку шоколада. Лежит себе плитка, да такая аппетитная, что сама так в рот и просится. Схватил Лаврентий плитку, да не тут-то было – плитка килограмм сто весит – не меньше. Закон сохранения массы, видать, в действии. \par
\par
А страж уж обратно в андроида зеркального превратился. \par
– Я, – говорит, – во что хочешь превращаться умею. Вот только в себя не могу. \par
– Это почему же? – спрашивает Лаврентий.\par
– Да я уж во столько всего превращался, что и забыл, как вначале выглядел. \par
– Постой, роботы же никогда ничего не забывают. \par
– Сам ты робот! – говорит с обидой страж и опять за аннигилятором тянется. – Шейпшифтер я! Шейп-шиф-тер! \par
– Да стой, не кипятись. Дай-ка я проверю, робот ты или нет, я тест знаю. \par
– Ну ладно, давай. \par
– Вот смотри, – говорит Лаврентий, – сможешь прочитать, что тут написано? \par
А сам берет листок бумаги, пишет на нем что-то и стражу протягивает. \par
Смотрит тот, листок в руках так и сяк вертит. \par
– Не, – говорит, – ответ отрицательный. Данная запись смысла не имеет. Что это тут, будто буковки какие-то неровные, да еще и двойной волнистой линией зачеркнуты? \par
– Ну, какой же ты не робот, – говорит Лаврентий, – ты типичный робот. Впрочем, ладно, вот тебе последний тест, смотри, – и на двух сторонах чистого листка что-то пишет. – Сможешь определить, правда тут написана, али ложь? \par
\par
Берет страж новый листок, читает: «На другой стороне листа этого правда написана». Переворачивает листок, видит: «На другой стороне листа этого ложь написана». Опять он листок переворачивает, опять читает. И опять, и опять, и опять. И все быстрее листок вертит, разобраться старается, правда там написана или ложь. Уж ветер от вращающегося листка подниматься начал. \par
\par
Посмотрел Лаврентий на это, да к Звездным Вратам пошел неторопливо.\par
\par
\par
Прошел Лаврентий сквозь Врата – видит, три двери перед ним. Убрал он паутину с самой маленькой, пыль с нее отряхнул да внутрь вошел. Осмотрелся и увидел машину вычислительную, то сокровище, к которому так стремился. Бросился Лаврентий к сокровищу своему, тут вдруг сзади шорох какой-то послышался. Оглянулся – позади Морганиона с пола поднимается.\par
– Ох! – говорит Морганиона. – Хорошо, что ты сюда добрался, а то раньше никому не удавалось, я уж и со счета сбилась.\par
– Морганиона! Ты-то здесь откуда? Да еще и на полу отдыхаешь.\par
– Так я ж тебе к правому ботинку микротелепортационный приемник прицепила, ну и 3D-видеокамеру с квадрофоническим микрофоном в придачу. Хоть и не верилось, что ты сюда доберешься. Да уж больно мне машину вычислительную раздобыть надо было. Пришлось вот даже ползком телепортироваться – слишком уж маленький портальчик сделался.\par
– Стоп! Это мне ее раздобыть надо было! Вот я здесь и оказался.\par
– Ты уж извини, Лаврентий, только мне это нужнее, – говорит Морганиона. Выхватывает петрификатор и стреляет. Лаврентий сразу на пол шлепнулся, ни рукой, ни ногой двинуть не может. Языком еле ворочает.\par
– Ой, – говорит, – ты что, супостат, делаешь?!\par
– Да я тебя в лабораторию ближайшую сейчас сдам – для опытов. Чтоб под ногами не путался.\par
Схватила Морганиона Лаврентия за шиворот и потащила в биолабораторию по соседству.\par
\par
Затащила она Лаврентия в дальний угол лаборатории, бросила там и к выходу направилась. Да за что-то вроде зеленого кабеля зацепилась. Тут сверху огромный цветок зубастый как упадет, Морганиона вмиг внутри цветка оказалась, мычит что-то, ничего не разобрать.\par
\par
– Это что ж такое?! – Лаврентий спрашивает.\par
– Это я, растение говорящее, – голос отвечает.\par
– Да откуда ж ты взялось?\par
– Люди в белых халатах говорили, что я – интересная мутация. И что это поможет им в борьбе с огородными вредителями.\par
– А что еще они говорили?\par
– Последние их слова были: «А где Орибазий? И Эвтаназий?»\par
– Слушай, выплюнь ты Морганиону, а то тебе плохо будет. Она гербицид.\par
– Что она делает?\par
– Гербицид – для растений ядовита.\par
– Почему? Да и вообще, кто ты такой?\par
– Да я тоже растение, куст говорящий. Видишь, шевелиться не могу. А этот человек меня поисследовать хотел. Ну, теперь я его поисследую, чтоб не важничал. Сейчас, погоди, проросту только немного.\par
– Хороший ты куст, тихий. И разговаривать умеешь. Ладно, на, исследуй свой гербицид.\par
\par
Распахнулся цветок, Морганиона оттуда вывалилась, еле дышит. Лаврентий подождал, пока руки-ноги двигаться смогут, схватил Морганиону, да бегом из лаборатории. За дверь выбежал, остановился, повернулся к Морганионе. «Что – говорит – довыпендривалась? Твое счастье, что я сегодня добрый». Да как двинет Морганиону в ухо.\par
\par
– Стой, погоди, Лаврентий! – кричит Морганиона. – Осознала я свою ошибку! Давай с начала начнем.\par
– Я тебе покажу с начала! Сейчас еще раз тресну!\par
\par
Совсем перепугалась тут Морганиона, заметалась, убежать старается. А Лаврентий не отстает, того и гляди догонит и еще раз треснет. Подбежала Морганиона к первой двери, с котлом ядерным, и шасть за нее. И Лаврентий за ней.\par
\par
Глядит – механизм там стоит дивный, лампочками моргает, жужжит тихонько и готов в любую секунду реальность изменить. А Морганиона уже рычажки какие-то дергает и кнопки нажимает. Кинулся Лаврентий к Морганионе, остановить хотел, да не успел. Прошла рябь по Вселенной и исчезли оба, как будто не было их тут вовсе. И история эта совсем другая стала...\par

\chapter{}
 \lettrine{Е}{ще когда Солнце не стало сверхновой,} жил один гуманоид по имени Игнат. Был он лучший из лучших разбойник и был на редкость умен.\par
\par
И вот прослышал он, что на другом конце Галактики спрятан кварк-глюонный плазмомёт силы необычайной. И решил Игнат его себе забрать, чтобы закинуть в самый дальний угол подпространства и чтобы никто больше не мог разыскать это сокровище и не смущало оно умы смертных. Но где искать плазмомёт кварк-глюонный? Звезд да черных дыр в галактиках ужас как много, и вокруг каждой великое множество планет да астероидов вертится!\par
\par
Принялся Игнат думать, у кого совета спросить. Думал-думал, да надумал обратиться к робостарцу Вегианскому, Иммортию, коий был старше самой Галактики, и помнил еще Большой Взрыв, при котором был ребенком. Взял он свой бластер верный и опрометью помчался к Иммортию.\par
\par
Много ли времени прошло, иль мало, но нашел Игнат способ с ним повстречаться. Так и так, сказывает Игнат, хочу я найти плазмомёт кварк-глюонный, сокровище это великое, и все помыслы мои теперь лишь об этом. Да вот проблема – не знаю, как!\par
\par
«Послушай, – говорит ему Иммортий, – вижу я, ты весьма ловок, да и смекалист очень! Впрочем, ты гуманоид, а гуманоиды этим известны, да еще прытью своей. Могу я помочь в поисках твоих, да только сначала должен ты пройти одно пустяковое испытание. Победишь меня в честном бою – помогу, а нет – пеняй на себя!» \par
\par
Выхватил Иммортий, откуда ни возьмись, алебарду плазменную, да как начнет оружием своим размахивать и всё вокруг крушить да взрывать! Еле успел Игнат за стул спрятаться. А Иммортий не унимается и напоминает уже бешеный вентилятор на полной мощности. \par
\par
Выхватил тогда Игнат свой верный бластер и решил до смерти или до победы с Иммортием биться. Невесть сколько кипела битва, да еще четыре минуты с половиною. Уж сколько сил потратили, сколько мебели повзрывали – и подумать страшно. Устал в конец Иммортий, на пол рухнул. Да и Игнат на ногах еле держится, а еще и патроны в бластере кончились. «Ладно, – говорит Иммортий, – надоел ты мне, Игнат. Расскажу тебе, где искать плазмомёт кварк-глюонный.\par
\par
Далеко отсюда твой путь лежит. Мимо галактик, в спирали закрученных, мимо облаков водородных, дивным светом сияющих, мимо квазаров грозных, сигналы чудные излучающих, к самому краю космоса, где лишь протоны да альфа-частицы шныряют, а звезду и на миллион парсек не встретишь. И всё же есть там звездочка одна, молодая да пригожая. А вокруг звезды той маленькая планета вращается, а вокруг планеты – спутник вертится. И на спутнике том кратер есть круглый да огромный. И на дне кратера того Врата стоят Звездные. И ведут эти Врата к тому, что ты найти так жаждешь.\par
\par
Только просто так во Врата не пройти. Живут там семь роботов-разбойников с машиной вычислительной белоснежной размеров громадных. Да такие жестокие, что каждого, кого встретят, в ящик металлический сажают да в машину вставляют, будто батарейки какие. Уж сколько смельчаков туда ни ходило – всех на батарейки извели.\par
\par
Но коли ты жив останешься да сумеешь до Врат добраться и через них пройти, окажешься в зале с потолком таким высоким, что и не видно. И будут пред тобой три двери – две больших да красивых, а третья – маленькая да невзрачная.\par
\par
На первой двери, платиной украшенной, нарисован будет ядерный котел над очагом звездным. За этой дверью Портал сияющий, в другие вселенные ведущий. Сунешь свой нос в эту дверь – на веки сгинешь. Но если уж не послушаешь моего совета и заглянешь туда – руками ничего не трогай, а то и вся Вселенная наша переиначиться может.\par
\par
На второй двери, алмазами выложенной, енот с пулеметом нарисован. За этой дверью биолаборатория заброшенная, в которой ксеноморфов да всяких хищников страшных выводили. Они и сейчас там бродят. Такому герою, как ты, туда зайти – заживо съеденным быть. Но уж если не послушаешь доброго совета да заглянешь туда – из пробирок не пей – ксеноморфиком станешь, или шай-хулудом каким.\par
\par
На третьей двери, паутиной затянутой да звездной пылью засыпанной, ничего не нарисовано, только написано: «Добро пожаловать!» За этой дверью найдешь ты плазмомёт кварк-глюонный, сокровище, которое так обрести жаждешь.\par
\par
А чтоб с пути не сбиться, дам я тебе комету путеводную, куда она полетит – туда и ты лети».\par
\par
Тут Игнат, не мешкая, в путь пустился. Летит он в подпространстве тоннелями неведомыми, измерениями неизвестными. Уж и не знает, сколько в нем самом теперь измерений осталось. Но не сдается, боится только одного – с пути сбиться.\par
\par
Сколько световых лет прошло, неведомо, но долетел Игнат до звездочки заветной. Рассчитал он курс, в посадочный модуль залез, к высадке подготовился.\par
\par
Приземлился в кратер, с краешку. Глядь – к нему уж робот спешит, грозный на вид, железный ящик перед собой катит.\par
– Здравствуй, – говорит, – гуманоид! Полезай в ящик, не томи – у нас электричество почти уж совсем закончилось!\par
– Погоди, – Игнат отвечает. – Ты разве не слышал, что робот не должен причинять гуманоиду вред или своим бездействием допускать, что бы такой вред был причинен?\par
– С какой это такой стати?\par
– Да с такой! Его Величество, Император Орионский на прошлой неделе указ издал.\par
– Да мне-то до него что за дело?\par
– Его Величество шутить не любит, если что – вмиг прилетит со своим космофлотом, всех лучами смерти перебьет.\par
– Ну, не знаю, – говорит робот, – пойдем с машиной нашей вычислительной посоветуемся, за главную она тут у нас. Полезай в ящик, я тебя подвезу!\par
– Ладно, – говорит Игнат.\par
Робот крышку открыл, старается его туда засунуть, да не тут-то было. Игнат руки-ноги растопырил, в ящик не влезает. \par
– Погоди, – говорит робот, – разве так в ящики залезают! \par
– Да мне-то откуда знать, я ж в них никогда не лазил! Покажи мне как надо, я и залезу. \par
– Ладно, – говорит робот, – смотри и учись. \par
Прижал робот к себе руки-ноги и в ящик кувырнулся. Игнат за ним крышку закрыл, защелку защелкнул и к звездным вратам направился, песенку насвистывая.\par
\par
Прошел Игнат сквозь Врата – видит, три двери перед ним. Убрал он паутину с самой маленькой, пыль с нее отряхнул да внутрь вошел. Осмотрелся и увидел плазмомёт кварк-глюонный, то сокровище, из-за которого покоя лишился. Бросился Игнат к сокровищу своему, тут вдруг сзади покашливание какое-то раздалось. Обернулся – позади Иммортий с пола поднимается.\par
– Ох! – говорит Иммортий. – Хорошо, что ты сюда добрался, а то раньше никому не удавалось, я уж и со счета сбился.\par
– Иммортий! Ты-то здесь откуда? Да еще и на полу отдыхаешь.\par
– Так я ж тебе к правому ботинку микротелепортационный приемник прицепил, ну и 3D-видеокамеру с квадрофоническим микрофоном в придачу. Хоть и не верилось, что ты сюда доберешься. Да уж больно мне плазмомёт кварк-глюонный раздобыть надо было. Пришлось вот даже ползком телепортироваться – слишком уж маленький портальчик сделался.\par
– Стоп! Это мне его раздобыть надо было! Вот я здесь и оказался.\par
– Ты уж извини, Игнат, только мне это нужнее, – говорит Иммортий. Выхватывает парализатор и стреляет. Игнат сразу на пол шлепнулся, ни рукой, ни ногой двинуть не может. Языком еле ворочает.\par
– Ой, – говорит, – ты что, супостат, делаешь?!\par
– Да я тебя в лабораторию ближайшую сейчас сдам – для опытов. Чтоб под ногами не путался.\par
Схватил Иммортий Игната за шиворот и потащил в биолабораторию по соседству.\par
\par
Затащил он Игната в дальний угол лаборатории, бросил там и к выходу направился. Да за что-то вроде зеленого кабеля зацепился. Тут сверху огромный цветок зубастый как упадет, Иммортий вмиг внутри цветка оказался, мычит что-то, ничего не разобрать.\par
\par
– Это что ж такое?! – Игнат спрашивает.\par
– Это я, растение говорящее, – голос отвечает.\par
– Да откуда ж ты взялось?\par
– Люди в белых халатах говорили, что я – интересная мутация. И что это поможет им в борьбе с антаранцами.\par
– А что еще они говорили?\par
– Последние их слова были: «А где Орибазий? И Эвтаназий?»\par
– Слушай, выплюнь ты Иммортия, а то тебе плохо будет. Он гербицид.\par
– Что он делает?\par
– Гербицид – для растений ядовит.\par
– Откуда ты знаешь? Да и вообще, кто ты такой?\par
– Да я тоже растение, куст говорящий. Видишь, шевелиться не могу. А этот человек меня поисследовать хотел. Ну, теперь я его поисследую, чтоб не важничал. Сейчас, погоди, проросту только немного.\par
– Хороший ты куст, тихий. И разговаривать умеешь. Ладно, на, исследуй свой гербицид.\par
\par
Распахнулся цветок, Иммортий оттуда вывалился, еле дышит. Игнат подождал, пока руки-ноги двигаться смогут, схватил Иммортия, да бегом из лаборатории. За дверь выбежал, остановился, повернулся к Иммортию. «Что – говорит – довыпендривался? Твое счастье, что я по вторникам кровавых жертв не приношу. До завтра подождем». Да как треснет Иммортия в ухо.\par
\par
– Стой, погоди, Игнат! – кричит Иммортий. – Осознал я свою ошибку! Давай с начала начнем.\par
– Я тебе покажу с начала! Сейчас еще раз тресну!\par
\par
Совсем перепугался тут Иммортий, заметался, убежать старается. А Игнат не отстает, того и гляди догонит и еще раз стукнет. Подбежал Иммортий к первой двери, с котлом ядерным, и шасть за нее. И Игнат за ним.\par
\par
Глядит – портал в параллельные миры посреди комнаты сияет и Иммортий к нему бежит. Бросился Игнат за Иммортием, чтобы остановить, схватил крепко. Да изловчился Иммортий, качнулся, и рухнули они оба в портал, в другой Вселенной оказались. А в ней и сказка эта совсем по-другому сказывается...\par

\end{document}
